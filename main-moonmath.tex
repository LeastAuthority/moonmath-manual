% with font size 10, I think there is just too much Information per page.
\documentclass[a4paper, 12pt]{scrreprt}

% border sizes
\setlength{\topmargin}{0.0in}
\setlength{\oddsidemargin}{0.33in}
\setlength{\textheight}{9.0in}
\setlength{\textwidth}{6.0in}

%fonts
% times is the font I (Mirco) thinks looks the most calm, ordered and math-textbook like
\usepackage{times}
% use times style for mathmode, too
\usepackage{mathptmx}
\usepackage[T1]{fontenc}
\usepackage[utf8]{inputenc}

% integer long division
\usepackage{longdivision}
% polynomial long division
\usepackage{polynom}

\usepackage{amssymb,amsmath,amsthm,mathrsfs,xspace,multicol}
\usepackage{polydiv}
\usepackage{amscd}
\usepackage[all]{xy}

\usepackage{algorithm}
\usepackage{algpseudocode}
% We use algorithmicX, since we need to customize commands (\algblockdefx)
% See https://ftp.agdsn.de/pub/mirrors/latex/dante/macros/latex/contrib/algorithmicx/algorithmicx.pdf for howTo customize
\usepackage{algorithmicx}


\usepackage{enumitem} % for personalizing enumeration with itemize

% the sagetex evironment
\usepackage{sagetex}
\usepackage{xcolor}
% color of the sagecommandline-tex-environment
\lstset{language=Sage,
commentstyle={\ttfamily\color{black}},
keywordstyle={\ttfamily\color{black}\bfseries},
stringstyle ={\ttfamily\color{black}\bfseries},
tabsize = 4,
basicstyle={\small \ttfamily},
backgroundcolor= \color{white},
}


%Load the following packages last
\usepackage{hyperref} % for all kinds of linke
\usepackage{cleveref}%must be loaded after `hyperref`


%margins
\usepackage[a4paper, margin=2.5cm]{geometry}


%bibliography management
\usepackage{natbib}
\bibliographystyle{unsrtnat}%we can discuss what style we want to use

%math typesetting
\usepackage{amsfonts}

% Table of Contents

\setcounter{tocdepth}{4}

%	Macros
% Fields and categories
\newcommand{\C}{\mathbb{C}}
\newcommand{\G}{\mathbb{G}}
\newcommand{\Prim}{\mathbb{P}}
\newcommand{\R}{\mathbb{R}}
\newcommand{\Z}{\mathbb{Z}}
\newcommand{\N}{\mathbb{N}}
\newcommand{\NN}{\mathbb{N}_0}
\newcommand{\K}{\mathbb{K}}
\newcommand{\F}{\mathbb{F}}
\newcommand{\X}{\mathfrak{X}}
\newcommand{\Oinf}{\mathcal{O}}

\newcommand{\Zdiv}[2]{#1 \text{ div } #2}
\newcommand{\Zmod}[2]{#1 \text{ mod } #2}
\newcommand{\kongru}[3]{#1 \equiv #2 \quad \text{( mod } #3 \text{ )}}

% inline code
\newcommand{\icode}[1]{\lstinline{#1}}

% theorem styles
\theoremstyle{plain}
\newtheorem{theorem}{Theorem}[subsection]
\newtheorem{prop}[theorem]{Proposition}
\newtheorem{lemma}[theorem]{Lemma}
\newtheorem{corollary}[theorem]{Corollary}
\newtheorem{definition}[theorem]{Definition}
\newtheorem{conjecture}{Conjecture}
\newtheorem{remark}{Remark}
\newtheorem{notation}{Notation and Symbols}

% Examples get their own counter
\newcounter{ExampleCounter}
\newtheorem{example}[ExampleCounter]{Example}

% Exercises get their own counter
\newcounter{ExerciseCounter}
\newtheorem{exercise}[ExerciseCounter]{Exercise}

% Jargon get their own counter
\newcounter{JargonCounter}
\newtheorem{jargon}[JargonCounter]{Jargon}

% algorithmicx commands
%\algnewcommand\algorithmicinput{\textbf{Statement:}}
%\algnewcommand\Statement{\item[\algorithmicinput]}

\title{Moonmath manual}
\author{TechnoBob and the Least Scruples crew}
\date{\today}

\begin{document}

\maketitle

Lorem \textbf{ipsum} dolor sit amet, consectetur adipiscing elit. Pellentesque semper viverra dictum.  Fusce interdum venenatis leo varius vehicula. Etiam ac massa dolor. Quisque vel massa faucibus, facilisis nulla nec, egestas lectus. Sed orci dui, egestas non felis vel, fringilla pretium odio. \textit{Aliquam} vel consectetur felis. Suspendisse justo massa, maximus eget nisi a, maximus gravida mi.

Here is a citation for demonstration: \cite{lamport1982the}

\chapter{Introduction}
% ATTENTION! THIS IS ALL COPY PASTED FROM SOMEWHERE; SO CAN ONLY BE USED AS GUIDENCE AND NEEDS TO BE REWRITTEN

% Climate Dao whitepaper

Zero knowledge proofs are a class of cryptographic protocols in which one can prove
honest computation without revealing the inputs to that computation. A simple high-level
example of a zero-knowledge proof is the ability to prove one is of legal voting age
without revealing the respective age. In a typical zero knowledge proof system, there
are two participants: a prover and a verifier. A prover will present a mathematical proof
of computation to a verifier to prove honest computation. The verifier will then confirm
whether the prover has performed honest computation based on predefined methods.
Zero knowledge proofs are of particular interest to public blockchain activities as the
verifier can be codified in smart contracts as opposed to trusted parties or third-party
intermediaries.

% https://docs.zkproof.org/reference.pdf

Zero-knowledge proofs (ZKPs) are an important privacy-enhancing tool from cryptography. Theyallow proving the veracity of a statement, related to confidential data, without revealing any in-formation beyond the validity of the statement. ZKPs were initially developed by the academiccommunity in the 1980s, and have seen tremendous improvements since then. They are now ofpractical feasibility in multiple domains of interest to the industry, and to a large community ofdevelopers and researchers. ZKPs can have a positive impact in industries, agencies, and for per-sonal use, by allowing privacy-preserving applications where designated private data can be madeuseful to third parties, despite not being disclosed to them. 

ZKP systems involve at least two parties: a prover and a verifier. The goal of the prover is toconvince the verifier that a statement is true, without revealing any additional information. Forexample, suppose the prover holds a birth certificate digitally signed by an authority. In orderto access some service, the prover may have to prove being at least 18 years old, that is, thatthere exists a birth certificate, tied to the identify of the prover and digitally signed by a trustedcertification authority, stating a birthdate consistent with the age claim. A ZKP allows this, withoutthe prover having to reveal the birthdate.


\section{Target audience}

% copy pasted from https://claritybook.netlify.app/ch00-00-introduction.html
% need adoption to our case
This book is accessible for both beginners and experienced developers alike. Concepts are gradually introduced in a logical and steady pace. Nonetheless, the chapters lend themselves rather well to being read in a different order. More experienced developers might get the most benefit by jumping to the chapters that interest them most. If you like to learn by example, then you should go straight to the chapter on Using Clarinet.

It is assumed that you have a basic understanding of programming and the underlying logical concepts. The first chapter covers the general syntax of Clarity but it does not delve into what programming itself is all about. If this is what you are looking for, then you might have a more difficult time working through this book unless you have an (undiscovered) natural affinity for such topics. Do not let that dissuade you though, find an introductory programming book and press on! The straightforward design of Clarity makes it a great first language to pick up.

\begin{comment}
\section{The Zoo of Zero-Knowledge Proofs}

{First, a list of zero-knowledge proof systems:

\begin{enumerate}
	\item Pinocchio (2013): \href{https://eprint.iacr.org/2013/279.pdf}{{Paper}}
	\begin{itemize}[label={--}]
		\item Notes: trusted setup
	\end{itemize}

	\item BCGTV (2013): \href{https://eprint.iacr.org/2013/507.pdf}{{Paper}}
	\begin{itemize}[label={--}]
		\item Notes: trusted setup, implementation
	\end{itemize}

	\item BCTV (2013): \href{https://eprint.iacr.org/2013/879.pdf}{{Paper}}
	\begin{itemize}[label={--}]
		\item Notes: trusted setup, implementation
	\end{itemize}

	\item Groth16 (2016): 	\href{https://eprint.iacr.org/2016/260.pdf}{Paper}
	\begin{itemize}[label={--}]
		\item Notes: trusted setup
		\item Other resources: \href{https://www.gakonst.com/zksummit2019.pdf}{Talk in 2019 by Georgios Konstantopoulos}
	\end{itemize}

	\item GM17 (207): 	\href{https://eprint.iacr.org/2017/540.pdf}{Paper}
	\begin{itemize}[label={--}]
		\item Notes: trusted setup
		\item Other resources: later \href{https://eprint.iacr.org/2018/187}{Simulation extractability in ROM, 2018}
	\end{itemize}

	\item Bulletproofs (2017): \href{https://eprint.iacr.org/2017/1066.pdf}{Paper}
	\begin{itemize}[label={--}]
		\item Notes: no trusted setup
		\item Other resources: \href{https://eprint.iacr.org/2016/263.pdf}{Polynomial Commitment Scheme on DL, 2016} and \href{https://www.iacr.org/archive/asiacrypt2010/6477178/6477178.pdf}{KZG10, Polynomial Commitment Scheme on Pairings, 2010}
	\end{itemize}

	\item Ligero (2017): \href{https://acmccs.github.io/papers/p2087-amesA.pdf}{Paper}
	\begin{itemize}[label={--}]
		\item Notes: no trusted setup
		\item Other resources: 
	\end{itemize}

	\item Hyrax (2017): \href{https://eprint.iacr.org/2017/1132.pdf}{Paper}
	\begin{itemize}[label={--}]
		\item Notes: no trusted setup
		\item Other resources: 
	\end{itemize}

	\item STARKs (2018): \href{https://eprint.iacr.org/2018/046.pdf}{Paper}
	\begin{itemize}[label={--}]
		\item Notes: no trusted setup 
		\item Other resources: 
	\end{itemize}

	\item Aurora (2018): \href{https://eprint.iacr.org/2018/828.pdf}{Paper}
	\begin{itemize}[label={--}]
		\item Notes: transparent SNARK
		\item Other resources:
	\end{itemize}

	\item Sonic (2019): \href{https://eprint.iacr.org/2019/099.pdf}{Paper}
	\begin{itemize}[label={--}]
		\item Notes: SNORK - SNARK with universal and updateable trusted setup, PCS-based
		\item Other resources: \href{https://www.benthamsgaze.org/2019/02/07/introducing-sonic-a-practical-zk-snark-with-a-nearly-trustless-setup/}{Blog post by Mary Maller from 2019} and \href{https://eprint.iacr.org/2018/280}{work on updateable and universal setup from 2018}
	\end{itemize}

	\item Libra (2019): \href{https://eprint.iacr.org/2019/317}{Paper}
	\begin{itemize}[label={--}]
		\item Notes: trusted setup
		\item Other resources:
	\end{itemize}

	\item Spartan (2019): \href{https://eprint.iacr.org/2019/550.pdf}{Paper}
	\begin{itemize}[label={--}]
		\item Notes: transparent SNARK
		\item Other resources:
	\end{itemize}

	\item PLONK (2019): \href{https://eprint.iacr.org/2019/953.pdf}{Paper}
	\begin{itemize}[label={--}]
		\item Notes: SNORK, PCS-based
		\item Other resources: \href{https://www.plonk.cafe/t/welcome-to-discussion-of-plonk-related-research/24}{Discussion on Plonk systems} and \href{https://github.com/Fluidex/awesome-plonk}{Awesome Plonk list}
	\end{itemize}

	\item Halo (2019): \href{https://eprint.iacr.org/2019/1021}{Paper}
	\begin{itemize}[label={--}]
		\item Notes: no trusted setup, PCS-based, recursive
		\item Other resources: 
	\end{itemize}

	\item Marlin (2019): \href{https://eprint.iacr.org/2019/1047.pdf}{Paper}
	\begin{itemize}[label={--}]
		\item Notes: SNORK, PCS-based
		\item Other resources: \href{https://github.com/arkworks-rs/marlin}{Rust Github}
	\end{itemize}

	\item Fractal (2019): \href{https://eprint.iacr.org/2019/1076.pdf}{Paper}
	\begin{itemize}[label={--}]
		\item Notes: Recursive, transparent SNARK
		\item Other resources: 
	\end{itemize}

	\item SuperSonic (2019): \href{https://eprint.iacr.org/2019/1229.pdf}{Paper}
	\begin{itemize}[label={--}]
		\item Notes: transparent SNARK, PCS-based
		\item Other resources: \href{https://eprint.iacr.org/2021/358}{Attack on DARK compiler in 2021}
	\end{itemize}

	\item Redshift (2019): \href{https://eprint.iacr.org/2019/1400}{Paper}
	\begin{itemize}[label={--}]
		\item Notes: SNORK, PCS-based
		\item Other resources: 
	\end{itemize}



\end{enumerate}

\textbf{Other resources on the zoo: } \href{https://github.com/matter-labs/awesome-zero-knowledge-proofs}{Awesome ZKP list on Github}, \href{https://zkp.science/}{ZKP community} with the \href{https://docs.zkproof.org/reference.pdf}{reference document}

}

\paragraph{To Do List}
\begin{itemize}
	\item Make table for prover time, verifier time, and proof size
	\item Think of categories - \textit{Achieved Goals}: Trusted setup or not, Post-quantum or not, \dots
	\item Think of categories - \textit{Mathematical background}: Polynomial commitment scheme, \dots
	\item \dots while we discuss the points above, we should also discuss a common notation/language for all these things. (E.g. transparent SNARK/no trusted setup/STARK)
\end{itemize}

\paragraph{Points to cover while writing}
\begin{itemize}
	\item Make a historical overview over the "discovery" of the different ZKP systems
	\item Make reader understand what paper is build on what result etc. - the tree of publications!
	\item Make reader understand the different terminology, e.g. SNARK/SNORK/STARK, PCS, R1CS, updateable, universal, $\dots$
	\item Make reader understand the mathematical assumptions - and what this means for the zoo.
	\item Where will the development/evolution go? What are bottlenecks?
\end{itemize}

\vspace*{1em}
{\footnotesize
\hspace*{-1em}\textbf{Other topics I fell into while compiling this list}
\begin{itemize}
	\item Vector commitments: \url{https://eprint.iacr.org/2020/527.pdf}
	\item Snarkl: \url{http://ace.cs.ohio.edu/~gstewart/papers/snaarkl.pdf}
	\item Virgo?: \url{https://people.eecs.berkeley.edu/~kubitron/courses/cs262a-F19/projects/reports/project5_report_ver2.pdf}
\end{itemize} 
}

\end{comment}



\chapter{The Zoo of Zero-Knowledge Proofs}

{\it First, a list of zero-knowledge proof systems:

\begin{enumerate}
	\item Pinocchio (2013): \href{https://eprint.iacr.org/2013/279.pdf}{{Paper}}
	\begin{itemize}[label={--}]
		\item Notes: trusted setup
	\end{itemize}

	\item BCGTV (2013): \href{https://eprint.iacr.org/2013/507.pdf}{{Paper}}
	\begin{itemize}[label={--}]
		\item Notes: trusted setup, implementation
	\end{itemize}

	\item BCTV (2013): \href{https://eprint.iacr.org/2013/879.pdf}{{Paper}}
	\begin{itemize}[label={--}]
		\item Notes: trusted setup, implementation
	\end{itemize}

	\item Groth16 (2016): 	\href{https://eprint.iacr.org/2016/260.pdf}{Paper}
	\begin{itemize}[label={--}]
		\item Notes: trusted setup
		\item Other resources: \href{https://www.gakonst.com/zksummit2019.pdf}{Talk in 2019 by Georgios Konstantopoulos}
	\end{itemize}

	\item GM17 (207): 	\href{https://eprint.iacr.org/2017/540.pdf}{Paper}
	\begin{itemize}[label={--}]
		\item Notes: trusted setup
		\item Other resources: later \href{https://eprint.iacr.org/2018/187}{Simulation extractability in ROM, 2018}
	\end{itemize}

	\item Bulletproofs (2017): \href{https://eprint.iacr.org/2017/1066.pdf}{Paper}
	\begin{itemize}[label={--}]
		\item Notes: no trusted setup
		\item Other resources: \href{https://eprint.iacr.org/2016/263.pdf}{Polynomial Commitment Scheme on DL, 2016} and \href{https://www.iacr.org/archive/asiacrypt2010/6477178/6477178.pdf}{KZG10, Polynomial Commitment Scheme on Pairings, 2010}
	\end{itemize}

	\item Ligero (2017): \href{https://acmccs.github.io/papers/p2087-amesA.pdf}{Paper}
	\begin{itemize}[label={--}]
		\item Notes: no trusted setup
		\item Other resources: 
	\end{itemize}

	\item Hyrax (2017): \href{https://eprint.iacr.org/2017/1132.pdf}{Paper}
	\begin{itemize}[label={--}]
		\item Notes: no trusted setup
		\item Other resources: 
	\end{itemize}

	\item STARKs (2018): \href{https://eprint.iacr.org/2018/046.pdf}{Paper}
	\begin{itemize}[label={--}]
		\item Notes: no trusted setup 
		\item Other resources: 
	\end{itemize}

	\item Aurora (2018): \href{https://eprint.iacr.org/2018/828.pdf}{Paper}
	\begin{itemize}[label={--}]
		\item Notes: transparent SNARK
		\item Other resources:
	\end{itemize}

	\item Sonic (2019): \href{https://eprint.iacr.org/2019/099.pdf}{Paper}
	\begin{itemize}[label={--}]
		\item Notes: SNORK - SNARK with universal and updateable trusted setup, PCS-based
		\item Other resources: \href{https://www.benthamsgaze.org/2019/02/07/introducing-sonic-a-practical-zk-snark-with-a-nearly-trustless-setup/}{Blog post by Mary Maller from 2019} and \href{https://eprint.iacr.org/2018/280}{work on updateable and universal setup from 2018}
	\end{itemize}

	\item Libra (2019): \href{https://eprint.iacr.org/2019/317}{Paper}
	\begin{itemize}[label={--}]
		\item Notes: trusted setup
		\item Other resources:
	\end{itemize}

	\item Spartan (2019): \href{https://eprint.iacr.org/2019/550.pdf}{Paper}
	\begin{itemize}[label={--}]
		\item Notes: transparent SNARK
		\item Other resources:
	\end{itemize}

	\item PLONK (2019): \href{https://eprint.iacr.org/2019/953.pdf}{Paper}
	\begin{itemize}[label={--}]
		\item Notes: SNORK, PCS-based
		\item Other resources: \href{https://www.plonk.cafe/t/welcome-to-discussion-of-plonk-related-research/24}{Discussion on Plonk systems} and \href{https://github.com/Fluidex/awesome-plonk}{Awesome Plonk list}
	\end{itemize}

	\item Halo (2019): \href{https://eprint.iacr.org/2019/1021}{Paper}
	\begin{itemize}[label={--}]
		\item Notes: no trusted setup, PCS-based, recursive
		\item Other resources: 
	\end{itemize}

	\item Marlin (2019): \href{https://eprint.iacr.org/2019/1047.pdf}{Paper}
	\begin{itemize}[label={--}]
		\item Notes: SNORK, PCS-based
		\item Other resources: \href{https://github.com/arkworks-rs/marlin}{Rust Github}
	\end{itemize}

	\item Fractal (2019): \href{https://eprint.iacr.org/2019/1076.pdf}{Paper}
	\begin{itemize}[label={--}]
		\item Notes: Recursive, transparent SNARK
		\item Other resources: 
	\end{itemize}

	\item SuperSonic (2019): \href{https://eprint.iacr.org/2019/1229.pdf}{Paper}
	\begin{itemize}[label={--}]
		\item Notes: transparent SNARK, PCS-based
		\item Other resources: \href{https://eprint.iacr.org/2021/358}{Attack on DARK compiler in 2021}
	\end{itemize}

	\item Redshift (2019): \href{https://eprint.iacr.org/2019/1400}{Paper}
	\begin{itemize}[label={--}]
		\item Notes: SNORK, PCS-based
		\item Other resources: 
	\end{itemize}



\end{enumerate}

\textbf{Other resources on the zoo: } \href{https://github.com/matter-labs/awesome-zero-knowledge-proofs}{Awesome ZKP list on Github}, \href{https://zkp.science/}{ZKP community} with the \href{https://docs.zkproof.org/reference.pdf}{reference document}

}

\paragraph{To Do List}
\begin{itemize}
	\item Make table for prover time, verifier time, and proof size
	\item Think of categories - \textit{Achieved Goals}: Trusted setup or not, Post-quantum or not, \dots
	\item Think of categories - \textit{Mathematical background}: Polynomial commitment scheme, \dots
	\item \dots while we discuss the points above, we should also discuss a common notation/language for all these things. (E.g. transparent SNARK/no trusted setup/STARK)
\end{itemize}

\paragraph{Points to cover while writing}
\begin{itemize}
	\item Make a historical overview over the "discovery" of the different ZKP systems
	\item Make reader understand what paper is build on what result etc. - the tree of publications!
	\item Make reader understand the different terminology, e.g. SNARK/SNORK/STARK, PCS, R1CS, updateable, universal, $\dots$
	\item Make reader understand the mathematical assumptions - and what this means for the zoo.
	\item Where will the development/evolution go? What are bottlenecks?
\end{itemize}

\vspace*{1em}
{\footnotesize
\hspace*{-1em}\textbf{Other topics I fell into while compiling this list}
\begin{itemize}
	\item Vector commitments: \url{https://eprint.iacr.org/2020/527.pdf}
	\item Snarkl: \url{http://ace.cs.ohio.edu/~gstewart/papers/snaarkl.pdf}
	\item Virgo?: \url{https://people.eecs.berkeley.edu/~kubitron/courses/cs262a-F19/projects/reports/project5_report_ver2.pdf}
\end{itemize} 
}

\chapter{Preliminaries}

Introduction and summary of what we do in this chapter

\section{Cryptological Systems}
The science of information security is referred to as \textit{cryptology}. In the broadest sense, it deals with encryption and decryption processes, with digital signatures, identification protocols, cryptographic hash functions, secrets sharing, electronic voting procedures and electronic money. EXPAND

\section{SNARKS}



\section{complexity theory}
Before we deal with the mathematics behind zero knowledge proof systems, we must first clarify what is meant by the runtime of an algorithm or the time complexity of an entire mathematical problem. This is particularly important for us when we analyze the various snark systems...

For the reader who is interested in complexity theory, we recommend, or example 
%\cite{BE} 
or 
%\cite{AB}
, as well as the references contained therein.

\subsection{Runtime complexity}
The runtime complexity of an algorithm describes, roughly speaking, the amount of elementary computation steps that this algorithm requires in order to solve a problem, depending on the size of the input data.

Of course, the exact amount of arithmetic operations required depends on many factors such as the implementation, the operating system used, the CPU and many more. However, such accuracy is seldom required and is mostly meaningful to consider only the asymptotic computational effort.

In computer science, the runtime of an algorithm is therefore not specified in individual calculation steps, but instead looks for an upper limit which approximates the runtime as soon as the input quantity becomes very large. This can be done using the so-called \textit{Landau notation} (also called big -$\mathcal{O}$-notation) A precise definition
would, however, go beyond the scope of this work and we therefore refer the reader to 
%\cite{AB}
.

For us, only a rough understanding of transit times is important in order to be able to talk about the security of crypographic systems. For example, $\mathcal{O}(n)$ means that the running time of the algorithm to be considered is linearly dependent on the size of the input set $n$, $\mathcal{O}(n^k)$ means that the running time is polynomial and $\mathcal{O}(2^n) $ stands for an exponential running time (%\cite{JB} 
chapter 2.4).


An algorithm which has a running time that is greater than a polynomial is often simply referred to as \textit{slow}.

A generalization of the runtime complexity of an algorithm is the so-called \textit{time complexity of a mathematical problem}, which is defined as the runtime of the fastest possible algorithm that can still solve this problem (
%\cite{AB} 
chapter 3.1).

Since the time complexity of a mathematical problem is concerned with the runtime analysis of all possible (and thus possibly still undiscovered) algorithms, this is often a very difficult and deep-seated question .

For us, the time complexity of the so-called discrete logarithm problem will be important. This is a problem for which we only know slow algorithms on classical computers at the moment, but for which at the same time we cannot rule out that faster algorithms also exist.

\section{Hash functions}
We assume that $H: \{0,1\}^* \to {0,1}^k$ is a \textbf{hash function} that maps binary strings of arbitrary length onto strings of length $k$. In addition we define a hash function to be $l$-bounded if it is only able to map from binary strings of length $l$ to binary strings of length $k$. 

STUFF ON CRYPTOGRAPHIC HASH FUNCTIOND

\paragraph{p\&{}p-hash}
In this example we define a $16$-bounded pen\&{}paper hash function that is simple enough to be computed without a computer. We call it the PaP-Hash and will use it throughout the book as a basic example whenever hashing is involved in other example.

The PaP-Hash $\mathcal{H}_{PaP}: \{0,1\}^{16}\to \{0,1\}^4$ is defined in the following way:
\begin{itemize}
\item Decompose the $16$-bit preimage $S=(s_0,s_1,\ldots,s_{15})$ into $4$ chunks $S_i=(s_{4i+0},\ldots,s_{4i+3})$ for $i\in \{0,1,2,3\}$.
\item For each chunk $S_i$ do a circular bitshift $\Zmod{s_{j}\to s_{j+1}}{4}$ for all $s_j\in S_i$
\item Xor all four chunks together $S = S_1\; XOR \; S_2\; XOR \; S_3\; XOR \; S_3$
\item Compute the result $\mathcal{H}_{PaP}(S) = S \; XOR\; (1001)$
\end{itemize}

\begin{example}
Lets compute our PaP-Hash on a concrete example string $S=(1110011101110011)$. Then the decomposition is $S_0=(1110)$, $S_1=(0111)$, $S_2=(0111)$ and $S_3=(0011)$ and after a circular bitshift we get $S'_0=(0111)$, $S'_1=(1011)$, $S'_2=(1011)$ and $S'_3=(1001)$. Xoring everything together we get $S= (0111) \; XOR \; (1011)\; XOR \; (1011)\; XOR \; (1001) = (1100) \; XOR\;
(0010) = (1110)$. So we get $\mathcal{H}_{PaP}(1010 0111 0110 0011) = (1110)$.
\end{example}


\section{Software Used in This Book}

It order to provide an interactive learning experience, and to allow getting hands-on with the concepts described in this book, we give examples for how to program them in the Sage programming language. Sage is a dialect of the learning-friendly programming language Python, which was extended and optimized for computing with, in and over algebraic objects. Therefore, we recommend installing Sage before diving into the following chapters.

The installation steps for various system configurations are described on the sage websit \footnote{\url{https://doc.sagemath.org/html/en/installation/index.html}}. Note however that we use Sage version 9, so if you are using Linux and your package manager only contains version 8, you may need to choose a different installation path, such as using prebuilt binaries.

We recommend the interested reader, who is not familiar with sagemath to read on the many tutorial before starting this book. For example 
%https://doc.sagemath.org/pdf/en/tutorial/SageTutorial.pdf

% Note: Logging Input and Output
% You can use this command to log all input you type, all output, and even play back that input in a future session (by simply reloadingthe log file).



\chapter{Arithmetics}
% Reason why we need a math-chapter in a book like this:
How much mathematics is needed to understand zero knowledge proofs? The answer, of course, depends on many things like the level of detail the reader want to understand them. For example it is possible to describe those proofs not using mathematics at all. However to read a foundational paper like [GROTH16], enough mathematics is needed to at least understand the basic concepts. 

Otherwise any student who is interested in learning the concepts, but who
has never seen or played with, say, a finite field, or an elliptic curve, may quickly become overwhelmed. This is not so much due to the complexity of the mathematics needs but perhaps more because of the vast amount of technical jargon, of unknown terms, obscure symbols that quickly makes a text ubreadable, desipte the concepts being actually not that hard. As a result the reader might either loose interest, or gain some dangerous smattering that in a worst case scenario materialize in inmature code. 

In this chapter we therefore derive the mathematical concepts needed to understand the basic concepts underlying snark development and we encourage the reader who is not familiar with basic number theory and lliptic curves to take the time and read this chapter until they are able to at least solve most of the simple exercises. 

If on the other hand the reader is already skilled in elliptic curve cryptography they might skip this section and only come back for reference and comparision. Maybe the most interesing parts are XXX.

We start at a very basic level and only really requie fundamntal concepts like integer arithmetics. At the same time we'll have a focus on teaching the reader how to think mathematically and to understand that there are numbers and methatical structures out there that appear to be very different from the stuff we learned in school and yet on a deeper level they are in deed very similar.

We want to stress however, that our introduction is informal, incomplete and optimized to enable the reader to understand zero knowledge concepts as efficient as possible. Our focus and design choices are so that we give as little theory as necessary but accompanied by a wealth of numerical examples. We found this on the believe, that such an informal, example-
driven approach to learning mathematics may ease the beginner’s digestion in the initial stages. 

For instance, a beginner would be likely to find it more beneficial to first compute a simple toy snark in a pen and paper style all the way through all steps before they dig deeper and actually devop real world production ready systems. Also having already a few simple examples in you head, its likely easier to only then read the actual academic papers. 

However in order to be able to derive those toy example, some mathematical groundwork is needed. This chapter therefore will help the reader to focus on what is important, while at the same time serve as first exercises the reader is encouraged to recompute themself. Every section usually then ends with a list of additional exercises in increasing difficulty order, to help the reader memorising and applying the concepts given. 

Overall the goal of this chapter is to provide a reader who is starting with nothing more than basic high school algebra, to be able to solve basic tasks in elliptic curve cryptography without the need of a computer.


%Summary of this chapter
We start with a brief recapitulation of basic integer aithmetics like long division, the greatest common divisior and Euklids algorithm. After that we introduce modular arithmetics as \textbf{the most important} skill to compute our pen and paper examples. We then introduce polynomials, compute their analogs to integer arithmetics and introduce the important concept of Lagrange interpolation.

After this practical warm up, we have to introduce some basic algebraic terms like groups and fields, because those terms are all over the place when reading academic papers in the context of zero knowledge proof. The beginner is good adviced to memorize those terms and think about them. We define these terms in the general abstract way of mathematics, hoping that the non mathematical trained reader will gradually learn to become comfortable with this style. We then give basic examples and do basic computations with these examples to get familiar with the concepts. 

\section{Integer Arithmetics}
\label{integer_arithmetics}
To some degree, most readers with probably remember integer arithmetics from school. It is however important for the rest of the book to be able to apply those concepts to understand and execute computations in the various pen and paper examples that are the main contribution of the moon math manual. We will therefore recapitulate those concepts filling up some knowledge gaps.

In what follows we applay standard mathematical notations and use the symbol $\mathbb{Z}$ for the set of all integers, that is we write
\begin{equation}
\label{integer_symbol}
\Z := \{\ldots, -3,-2,-1,0,1,2,3,\ldots\}
\end{equation}
So whenever you see the symbol $\mathbb{Z}$, think of the set of all integers. If $a \in \Z$ is an integer, we write $|a|$ for the \textit{absolute value} of $a$, that is the the non-negative value of $a$ without regard to its sign. In addition we will use the symbol $\N$ for the set of all counting numbers, that is we write 
\begin{equation}
\label{integer_symbol}
\N := \{0,1,2,3,\ldots\}
\end{equation}
including the number $0$. So whenever you see the symbol $\mathbb{N}$, think of the set of all non negative integers. 

To make it easier to memorize new concepts and symbols, we might frequently link to definitions (See \ref{integer_symbol} for a definition of $\Z$) in the begining, but as to many links render a text unreadable, we will assume the reader will become familiar with definitions as the text proceeds at which point we will not link them anymore. 

Both sets $\N$ and $\Z$ have a notion of addition as well as multiplication dedined on them and also most of us are probably able to do many integer computations in their head, we will frequently invoke the sagemath system (\ref{sagemath_setup}) for more complicated computations. One way to invoke the integer-type in sage is:
\begin{sagecommandline}
sage: ZZ # A sage notation for the integer type
sage: NN # A sage notation for the counting number type
sage: ZZ(5) # Get an element from the Ring of integers
sage: ZZ(5) + ZZ(3)
sage: ZZ(5) * NN(3)
sage: ZZ.random_element(10**50)
sage: ZZ(27713).str(2) # Binary string representation
sage: NN(27713).str(2) # Binary string representation
sage: ZZ(27713).str(16) # Hexadecimal string representation
\end{sagecommandline}
Of particular interest for us are the so called \textit{prime numbers}, which are counting numbers $ p \in \N $ with $ p \geq 2 $, which are divisible by themself and by $ 1 $ only. Prime numbers are called \textit{odd} if they are not the number $ 2 $. We write $ \Prim $ for the set of all prime numbers and $ \Prim _{\geq 3} $ for the set of all odd prime numbers.
$\Prim$ is infinite and can be ordered according to size, so that we can writem them as
\begin{equation}
\label{eq: primenumber_sequence}
2, 3, 5, 7, 11, 13, 17, 19, 23, 29, 31, 37, 41, 43, 47, 53, 59, 61, 67, \ldots
\end{equation}
which is sequence $ A000040 $ in OEIS. In particular, we can talk about small and large prime numbers.

As the \textit{fundamental theorem of arithmetics} tells us, prime numbers are in a certain sense the basic building blocks from which all other natural numbers are composed. To see that, let $ n \in \N_{\geq 2} $ be any natural number. Then there are always prime numbers $ p_1, p_2, \ldots, p_k \in \Prim $, such that
\begin{equation}
n = p_1 \cdot p_2 \cdot \ldots \cdot p_k \;.
\end{equation}
This representation is unique, except for permutations in the factors and is called the \textbf{prime factorization} of $n$.
\begin{example}[Prime Factorization] To see what we mean by integer factorization, lets look at the number $19214758032624000$. To get its prime factors, we can sucessively divide it by all prime numbers in ascending order starting with $2$. We get
\begin{equation*}
19214758032624000 = 2\cdot 2\cdot 2\cdot 2\cdot 2\cdot 2\cdot 2 \cdot 3\cdot 3\cdot 3\cdot 5\cdot 5\cdot 5\cdot 7 \cdot 11 \cdot 17\cdot 17 \cdot 23 \cdot 43\cdot 43 \cdot 47
\end{equation*}
We can double check our findings invoking sage, which provides an algorith to factor counting numbers:
\begin{sagecommandline}
sage: n = ZZ(19214758032624000)
sage: factor(n)
\end{sagecommandline}
\end{example}
Having done the computation from the previous example, reveals an important observation: Computing the factorization was computationally expensive, while on the other hand, giving a string of prime numbers, computing their product is fast. 

From this an important question arises: How fast we can compute the prime factorization of a natural number? This is the famous \textit{factorization problem} and as far as we know, there is no method on a classical Turing machine that is able to compute this representation in polynomial time. The fastest algorithms known today run sub-exponentially, with $\mathcal{O}((1+ \epsilon)^n)$ and some $ \epsilon> 0 $.

It follows that integer factorization $\Leftrightarrow$ prime number multiplication is an example of, what is called a one-way.function. Something that is easy to compute in one direction, but hard to compute in the other direction. Existence of one way functions like this are basic crytographic assumptions, that the security of many crypto systems is based on.

It should be pointed out however hat the American mathematician Peter Williston Shor developed an algorithm in 1994 which can calculate the prime factor representation of a natural number in polynomial time on a quantum computer. The consequence of this is, of course, that cryosystems, which are based on the time complexity of the prime factor problem, are unsafe as soon as practically usable quantum computers are available.
\begin{exercise}
Compute the factorization of $6469693230$ and double check your results using sage.
\end{exercise}

\paragraph{Euklidean Division}
\label{Euklidean_division}
In general there is no division defined in the usual sense for integers, as for example $7$ divided by $3$ will not be an integer again. However it is possible to devide any two integers with a remainder. So for example $7$ divided by $3$ is equal to $2$ with a remainder of $1$, since $7 = 2\cdot 3 + 1$. 

Doing integer division like this is probably something many of us remember from school. It is 
usually called \textit{Euclidean division}, or division with remainder and it is an important technique, that every reader must become familiar with to understand many concepts in this book. The precise definition is as follows:

Let $ a \in \Z $ and $ b \in \Z $ be two integers. Then there is always another integer $ m \in \Z $ and a counting number $ r \in \N $, with $ 0 \leq r <|b| $ such that
\begin{equation}
\label{eq_euklidean_division}
a = m \cdot b + r
\end{equation}
This decomposition of $a$ given $b$ is called \textit{Euklidean division}, where $ a $ is called the \textit{divident}, $ b $ is called the \textit{divisor}, $m$ is called the \textit{quotient} and $r$ is called the \textit{remainder}. 
\begin{notation}
\label{eq_euklidean_division_notation}
Suppose that the numbers $ a, b, m $ and $ r $ satisfy equation (\ref{eq_euklidean_division}). Then we often write 
\begin{equation}
\label{def_integer_division_and_modulus}
\begin{array}{lcr}
\Zdiv{a}{b}: = m, & & \Zmod{a}{b}: = r 
\end{array}
\end{equation}
to describe the quotient and the remainder of the Euklidean division. We also say, that an integer $ a $ is divisible by another integer $ b $ if $ \Zmod{a}{b} = 0 $ holds. In this case we also write $ a | b $.
\end{notation}
So in a Nutshell Euclidean division is a process of dividing one integer by another, in a way that produces a quotient and a non negative remainder the latter of which is smaller than the absolute value of the divisor. It can be shown, that both the quotient and the remainder always exist and are unique, as long as the divident is different from $0$.

A special situation occures, is the remainder is zero, because in this special case the divident \textit{is} divisible by the divisor. Our notation $a|b$ refelcts that. 


\begin{example} Applying Euklidean division and our previously defined notation \ref{def_integer_division_and_modulus} to the divisor $-17$ and the divident $4$, we get 
\begin{equation*}
\begin{array}{lcr}
\Zdiv{-17}{4} = - 5, & & \Zmod{-17}{4} = 3
\end{array}
\end{equation*}
because $ -17 = -5 \cdot 4 + 3 $  is the Euklidean division of $-17$ and $4$ (Since the remainder is by definition a non-negative number). In this case $4$ does not divide $-17$ as the reminder is not zero. Writing $-17 | 4$ therefore has no meaning. On the other hand we can write $12 | 4$, since $4$ divides $12$, as $ \Zmod{12}{4} = 0 $. We can invoke sagemath to do the computation for us. We get
\begin{sagecommandline}
sage: ZZ(-17) // ZZ(4) # Integer quotient 
sage: ZZ(-17) % ZZ(4) # remainder 
sage: ZZ(4).divides(ZZ(-17)) # self divides other
sage: ZZ(4).divides(ZZ(12))
\end{sagecommandline}
\end{example}
Methods to compute Euklidean division for integers are called \textit{integer division algorithms}. Probably the best known algorithm is the so called \textit{long division}, that most of us might have learned in school. It should be noted however that there are faster methods like \textit{Newton–Raphson division}.

As long division is the standard method used for pen-\&-paper division of multi-digit numbers expressed in decimal notation, the reader should become familiar with it as we use it all over this book when we do simple pen-and-paper computations. However instead of defining the algorithm formally, we rather give some examples, that hopelly will make the process clear

\begin{example}[Integer Long Division] To give an example of integer long division algorithm, lets divide the integer $a=143785$ by the number $b=17$. Our goal is therfore to find solutions to equation \ref{eq_euklidean_division}, that is we need to find the quotient $m\in\Z$ and the reminder $r \in \N$ such that $143785 = m\cdot 17 + r$. Using a notation that is mostly used in Commonwealth countries, we compute
\begin{equation}
\intlongdivision{143785}{17}
\end{equation}
We therefore get $m=8457$ as well as $r=16$ and indeed we have $143785 = 8457\cdot 17 + 16$, which we can double check invoking sage:
\begin{sagecommandline}
sage: ZZ(143785).quo_rem(ZZ(17)) # Euclidean Division
sage: ZZ(143785) == ZZ(8457)*ZZ(17) + ZZ(16) # check
\end{sagecommandline}
In a nutshell, the algorithm loops through the digits of the divident from the left to right, subtracting the largest possible multiple of the divisor (at the digit level) at each stage; the multiples then become the digits of the quotient, and the remainder is the first digit of the divident.
\end{example}
\begin{exercise}[Integer Long Division]
Find an $m\in\Z$ as well as an $r\in\N$ such that $a= m\cdot b +r$ (See equation \ref{eq_euklidean_division}) holds for the folling pairs $(a,b) = (27,5)$, $(a,b)=(27,-5)$, $(a,b)=(127,0)$, $(a,b)= (-1687, 11)$. In which cases are your solutions unique?
\end{exercise}
\begin{exercise}[Long Division Algorithm]
Write an algorithm in pseudocode that computes integer long division, handling all edge cases properly.
% https://en.wikipedia.org/wiki/Division_algorithm
\end{exercise}

\paragraph{The Extended Euklidean Algorithm}
One of the most critical parts in this book is modular arithmetics XXX and its application in the computations in so called finite fields, as we explain in XXX. In modular arithmetics it is sometimes possible to define actual division and multiplicative inverses of numbers, that is very different from inverses as we know them from other systems like factional numbers. 

However, to actually compute those inverses we have to get familar with the so-called \textit{extended Euclidean algorithm}. To recapitulate jargon first, the \textit{greatest common divisor} (GCD) of two nonzero integers $a$ and $b$ is the greatest non-zero counting number $d$ such that $d$ divides both $a$ and $b$; that is $d|a$ as well as $d|b$. We write $ gcd (a, b):=d $ for this number. In addition two counting numbers are called \textbf{relative prime}, if their greates common divisor is $1$.

The extended Euclidean algorithm is then a method to calculate the greatest common divisor of two counting numbers $ a $ and $ b \in \N $, as well as two additional integers $ s, t \in \Z $, such that the equation
\begin{equation}
\label{eq: erw_Eukl_algo}
gcd (a, b) = s \cdot a + t \cdot b
\end{equation}
holds. The following pseudocode shows in detail how to calculate these numbers with the extended Euclidean algorithm (\cite{JB} chapter 2.9):

\begin{algorithm}\caption{Extended Euklidean Algorithm}
\label{alg_ext_euclid_alg}
\begin{algorithmic}[0]
\Require $a,b \in \N$ with $a\geq b$
\Procedure{Ext-Euclid}{$a,b$}
\State $r_0\gets a$
\State $r_1\gets b$
\State $s_0\gets 1$
\State $s_1\gets 0$
\State $k\gets 1$
\While{$ r_{k} \neq 0 $}
\State $ q_k\gets \Zdiv{r_{k-1}}{r_k} $
\State $ r_{k + 1}\gets r_{k-1} -q_k \cdot r_k $
\State $ s_{k + 1}\gets s_{k-1} -q_k \cdot s_k $
\State $ k \gets k + 1 $
\EndWhile
\State \textbf{return} $gcd(a,b)\gets r_{k-1}$, $s\gets s_{k-1}$ and $ t: = \Zdiv{(r_{k-1}-s_{k-1} \cdot a)}{b} $ 
\EndProcedure
\Ensure $ gcd (a, b) = s \cdot a + t \cdot b $
\end{algorithmic}
\end{algorithm}
The algorithm is simple enough to be done effectively in pen-\&-paper examples, where it is common to write it as a table where the rows represent the while-loop and the colums represent the values of the the array $r$, $s$ and $t$ with index $k$. The following example provides a simple execution:
\begin{example} To illustrate the algorithm, lets apply it to the numbers $a=12$ and $b=5$. Since $12,5\in \N$ as well as $12\geq 5$ all requirements are meat and we compute
\begin{center}
  \begin{tabular}{c | c c l}
    k & $ r_k $ & $ s_k $ & $ t_k = \Zdiv{(r_k-s_k \cdot a)}{b} $ \\\hline
    0 & 12 & 1 & 0 \\
    1 & 5 & 0 & 1 \\
    2 & 2 & 1 & -2 \\
    3 & 1 & -2 & 5 \\
  \end{tabular}
\end{center}
From this we can see that $ 12 $ and $ 5 $ are relatively prime (coprime), since their greatest common divisor is $ gcd (12, 5) = 1 $ and that the equation $ 1 = (-2) \cdot 12 + 5 \cdot 5 $ holds. We can also invoke sage to double check our findings:
\begin{sagecommandline}
sage: ZZ(12).xgcd(ZZ(5)) # (gcd(a,b),s,t)
\end{sagecommandline}
\end{example}
\begin{exercise}[Extended Euklidean Algorithm]
Find integers $s,t\in\Z$ such that $gcd(a,b)= s\cdot a +t\cdot b$ holds for the folling pairs $(a,b) = (45,10)$, $(a,b)=(13,11)$, $(a,b)=(13,12)$. What pairs $(a,b)$ are coprime?
\end{exercise}
\begin{exercise}[Towards Prime fields]
Let $n\in \N$ be a counting number and $p$ a prime number, such that $n<p$. What is the greatest common divisor $gcd(p,n)$?
\end{exercise}

\section{Modular arithmetic}
% TODO: Introduce the term residue class
In mathematics, so called \textit{modular arithmetic} is a system of arithmetic for integers, where numbers "wrap around" when reaching a certain value, much like calculations on a clock wrap around whenever the value exceeds the number $12$, $24$ or $60$, depending on your clock. For example if the clock shows that it is $11$ o'clock, then $20$ hours later it will be $7$ o'clock, not $31$ o'clock. The letter of which has no meaning on a normal clock that shows hours. 

The number at which the wrap occures is called the \textit{modulus}. Modular arithmetics generalizes the clock example to arbitrary moduli and studies equations and phenomena that arizes in this new kind of arithmetics. It is of central importance for understanding most modern crypto systems, in large parts because the exponentiation function has an inverse with respect to certain moduli, that is hard to compute. In addition we will see that it provides the foundation of what is called finite fields (See XXX)

Also it will turn out that modular arithmetic appears very different from ordinary integer arithmetic that we are all familiar with, we encouurage the interested reader to work through the example and to discover that, once they accept that this is a new kind of calculations, its actually not that hard.
\paragraph{Congurency}
In what follows, let $n\in\N$ with $n\geq 2$ be a fixed counting number, that we will call the \textit{modulus} of our modular arithmetics system. With such an $n$ given, we can then group integers into classes, by saying that two integers are in the same class, whenever their Euklidean division \ref{Euklidean_division} by $n$ will give the same remainder. We then say that two numbers are \textit{congruent} whenever they are in the same class.

\begin{example}
If we choose $n=12$ as in our clock example, then the integers $-7$, $5$, $17$ and $29$ are all congruent with respect to $12$, since all of them have the remainder $5$ if we Euklidean divide them by $12$. In the picture of an analog $12$-hour clock, starting at $5$ o'clock, when we add $12$ hours we are again at $5$ o'clock, representing the number $17$. On the other hand when we subtract $12$ hours, we are at $5$ o'clock again, representing the number $-7$. 
\end{example}
We can formulize this intuition of what congruency should be into a proper definition utilizing Euklidean division as explained previously \ref{integer_arithmetics}: Let $ a $, $ b \in \Z $ be two integers and $ n \in \N $ a natural number.
Then $ a $ and $ b $ are said to be \textbf{congruent with respect to the modulus} $ n $, if and only if the equation
\begin{equation}
\Zmod{a}{n} = \Zmod{b}{n}
\end{equation}
holds. If on the other hand two numbers are not congruent with respect to a given modulus $n$, we call them \textit{incongruent} w.r.t. $n$. 

A \textit{congruency} is then nothing but an equation "up to congruency", which means that the equation only needs to hold if we take the modulus on both sides. In which case we write 
\begin{equation}
\kongru{a}{b}{n} 
\end{equation}
\paragraph{Modular Arithmetics}
On particulary nice thing about congruencies is, that we can do calculations (aithmetics), much like we can with integer equations. That is we can add or multiply numbers on both sides. The main difference is probably that the congruency $\kongru{a}{b}{n}$ is only equivalent to the congruency $\kongru{k\cdot a}{k\cdot b}{n}$ for some non zero integer $k\in \Z$, whenever $k$ and the modulus $n$ are coprime. The following list gives a set of useful rules:

Suppose that the congurencies $\kongru{a_1}{b_1}{n}$ as well as $\kongru{a_2}{b_2}{n}$ are satisfied for integers $a_1,a_2,b_1,b_2\in\Z$ and that $k\in\Z$ is another integer. Then:
\begin{itemize}
\item $\kongru{a_1+k}{b_1+k}{n}$ (compatibility with translation)
\item $\kongru{k\cdot a_1}{k\cdot b_1}{n}$ (compatibility with scaling)
\item $\kongru{a_1+a_2}{b_1+b_2}{n}$ (compatibility with addition)
\item $\kongru{a_1\cdot a_2}{b_1\cdot b_2}{n}$ (compatibility with multiplication)
\end{itemize}
Other rules like compatibility with subtraction and exponentiation follow from this rule, as for example compatibility with subtraction is compatibility with scaling by $k=-1$ and compatibility with addition.

Note that the previous rules are implications not equivalences, which means that you can not necessarily reverse those rules. The following rules makes this precise:
\begin{itemize}
\item If $\kongru{a_1+k}{b_1+k}{n}$, then $\kongru{a_1}{b_1}{n}$
\item If $\kongru{k\cdot a_1}{k\cdot b_1}{n}$ and $k$ is coprime with $n$, then $\kongru{a_1}{b_1}{n}$
\item If $\kongru{k\cdot a_1}{k\cdot b_1}{k\cdot n}$ , then $\kongru{a_1}{b_1}{n}$
\end{itemize}
Another property of congruencies, not known in the traditional arithmetics of integers is the so called \textit{Fermat's Little Theorem}. In simple words, it says that in modular arithmetics every number raised to the power of a prime number modulus is congruent to the number itself. Or, to be more precise, if $ p \in \Prim $ is a prime number and $ k \in \mathbb{Z} $ is an integer, then:
\begin{equation}
\kongru{k ^ p}{k}{p} \;,
\end{equation}
If $k$ is coprime to $p$, then we can divide both sides of this congruency by $k$ and rewrite the expression into the equivalent form 
\begin{equation}
\label{eq_fermat_lt_2}
\kongru{k ^{p-1}}{1}{p}
\end{equation} 
We can invoke sage, to compute examples for both $k$ being coprime and not coprime to $p$:
\begin{sagecommandline}
sage: ZZ(137).gcd(ZZ(64))
sage: ZZ(64)** ZZ(137) % ZZ(137) == ZZ(64) % ZZ(137)
sage: ZZ(64)** ZZ(137-1) % ZZ(137) == ZZ(1) % ZZ(137)
sage: ZZ(1918).gcd(ZZ(137))
sage: ZZ(1918)** ZZ(137) % ZZ(137) == ZZ(1918) % ZZ(137)
sage: ZZ(1918)** ZZ(137-1) % ZZ(137) == ZZ(1) % ZZ(137)
\end{sagecommandline}
\begin{remark}
Congruency \ref{eq_fermat_lt_2} has a nice interpretation, that gives a first glimpse of the idea of a prime field as we will describe it in XXX: Whenever the modulus is a prime number and $k<p$ then $k$ and $p$ are coprime (Exercise XXX) and hence there is a notion of division by $k$, that is absent in integer arithmetics. If $a$ is another integer we could define division by $k$ as follows $a/k := a\cdot k^{p-2}$, which is always defined. This makes sense because "division by something" should be defined as "multiplication by the inverse" and since $\kongru{k\cdot k ^{p-2}}{1}{p}$, the number $k^{p-2}$ behaves like a multiplicative inverse for $k$ in modular arithmetics for prime number moduli and coprime $k$.
\end{remark}

Now, since this was a lot to digest for a reader who has never encountered modular arithmetics before, lets compute an example that contains most of the stuff we just described:   
   
\begin{example}Assume that we choose the modulus $17$ and that our task is to solve the following congruency for $x\in \Z$
$$\kongru{7\cdot(2x+21) + 11}{x-102}{17}$$
As many rules for congruencies are more or less same as for integers, we can proceed in a way similar, as we would if we had an equation to solve. 
The first thing we notice, is that $7\cdot(2x+21) + 11= 14x +158$, since both sides of a congruency contain ordinary integers. We can therefore rewrite the congruency into the equivalent form
$$\kongru{14x +158}{x-102}{17}$$
In a next step we want to shift all encounters of $x$ to left and every other term to the right. So we applay the "compatibility with translation" rules two times. In a first step we choose $k=-x$ and in a second step we choose $k=-158$. Since "compatibility with translation" transforms a congruency into an equivalent form, the solution set will not change and we get 
\begin{multline*}
\kongru{14x +158}{x-102}{17} \Leftrightarrow\\
\kongru{14x-x +158-158}{x-x-102-158}{17} \Leftrightarrow \\
\kongru{13x}{-260}{17}
\end{multline*}
If our congruency would just be a normal integer equation, we would divide both sides by $13$ to get $x=-20$ as our solution. However in case of a congruency we need to make sure that the modulus and the number we want to divide by are coprime first. Only then will we get an equivalent expression. So we need to the greatest common divisor $gcd(17,13)$ and since both numbers are prime, we know $gcd(17,13)=1$, so both numbers are indeed coprime. We therefore compute 
$$
\kongru{13x}{-260}{17} \Leftrightarrow \kongru{x}{-20}{17}
$$
Our task is now to find all integers $x$, such that $x$ is congruent to $-20$ with respect to the modulus $17$. So we have to find all $x$ such
$$
\Zmod{x}{17} = \Zmod{-20}{17}
$$
Since $-2\cdot 17 +14 = -20$ we know $ \Zmod{-20}{17} = 14$ and hence we know that $x=14$ is a solution. However $31$ is another solution since $ \Zmod{31}{17} = 14$ as well and so is $-20$. In fact there are infinite many solutions given by the set
$$
\{\ldots, -20,-3, 14,31, 48,\ldots\} = \{14+k\cdot 17 \;|\; k\in \Z\}
$$
Putting all this together we have shown that the every $x$ from the set $\{x=14+k\cdot 17 \;|\; k\in \Z\}$ is a solution to the congruency $\kongru{7\cdot(2x+21) + 11}{x-102}{17}$. We double ckeck for, say, $x=14$ as well as $x=14 + 12\cdot 17 = 218$ using sage:
\begin{sagecommandline}
sage: (ZZ(7)* (ZZ(2)*ZZ(14) + ZZ(21)) + ZZ(11))  % ZZ(17) == (ZZ(14) - ZZ(102))  % ZZ(17)
sage: (ZZ(7)* (ZZ(2)*ZZ(218) + ZZ(21)) + ZZ(11))  % ZZ(17) == (ZZ(218) - ZZ(102))  % ZZ(17)
\end{sagecommandline}
\end{example}
\begin{example}[Mudular multiplicative inverse] In the previous example we encounter the "lucky coincidence", that we could solve the congruency $\kongru{13x}{-260}{17}$ by division by $13$, since $260$ is divisible by $13$ as integers. But what if this is not the case?
So see how this can be solved in modular arithmetics lets consider the following modification of the congruency in the previous example:
$$
\kongru{7x}{-260}{17}
$$ 
In this example we can not integer divide both sides of the congruency by $7$, since $260/7$ is not an integer. However we recall from Fermats little theorem, that $\kongru{k\cdot k^{p-2}}{1}{p}$ for every integer $k$ that coprime to every prime modulus $p$. 

In our example, since $7$ is coprime to $17$ (both are prime numbers), we can exploit Fermats theorem, in combination with the "compatibility with scaling" rule and multiply both sides of the congruency with $7^{17-2}= 4747561509943$. We then get
\begin{multline*}
\kongru{7x}{-260}{17} \Leftrightarrow \\
\kongru{7^{17-2}\cdot 7\cdot x}{-260\cdot 7^{17-2}}{17}\Leftrightarrow \\
%\kongru{x}{-260\cdot 4747561509943}{17} \Leftrightarrow \\
\kongru{x}{-1234365992585180}{17} 
\end{multline*}
And since $\Zmod{-1234365992585180}{17}=9$, we know that the solution is given by the set of all numbers $x\in \{9+m\cdot 17\;|\; m\in\Z\}$. We double ckeck for, say, $x=9$ as well as $x=9 + 101\cdot 17 = 1726$ using sage:
\begin{sagecommandline}
sage: (ZZ(7)* ZZ(9))  % ZZ(17) == (-ZZ(260))  % ZZ(17)
sage: (ZZ(7)* ZZ(1726))  % ZZ(17) == (-ZZ(260))  % ZZ(17)
\end{sagecommandline}
\end{example}
\begin{remark}
The discouraged reader, who at this point thinks that modular aithmetics is to complicated, might consider two thinks: First, computing congruencies in modular arithmetics is not really more complicated then computations in more familiar number systems like fractional numbers. Its just a matter of getting used to it. Second, the theory of prime fields (and more general residue class rings) takes a different view on modular rithmetics with the attempt to simplify thinks. In other words, once we understand prime field arithmetics, thinks become conceptually cleaner and more easy to compute.
\end{remark}
\paragraph{The Chinese Remainder Theorem} We have seen in the previous paragraph how to solve congruencies in modular arithmetic. However one question that remains is, how to solve systems of congruencies, whith different moduli? The answer is given by the so called \textit{Chinese raimainder theorem}, which tells us, that for any $ k \in \N $ and coprime natural numbers $ n_1, \ldots n_k \in \N $ as well as integers $ a_1, \ldots a_k \in \Z $, the so-called \textit{simultaneous congruency}
\begin{equation}
\label{eq_simultaneous_congruency}
\begin{array}{c}
\kongru{x}{a_1}{n_1} \\
\kongru{x}{a_2}{n_2} \\
\cdots \\
\kongru{x}{a_k}{n_k} \\
\end{array}
\end{equation}
has a solution and all possible solutions of this congruence system are congruent modulo
the product $N= n_1 \cdot \ldots \cdot n_k $. In fact, the following algorithm computes the solution set:
\begin{algorithm}\caption{Chinese Reminder Theorem}
\label{alg_ext_euclid_alg}
\begin{algorithmic}[0]
\Require $n_0,\ldots,n_{k-1} \in \N$ coprime
\Procedure{Congruency-Systems-Solver}{$k,a_{0,\ldots,k-1},n_{0,\ldots,k-1}$}
\State $N\gets n_0\cdot \ldots \cdot n_{k-1}$
\While{$j< k $}
\State $N_j\gets N/n_j$
\State $(\_,s_j,t_j)\gets EXT-EUCLID (N_j,n_j)$ 
  \Comment{$1 = s_j\cdot N_j + t_j\cdot n_j$}
\EndWhile
\State $x'\gets \sum_{j=0}^{k-1}a_j\cdot s_j\cdot N_j$
\State $x\gets \Zmod{x'}{N}$
\State \textbf{return} $\{x+ m\cdot N\;|\; m\in \Z\}$
\EndProcedure
\Ensure $\{x+ m\cdot N\;|\; m\in \Z\}$ is the complete solution set to \ref{eq_simultaneous_congruency}.
\end{algorithmic}
\end{algorithm}

This is the classical Chinese remainder theorem as it was already known in ancient China. Under certain circumstances, the theorem can be extended to non-coprime moduli $ n_1, \ldots, n_k $ but we don't need that extension in the book.
\begin{example} To illustrate how to solve simultaneous congruences using the Chinese remainder theorem, let's look at the following system of congruencies:
$$
\begin{array}{c}
\kongru{x}{4}{7} \\
\kongru{x}{1}{3} \\
\kongru{x}{3}{5} \\
\kongru{x}{0}{11} \\
\end{array}
$$
Clearly all moduli are coprime and we have $ N = 7 \cdot 3 \cdot 5 \cdot 11 = 1155 $, as well as
$ N_1 = 165 $, $ N_2 = 385 $, $ N_3 = 231 $ and $ N_4 = 105 $. From this we calculate with the extended Euclidean algorithm
$$
\begin{array}{cccc}
 1 = & 2 \cdot 165  & + & -47 \cdot 7 \\
 1 = & 1 \cdot 385  & + &  -128 \cdot 3 \\
 1 = & 1 \cdot 231  & + &  -46 \cdot 5 \\
 1 = & 2 \cdot 105  & + &  -19 \cdot 11 \\
\end{array}
$$
so we have
$x = 4 \cdot 2 \cdot 165 + 1 \cdot 1 \cdot 385 + 3 \cdot 1 \cdot 231 + 0 \cdot 2 \cdot 105 = 2398$
as one solution. Because $ \Zmod{2398}{1155} = 88 $ the set of all solutions is
$ \{\ldots, -2222, -1067,88,1243, 2398, \ldots \} $. In particular, there are infinitely many different solutions. We can invoke sage's computation of the Chinese Remainder Theorem (CRT) to double check our findings:
\begin{sagecommandline}
sage: CRT_list([4,1,3,0], [7,3,5,11])
\end{sagecommandline}
\end{example}
As we have seen in various examples before, computing congruencies can be cumersome and solution sets are huge in general. It is therefore advantagous to find some kind of simplification for modular arithmetic. Fortunately this is possible if we consider all integers that have the same remainder with respect to a given modulus $n$ to be the same. It then follows from the properties of Euclidean division, that there are exactly $ n $ different such sets for every moduls. 

If we go a step further and identify each such set (equivalence class) with the corresponding remainder of the Euclidean division, we get a new set, where integer addition and multiplication can be projected to a new kind of addition and multiplication on the equivalence classes. 

Roughly speaking the new rules for addition and multiplication are then computed by taking any element of the firsr equivalence class and some element of the second, then add or multiply them in the usual way and see in which equivalence class the result is contained.
The following example makes the abstract idea more concrete
\begin{example} [Arithmetics modulo $6$]
\label{def_residue_ring_z_6}
Choosing the modulus $ n = 6 $ we have six equivalence classes of integers which are congruent modulo $ 6 $ (which have the same remainder when divided by $6$). We write
$$
\begin{array}{lll}
0: = \{\ldots, -6,0,6,12, \ldots \}, &
1: = \{\ldots, -5,1,7,13, \ldots \}, &
2: = \{\ldots, -4,2,8,14, \ldots \} \\
3: = \{\ldots, -3,3,9,15, \ldots \}, &
4: = \{\ldots, -2,4,10,16, \ldots \}, &
5: = \{\ldots, -1,5,11,17, \ldots \}
\end{array}
$$
Now to compute the addition of those equivalence classes, say $2+5$, one chooses arbitrary elements from both sets say $14$ and $-1$, adds those numbers in the usual way and then looks in which equivalence class the result will be. 

So we have $14+(-1)=13$ and $13$ is in the equivalence class (of) $1$. Hence we find that $2+5=1$ in modular $6$ aithmetics, which is a more readable way to write the congruency $\kongru{2+5}{1}{6}$.

Applying the same reasoning to all equivalence classes, addition and multiplication can  be transferred to the equivalence classes and the results are summarized in the following addition and multiplication tables for modulus $6$ aithmetics:
\begin{center}
  \begin{tabular}{c | c c c c c c}
    + & 0 & 1 & 2 & 3 & 4 & 5\\\hline
    0 & 0 & 1 & 2 & 3 & 4 & 5 \\
    1 & 1 & 2 & 3 & 4 & 5 & 0\\
    2 & 2 & 3 & 4 & 5 & 0 & 1\\
    3 & 3 & 4 & 5 & 0 & 1 & 2\\
    4 & 4 & 5 & 0 & 1 & 2 & 3\\
    5 & 5 & 0 & 1 & 2 & 3 & 4
  \end{tabular} \quad \quad \quad \quad
  \begin{tabular}{c | c c c c c c}
$ \cdot $ & 0 & 1 & 2 & 3 & 4 & 5 \\\hline
        0 & 0 & 0 & 0 & 0 & 0 & 0\\
        1 & 0 & 1 & 2 & 3 & 4 & 5\\
        2 & 0 & 2 & 4 & 0 & 2 & 4\\
        3 & 0 & 3 & 0 & 3 & 0 & 3\\
        4 & 0 & 4 & 2 & 0 & 4 & 2\\
        5 & 0 & 5 & 2 & 3 & 2 & 1
  \end{tabular}
\end{center}

These two tables are all you need to be able to calculate in modular $6$ aithmetics. 

To see how this simplifies congruency computations, lets look at the congruency 
$$\kongru{7\cdot(2x+21) + 11}{x-102}{6}$$
from example XXX again (But this time in modular $6$ arithmetics to use our addition and multiplication tables). Since $\Zmod{21}{6}= 3$ and $\Zmod{102}{6}= =$ We can rewrite the congruency into the equation 
$7\cdot(2x+3) + 11=x$
AND DO STUFF



For example, to determine the multiplicative inverse of a remainder class, look for the entry that results in $ 1 $ in the product table. For example the multiplicative inverse of $ 5 $ is $ 5 $ itself, since $5\cdot 5 = 1$. Similar to the integers not all numbers have inverses. For example there is no element, that when multiplied with $4$ will give $1$. 
However in contrast to what we know from integers, there are non zero numbers, that, when multiplied gives zero (e.g $4\cdot 4 =0$).

One can show that distributivity, associativity and commuativity hold (See section XXX for a precise definition)

\begin{sagecommandline}
sage: Z6=Integers(6) # Define integers modulo 6 
sage: Z6(2)+Z6(5) # standard representatives of a class
sage: Z6(14)+Z6(-1) # different representatives for same class
sage: - Z6(2) # additive inverse
sage: Z6(5)**(-1) # multiplicative inverse if exists
\end{sagecommandline}
\end{example}

TODO:

Barrett reduction

Montgomery modular multiplication (Montgomery domain)

\begin{jargon}[$k$-bit modulus] In cryptographic papers, we can sometimes read phrases like "$[\ldots]$ using a 4096-bit modulus". This means that the underlying modulus $n$ of the modular arithmetic used in the system has a binary representation with a length of $4096$ bits. For example, the number $6$ has the binary representation $110$ and henxe example describes a $3$-bit modulus arithmetics system.   
\end{jargon}
\begin{exercise}
Let $a,b,k$ be integers, such that $\kongru{a}{b}{n}$ holds. Show $\kongru{a^k}{b^k}{n}$.
\end{exercise}
\begin{exercise}
Let $a,n$ be integers, such that $a$ and $n$ are not coprime. For which $b\in\Z$ does the 
congruency $\kongru{a\cdot x}{b}{n}$ have a solution $x$ and how does the solution set look in that case?
\end{exercise}
\paragraph{Modular Inverses} As we know integers can be added, subtracted and multiplied, but not divided in general, as for example $3/2$ is not an integer anymore. To see why this is, from a more theoretical perspective, lets consider the definition of a multiplicative inverse first. When we have a set that has some kind of multiplication defined on it and we have a distinguished element of that set, that behaves neutral with respect to that multiplication (doesn't change anything when multiplied with any other element), then we can define \textit{multiplicative inverses} in the following way:

Let $S$ be our set that has some notion $a\cdot b$ of multiplication and a \textit{neutral element} $1\in S$, such that $1\cdot a = a$ for all elements $a\in S$. Then a \textit{multiplicative inverse} $a^{-1}$ of an element $a\in S$ is defined by
\begin{equation}
a\cdot a^{-1} = 1
\end{equation}
So roughly speaking a multiplicative element is defined in such a way, that it cancels the original element to give $1$, whenever they are multiplied. 

Numbers that have multiplicative inverses are of particular interest, because they immediately lead to the definition of division by those numbers. In fact if $a$ is number, such that the multiplicative inverse $a^{-1}$ exist, then we define \textit{division} by $a$ simply as multiplication by the inverse, i.e
\begin{equation}
\frac{b}{a}:= b\cdot a^{-1}
\end{equation}
\begin{example} Consider the set of rational numbers $\mathbb{Q}$, that is the set of all fractions. Then the neural element of multiplication is $1$, since $1\cdot a = a$ for all rational numbers. For example $1\cdot 4=4$, $1\cdot \frac{1}{4}=\frac{1}{4}$, or $1\cdot 0 =0$ and so on.

Then every rational number $a\neq 0$ has a multiplicative inverse, given by $\frac{1}{a}$. 
For example the multiplicative inverse of $3$ is $\frac{1}{3}$, since $3\cdot \frac{1}{3}=1$, the multiplicative inverse of $\frac{5}{7}$ is $\frac{7}{5}$, since $\frac{5}{7}\cdot \frac{7}{5}=1$ and so on. 
\end{example}
\begin{example}Looking at  the set $\Z$ of integers, we see that with respect to multiplication the neutral element is the number $1$ and we notice, that no integer $a\neq 1$ has a multiplicative inverse, since the equation $a\cdot x =1$ has no integer solutions for $a\neq 1$. 

The definition of multiplicative inverse works verbtim for addition, too. In the case of integers, the neutral element with respect to addition is $0$, since $a+0=0$ for all integers $a\in\Z$. The additive inverse then always exist and is given by the negative number $-a$, since $a+(-1)=0$.  
\end{example}
\begin{example} Looking at the set $\Z_6$ of residuel classes modulo $6$ from example XXX, we can use the multiplication table to find multiplicative inverses. To see that we look at the row of the element and then find the entry equal to $1$. If such an entry exist, the element of that column is the multiplicative inverse. If on the other hand the row has no entry equal to $1$, we know that the element has no multiplicative inverse.

For example in $\Z_6$ the multiplicative inverse of $5$ is $5$ itself, since $5\cdot 5=1$. We can moreover see that $5$ and $1$ are the only elements that have multiplicative inverses in $\Z_6$. 

Now since $5$ has a multiplicative inverse modulo $6$, it makes sense to "divide by $5$ in $\Z_6$. For example
$$
\frac{4}{5}= 4\cdot 5^{-1} = 4\cdot 5 = 2
$$ 
\end{example}
From the last example we can make the interesting observation, that while $5$ has no multiplicative inverse as an integer, it has a multiplicative inverse in modular $6$ arithmetics. 

So the question remains, to understand, what elements have multiplicative inverses in modular arithmetics. The answer is, that in modular $n$ arithmetics, a residue class $r$ has a multiplicative inverse, if and only if $n$ and $r$ are coprime. Since $ggt(n,r)=1$ in that case, we know from the extended Euklidean algorithm, that there are numbers $s$ and $t$, such that 
\begin{equation}
\label{eq_compute_multiplicative_inverse}
1 = s\cdot n + t\cdot r
\end{equation}
and if we take the modulus $n$ on both sides the term $s\cdot n$ vanishes, which tells us that $\Zmod{t}{n}$ is the multiplicative inverse of $r$ in modular $n$ arithmetics.
\begin{example}[Multiplicative inverses in $\Z_6$] In the previous example we have looked up multiplicative inverses in $\Z_6$ from lookup-table XXX. In real world examples, it is of course usually impossible to write down those lookup tables as the modulus is way to large and the sets occasionally contain more elements, then there are atoms in the observable universe.

No to see that $2\in \Z_6$ has no multiplicative inverse in $\Z_6$ without using the lookup table, we immediately observe that $2$ and $6$ are not coprime since their greatest common divisor is $2$. If follows that equation \ref{eq_compute_multiplicative_inverse} has no solutions $s$ and $t$ and hence $2$ has no multiplicative inverse.

The same reasoning works for $3$ and $4$, too as both are not coprime with $6$ and the only case that is different is $5$, since $ggt(6,5)=1$. To compute the multiplicative inverse of $5$ we use the extended Euklidean algorithm and compute   
\begin{center}
  \begin{tabular}{c | c c l}
    k & $ r_k $ & $ s_k $ & $ t_k = \Zdiv{(r_k-s_k \cdot a)}{b} $ \\\hline
    0 & 6 & 1 & 0 \\
    1 & 5 & 0 & 1 \\
    2 & 1 & 1 & -1 \\
    3 & 0 & . & . \\
  \end{tabular}
\end{center}
So we get $s=1$ as well as $t=-1$ and have $1 = 1\cdot 6 -1\cdot 5$. From this follows that $\Zmod{-1}{6}=5$ is the multiplicative inverse of $5$ in modular $6$ arithmetics. We can double check using sage:
\begin{sagecommandline}
sage: ZZ(6).xgcd(ZZ(5))
\end{sagecommandline}
\end{example}
At this point the attentive reader might notice, that the situation, where the modulus is a prime number is of particular interest, since we know from exercise XXX, that in this cases all remainder classes must have modular inverses, since $ggt(r,n)=1$ for prime $n$ and $r<n$. We will look at this central situation in XXX. 

\begin{example} [Modular $5$ arithmetics] 
\label{primfield_z_5}
To see the unique properties of mudular arithmetics whenever the modulus is prime numbers, we will parallel our findings from example XXX, but this time for the prime modulus $5$.
For $ n = 5 $ we have five equivalence classes of integers which are congruent modulo $ 5 $. We write
$$
\begin{array}{ccc}
0: = \{\ldots, -5,0,5,10, \ldots \}, &
1: = \{\ldots, -4,1,6,11, \ldots \}, &
2: = \{\ldots, -3,2,7,12, \ldots \} \\
3: = \{\ldots, -2,3,8,13, \ldots \}, &
4: = \{\ldots, -1,4,9,14, \ldots \}
\end{array}
$$
Addition and multiplication can now be transferred to the equivalence classes. This results in the following addition and multiplication tables in $ \Z_5 $:
\begin{center}
  \begin{tabular}{c | c c c c c}
    + & 0 & 1 & 2 & 3 & 4 \\\hline
    0 & 0 & 1 & 2 & 3 & 4 \\
    1 & 1 & 2 & 3 & 4 & 0 \\
    2 & 2 & 3 & 4 & 0 & 1 \\
    3 & 3 & 4 & 0 & 1 & 2 \\
    4 & 4 & 0 & 1 & 2 & 3 \\
  \end{tabular} \quad \quad \quad \quad
  \begin{tabular}{c | c c c c c}
$ \cdot $ & 0 & 1 & 2 & 3 & 4 \\\hline
      0 & 0 & 0 & 0 & 0 & 0 \\
      1 & 0 & 1 & 2 & 3 & 4 \\
      2 & 0 & 2 & 4 & 1 & 3 \\
      3 & 0 & 3 & 1 & 4 & 2 \\
      4 & 0 & 4 & 3 & 2 & 1 \\
  \end{tabular}
\end{center}

These two tables are all you need to be able to calculate in $ \Z_5 $. For example, to determine the multiplicative inverse of an element, look for the entry that results in $ 1 $ in the product table. This is the multiplicative inverse. For example the multiplicative inverse of $ 2 $ is $ 3 $ and the multiplicative inverse of $4$ is $4$ itself, since $4\cdot 4=1$. As we can see indeed every element $\neq 0$ has a multiplicative inverse.
\end{example}
\section{Polynomial Arithmetics}
A polynomial is an expression consisting of variables (also called indeterminates) and coefficients, that involves only the operations of addition, subtraction, multiplication, and non-negative integer exponentiation of variables. All coefficients of a polynomial must have the same type, e.g. being integers or fractions etc. To be more precise a \textit{polynomial} is an expression
\begin{equation}
P(x) := \sum _{j = 0} ^{m}{a} _{j}{t} ^{j} ={a} _{m}x^m +{a} _{m-1} x^{m-1} + \dots + a_1 x + a_0 \;,
\end{equation}
where $x$ is called the \textit{indeterminate}, each $ a_j$ is called a \textit{coefficient}. If $R$ is the type of the coefficients then the set of all \textbf{polynomials with coefficients in $R$} is written as $R[x]$. We often simply write $ P (x) \in R[x]$ for a polynomial and denote the constant term as $ P(0)$. 

A polynomial is called the \textit{zero polynomial} if all coefficients are zero and a polynomial is called the \textit{one polynomial} if the constant term is $1$ and all other coefficients are zero.

If a polynomial $P(x)=\sum_{j=0}^m a_jx^j$ is given and is not the zero polynomial, we call 
\begin{equation}
deg(P):=m
\end{equation}
the \textit{degree} of $P$ and define the degree of the zero polynomial to be $-\infty$. In addition we write 
\begin{equation}
\label{def_leading_coefficient}
Lc(P):=a_m
\end{equation}
and call it the \textit{leading coefficient} of the polynomial $P$. We can restrict the set $R[x]$ of \textit{all} polynomials with coefficients in $R$, to the set of all such polynomials that have a degree that does not exceed a certain value. If $m$ is the maximum degree allowed, we write $R_{\leq m}[x]$ for the set of all poynomials with a degree less or equal to $m$.
\begin{example}[Integer Polyinomials] The coefficients of a polynomial must all have the same type. The set of polynomials with integer coefficients is written as $\Z[x]$. Examples of such polynomialse are:
\begin{itemize}
\item $P_1(x)= 2x^2 -4x +17$, with degree $deg(P_1)=2$.
\item $P_2(x)= x^{23}$, with degree $deg(P_2)=23$.
\item $P_3(x)= x$, with degree $deg(P_3)=1$.
\item $P_4(x)= 174$, with degree $deg(P_4)=0$.
\item $P_5(x)= 1$, with degree $deg(P_5)=0$.
\item $P_6(x)=0$, with degree $deg(P_5)=-\infty$.
\item $P_7(x)= (x-2)(x+3)(x-5)$, with degree $deg(P_3)=3$.
\end{itemize}
In particular every integer can be seen as an integer polynomial of degree zero. $P_7$ is a polynomial, because we can expand its definition into $P_7(x)=x^3 - 4 x^2 - 11 x + 30$, which is polynomial of degree $3$. The following expressions are not integer polynomial
\begin{itemize}
\item $Q_1(x)= 2x^2 + 4 + 3x^{-2}$
\item $Q_2(x)= 0.5x^4 -2x$
\item $Q_3(x)=1/x$
\end{itemize}
\end{example}
\begin{example}[Polynomials over $\mathbb{Z}_6$] Recall our definition of the residue classes $\Z_6$ and their arithmetics as defined in \ref{def: residual class ring}. The set of all polynomials with indeterminant $x$ and coefficients in $\Z_6$ is symbolized as $\Z_6[x]$. Example of polynomials from $\Z_6$ are:
\begin{itemize}
\item $P_1(x)= 2x^2 -4x +5$, with degree $deg(P_1)=2$.
\item $P_2(x)= x^{23}$, with degree $deg(P_2)=23$.
\item $P_3(x)= x$, with degree $deg(P_3)=1$.
\item $P_5(x)= 1$, with degree $deg(P_5)=0$.
\item $P_6(x)=0$, with degree $deg(P_5)=-\infty$.
\item $P_7(x)= (x-2)(x+3)(x-5)$, with degree $deg(P_3)=3$.
\end{itemize}
As in the previous example $P_7$ is a polynomial. However since we are working with coefficients from $\Z_6$ now the expension of $P_7$ is computed differently, as we have to invoke addition and multiplication in $\Z_6$ as defined in XXX. We get:
$$
\begin{array}{l c l r}
(x-2)(x+3)(x-5) & = & (x+4)(x+3)(x+1) & \text{\# additive inverses in } \Z_6 \\
                & = & (x^2+4x+3x+3\cdot 4)(x+1) & \text{\# bracket expansion} \\
                & = & (x^2+1x+0)(x+1) & \text{\# compuation in } \Z_6 \\
                & = & (x^3+x^2+x^2+x) & \text{\# bracket expansion}\\
                & = & (x^3+2x^2+x) &
\end{array} 
$$
\end{example}
We can invoke sage to do computations with polynomials. To do so we have to specify the symbol for the indertimate and the type for the coefficients. Note however that sage defines the degree of the zero polynomial to be $-1$.
\begin{sagecommandline}
sage: Zx = ZZ['x'] # integer polynomials with indeterminate x
sage: Zt.<t> = ZZ[] # integer polynomials with indeterminate t
sage: Zx
sage: Zt
sage: p = Zx([1,2,3,4])
sage: q = Zt([1,2,3,4])
sage: p
sage: q
sage: p.degree()
sage: zero = Zx([0])
sage: zero.degree()
\end{sagecommandline}
Given some element from the same type as the coefficients of a polynomial, the polynomial can be evaluated at that element, which means that we insert the element for every occurence of $x$ in the polynomial expression. To be more precise let $P\in R[x]$, with $P(x)=\sum_{j=0}a_j x^j$ be some polynomial with coefficient of type $R$ and let $b\in R$ be an element of that type. Then the \textit{evaluation} of $P$ at $b$ is given by
\begin{equation}
P(a) = \sum_{j=0} a_j b^j
\end{equation}
\begin{example} We evaluate the integer polynomials from example XXX at certain integers. We get:
\begin{itemize}
\item $P_1(2)= 2\cdot 2^2 -4\cdot 2 +17 = 17$
\item $P_2(3)= 3^{23}=94143178827$
\item $P_3(-4)= -4 = -4$.
\item $P_4(15)= 174$.
\item $P_5(0)= 1$.
\item $P_6(1274)=0$.
\item $P_7(-6)= (-6-1)(-6+2)(-6-4) = -280$.
\end{itemize}
Note however that is not possible to evaluate any of those polynomial on values of different type. It is for example strictly speaking wrong to write $P_1(0.5)$, since $0.5$ is not an integer.
\end{example}
\begin{example} We evaluate the polynomials from example XXX at certain values from $\Z_6$. We get:
\begin{itemize}
\item $P_1(2)= 2\cdot 2^2 -4\cdot 2 +5 = 2 - 2 + 5 = 5$
\item $P_2(3)= 3^{23}=3$, since $3\cdot 3=3$ in $\Z_6$.
\item $P_3(-4)= P_3(2) = 2$.
\item $P_5(0)= 1$.
\item $P_6(4)=0$.
\end{itemize}
\end{example}
\begin{exercise}
Compare both expansions of $P_7$ from $\Z[x]$ and from $\Z_6[x]$ in example XXX and example XXX and consider the definition of $\Z_6$ as given in example XXX. Can you see how the definition of $P_7$ over $\Z$ projects to the definition over $\Z_6$ if you consider the residue classes of $\Z_6$?
\end{exercise}
\paragraph{Polynomial Aithmetics}
Polynomials behave like integers in many ways. In particular they can be added, subtracted and multiplied. In addition they have their own notion of Euklidean division. Roughly speaking two polynomials are added by simply adding the coefficients of the same index and they are multiplied by applying the distributive property, that is by multiplying every term of the left factor with every term of the right factor and add the results together.

To be more precise let $ \sum _{n = 0} ^{m_1}{a} _{n}{x} ^{n} $ and
$ \sum _{n = 0} ^{m_2}{b} _{n}{x} $ be two polynomials from $ R[x]$. Then the \textit{sum} and the \textit{product} of these polynomials is defined as:
\begin{equation}
\sum _{n = 0} ^{m_1}{a} _{n}{x} ^{n} + \sum _{n = 0} ^{m_2}{b} _{n}{x } ^{n} = \sum _{n = 0} ^{may(m_1,m_2)}{({a} _{n} +{b} _{n})}{x} ^{n}
\end{equation}
\begin{equation}
\bigg (\sum _{n = 0} ^{m_1}{a} _{n}{x} ^{n} \bigg) \cdot \bigg (\sum _{n = 0} ^{m_2 }{b} _{n}{x} ^{n} \bigg) = \sum _{n = 0} ^{m_1+m_2} \sum _{i = 0} ^{n}{a} _{i }{{b} _{ni}}{x} ^{n}
\end{equation}
A rule for polynomial subtraction can be deduced from these two rules by first muliplying the subtrahend with (the polynomial) $-1$ and then add the result to the minuend.

Regarding ower definition of the degree of a polynomial, we see that the degree of the sum is always the maximum of the degrees of both summands and the degree of the product is always the degree of the factors, providing that $-\infty \cdot m= \infty$ for every integer $m\in\Z$. Using sage's definition of degree, this would not hold, as the zero polynomials degree is $-1$ is sage, which would violate this rule.
\begin{example} To given an example of how polynomial aithmetics work, consider the following two integer polynomials $P,Q\in \Z[x]$ with $P(x)= 5x^2 -4x +2$ and $Q(x)=x^3-2x^2 +5$. The sum of these two polynomials is computed by adding the coefficients of each term with equal exponent in $x$. This gives
$$
\begin{array}{lcl}
(P+Q)(x) & = & (0+1)x^3 + (5-2)x^2 + (-4 +0) x +(2+5) \\
         & = & x^3 +3x^2 -4x +7
\end{array}
$$
The product of thess two polynomials is computed by multiplication of each term in the first factor with each term in the second factor. We get
$$
\begin{array}{lcl}
(P\cdot Q)(x) & = & (5x^2 -4x +2)\cdot (x^3-2x^2 +5) \\
              & = & (5 x^5 -10 x^4 +25 x^2)+ (-4x^4 +8 x^3 -20x) + (2x^3 -4x^2+10) \\
              & = & 5 x^5 -14x^4 +10x^3+21x^2-20x +10
\end{array}
$$
We can double check our computations by invoking sage
\begin{sagecommandline}
sage: Zx = ZZ['x'] 
sage: P = Zx([2,-4,5])
sage: Q = Zx([5,0,-2,1])
sage: P
sage: Q
sage: P+Q
sage: P*Q
\end{sagecommandline}
\end{example}
\begin{example} Lets consider the polynomials of the previous example but interpreted in modular $6$ aithmetics. So we consider $P,Q\in \Z_6[x]$ again with $P(x)= 5x^2 -4x +2$ and $Q(x)=x^3-2x^2 +5$. This time we get
$$
\begin{array}{lcl}
(P+Q)(x) & = & (0+1)x^3 + (5-2)x^2 + (-4 +0) x +(2+5) \\
         & = & (0+1)x^3 + (5+4)x^2 + (2 +0) x +(2+5) \\
         & = & x^3 +3x^2 +2x +1\\
         \\
(P\cdot Q)(x) & = & (5x^2 -4x +2)\cdot (x^3-2x^2 +5) \\
              & = & (5x^2 +2x +2)\cdot (x^3+4x^2 +5) \\
              & = & (5 x^5 +2 x^4 +1x^2)+ (2x^4 +2x^3 +4x) + (2x^3 +2x^2+4) \\
              & = & 5 x^5 +4x^4 +4x^3+3x^2+4x +4
\end{array}
$$
We can double check our computations by invoking sage
\begin{sagecommandline}
sage: Z6 = Integers(6)['x'] 
sage: P = Z6([2,-4,5])
sage: Q = Z6([5,0,-2,1])
sage: P
sage: Q
sage: P+Q
sage: P*Q
\end{sagecommandline}
\end{example}

\paragraph{Euklidean Division}
The ring of polynomials shares a lot of properties with the integers. In particular there is also the concept of Euclidean division and the algorithm of long division defined for polynomials. Recalling from Euklidean division of integers XXX, we know, that given two integers $a$ and $b\neq 0$ there is always another integer $m$ and a counting number $r$ with $r<|b|$, such that $a = m\cdot b +r$ holds.

We can generalize this to polynomials, whenver the coefficients of the polynomials have a notion of division. In fact given two polynomials $A$ and $B\neq 0$ from $R[x]$, there exist two polynomials $M$ (the quotient) and $R$ (the remainder), such that
\begin{equation}
A = M\cdot B + R
\end{equation}
and $deg(R) < deg(B)$. Similar to integer Euklidean division both $M$ and $R$ are uniquely defined by these relations. 
\begin{notation}
\label{eq_polynomial_euklidean_division_notation}
Suppose that the polynomials $ A, B, M $ and $ R $ satisfy equation XX. Then we often write 
\begin{equation}
\label{def_integer_division_and_modulus}
\begin{array}{lcr}
\Zdiv{A}{B}: = M, & & \Zmod{A}{B}: = R 
\end{array}
\end{equation}
to describe the quotient and the remainder polynomials of the Euklidean division. We also say, that a polynomial $ A $ is divisible by another polynomial $ B $ if $ \Zmod{A}{B} = 0 $ holds. In this case we also write $ B | A $ and call $B$ a \textit{factor} of $A$.
\end{notation}
Analog to integers, methods to compute Euklidean division for polynomials are called \textit{polynomial division algorithms}. Probably the best known algorithm is the so called \textit{polynomial long division}. 
\begin{algorithm}\caption{Polynomial Euklidean Algorithm}
\label{alg_ext_euclid_alg}
\begin{algorithmic}[0]
\Require $A,B \in R[x]$ with $B\neq 0$, such that $Lc(B)^{-1}$ exists in $R$
\Procedure{Poly-Long-Division}{$A,B$}
\State $M \gets 0$
\State $R \gets A$
\State $d \gets deg(B)$
\State $c \gets Lc(B)$
\While{$ deg(R) \geq d$}
\State  $S := \frac{Lc(R)}{c}\cdot x^{deg(R)-d}$
\State $M \gets M + S$
\State $R \gets R - S\cdot B$
\EndWhile
\State \textbf{return} $(Q, R)$ 
\EndProcedure
\Ensure $ A=  M \cdot B + R$
\end{algorithmic}
\end{algorithm}

% https://math.stackexchange.com/questions/2140378/division-algorithm-for-polynomials-in-rx-where-r-is-a-commutative-ring-with-u
This algorithm works only when there is a notion of division by the leading coefficient of $B$. It can be generalized, but we will only need this somewhat simpler method in what follows.
\begin{example}[Polynomial Long Division] To give an example of how the previous algorithm works, lets divide the polynomial $A(x)=x^5+2x^3-9$ by the polynomial $B(x)=x^2+4x-1$. Since $B$ is not the zero polynomial and the leading coefficient of $B$ is $1$, which is invertible as an integer, we can applay algorithm XXX and our goal is to find solutions to equation \ref{xxx}, that is we need to find the quotient polynomial $M\in\Z[x]$ and the reminder polynomial $R \in \Z[x]$ such that $x^5+2x^3-9 = M(x)\cdot (x^2+4x-1) + R$. Using a notation that is mostly used in Commonwealth countries, we compute
\begin{equation}
\polylongdiv{X^5+2X^3-9}{X^2+4X-1}
\end{equation}
We therefore get $M(x)=x^3-4x^2+19x-80$ as well as $R(x)=339x-89$ and indeed we have $x^5+2x^3-9 = (x^3-4x^2+19x-80)\cdot (x^2+4x-1) + (339x-89)$, which we can double check invoking sage:
\begin{sagecommandline}
sage: Zx = ZZ['x']
sage: a = Zx([-9,0,0,2,0,1])
sage: b = Zx([-1,4,1])
sage: m = Zx([-80,19,-4,1])
sage: r = Zx([-89,339])
sage: a == m*b +r
\end{sagecommandline}
\end{example}
\begin{example} In the previous example polynomial division gave a non trivial (non vanishing, i.e non-zero) remainder. Of special interest are divisions that don't give a remainer. Sutable divisiors are called factors in that case. For example consider the polynomial $P_7$ from example XXX again. As we have shown, it can be written both as $x^3 - 4 x^2 - 11 x + 30$ as well as $(x-2)(x + 3)(x-5)$. From this we can see that the polynomials $F_1(x)=(x-2)$, $F_2(x)=(x+3)$ and $F_3(x)=(x-5)$ are all the factors of $x^3 - 4 x^2 - 11 x + 30$.
\end{example}

\paragraph{Prime Factors}
As we have seen in example XXX, points where a polynomial evaluates to zero, i.e points $x_0\in R$ with $P(x_0)=0$ are of special interest, since the polynomial $F(x)=(x-x_0)$ is a factor of $P$. Those points are called the \textit{roots} of $P$. To be more precise, let $P\in R[x]$ be a polynomial. Then the set of all roots of $P$ is defined as
\begin{equation}
R_0(P):=\{x_0\in R\;|\; P(x_0)=0\}
\end{equation}
Finding the roots of a polynomial is sometimes called solving the polynomial. It is a hard problem and has been the subject of much research throughout history. In fact it is well known that for polynomials of degree $5$ or higher there is, in general, no closed expression, from which the roots can be deduced. 

Therefore, except for very low degrees, root finding of polynomials consists of finding approximations of the roots.
\begin{example}
Consider our integer polynomial $P_7(x)=x^3 - 4 x^2 - 11 x + 30$ from example XXX again. We know that it's set of roots is given by $R_0(P_7)=\{-3,2,5\}$.

On the other hand consider the integer polynomial $P(x)=x^2-2$. We know that it can not have any integer root, since $x^2-2=0$ implies that any root $x_0$ would have to satisfy $x_0^2=2$, but there is no integer that squares into $2$ ($\sqrt{2}$ is not an integer).
\end{example}
It can be shown, that if $m$ is the degree of a polynomial $P$, then $P$ can not have more then $m$ roots. However in general polynomials can have less then $m$ roots. A special case occures if a polynomial has no roots at all in which case it is called \textit{irreducible}. 

In a sense, irreducible polynomials are for polynomials what prime numbers are for integers. They are the basic building blocks from which all other polynomials can be constructed. This can be expressed in a polynomial analog of the fundamental theorem of arithmetics XXX. To be more precise, let $P \in R[x]$ be any polynomial. Then there are always irreducible polynomials $F_1, F_2, \ldots, F_k \in R[x]$, such that
\begin{equation}
P = F_1 \cdot F_2 \cdot \ldots \cdot F_k \;.
\end{equation}
This representation is unique, except for permutations in the factors and is called the \textbf{prime factorization} of $P$.

Every root $x_0$ of a polynomial defines a prime factor $(x-x_0)$ of $P$. The converse however is not necessarily true, because a polynomial can have irreducible prime factors. 
\begin{example}[Prime Factorization] To give an example consider the polynomial 
$P=x^7 + 3 x^6 + 3 x^5 + x^4 - x^3 - 3 x^2 - 3 x - 1$. We can check that this polynomial has the following roots $R_0(P)=\{-3,2\}$ and that the irreducible polynomial $F_1(x)=x^2-2$ is a factor of $P$. We can use this data to compute the prime factorization of $P$, by succesive polynomial division of $P$ by $(x+3)$, $(x-2)$ as well as $x^2-2$. We get:
$$
P= (x^2-2)(x^4-4)(x-2)(x+3)
$$
\end{example}
\paragraph{Lange interpolation}
One particulary nice property of polynomials is that a polynomial of degree $m$ is completely determined on $m+1$ evaluation points. Seeing this from a different angle, we can (sometimes) uniquely derive a polynomial of degree $m$ from a set 
\begin{equation}
\label{def_lagrange_interpolation_set}
S= \{(x_0,y_0), (x_1,y_1),\ldots,(x_m,y_m)\;|\; x_i\neq x_j\text{ for all indices i and j}\}
\end{equation}
This "few to many" property of polynomials is used in many places, like for example in erasure codes. It is also of importance in snarks and we therefore need to understand a method to actually compute a polynomial from a set of points. 

If the coefficients of the polynomial we want to find have a notion of multiplicative inverse, it is always possible to find such a polynomial and one method is called \textit{Lagrange interpolation}. It works as follows:
Give a set like \ref{def_lagrange_interpolation_set}, a polynomial $P$ of degree $m+1$ with $P(x_i)=y_i$ for all pairs $(x_i,y_i)$ from $S$ is given by the following algorithm:

\begin{algorithm}\caption{Lagrange Interpolation}
\label{alg_lagrange_interplation}
\begin{algorithmic}[0]
\Require $R$ must have multiplicative inverses
\Require $S= \{(x_0,y_0), (x_1,y_1),\ldots,(x_m,y_m)\;|\; x_i,y_i\in R, x_i\neq x_j\text{ for all indices i and j}\}$  
\Procedure{Lagrange-Interpolation}{$S$}
\For{$j \in (0\ldots m)$}
\State  $l_j(x) \gets \Pi_{i=0;i\neq j}^{m}\frac{x-x_i}{x_j-x_i} = \frac{(x-x_0)}{(x_j-x_0)} \cdots \frac{(x-x_{j-1})}{(x_j-x_{j - 1})} \frac{(x-x_{j+1})}{(x_j-x_{j+1})} \cdots \frac{(x-x_m)}{(x_j-x_m)}$
\EndFor
\State $P\gets \sum_{j=0}^m y_j\cdot l_j$
\State \textbf{return} $P$ 
\EndProcedure
\Ensure $P\in R[x]$ with $deg(P)=m$
\Ensure $P(x_j)=y_j$ for all pairs $(x_j,y_j)\in S$
\end{algorithmic}
\end{algorithm}

\begin{example} Lets consider the set $S=\{(0,4),(-2,1),(2,3)\}$ of fractional numbers. We know that this set determines a unique integer polynomial of degree $2$ in $\mathbb{Q}[x]$. To compute this polynomial, we use the Lagrange interpolation algorithm from XXX. We compute 
$$
\begin{array}{lcccl}
l_0(x) & = & \frac{x-x_1}{x_0-x_1}\cdot\frac{x-x_2}{x_0-x_2}
         = \frac{x+2}{0+2}\cdot\frac{x-2}{0-2}
         =  -\frac{(x+2)(x-2)}{4}
       & = & -\frac{1}{4}(x^2-4)\\
         \\
l_1(x) & = & \frac{x-x_0}{x_1-x_0}\cdot\frac{x-x_2}{x_1-x_2}
         = \frac{x-0}{-2-0}\cdot \frac{x-2}{-2-2}
         = \frac{x(x-2)}{8}
       & = & \frac{1}{8}(x^2-2x)\\
         \\
l_2(x) & = & \frac{x-x_0}{x_2-x_0}\cdot\frac{x-x_1}{x_2-x_1}
         = \frac{x-0}{2-0}\cdot\frac{x+2}{2+2}
         = \frac{x(x+2)}{8}
       & = & \frac{1}{8}(x^2+2x)\\
       \\
       \\
P(x)   & = &  4\cdot (-\frac{1}{4}(x^2-4)) + 1\cdot \frac{1}{8}(x^2-2x) + 3\cdot \frac{1}{8}(x^2+2x) 
       & & \\
       & = & -x^2+4 + \frac{1}{8}x^2-\frac{1}{4} x + \frac{3}{8}x^2+\frac{3}{4} x 
       & = & -\frac{1}{2} x^2 +\frac{1}{2} x + 4        
\end{array}
$$
And indeed evaluation of $P$ on the $x$-values of $S$ gives the correct points, since $P(0)=4$, $P(-2)=1$ and $P(2)=3$.
\end{example}
\begin{example} To give another example, more relevant to the topics of this book, lets consider the same set $S=\{(0,4),(-2,1),(2,3)\}$ as in the pevious example but this times interpreted as residue classes modulo $5$ as we derive in XXX. Since we know
that multiplicative inverses exist in $\Z_5$, algorithm XXX applies and we can compute a unique polynomial of degree 2 in $\Z_5[x]$ from $S$. We can use the lookup tables XXX for computation in $\Z_5$ and get
$$
\begin{array}{lcccl}
l_0(x) & = & \frac{x-x_1}{x_0-x_1}\cdot\frac{x-x_2}{x_0-x_2}
         = \frac{x+2}{0+2}\cdot\frac{x-2}{0-2}
         =  \frac{(x+2)(x-2)}{-4}
         =  \frac{(x+2)(x+3)}{1}
       & = & x^2+1\\
         \\
l_1(x) & = & \frac{x-x_0}{x_1-x_0}\cdot\frac{x-x_2}{x_1-x_2}
         = \frac{x-0}{-2-0}\cdot \frac{x-2}{-2-2}
         = \frac{x}{3}\cdot \frac{x+3}{1}
         = 2(x^2+3x)
       & = & 2x^2+x\\
         \\
l_2(x) & = & \frac{x-x_0}{x_2-x_0}\cdot\frac{x-x_1}{x_2-x_1}
         = \frac{x-0}{2-0}\cdot\frac{x+2}{2+2}
         = \frac{x(x+2)}{3}
         = 2(x^2+2x)
       & = & 2x^2+4x\\
       \\
       \\
P(x)   & = &  4\cdot (x^2+1) + 1\cdot (2x^2+x) + 3\cdot (2x^2+4x) 
       & & \\
       & = & 4x^2+4 + 2x^2 +x + x^2+2x
       & = & 2x^2 +3x +4       
\end{array}
$$
And indeed evaluation of $P$ on the $x$-values of $S$ gives the correct points, since $P(0)=4$, $P(-2)=1$ and $P(2)=3$.
\end{example}

\begin{exercise}
Consider example XXX and example XXX again. Why is it not possible to applay algorithm XXX if we consider $S$ as a set of integers, nor as a set in $\Z_6$?
\end{exercise}




\chapter{Algebra}
We gave an introduction to the basic computational skills needed for a pen \& paper approach to SNARKS in the previous chapter. In this chapter we get a bit more abstract and clarify a lot of mathematical terminology and jargon.

When you read papers about cryptography or mathematical papers in general, you will frequently stumble across algebraic terms like \textit{groups}, \textit{fields},\textit{rings} and similar. To understand what is going on, it is necessary to get at least some understanding of these terms. In this chapter we therefore with a short introduction to those terms.

In a nutshell, algebraic types like groups or fields define sets that are analog to numbers to various extend, in the sense that you can add, subtract, multiply or divide on thoses sets. 

We know many example of sets that fall under those categories, like the natural numbers, the integers, the ratinal or the real numbers. they are in some sense already the most fundamental examples.

\section{Groups} Groups are abstractions that capture the essence of mathematical phenomena, like addition and subtraction, multiplication and division, permutations, or symmetries.

To understand groups, remember back in school when we learned about addition and subtraction of integers (Forgetting about integer multiplication for a moment). We learned that we can always add two integers and that the result is guranteed to be an integer again. We also learned how to deal wih brackets, that nothing happens, when we add zero to any integer, that it doesn't matter in which order we add a given set of integers and that for every integer there is always another integer (the negative), such that when we add both together we get zero. 

These conditions are the defining properties of a group and mathematicians have recognozed that the exact same set of rules can be found in very different mathematical structures. It therefore makes sense to give a formulation of what a group should be, detached from any concrete example. This allows one to handle entities of very different mathematical origins in a flexible way, while retaining essential structural aspects of many objects in abstract algebra and beyond. 

Distilling these rules to the smallest independend list of properties and making them abstract we arrive at the definition of a group:

A \textbf{group} $(\G,\cdot) $ is a set $ \G$, together with a map $ \cdot: \G \times \G \to \G $, called the group law, such that the following properties hold:
\begin{itemize}
\item (Existence of a neutral element) There is a $e\in\G$ for all $g\in\G$, such that $e\cdot g=g$ as well as $g\cdot e = g$.
\item (Existence of an inverse) For every $g\in\G$ there is a $g^{-1}\in\G$, such that $g\cdot g^{-1}=e$ as well as $g^{-1}\cdot g = e$.
\item (Associativity) For every $g_1,g_2,g_3\in\G$ the equation 
$g_1\cdot(g_2\cdot g_3) = (g_1\cdot g_2)\cdot g_3$ holds.
\end{itemize}
Rephrasing the abstract definition in more laymans terms, a group is something, where we can do computations that resembles the behaviour of addition of integers. Therefore when the reader reads the term group they are adviced to think of something where can combine some element with another element into a new element in a way that is reversable and where the order of combining many elements doesn't matter.
\begin{notation}
Let $(\mathbb{G}\cdot)$ be a finite group. If there is no risk of ambigously we frquently drop the symbol $\cdot$ and simply write $\mathbb{G}$ as a notation for the group keeping the group law implicit.
\end{notation}
As we will see in what follows, groups are all over the place in cryptography and in SNARKS. In particular we will see in XXX, that the set of points on an elliptic curve define a group, which is the most important example in this book. To give some more familiar examples first:
\begin{example}[Integer Addition and Subtraction]
The set $(\Z,+)$ of integers together with integer addition is the archetypical example of a group, where the group law is traditionally written as $+$ (instead of $\cdot$). To compare integer addition against the abstract axioms of a group, we first see that the neutral element $e$ is the number $0$, since $a+0=a$ for all integers $a\in $ and that the inverse of a number is the negative, since $a+(-a)=0$, for all $a\in\Z$. In addition we know that $(a+b)+c=a+(b+c)$, so integers with addition are indeed a group in the abstract sense.
\end{example}
\begin{example}[The trivial group]
The most basic example of a group, is group with just one element $\{\bullet\}$ and the group law $\bullet\cdot \bullet=\bullet$. 
\end{example}
%\begin{example}[Rotations]
%To give an example of a group that has effects in the real world, consider a dice. Then our group is the set of all possible ways to rotate the dice by 90 degrees alonng an imagined axix through two opposite faces. The group law is composition of rotations. So say hold the dice with two fingers at $1$ and $6$. $1$ is the face that points towards you and $5$ is the top face. Then you first rotate the dice along the $1$-$6$ axix by 90 degrees clockwise, such that now $3$ is the top face. Then you hold the dice at $5$ and 
%\end{example}
\paragraph{Commutative Groups} When we look at the general definition of a group we see that it is somewhat different from what we know from integers. For integers we know, that it doesnt matter in which order we add two integers, as for example $4+2$ is the same as $2+4$. However we also know from example XXX, that this is not always the case in groups. 

To capture the special case of a group where the order in which the group law is executed doesn't matter, the concept of so called a \textbf{commutative group} is introduced. To be more precise a group is called commutative if  $g_1\cdot g_2 = g_2 \cdot g_1$ holds for all $g_1,g_2\in\G$. 
\begin{notation}
In case $(\G,\cdot)$ is a commutative group, we frequently use the so called \textit{additive notation} $(\G,+)$, that is we write $+$ instead of $\cdot$ for the group law and $-g:=g^{-1}$ for the inverse of an element $g\in\G$.
\end{notation}
\begin{example} Consider the group of integers with integer addition again.
Since $a+b=b+a$ for all integers, this group is the archetypical example of a commutative group. Since there are infinite many integers, $(\Z,+)$ is not a finite group.
\end{example}
\begin{example} Consider our definition of modulo $6$ residue classes $(\Z_6,+)$ as defined in the addition table from example XXX. As we see the residue class $0$ is the neutral element in modulo $6$ arithmetics and the inverse of a residue class $r$ is given by $6-r$, since $r+(6-r)=6$, which is congruent to $0$, since $\Zmod{6}{6}=0$. Moreover $(r_1+r_2)+r_3=r_1+(r_2+r_3)$ is inherited from integer arithmetic.  

We therefore see that $(\Z_6,+)$ is a group and since addition table XX is symmetric, we see $r_1+r_2 = r_2+r_1$ which shows that $(\Z_6,+)$ is commutative. 
\end{example}
The previous example provided us with an important example of commuative groups that are important in this book. Abstracting from this example and considering residue classes $(\Z_n,+)$ for arbitrary moduli $n$, it can be shown that $(\Z,+)$ is a commutative group with neutral element $0$ and additive inverse $n-r$ for any element $r\in\Z_n$. We call such a group the \textit{reminder class groups} of modulus $n$.

Of particular importance for pairing based cryptography in general and snarks in particular are so called \textit{pairing maps} on commutative groups. To be more precise let $\G_1$, $\G_2$ and $\G_3$ be three commutative groups. For historical reasons, we write the group law on $\G_1$ and $\G_2$ in additive notation and the group law on $\G_3$ in multiplicative notation. Then a \textbf{pairing map} is a function
\begin{equation}
e(\cdot,\cdot): \G_1 \times \G_2 \to \G_3
\end{equation}
that takes pairs $(g_1,g_2)$ (products) of elements from $\G_1$ and $\G_2$ and maps them somehow to elements from $\G_3$, such that the \textit{bilinearity} property holds: For all $g_1,g_1'\in \G_1$ and $g_2\in \G_2$ we have $e(g_1+ g_1',g_2)= e(g_1,g_2)\cdot e(g_1',g_2)$ and for all $g_1\in \G_1$ and $g_2, g_2'\in \G_2$ we have $e(g_1,g_2+ g_2')= e(g_1,g_2)\cdot e(g_1,g_2')$. 

A pairing map is called \textit{non-degenerated}, if whenever the result of the pairing is the neutral element in $\G_3$, one of the input values must be the neutral element of $\G_1$ or $\G_2$. To be more precise $e(g_1,g_2)=e_{\G_3}$ implies $g_1=e_{\G_1}$ or $g_2=e_{\G_2}$.

So roughly speaking bilinearity means, that it doesn't matter if we first execute the group law on any side and then apply the bilinear map of if we first applay the bilinear map and then apply the group law. Moreover non-degeneray means that the result of the pairing is zero, only if at least one of the input values is zero.
\begin{example}Maybe the most basic example of a non-degenerate pairing is optained, if we take $\G_1$, $\G_2$ and $\G_3$ all to be the group of integers with addition $(\Z,+)$. Then the following map 
$$
e(\cdot,\cdot): \Z \times \Z \to \Z \; (a,b)\mapsto a\cdot b
$$
defines aa non-degenerate pairing. To see that observe, that bilinearity follows from the distriutive law of integers, since for $a,b,c\in \Z$, we have $e(a+b,c)=(a+b)\cdot c = a\cdot c + b\cdot c = e(a,c)+ e(b,c)$ and the same reasoning is true for the second argument.

To the that $e(\cdot,\cdot)$ is non degenrate, assume that $e(a,b)=0$. Then a$\cdot b =0$ and this implies that $a$ or $b$ must be zero.
\end{example} 

\begin{exercise} Consider example XXX again and let $\F_5^*$ be the set of all remainder classes from $\F_5$ without the class $0$. Then $\F_5^*=\{1,2,3,4\}$. Show that $(\F_5^*,\cdot)$ is a commutative group. 
\end{exercise}
\begin{exercise} Generalizing the previous exercise, consider general moduli $n$ and let $\Z_n^*$ be the set of all remainder classes from $\Z_n$ without the class $0$. Then $\Z_n^*=\{1,2,\ldots,n-1\}$. Give a counter example to show that $(\Z^*_n,\cdot)$ is not a group in general. 

Find a condition, such that $(\Z^*_n,\cdot)$ is a commutative group, compute the neutral element, give a closed form for the inverse of any element and proof the commutative group axioms.
\end{exercise}
\begin{exercise} Consider the remainder class groups $(\Z_n,+)$ for some modulus $n$. Show that the map
$$
e(\cdot,\cdot): \Z_n \times \Z_n \to \Z_n \; (a,b)\mapsto a\cdot b
$$
is bilinear. Why is it not a pairing in general and what condition must be imposed on $n$, such that the map is a pairing?
\end{exercise}
\paragraph{Finite groups} As we have seen in the previous examples, groups can either contain infinite many elements (as the integers) or finitely many elements as for example the remainder class groups $(\Z_n,+)$. To capture this distinction a group is called a \textit{finite group}, if the underlying set of elements is finite. In that case the number of elements of that group is called its \textbf{order}.
\begin{notation}
Let $\mathbb{G}$ be a finite group. Then we frquently write $ord(\mathbb{G})$ or  $|\mathbb{G}|$ for the order of $\mathbb{G}$.
\end{notation}
\begin{example}
Consider the remainder class groups $(\Z_6,+)$ and $(\F_5,+)$ from example XXX and example XXX and the group $(\F_5^*,\cdot)$ from exercise XX. We can easily see that the order of $(\Z_6,+)$ is $6$, the order of $(\F_5,+)$ is five and the order of $(\F_5^*,\cdot)$ is $4$.

To be more general, considering arbitrary moduli $n$, then we know from Euklidean division, that the order of the remainder class group $(\Z_n,+)$ is $n$. 
\end{example}
\begin{exercise}The RSA crypto system is based on a modulus $n$ that is typically the product of two prime numbers of size $2048$-bits. What is (approximately) the order of the rainder class group $(\Z_n,+)$ in this case? 
\end{exercise}
\paragraph{Generators} Of special interest, when working with groups are sets of elements that can generate the entire group, by applying the group law repeadly to those elements or their inverses only. 

Of course every group $\G$ has trivially a set of generators, when we just consider every element of the group to be in the generator set. So the more interesting question is to find the smallest set of generators. Of particular interest in this regard are groups that have a single generator, that is there exist an element $g\in\G$, such that every other element from $\G$ can be computed by repeated combination of $g$ and its inverse $g^{-1}$ only. Those groups are called \textbf{cyclic groups}.
\begin{example} The most basic example of a cyclic group are the integers $(\Z,+)$ with integer addition. To see that observe that $1$ is a generator of $\Z$, since every integer can be obtained by repeadly add either $1$ or its inverse $-1$ to itself. For example 
$-4$ is generated by $-1$, since $-4=-1+(-1)+(-1)+(-1)$. 
\end{example}
\begin{example} Consider a modulus $n$ and the remainder class groups $(\Z_n,+)$ from example XXX. These groups are cyclic, with generator $1$, since every other element of that group can be constructed by repeadly adding the remainder class $1$ to itself. Since $\Z_n$ is also finite, we know that $(\Z_n,+)$ is a finite cyclic group of order $n$.
\end{example}
\begin{example} Let $p\in\P$ be prime number and $(\F_p^*,\cdot)$ the finite group from exercise XXX. Then $(\F_p^*,\cdot)$ is cyclic and every element $g\in\F_q^*$ is a generator. 
\end{example}
\paragraph{The discrete Logarithm problem}
In cryptography in general and in snark development in particular, we often do computations "in the exponent" of a generator. To see what this means, observe, that when 
$\G$ is a cyclic group of order $n$ and $g\in \G$ is a generator of $\G$, then there is a map, called the \textbf{exponential map} with respect to the generator $g$
\begin{equation}
g^{(\cdot)}: \Z_n \to \G\; x \mapsto g^x
\end{equation}
where $g^x$ means "multiply $g$ $x$-times by itself and $g^0=e_{\G}$. This map has the remarkable property maps the additive group law of the remainder class group $(\Z_n,+)$ in a one-to-one correspondence to the group law of $\G$. 

To see that first observe, that since $g^0:=e_{\G}$ by definition, the neutral element of $\Z_n$ is mapped to the neutral element of $\G$ and since $g^{x+y}=g^x\cdot g^y$, the map respects the group laws. 

Since the exponential map respects the group law, it doesn't matter if we do our computation in $\Z_n$ before we write the result into the exponent of $g$ or afterwards. The result will be the same. This is what is usually meant by saying we do our computations "in the exponent".
\begin{example} Consider the multiplicative group $(\F_{5}^*,\cdot)$ from example XXX. We know that $\F_{5}^*$ is a cyclic group of order $4$ and that every element is a generator. Choose $3\in\F_5^*$, we then know that the map
$$
3^{(\cdot)}: \Z_4 \to \F_5^* \; x \mapsto 3^x
$$
respects the group law of addition in $\Z_4$ and the group law of multiplication in $\F_5^*$.
And indeed doing a computation like 
\begin{align*}
3^{2+3-2} &=3^{3}\\
          & = 2
\end{align*}
in the exponent gives the same result as doing the same computation in $\F*_5$, that is 
\begin{align*}
3^{2+3-2} &= 3^2 \cdot 3^3 \cdot 3^{-2}\\
          &= 4\cdot 2 \cdot (-3)^2\\
          &= 3\cdot 2^2\\
          &= 3\cdot 4 \\
          &= 2
\end{align*}
\end{example}
Since the exponential map is a one-to-one correspondence, that respects the group law, it can be shown that this map has an inverse
\begin{equation}
log_g(\cdot): \G \to \Z_n\; x \mapsto log_g(x)
\end{equation}
which is called the \textbf{discrete logarithm} map with respect to the base $g$. Discrete logarithms are highly importsnt in cryptography as there are groups, such that the exponential map and its inverse the discrete logarithm, are believed to be one way functions, that is while it is possible to compute the exponential map in polynomial time, computing the discrete log takes (sub)-exponential time. 

Now consider a finite cyclic group $\G$ of order $n$ and a generator $g$ of $\G$. The \textbf{discrete logarithm problem} is then the task, to find a solution $x\in\Z_n$, to the equation 
\begin{equation}
h = g^x
\end{equation}
for some given $h\in\G$. In groups where the expontial map and the discrete logarithm map are believed to be examples of one way functions, it is computationally hard to find solutions to this equation.
\section{Commutative Rings}
Thinking of integers again, we know, that there are actually two operations addition and multiplication and as we know addition defines a group structure on the set of integers. However multiplication does not define a group structure as we know that integers in general don't have multiplicative inverses. 

Combinations like this are captured by the concept of a so called \textit{commutative ring with unit}. To be more precise, a commutative ring with unit $ (R, +, \cdot, 1) $ is a set $R$, provided with two maps $ +: R \cdot R \to R $ and $ \cdot: R \cdot R \to R $, called \textit{addition} and \textit{multiplication}, such that the following conditions hold:
\begin{itemize}
\item $ \left (R, + \right) $ is a commutative group, where the neutral element is denoted  with $ 0 $.
\item (Commuativity of the multiplication) We have $r_1\cdot r_2 = r_2\cdot r_1$ for all $r_1, r_2\in R$. 
\item (Existence of a unit) There is an element $1\in R$, such that $1\cdot g$ holds for all $g\in R$, 
\item (Associativity) For every $g_1,g_2,g_3\in\G$ the equation 
$g_1\cdot(g_2\cdot g_3) = (g_1\cdot g_2)\cdot g_3$ holds. 
\item (Distributivity) For all $ g_1, g_2, g_3 \in R $ the distributive laws
$ g_1 \cdot \left (g_2 + g_3 \right) = g_1 \cdot g_2 + g_1 \cdot g_3$ holds.
\end{itemize}
\begin{example}[The Ring of Integers] The set $\Z$ of integers with the usual addition and multiplication is the archetypical example of a commutative ring with unit $1$. 
\end{example}
\begin{example}[Underlying commutative group of a ring] Every commutative ring with unit $(R,+,\cdot,1)$ gives rise to group, if we just forget about the multiplication
\end{example}
The following example is more interesting. The motivated reader is encuraged to think through this example, not so much because we need this in what follows, but more so as it helps to detach the reader from familiar styles of computation. 
\begin{example} Let $S:=\{\bullet,\star,\odot,\otimes\}$ be a set that contains four elements and let adiition and multiplication on $S$ be defined as follows:
\begin{center}
  \begin{tabular}{c | c c c c c c}
    $\cup$ & $\bullet$ & $\star$ & $\odot$ & $\otimes$ \\\hline
    $\bullet$ & $\bullet$ & $\star$ & $\odot$ & $\otimes$ \\
    $\star$ & $\star$ & $\odot$ & $\otimes$ & $\bullet$ \\
    $\odot$ & $\odot$ & $\otimes$ & $\bullet$ & $\star$ \\
    $\otimes$ & $\otimes$ & $\bullet$ & $\star$ & $\odot$ \\
  \end{tabular} \quad \quad \quad \quad
  \begin{tabular}{c | c c c c c c}
$ \circ $ & $\bullet$ & $\star$ & $\odot$ & $\otimes$ & \\\hline
        $\bullet$ & $\bullet$ & $\bullet$ & $\bullet$ & $\bullet$ &\\
        $\star$ & $\bullet$ & $\star$ & $\odot$ & $\otimes$ &\\
        $\odot$ & $\bullet$ & $\odot$ & $\bullet$ & $\odot$ &\\
        $\otimes$ & $\bullet$ & $\otimes$ & $\odot$ & $\star$ &\\
  \end{tabular}
\end{center}
Then $(S,\cup,\circ)$ is a ring with unit $\star$ and zero $\bullet$. It therefore makes sense to ask for solutions to equations like this one:
Find $x\in S$ such that
$$
\otimes \circ (x \cup \odot ) = \star
$$
To see how such a "moonmath equation" can be solved, we have to keep in mind, that rings behaves mostly like normal number when it comes to bracketing and computation rules. The only differences are the symbols and the actual way to add and multiply. With this we solve the equation for $x$ in the "usual way"
\begin{align*}
\otimes \circ (x \cup \odot ) &= \star & \text{ \# aplly the distributive law}\\
\otimes \circ x \cup \otimes \circ \odot  &= \star &\# \otimes \circ \odot = \odot\\
\otimes \circ x \cup \odot  &= \star & \text{\# concatenate the $\cup$ inverse of $\odot$ to both sides}\\
\otimes \circ x \cup \odot \cup -\odot  &= \star \cup -\odot & \# \odot \cup -\odot = \bullet\\
\otimes \circ x \cup \bullet &= \star \cup -\odot & \text{\# $\bullet$ is the $\cup$ neutral element}\\
\otimes \circ x &= \star \cup -\odot & \text{\# for $\cup$ we have $-\odot = \odot$} \\
\otimes \circ x &= \star \cup \odot &\# \star \cup \odot = \otimes \\
\otimes \circ x &= \otimes  &\text{\# concatenate the $\circ$ inverse of $\otimes$ to both sides}\\
(\otimes)^{-1}\circ \otimes \circ x &= (\otimes)^{-1}\circ \otimes & \text{\# multiply with the multiplicative inverse}\\
\star \circ x &= \star\\
x &= \star
\end{align*}
So even despite this equation looked really alien on the surface, computation was basically exactly the way "normal" equation like for fractional numbers are done.

Note however that in a ring, things can be very different, then most are used to, whenever a multiplicative inverse would be needed to solve an equation in the usual way. For example the equation
$$
\odot \circ x = \otimes
$$
can not be solved for $x$ in the usual way, since there is no multiplicative inverse for $\odot$ in our ring. And in fact looking at the multiplication table we see that no such $x$ exits. On another example the equation
$$
\odot \circ x = \odot
$$
can has not a single solution but two $x\in\{\star, \otimes\}$. Having no or two solutions is certainly not something to expect from types like $\mathbb{Q}$. 
\end{example}
\begin{example} Considering polynomials again, we note from their definition, that what we have called the type $R$ of the coefficients, must in fact be a commutative ring with unit, since we need addition, multiplication, commutativity and the existence of a unit for $R[x]$ to have the properties we expect. 

Now considering $R$ to be a ring, addition and multiplication of polynomials as defined in XXX, actually makes $R[x]$ into a commutative ring with unit, too, where the polynomial $1$ is the multiplicative unit.
\end{example}
\begin{example} Let $n$ be a modulus and $(\Z_n,+,\cdot)$ the set of all remainder classes of integers modulo $n$, with the projection of integer addition and multiplication as defined in XXX. It can be shown that $(\Z_n,+,\cdot)$ is a commutative ring with unit $1$.
\end{example}
Considering the exponential map from XXX again, let $\G$ be a finite cyclic group of order $n$ with generator $g\in\G$. Then the ring structure of $(\Z_n,+,\cdot)$ is mapped onto the group structure of $\G$ in the following way:
\begin{align*}
g^{x+y} &= g^x\cdot g^y & \text{for all } x,y\in\Z_n\\
g^{x\cdot y} &= \left( g^x\right)^y & \text{for all } x,y\in\Z_n
\end{align*}
This of particular interest in cryptographic and snarks, as it allows for the evaluation of polynomials with coefficients in $\Z_n$ to be evaluated "in the exponent". To be more precise let $p\in \Z_n[x]$ be a polyninomial with $p(x)=a_m\cdot x^m+a_{m-1}x^{m-1}+\ldots + a_1x +a_0$. Then the previously defined exponential laws XXX imply that
\begin{align*}
g^{p(x)} & = g^{a_m\cdot x^m+a_{m-1}x^{m-1}+\ldots + a_1x +a_0}\\
         & = \left(g^{x^m}\right)^{a_m}\cdot \left(g^{x^{m-1}}\right)^{a_{m-1}}\cdot \ldots\cdot \left(g^{x}\right)^{a_1}\cdot g^{a_0}
\end{align*}
and hence to evaluate $p$ at some point $s$ in the exponent, we can insert $s$ into the right hand side of the last equation and evaluate the product.
 
As we will see this is a key insight to understand many snark protocols like e.g. Groth16 or XXX.
\begin{example} To give an example for the evaluation of a polynomial in the exponent of a finite cyclic group, xonsider the exponential map 
$$
3^{(\cdot)}: \Z_4 \to \F_5^* \; x \mapsto 3^x
$$
from example XXX. Choosing the polynomial $p(x)= 2x^2 +3x +1$ from $\Z_4[x]$, we can evaluate the polynomial at say $x=2$ in the exponent of $3$ in two different ways. On the one hand side we can evaluate $p$ at $2$ and then write the result into the expinent, which gives
\begin{align*}
3^{p(2)} &=3^{2\cdot 2^2+3\cdot 2 +1}\\
          & = 3^{2\cdot 0 +2 +1}\\
          & = 3^{3}\\
          & = 2
\end{align*}
and on the other hand we can use the right hand side of equation to evaluate $p$ at $2$ in the exponent of $3$, which gives: 
\begin{align*}
3^{p(2)} &= \left(3^{2^2}\right)^2 \cdot \left(3^{2}\right)^3\cdot 3^1\\
         &= \left(3^{0}\right)^2 \cdot 3^3\cdot 3\\
         &= 1^2 \cdot 2 \cdot 3\\
         &= 2 \cdot 3\\
         &= 2
\end{align*}
\end{example}

\section{Fields}
In this chapter we started with the definition of a group, which we the expended into the definition of a commutative ring with unit. Those rings generalize the behaviour of integers. In this section we will look at the special case of commutative rings, where every element, other then the neutral element of addition, has a multiplicative inverse. Those structures behave very much like the rational numbers $\mathbb{Q}$, which are in a sense an extension of the ring of integers, that is constructed by just including newly defined multiplicative inverses (the fractions) to the integers. 

Now considering the definition of a ring XXX again, we define a \textbf{field} $ (\F, +, \cdot) $ to be a set $ \F$, together with two maps $ +: \F \cdot \F \to \F $ and $ \cdot: \F \cdot \F \to \F $, called \textit{addition} and \textit{multiplication}, such that the following conditions holds
\begin{itemize}
\item $ \left (\F, + \right) $ is a commutative group, where the neutral element is denoted by $ 0 $.
\item $ \left (\F \setminus \left \{0 \right \}, \cdot \right) $ is a commutative group, where the neutral element is denoted by $ 1 $.
\item (Distributivity) For all $ g_1, g_2, g_3 \in \F $ the distributive law
$g_1 \cdot \left (g_2 + g_3 \right) = g_1 \cdot g_2 + g_1 \cdot g_3$ holds.
\end{itemize}
If a field is iven and the definition of its addition and multiplication is not ambiguous, we will often simple write $\F$ instead of $(\F,+,\cdot)$ to describe it. We moreover write $\F^*$ to describe the multiplicative group of the field, that is the set of elements, except the neutral element of addition, with the multiplication as group law.

The \textbf{characteristic} $char(\F)$ of a field $ \F $ is the smallest natural number $ n \geq 1 $, for which the $ n $ -fold sum of $ 1 $ equals zero, i.e. for which $ \sum_{i = 1} ^ n 1 = 0 $. If such a $ n> 0 $ exists, the field is also called to have a \textit{finite characteristic}. If, on the other hand, every finite sum of $1$ is not equal to zero, then the field is defined to have characteristic $ 0 $.
\begin{example}[Field of rational numbers] Probably the best known example of a field is the set of rational numbers $\mathbb{Q}$ together with the usual definition of addition, subtraction, multiplication and division. Since there is no counting number $n\in \N$, such that $\sum_{j=0}^n 1 =0$ in the rational numers, the characteristic $char(\mathbb{Q})$ of the field $\mathbb{Q}$ is zero. In sage rational numbers are called like this
\begin{sagecommandline}
sage: QQ
sage: QQ(1/5) # Get an element from the field of rational numbers
sage: QQ(1/5) / QQ(3) # Division
\end{sagecommandline}
\end{example}
\begin{example}[Field with two elements] It can be shown that in any field, the neutral element $0$ of addition must be different from the neutral element $1$ of multiplication, that is we always have $0\neq 1$ in a field. From this follows that the smallest field must contain at least two elements and as the following addition and multiplication tables show, there is indeed a field with two elements, which is usually called $\F_2$:

Let $\F_2:=\{0,1 \}$ be a set that contains two elements and let addition and multiplication on $\F_2$ be defined as follows:
\begin{center}
  \begin{tabular}{c | c c c}
    + & 0 & 1 \\\hline
    0 & 0 & 1\\
    1 & 1 & 0 \\
  \end{tabular} \quad \quad \quad \quad
  \begin{tabular}{c | c c c}
$\cdot$ & 0 & 1 \\\hline
      0 & 0 & 0 \\
      1 & 0 & 1 \\
  \end{tabular}
\end{center}
Since $1+1=0$ in the field $\F_2$, we know that the characteristic of $\F_2$ is there, that is we have $char(\F_2)=0$.

For reasons we will understand better in XXX, sage defines this field as a so called Galois field with 2 elements. It is called like this:
\begin{sagecommandline}
sage: F2 = GF(2)
sage: F2(1) # Get an element from GF(2)
sage: F2(1) + F2(1) # Addition
sage: F2(1) / F2(1) # Division
\end{sagecommandline}
\end{example}
\begin{example}
Both the real numbers $\mathbb{R}$ as well as the complex numbers $\mathbb{C}$ are well known examples of fields.
\end{example}
\begin{exercise}
Consider our remainder class ring $(\F_5,+,\cdot)$ and show that it is a field. What is the characteristic of $\F_5$?
\end{exercise}
\paragraph{Prime fields}
As we have seen in the variou examples of the previous sections, modular arithmetics behaves in many ways similar to ordinary arithmetics of integers, which is due to the fact that remainder class sets $\Z_n$ are commutative rings with units.

However at the same time we have seen in XXX, that, whenever the modulus is a prime number, every remainder class other then the zero class, has a modular multiplicative inverse. This is an important observation, since it immediately implies, that in case of a prime number, the modulus $\Z_n$ is not just a ring but actually a \textit{field}. Moreover since $\sum_{j=0}^n 1 = 0$ in $\Z_n$, we know that those fields have finite characteristic $n$ 

To distinguish this important case from arbitrary reminder class rings, we write  $ (\F_p, +, \cdot) $ for the field of all remainder classes for a prime number modulus $p \in \Prim$ and call it the \textbf{prime field} of characteristic $p$.

Prime fields are the foundation for many of the contemporary algebra based cryptographic systems, as they have many desireable properties. One of them is, that since these sets are finite and a prime field of characteristic $p$ can be represented on a computer in roughly $log_2(p)$ amount of space, no precision problems occure, that are for example unavoidable for computer representations of rational numbers or even the integers.

Since prime fields are special cases of remainder class rings, all computations remain the same. Addition and multiplication can be computed by first doing normal integer addition and multiplication and then take the remainder modulus $p$. Subtraction and division can be computed by addition or multiplication with the additive or the multiplicative inverse, respectively. The additive inverse $-x$ of a field element $x\in\F_p$ is given by $p-x$ and the multiplicative inverse of $x\neq 0$ is given by $x^{p-2}$, or can be computed using the extended Euclidean algorithm. 

Note however that these computations might not be the fastest to implement on a computer. They are however useful in this book as they are easy to compute for small prime numbers.
\begin{example}
The smallest field is the field $\F_2$ of characteristic $2$ as we have seen it in example XXX. It is the prime field of the prime number $2$.
\end{example}
\begin{example}
To summarize the basic aspects of computation in prime fields, lets consider the prime field $\F_5$ and simplify the following expression 
$$\left(\frac{2}{3} - 2\right)\cdot 2 $$
A first thing to note is that since $\F_5$ is a field all rules like bracketing (distributivity), summing ect. are identical to the rules we learned in school when we where dealing with rational, real or complex numbers. We get
\begin{align*}
\left(\frac{2}{3} - 2\right)\cdot 2 &= 
 \frac{2}{3}\cdot 2 - 2\cdot 2 & \text{\# distributive law}\\
 &= \frac{2\cdot 2}{3} - 2\cdot 2 & \Zmod{4}{5}=4 \\
 &= \frac{4}{3} - 4 & \text{\# multiplicative inverse of 3 is } \Zmod{3^{5-2}}{5}=2\\
 &= 4\cdot 2 - 4 & \text{\# additive inverse of 4 is } 5-4=1\\
 &= 4\cdot 2 +1 & \Zmod{8}{5}=3\\
 &= 3 +1 & \Zmod{4}{5}=4\\
 &= 4
\end{align*}
In this computation we computed the multiplicative inverse of $3$ using the identity
$x^{-1}=x^{p-2}$ in a prime field. This impractical for large prime numbers. Recall that another way of computing the multiplicative inverse is the Extended Euclidean algorithm.  To see that again, the task is to compute $x^{-1}\cdot 3 + t \cdot 5 =1$, but $t$ is actulally irrelevant. We get
\begin{center}
  \begin{tabular}{c | c c l}
    k & $ r_k $ & $ x^{-1}_k $ & $ t_k = \Zdiv{(r_k-s_k \cdot a)}{b} $ \\\hline
    0 & 3 & 1 & $\cdot$\ \\
    1 & 5 & 0 & $\cdot$ \\
    2 & 3 & 1 & $\cdot$ \\
    3 & 2 &-1 & $\cdot$ \\
    4 & 1 & 2  & $\cdot$ \\
  \end{tabular}
\end{center}
So the multiplicative inverse of $3$ in $\Z_5$ is $2$ and indeed if compute $3\cdot 2$ we get $1$ in $\F_5$. 
\end{example}
\paragraph{Square Roots}
In this part we deal with square numbers also called \textit{quadratic residues} and \textit{square roots} in prime fields. This is of particular importance in our studies on elliptic curves as only square numbers can actually be points on an elliptic curve. 

To make the intuition of quadratic risidues and roots precise, let $p \in \Prim $ be a prime number and $\F_p $ its associate prime field. Then a number $x\in \F_p$ is called a \textbf{square root} of another number $y\in\F_p$, if $x$ is a solution to the equation
\begin{equation}
x^2 = y
\end{equation}
In this case $y$ is called a \textbf{quadratic residue}. On the other hand, if $y$ is given and the quadratic equation has no $x$ solution, we call $ y $ as \textbf{quadratic non-residue}. For any $ y \in \F_p $ we write
\begin{equation}
\sqrt{y}: = \{x \in \F_p \; | \; x^2 = y \}
\end{equation}
for the set of all square roots of $ y $ in the prime field $ \F_n $. (If $ y $ is a quadratic non-residue, then $ \sqrt{y} = \emptyset $ and if $ y = 0 $, then $ \sqrt{y} = \{0 \} $)

So roughly speaking, quadratic residues are numbers such that we can take the square root from them and quadratic non-residues are numbers that don't have square roots. The situation therefore parallels the know case of integers, where some integers like $4$ or $9$ have square roots and others like $2$ or $3$ don't (as integers).

It can be shown that in any prime field every non zero element has either no square root or two of them. We adopt the convention to call the smaller one (when interpreted as an integer) as the \textbf{positive} square root and the larger one as the \textbf{negative}. This makes sense, as the larger one can always be computed as the modulus minus the smaller one, which is the definition of the negative in prime fields. 


\begin{example} [Quadratic (Non)-Residues and roots in $ \F_5 $] Let us consider our example prime field $\F_5$ again. All square numbers can be found on the main diagonal of the multiplication table XXX. As you can see, in $ \Z_5 $ only the numbers $ 0 $, $ 1 $ and $ 4 $ have square roots and we get $ \sqrt{0} = \{0 \} $, $ \sqrt{1} = \{1,4 \} $, $ \sqrt{2} = \emptyset $, $ \sqrt{3} = \emptyset $ and $ \sqrt{4} = \{2,3 \} $. The numbers $0$, $1$ and $4$ are therefore quadratic residues, while the numbers $2$ and $3$ are quadratic non-residues.
\end{example}
In order to describe whether an element of a prime field is a square number  or not, the so called Legendre Symbol can sometimes be found in the literature, why we will recapitulate it here:

Let $ p \in \Prim $ be a prime number and $ y \in \F_p $ an element from the associated prime field. Then the so-called \textit{Legendre symbol} of $ y $ is defined as follows:
\begin{equation}
\label{eq: Legendre-symbol}
\left (\frac{y}{p} \right): =
\begin{cases}
1 & \text{if $ y $ has square roots} \\
-1 & \text{if $ y $ has no square roots} \\
0 & \text{if $ y = 0 $}
\end{cases}
\end{equation}
\begin{example}
Look at the quadratic residues and non residues in $\F_5$ from example XXX again, we can deduce the following Legendre symbols, from example XXX.
$$
\begin{array}{ccccc}
\left (\frac{0}{5} \right) = 0, &
\left (\frac{1}{5} \right) = 1, &
\left (\frac{2}{5} \right) = -1, &
\left (\frac{3}{5} \right) = -1, &
\left (\frac{4}{5} \right) = 1 \;.
\end{array}
$$
\end{example}
The legendre symbol gives a criterion to decide wheather or not an element from a prime field has a quadratic root or not. This however is not just of theoretic use, as the following so called \textit{Euler criterion} gives a compact way to actually compute the Legendre symbol. To see that, let $ p \in \Prim_{\geq 3} $ be an odd 
Prime number and $ y \in \F_p $. Then the Legendre symbol can be computed as 
\begin{equation}
\label{eq: Euler_criterium}
\left (\frac{y}{p} \right) = y^{\frac{p-1}{2}} \;.
\end{equation}
\begin{example}
Look at the quadratic residues and non residues in $\F_5$ from example XXX again, we can compute the following Legendre symbols using the Euler criterium:
\begin{align*}
\left (\frac{0}{5} \right) &= 0^{\frac{5-1}{2}}= 0^2=0\\
\left (\frac{1}{5} \right) &= 1^{\frac{5-1}{2}}= 1^2=1\\
\left (\frac{2}{5} \right) &= 2^{\frac{5-1}{2}}= 2^2=4 = -1\\
\left (\frac{3}{5} \right) &= 3^{\frac{5-1}{2}}= 3^2=4 =-1\\
\left (\frac{4}{5} \right) &= 4^{\frac{5-1}{2}}= 4^2=1
\end{align*}

\end{example}

% I think this isn't needed. Will just leave it here in case this changes
%
%So the question remains how to actually compute square roots in prime field. The following algorithms give a solution
%\begin{definition}[Tonelli-Shanks algorithm]
%\label{def: Tonelli-Shanks}
%Let $ p $ be an odd prime number $ p \in \Prim _{\geq 3} $ and $ y $ a quadratic residue in $ \Z_p $. Then the so-called Tonneli \cite{TA} and Shanks \cite{SD} algorithm computes the two square roots of $ y $. It is defined as follows:
%\begin{enumerate}
%\item Find $ Q, S \in \Z $ with $ p-1 = Q \cdot 2 ^ S $ such that $ Q $ is odd.
%\item Find an arbitrary quadratic non-remainder $ z \in \Z_p $.
%\item
%\begin{algorithmic}
%\State $ \begin{array}{ccccc}
%M: = S, & c: = z ^ Q, & t: = y ^ Q, & R: = y ^{\frac{Q + 1}{2}}, & M, c, t, R \in \Z_p
%\end{array} $
%\While{$ t \neq 1 $}
%\State Find the smallest $ i $ with $ 0 <i <M $ and $ t ^{2 ^ i} = 1 $
%\State $ b: = c ^{2 ^{M-i-1}} $
%\State $ \begin{array}{ccccc}
%M: = i, & c: = b ^ 2, & t: = tb ^ 2, & R: = R \cdot b
%\end{array} $
%\EndWhile
%\end{algorithmic}
%The results are then the square roots $ r_1: = R $ and $ r_2: = p-R $ of $y$ in $\F_p$.
%\end{enumerate}
%\end{definition}

%\begin{remark}The algorithm (\ref{def: Tonelli-Shanks}) works in prime fields for any odd prime numbers. From a practical point of view, however, it is efficient only if the prime number is congruent to $ 1 $ modulo $ 4 $, since in the other case the formula from the proposition \ref{theorem: square_roots}, which can be calculated more quickly, can be used.\end{remark}

\paragraph{Exponentiation} TO APPEAR...
\paragraph{Extension Fields}
% references https://blog.plover.com/math/se/finite-fields.html
We have define prime fields in the previous section. They are basic building locks for cryptography in general ans snarks in particular. However so called \textit{pairing based} snark systems are crcially dependend on group pairings XXX defined over elliptic curves. For those pairings to be non-trivial the elliptic curve must not only be defined over a prime field but over a so called \textit{field extension} of a given prime field.

We therefore have to understand field extensions. To understand them let $p\in \Prim$ be a prime number and $m\in\N$ a natural number. Then there is a field $\F_{p^m}$ with characteristic $p$ and $p^m$ elements. Such a field is called an \textbf{extension field} of the prime field $\F_p$, because it contains $\F_p$ as a subfield. 

Similar to how prime fields $\F_p$ are generated by starting with the ring of integers and then divide by a prime number $p$ and keep the remainder, prime field extensions $\F_{p^m}$ are generated by starting with the ring $\F_p[x]$ of polynomials and then divide them by an irreducible polynomial of degree $m$ and keep the remainder. 

To be more precise let $P\in F_p[x]$ be an irreducible polynomial of degree $m$ with coefficient from our prime field $\F_p$. Then the set of the extension field is given by  the set of all polynomials with a degree less then $m$:
\begin{equation}
\F_{p^m}:=\{a_{m-1} x^{m-1}+a_{k-2}x^{k-2}+\ldots+a_1 x+a_0\;|\; a_i\in \F_p\}
\end{equation}
which can be shown to be the set of all remainders when dividing any polynomial $Q\in \F_p[x]$ by $P$. So elements of the extension field are polynomials of degree less then $m$. This is analog to how $\F_p$ is the set of all remainders, when dividing integers by $p$.   

Addition in then inheritec from $\F_p[x]$, which means that addition on $\F_{p^m}$ is defined as normal addition of polynomials. To be more precise, we have
\begin{equation}
+:\; \F_{p^m}\times \F_{p^m} \to \F_{p^m}\; ,\; (\sum_{j=0}^m a_j x^j,\sum_{j=0}^m b_j x^j)\mapsto \sum_{j=0}^m (a_j+b_j) x^j 
\end{equation}
and we can see that the neutral element is (the polynomial) $0$ and that the additive inverse is given by the polynomial with all negative coefficients.

Multiplication in inheritec from $\F_p[x]$, too, but we have to divide the result by our modulus polynomial $P$, whenever the degree of the resulting polynomial is equal or greater to $m$. To be more precise, we have
\begin{equation}
+:\; \F_{p^m}\times \F_{p^m} \to \F_{p^m}\; ,\; (\sum_{j=0}^m a_j x^j,\sum_{j=0}^m b_j x^j)\mapsto \Zdiv{\sum _{n = 0} ^{2m} \sum _{i = 0} ^{n}{a} _{i }{{b} _{n-i}}{x} ^{n}}{P} 
\end{equation}
and we can see that the neutral element is (the polynomial) $1$. It is however not obvious from this definition how the multiplicative looks.

We can easily see from the definition of $\F_{p^m}$ that the field is of characteristic $p$, since the multiplicative neutral element $1$ is equivalent to the multiplicative element $1$ from the underlying prime field and hence $\sum_{j=0}^p 1=0$. Moreover $\F_{p^m}$ is finite and contains $p^m$ many elements, since elements are polynomials of degree $<m$ and every coefficient $a_j$ can have $p$ different values.

One key point is that the construction of $\F_{p^m}$ depends on the choice of an irreducible polynomial and in fact different choices will give different multiplication tables, since the remainders from dividing a product by $P$ will be different..

It can however be shown, that the fields for different choices of $P$ are isomorphic, which means that there is a one to one identification between all of them and hence from an abstract point of view they are the same thing. From an implementations point of view however some choices are better, because they allow for faster computations.

\begin{example}[The Extension field $\F_{3^2}$]In (XXX) we have constructed the prime field $\F_3$. In this example we apply the definition (XXX) of a field extension to construct $\F_{3^2}$. We start by choosing an irreducibe polynomial of degree $2$ with coefficients in $\F_3$. We try 
$P(t)=t^2+1$. Maybe the fastest way to show that $P$ is indeed irreducible is to just insert all elements from $\F_3$ to see if the result is never zero. WE compute
\begin{align*}
P(0) = 0^2+1 &= 1\\
P(1) = 1^2+1 &= 2\\
P(2) = 2^2+1 &=  1+1  = 2
\end{align*}
This implies, that $P$ is irreducible. The set $\F_{3^2}$ then contains all poynomials of degrees lower then two with coefficients in $\F_3$, which is precisely
$$
\F_{3^2}=\{0,1,2,t,t+1,t+2,2t,2t+1,2t+2\}
$$
So our extension field contains $9$ elements as expected. Addition is  defined as addition of polynomials. For example $(t+2) + (2t+2)= (1+2)t +(2+2)= 1$. Doing this computation for all elements give the following addition table
\begin{center}
  \begin{tabular}{c | c c c c c c c c c}
    + & 0    & 1    & 2    & t    & t+1  & t+2  & 2t   & 2t+1 & 2t+2 \\\hline
    0 & 0    & 1    & 2    & t    & t+1  & t+2  & 2t   & 2t+1 & 2t+2 \\
    1 & 1    & 2    & 0    & t+1  & t+2  & t    & 2t+1 & 2t+2 & 2t   \\
    2 & 2    & 0    & 1    & r+2  & t    & t+1  & 2t+2 & 2t   & 2t+1 \\
    t & t    & t+1  & t+2  & 2t   & 2t+1 & 2t+2 & 0    & 1    & 2    \\
  t+1 & t+1  & t+2  & t    & 2t+1 & 2t+2 & 2t   & 1    & 2    & 0    \\
  t+2 & t+2  & t    & t+1  & 2t+2 & 2t   & 2t+1 & 2    & 0    & 1    \\
   2t & 2t   & 2t+1 & 2t+2 & 0    & 1    & 2    & t    & t+1  & t+2  \\
 2t+1 & 2t+1 & 2t+2 & 2t   & 1    & 2    & 0    & t+1  & t+2  & t    \\
 2t+2 & 2t+2 & 2t   & 2t+1 & 2    & 0    & 1    & t+2  & t    & t+1
  \end{tabular}
\end{center}
As we can see, the group $(\F_3,+)$ is a subgroup of the group $(\F_{3^2},+)$, obtained by only considering the first three rows and columns of this table.

As it was the case in previozs examples, we can use the table to deduce the negative of any element from $\F_{3^2}$. For example in $\F_{3^2}$ we have $-(2t+1)= t+2$, since $(2t+1) + (t+2)=0$ and the negative of an element is that other element, such that the sum gives the additive neutral element.

Multiplication needs a bit more computation, as we first have to multiply the polynomials and then divide the result by $P$ and keep the remainder. To see how this works compute the product of $t+2$ and $2t+2$ in $\F_{3^2}$
\begin{align*}
(t+2) \cdot (2t+2) &= \Zmod{2t^2 + 2t + t + 1}{t^2+1} \\
                   &= \Zmod{2t^2+1}{t^2+1} & \#\; 2t^2+1:t^2+1= 2 + \frac{2}{t^2+1} \\
                   &= 2 
\end{align*}
So the product of $t+2$ and $2t+2$ in $\F_{3^2}$ is $2$. Doing this computation for all elements give the following multiplication table:
\begin{center}
  \begin{tabular}{c | c c c c c c c c c}
$\cdot$ & 0    & 1    & 2    & t    & t+1  & t+2  & 2t   & 2t+1 & 2t+2 \\\hline
      0 & 0    & 0    & 0    & 0    & 0    & 0    & 0    & 0    & 0 \\
      1 & 0    & 1    & 2    & t    & t+1  & t+2  & 2t   & 2t+1 & 2t+2\\
      2 & 0    & 2    & 1    & 2t   & 2t+2 & 2t+1 & t    & t+2  & t+1 \\
      t & 0    & t    & 2t   & 2    & t+2  & 2t+2 & 1    & t+1  & 2t+1  \\
    t+1 & 0    & t+1  & 2t+2 & t+2  & 2t   & 1    & 2t+1 & 2    & t   \\
    t+2 & 0    & t+2  & 2t+1 & 2t+2 & 1    & t    & t+1  & 2t   & 2    \\
     2t & 0    & 2t   & t    & 1    & 2t+1 & t+1  & 2  & 2t+2 & t+2\\
   2t+1 & 0    & 2t+1 & t+2  & t+1  & 2    & 2t   & 2t+2 & t    & 1    \\
   2t+2 & 0    & 2t+2 & t+1  & 2t+1 & t    & 2    & t+2  & 1     & 2t
  \end{tabular}
\end{center}
As it was the case in previous examples, we can use the table to deduce the multiplicative inverse of any non-zero element from $\F_{3^2}$. For example in $\F_{3^2}$ we have $(2t+1)^{-1}= 2t+2 $, since $(2t+1) \cdot (2t+2)=1$.

From the multiplication table we can also see, that the only quadratic residues in $\F_{3^2}$ are the set $\{0,1,2, t, 2t\}$, with
$\sqrt{0}=\{0\}$, $\sqrt{1}=\{1,2\}$, $\sqrt{2}=\{t, 2t\}$, $\sqrt{t}=\{t+2,2t+1\}$ and $\sqrt{2t}=\{t+1,2t+2\}$.    

Computations in extension fields are arguably on the edge of what can reasonbly be done with pen and paper. Fortunately sage provides us with a simple way to do the computations.
\begin{sagecommandline}
sage: Z3 = GF(3) # prime field
sage: Z3t.<t> = Z3[] # polynomials over Z3
sage: P = Z3t(t^2+1)
sage: P.is_irreducible()
sage: F3_2.<t> = GF(3^2, name='t', modulus=P)
sage: F3_2
sage: F3_2(t+2)*F3_2(2*t+2) == F3_2(2)
sage: F3_2(2*t+2)^(-1) # multiplicative inverse
\end{sagecommandline}
\end{example}


%\paragraph{Hash to Prime fields} 
% https://crypto.stackexchange.com/questions/78017/simple-hash-into-a-prime-field
%An important problem in elliptic curve cryptography and in its implementations as a snark is the ability to hash to (various subsets) of elliptic curves. As we will see in XXX those curves are usually defined over prime fields and hashing to a curve often starts with hashing to the prime field. In this paragraph we therefore look at common techniques to hash to a prime field.

%In what follows let $\F_{q}$ be a prime field, such that $q$ is a prime number with $m$-digits in its binary representation, i.e. $|p_{base_2}|=m$ and let $H:\{0,1\}^* \to \{0,1\}^k$ be a hash function. The methods to map $H$ onto $\F_{q}$ depend on $k$. 

%If $k\leq m-1$, then every image $H(data)$ if interpreted as an integer in its base-2 representation, is smaller then $p$ and hence can directly be interpreted as an element of $\F_{q}$. So in this case $H:\{0,1\}^* \to \F_q$ can be used unchanged. The drawback of this simple method, is that the bigger the difference between $k$ and $m$ is, the more will the distribution of $H$ deviate from uniformity.

%For example in the extreme case $k=1$, $H$ only maps to $\{0,1\}\subset \F_q$. The best possible case is therefore $k=m-1$. In that case only the highest XXX numbers (DO THE COMPUTATION) are missing from the distribution.

%On the other hand if $k\geq m$, then there are basically two commonly used methods to map the output of $H$ onto $\F_q$. The first is to simply forget all leading $k-m+1$-bits from the image of $H$, which brings you back to our previous consideration.

%The second method is to interpret $H(data)$ as an integer and then compute the modulus $ \Zmod{H(data)}{p}$. This also introduces a small bias (COMPUTATION FROM THE FORUM ENTRY). 

%\begin{example}[p\&{}p-$\F_{13}$-mod-hash]
%Consider our pen\&paper hash function from XXX. We want to use this hash function, to define a $16$-bounded hash function that maps into the prime field $\F_{13}$. We define:
%$$\mathcal{H}_{mod}^{13}: \{0,1\}^{16}\to \F_{13}: S \mapsto \Zmod{\mathcal{H}_{PaP}(S)}{13}$$
%Considering the string $S=(1110011101110011)$ from example XXX again we know $\mathcal{H}_{PaP}(S)=(1110)$ and since $(1110)_{10}=14$ and $\Zmod{14}{13}=1$ we get $\mathcal{H}_{mod}^{13}(S)=1$.
%\end{example}

%\begin{example}[p\&{}p-$\F_{13}$-drop-hash]We can consider the same pen\&paper hash function from XXX and define another hash into $\F_{13}$, by deleting the first leading bit from the hash. The result is then a $3$-digit number and therefore guaranteed to be smaller then $13$, since $13$ is equal to $(1101)$ in base $2$.  

%Considering the string $S=(1110011101110011)$ from example XXX again we know $\mathcal{H}_{PaP}(S)=(1110)$ and stripping of the leading bit we get $(110)_{10}=6$ as our hash value.  

%As we can see this hash function has the drawback of an uneven distribution in $\F_{13}$. In fact this hash function is unable to map to values from $\{8,9,10,11,12\}$ as those numbers have a $1$-bit in position $4$. However as we will see in XXX, this hash is cheaper to implement as a circuit as no expensive modulus operation has to be used.
%\end{example}

\chapter{Elliptic Curves}\label{chap:elliptic-curves}

\sme{TODO: Elliptic Curve asymmetric cryptography examples. Private key, generator, public key.} 
%references http://infosec.pusan.ac.kr/wp-content/uploads/2019/09/Pairings-For-Beginners.pdf
Generally speaking, elliptic curves are ``curves'' defined in geometric planes like the Euclidean or the projective plane over some given field. One of the key features of elliptic curves over finite fields from the point of view of cryptography is that their set of points has a group law such that the resulting group is finite and cyclic, and it is believed that the discrete logarithm problem on these groups is hard. 

A special class of elliptic curves are so-called \term{pairing-friendly curves}, which have a notation of a group pairing as defined in XXX\sme{add reference}. This pairing has cryptographically advantageous properties. Those curve are useful in the development of SNARKs, since they allow to compute so-called R1CS-satisfiability ``in the exponent'' \smelong{MIRCO: (THIS HAS TO BE REWRITTEN WITH WAY MORE DETAIL)}

In this chapter, we introduce epileptic curves as they are used in pairing-based approaches to the construction of SNARKs. The elliptic curves we consider are all defined over prime fields or prime field extensions and the reader should be familiar with the contend of the previous section on those fields.

\smelong{In its most generality elliptic curves are defined as a smooth projective curve of genus 1 defined over some field $\F$ with a distinguished $\F$-rational point, but this definition is not very useful for the introductory character of this book.}\sme{maybe remove this sentence?} We will therefore look at $3$ more practical definitions in the following sections, by introducing Weierstraß, Montgomery and Edwards curves. All of them are widely used in cryptography, and understanding them is crucial to being able to follow the rest of this book.

\section{Elliptic Curve Arithmetics}

\subsection{Short Weierstraß Curves}
In this section, we introduce \term{short Weierstraß }curves, which are the most general types of curves over finite fields of characteristic greater than $3$. 

We start with their representation in \uterm{affine space}. This representation has the advantage that affine points correspond to pairs of numbers, which makes it more accessible for beginners. However, it has the disadvantage that a special ``point at infinity'', that is not a point on the curve, is necessary to describe the group structure. We introduce the elliptic curve group law and describe elliptic curve scalar multiplication, which is an instantiation of the exponential map from general cyclic groups.

Then we look at the projective representation of short Weierstraß curves. This has the advantage that no special symbol is necessary to represent the point at infinity but comes with the drawback that projective points are classes of numbers, which might be a bit unusual for a beginner.

We finish this section with an explicit equivalence that transforms affine representations into projective ones and vice versa.

\paragraph{Affine short Weierstraß form} Probably the least abstract and most straight-forward way to introduce elliptic curves for non-mathematicians and beginners is the so-called affine representation of a short Weierstraß curve. To see what this is, let $\F$ be a finite field of order $q$ and $a,b\in \F$ two field elements such that $\Zmod{4a^3+ 27b^2}{q}\neq 0$. Then a \term{short Weierstraß elliptic curve} $E(\F)$ over $\F$ in its affine representation is the set of all pairs of field elements $(x,y)\in \F\times \F$ that satisfy the short Weierstraß cubic equation $y^2=x^3+a\cdot x+b$, together with a distinguished symbol $\Oinf$, called the \term{point at infinity}:

\begin{equation}
\label{def_short_weierstrass_curve}
E(\F) = \{(x,y)\in \F\times \F\;|\; y^2=x^3+a\cdot x+b \} \bigcup \{\Oinf\}
\end{equation}
\begin{notation}
In the literature, the set $E(\F)$, which includes the symbol $\mathcal{O}$, is often called the set of \term{rational points} of the elliptic curve, in which case the curve itself is usually written as $E/\F$. However, in what follows, we will frequently identify an elliptic curve with its set of rational points and therefore use the notation $E(\F)$ instead. This is possible in our case, since we only the group structure of the curve in consideration is relevant for us.
\end{notation}
The term ``curve'' is used here because, in the ordinary 2 dimensional plane $\R^2$,
the set of all points $(x,y)$ that satisfy $y^2 = x^3 +a\cdot x +b$ looks like a curve. We should note however that visualizing elliptic curves over finite fields as ``curves'' has its limitations, and we will therefore not stress the geometric picture too much, but focus on the computational properties instead. To understand the visual difference, consider the following two elliptic curves: 

\medskip

% let Sage draw some elliptic curve in R^2, but show only the picture
\begin{sagesilent}
E1 = EllipticCurve([-2,1])
C1 = E1.plot()
F = GF(9973)
E2 = EllipticCurve(F, [-2,1])
C2 = E2.plot()
\end{sagesilent}
\begin{minipage}{0.48\textwidth}
\sageplot[scale=.48]{C1} 
\end{minipage}
%
\begin{minipage}{0.48\textwidth}
\sageplot[scale=.5]{C2}
\end{minipage}

% TODO: PLOT BOTH CURVES ON THE SAME LINE NOT BELOW EACH OTHER
Both elliptic curves are defined by the same short Weierstraß equation $y^2 = x^3-2x+1$, but the first curve is defined in the real affine plane $\mathbb{R}^2$, that is, the pair $(x,y)$ contains real numbers, while the second one is defined in the affine plane $\F_{9973}^2$, which means that both $x$ and $y$ are from the prime field $\F_{9973}$. Every blue dot represents a pair $(x,y)$, that is a solution to $y^2 = x^3-2x+1$. As we can see, the second curve hardly looks like a geometric structure one would naturally call a curve. This shows that our geometric intuitions from $\R^2$ are obfuscated in curves over finite fields.

The identity $\Zmod{6\cdot(4a^3+ 27b^2)}{q}\neq 0$ ensures that the curve is  non-singular, which basically means that the curve has no \uterm{cusps} or \uterm{self-intersections}.

Throughout this book, the reader is advised to do as many computations in a pen-and-paper fashion as possible, as this is helps getting a deeper understanding of the details. However, when dealing with elliptic curves, computations can quickly become cumbersome and tedious, and one might get lost in the details. Fortunately, Sage is very helpful in dealing with elliptic curves. This book to introduces the reader to the great elliptic curve capabilities of Sage. The following snippet shows a way to define elliptic curves and work with them in Sage:

\begin{sagecommandline}
sage: F5 = GF(5) # define the base field
sage: a = F5(2) # parameter a
sage: b = F5(4) # parameter b
sage: # check non-sigularity
sage: F5(6)*(F5(4)*a^3+F5(27)*b^2) != F5(0)
sage: # short Weierstrass curve 
sage: E = EllipticCurve(F5,[a,b]) # y^2 == x^3 + ax +b 
sage: P = E(0,2) # 2^2 == 0^3 + 2*0 + 4
sage: P.xy() # affine coordinates
sage: INF = E(0) # point at infinity
sage: try: 	# point at infinity has no affine coordinates
....:     INF.xy()
....: except ZeroDivisionError:
....:     pass
sage: P = E.plot() # create a plotted version 
\end{sagecommandline}
The following three examples give a more practical understanding of what an elliptic curve is and how we can compute it. The reader is advised to read them carefully, and ideally, to also carry out the computation themselves. We will repeatedly build on these examples in this chapter, and use the second example throughout this book.
\begin{example}\label{E1F5}To provide the reader with an example of a small elliptic curve where all computation can be done with pen and paper, consider the prime field $\F_5$ from example \ref{prime-field-F5}\sme{check reference} (page \pageref{prime-field-F5}).  quite familiar  to readers who had worked through the examples and exercises in the previous chapter.

To define an elliptic curve over $\F_5$, we have to choose to numbers $a$ and $b$ from that field. Assuming we choose $a=1$ and $b=1$ then $\kongru{4a^3+ 27b^2}{1}{5}$ from which follows that the corresponding elliptic curve $E_1(\F_5)$ is given by the set of all pairs $(x,y)$ from $\F_5$ that satisfy the equation $y^2=x^3+x+1$, together with the special symbol $\Oinf$, which represents the ``point at infinity''. 

To get a better understand of that curve, observer that if we choose arbitrarily the pair $(x,y)=(1,1)$, we see that $1^2 \neq 1^3+1 + 1$ and hence $(1,1)$ is not an element of the curve $E_1(\F_5)$. On the other hand choosing for example $(x,y)=(2,1)$ gives $1^2 = 2^3 + 2 + 1$ and hence the pair $(2,1)$ is an element of $E_1(\F_5)$ (Remember that all computations are done in modulo $5$ arithmetics).

Now since the set $\F_5\times \F_5$ of all pairs $(x,y)$ from $\F_5$ contains only $5\cdot 5=25$ pairs, we can compute the curve, by just inserting every possible pair $(x,y)$ into the short Weierstraß equation $y^2 = x^3 + x +1$. If the equation holds, the pair is a curve point, if not that means that the point is not on the curve. Combining the result of this computation with the point at infinity gives the curve as follows:
$$
E_1(\F_5) = \{\Oinf, (0,1),(2,1),(3,1),(4,2),(4,3),(0,4),(2,4),(3,4)\}
$$
This means that our elliptic curve is a set of $9$ elements, $8$ of which are pairs of numbers and one special symbol $\Oinf$. Visualizing $E1$ gives the following plot:
\begin{sagesilent}
F5 = GF(5)
E1 = EllipticCurve(F5, [1,1])
C1 = E1.plot()
\end{sagesilent}
\begin{center} 
\sageplot[scale=.5]{C1}
\end{center}
% sage: AffinePoints = [P.xy() for P in E1.points() if P.order > 1]
\end{example}
In the development of SNARKs, it is sometimes necessary to do elliptic curve cryptography ``in a circuit", which basically means that the elliptic curves need to be implemented in a certain SNARK-friendly way. We will look at what this means in chapter \ref{chap:circuit-compilers}\sme{check reference}. To be able to do this efficiently, it is desirable to have curves with special properties. The following example is a pen-and-paper version of such a curve, called \term{Baby-\comms{jubjub}}, which resembles cryptographically secure curves extensively used in real-world SNARKs. The interested reader is advised to study this example carefully, as we will use it and build on it in various places throughout the book. \commslong{I feel like a lot of people won't get the Lewis Carroll reference unless we make it more explicit. The term Baby-JubJub is actually the name of a curve used in zCash and Ethereum a lot. IDK why they choosed that name}
\begin{example}[Pen-JubJub]\label{PJJ13} Consider the prime field $\F_{13}$ from exercise \ref{prime-field-F13} (page \pageref{prime-field-F13}.\sme{check reference} If we choose $a=8$ and $b=8$, then $\kongru{4a^3+ 27b^2}{6}{13}$ and the corresponding elliptic curve is given by all pairs $(x,y)$ from $\F_{13}$ such that $y^2=x^3+8x+8$ holds. We call this curve the \term{Pen-JubJub} curve, or  $\mathit{PJJ\_13}$ for short.

Now, since the set $\F_{13}\times \F_{13}$ of all pairs $(x,y)$ from $\F_{13}$ contains only $13\cdot 13=169$ pairs, we can compute the curve by just inserting every possible pair $(x,y)$ into the short Weierstraß equation $y^2 = x^3 +8x +8$.  We get the following result:
\begin{equation}\label{eq:PJJ13-weierstrass}
\begin{split}
\mathit{PJJ\_13} = \{\Oinf, (1, 2), (1, 11), (4, 0), (5, 2), (5, 11), (6, 5), (6, 8), (7,2), (7, 11), (8, 5), (8, 8),\\ (9, 4), (9, 9), (10, 3), (10,
10), (11, 6), (11, 7), (12, 5), (12, 8)\}
\end{split}
\end{equation}
As we can see, the curve consists of $20$ points; $19$ points from the \uterm{affine plane} and the point at infinity.
To get a visual impression of the $\mathit{PJJ\_13}$ curve, we might plot all of its points (except the point at infinity) in the $\F_{13}\times \F_{13}$ affine plane. We get the following plot: 
\begin{sagesilent}
F13 = GF(13)
PJJ_13 = EllipticCurve(F13, [8,8])
CPJJ_13 = PJJ_13.plot()
\end{sagesilent}
\begin{center} 
\sageplot[scale=.5]{CPJJ_13}
\end{center}
As we will see in what follows, this curve is rather special, as it is possible to represent it in two alternative forms called the \term{Montgomery} and the \term{twisted Edwards form} (See sections \ref{sec:montgomery}\sme{check reference} and \ref{sec:edwards}, respectively\sme{check reference}).
\end{example}
Now that we have seen two pen-and-paper friendly elliptic curves, let us look at a curve, that is used in actual cryptography. Cryptographically secure elliptic curves are not \hilight{qualitatively} different from the curves we looked at so far, but the prime number modulus of their prime field is much larger. Typical examples use prime numbers that have binary representations in the magnitude of more than double the size of the desired security level. If, for example, a security of $128$ bits is desired, a prime modulus of binary size $\geq 256$ is chosen. The following example provides such a curve. 

\begin{example}[Bitcoin's Secp256k1 curve]\label{Secp256k1}
To give an example of a real-world, cryptographically secure curve, let us look at curve Secp256k1, which is famous for being used in the public key cryptography of Bitcoin. The prime field $\F_p$ of Secp256k1 is defined by the following prime number:
$$
p = \scriptstyle 115792089237316195423570985008687907853269984665640564039457584007908834671663
$$
 
 The binary representation of this number needs $256$ bits, which implies that the prime field $\F_p$  contains approximately $2^{256}$ many elements, which is considered quite large. To get a better impression of how large the base field is, consider that the number $2^{256}$ is approximately in the same order of magnitude as the estimated number of atoms in the observable universe. 

The curve Secp256k1 is defined by the parameters $a,b\in \F_p$ with $a=0$ and $b=7$. Since $\Zmod{4\cdot 0^3 + 27\cdot 7^2}{p}=1323$, those parameters indeed define an elliptic curve given as follows:
$$
\mathit{Secp256k1} = \{(x,y)\in \F_p\times \F_p\;| y^2 = x^3 +7\;\} 
$$
Clearly, the Secp256k1 curve is too large to do computations by hand, since it can be shown that  the number of its elements is a prime number $r$ that also has a binary representation of $256$ bits:
$$
r = \scriptstyle 11579208923731619542357098500868790785283756427907490438260516
3141518161494337
$$
Cryptographically secure elliptic curves are therefore not useful in pen-and-paper computations. Fortunately, Sage handles large curves efficiently:
\begin{sagecommandline}
sage: p = 115792089237316195423570985008687907853269984665640564039457584007908834671663
sage: # Hexadecimal representation
sage: p.str(16)
sage: p.is_prime()
sage: p.nbits()
sage: Fp = GF(p)
sage: Secp256k1 = EllipticCurve(Fp,[0,7])
sage: r = Secp256k1.order() # number of elements
sage: r.str(16)
sage: r.is_prime()
sage: r.nbits()
\end{sagecommandline}
%\seqsplit{115792089237316195423570985008687907853269984665640564039457584007908834671663}
\end{example}
\begin{exercise}
Look up the definition of curve BLS12-381, implement it in Sage and compute its order.
\end{exercise}

\paragraph{Affine compressed representation} As we have seen in example \ref{Secp256k1}\sme{check reference}, cryptographically secure elliptic curves are defined over large prime fields, where elements of those fields typically need more than $255$ bits of storage on a computer. Since elliptic curve points consist of pairs of those field elements, they need double that amount of storage.

However, we can reduce the amount of space needed to represent a curve point by using a technique called \term{point compression}. Note that, up to a \uterm{sign}, the $y$ coordinate of a curve point can be computed from the $x$ coordinate by simply inserting $x$ into the Weierstraß equation and then computing the roots of the result. This gives two results, and it means that we can represent a curve point in \term{compressed form} by simply storing the $x$ coordinate together with a single sign bit only, the latter of which deterministically decides which of the two roots to choose. One convention could be to always choose the root closer to $0$ when the sign bit is $0$, and the root closer to the order of $\F$ when the sign bit is $1$. In case the $y$ coordinate is zero, both sign bits give the same result.\sme{more explanation of what the sign is}



\begin{example}[Pen-jubjub] To understand the concept of compressed curve points a bit better, consider the $\mathit{PJJ\_13}$ curve from example \ref{PJJ13}\sme{check reference} again. Since this curve is defined over the prime field $\F_{13}$, and numbers between $0$ and $13$ need approximately $4$ bits to be represented, each $\mathit{PJJ\_13}$ point on this curve needs $8$ bits of storage in uncompressed form. The following set represents the uncompressed form of the points on this curve:
\begin{multline*}
\mathit{PJJ\_13} = \{\Oinf, (1, 2), (1, 11), (4, 0), (5, 2), (5, 11), (6, 5), (6, 8), (7,2), (7, 11), \\ (8, 5), (8, 8), (9, 4), (9, 9), (10, 3), (10,
10), (11, 6), (11, 7), (12, 5), (12, 8)\}
\end{multline*}
Using the technique of point compression, we can reduce the bits needed to represent the points on this curve to  $5$ per point. To achieve this, we can replace the $y$ coordinate in each $(x,y)$ pair by a sign bit indicating whether or not $y$ is closer to $0$ or to $13$. As a result $y$ values in the range $[0,\ldots,6]$ will have the sign bit $0$, while $y$-values in the range $[7,\ldots,12]$ will have the sign bit $1$. Applying this to the points in $\mathit{PJJ\_13}$ gives the compressed representation as follows:
\begin{multline*}
\mathit{PJJ\_13} = \{\Oinf, (1, 0), (1, 1), (4, 0), (5, 0), (5, 1), (6, 0), (6, 1), (7,0), (7, 1), \\ (8, 0), (8, 1), (9, 0), (9, 1), (10, 0), (10,1), (11, 0), (11, 1), (12, 0), (12, 1)\}
\end{multline*} 
Note that the numbers $7,\ldots, 12$ are the negatives (additive inverses) of the numbers $1,\ldots, 6$ in modular $13$ arithmetics and that $-0=0$. \smelong{Calling the compression bit a ``sign bit'' therefore makes sense.}\sme{S: I don't follow this at all}

To recover the uncompressed counterpart of, say, the compressed point $(5,1)$, we insert the $x$ coordinate $5$ into the Weierstraß equation and get $y^2 = 5^3 + 8\cdot 5 +8 = 4$. As expected, $4$ is a quadratic residue in $\F_{13}$ with roots $\sqrt{4}= \{2,11\}$. Since the sign bit of the point is $1$, we have to choose the root closer to the modulus $13$, which is $11$. The uncompressed point is therefore $(5,11)$. 
\end{example}
Looking at the previous examples, the compression rate does not look very impressive. However, looking at the real-life example of the Secp256k1 curve shows that compression is has significant practical advantages.
\begin{example}
Consider the Secp256k1 curve from example \ref{Secp256k1}\sme{check reference} again. The following code invokes Sage to generate a random affine curve point, then applies our compression method to it:
\begin{sagecommandline}
sage: P = Secp256k1.random_point().xy()
sage: P
sage: # uncompressed affine point size
sage: ZZ(P[0]).nbits()+ZZ(P[1]).nbits()
sage: # compute the compression
sage: if P[1] > Fp(-1)/Fp(2):
....:     PARITY = 1
....: else:
....:     PARITY = 0
sage: PCOMPRESSED = [P[0],PARITY]
sage: PCOMPRESSED
sage: # compressed affine point size
sage: ZZ(PCOMPRESSED[0]).nbits()+ZZ(PCOMPRESSED[1]).nbits()
\end{sagecommandline}
\end{example}\sme{add explanation of how this shows what we claim}


\paragraph{Affine group law}
%group law
% http://wwwmayr.informatik.tu-muenchen.de/konferenzen/Jass07/courses/1/Lukyanenko/Lukyanenk_Paper.pdf
One of the key properties of an elliptic curve is that it is possible to define a group law on the set of its rational points such that the point at infinity serves as the neutral element and inverses are reflections on the $x$-axis.

The origin of this law can be understood in a geometric picture and is known as the \term{chord-and-tangent rule}. In the affine representation of a short Weierstraß curve, the rule can be described in the following way:

\begin{definition}{\deftitle{Chord-and-tangent rule}}\label{def:chord-and-tangent}
\begin{itemize}
\item (Point at infinity) We define the point at infinity $\Oinf$ as the neutral element of addition, that is, we define $P+\Oinf = P$ for all points $P\in E(\F)$.\sme{should this def. be moved even earlier?}
\item (Point addition) Let $P, Q\in E(\F)\textbackslash \{\Oinf\}$ with $P\neq Q$ be two distinct points on an elliptic curve, neither of them the point at infinity. The sum of $P$ and $Q$ is defined as follows:\\
Consider the line $l$ which intersects the curve in $P$ and $Q$. If $l$ intersects the elliptic curve at a third point $R'$, define the sum $R=P\oplus Q$ of $P$ and $Q$ as the reflection of $R'$ at the $x$-axis. If the line $l$ does not intersect the curve at a third point, define the sum to be the point at infinity $\Oinf$. It can be shown that no such \uterm{chord line} will intersect the curve in more than three points, so addition is not ambiguous.
\item (Point doubling) Let $P \in E(\F)\textbackslash \{\Oinf\}$ be a point on an elliptic curve, that is not the point at infinity. The sum of $P$ with itself (the doubling of $P$) is defined as follows:\\
Consider the line which is \uterm{tangential} to the elliptic curve at $P$. If this line intersects the elliptic curve at a second point $R'$, the sum $2P=P+P$ is the reflection of $R'$ at the $x$-axis. If it does not intersect the curve at a third, point define the sum to be the point at infinity $\Oinf$. It can be shown that no such \uterm{tangent line} will intersect the curve in more than two points, so addition is not ambiguous.
\end{itemize}
\end{definition}

It can be shown that the points of an elliptic curve form a commutative group with respect to the tangent-and-chord rule such that $\Oinf$ acts the neutral element, and the inverse of any element $P\in E(\F)$ is the reflection of $P$ on the $x$-axis.

To translate the geometric description into algebraic equations, first observe that, for any two given curve points $(x_1,y_1), (x_2,y_2)\in E(\F)$, it can be shown that the identity $x_1=x_2$ implies $y_2=\pm y_1$, which shows that the following rules are a complete description of the affine addition law.

\begin{definition}{\deftitle{Chord-and-tangent rule: algebraic equations}}\label{def:chord-tangent-algebra}
\begin{itemize}
\item (Neutral element) The point at infinity $\Oinf$ is the neutral element.
\item (Additive inverse ) The additive inverse of $\mathcal{O}$ is $\mathcal{O}$. For any other curve point $(x,y) \in E(\F)\textbackslash \{\mathcal{O}\}$, the additive inverse is given by $(x,-y)$.
\item (Addition rule) For any two curve points $P, Q \in E(\F)$, addition is defined by one of the following three cases:
\begin{enumerate}
\item (Adding the neutral element) If $Q=\Oinf$, then the sum is defined as $P\oplus Q=P$.
\item (Adding inverse elements)  If $P=(x,y)$ and $Q=(x,-y)$, the sum is defined as $P\oplus Q=\Oinf$.
\item (Adding non-self-inverse equal points) If $P=(x,y)$ and $Q=(x,y)$ with $y\neq 0$, the sum $2P=(x',y')$ is defined as follows:\sme{remove $Q$?}\smelong{We only referred to $P$ in the definition of point doubling above so $Q$ seems a bit confusing here even though it's defined as equal to $P$}
$$
\begin{array}{llr}
x' = \left(\frac{3x^2+a}{2y}\right)^2 -2x &,&
y' = \left(\frac{3x^2+a}{2y}\right)^2\left(x-x'\right) - y
\end{array} 
$$
\item (Adding non-inverse different points) If $P=(x_1,y_1)$ and $Q=(x_2,y_2)$ such that $x_1 \neq x_2$, the sum $R=P+Q$ with $R=(x_3,y_3)$ is defined as follows:
$$
\begin{array}{llr}
x_3 = \left(\frac{y_2-y_1}{x_2-x_1}\right)^2 -x_1-x_2 &, &
y_3 = \left(\frac{y_2-y_1}{x_2-x_1} \right)\left(x_1-x_3\right) - y_1
\end{array} 
$$
\end{enumerate}
\end{itemize} 
\end{definition}
Note that short Weierstraß curve points $P$ with $P=(x,0)$ are inverses of themselves, which implies $2P=\Oinf$ in this case.
\begin{notation}
Let $\F$ be a field and $E(\F)$ be an elliptic curve over $\F$. We write $\oplus$ for the group law on $E(\F)$ and $(E(\F),\oplus)$ for the group of rational points.
\end{notation}
As we can see, it is very efficient to compute inverses on elliptic curves. However, computing the addition of elliptic curve points in the affine representation needs to consider many cases and involves extensive finite field divisions. As we will see \smelong{in the next paragraph}\sme{where?}, this can be simplified in projective coordinates.

To get some practical impression of how the group law on an elliptic curve is computed, let's look at some actual cases: 
\begin{example}\label{ex:01+42}
Consider the elliptic curve $E_1(\F_5)$ from example \ref{E1F5}\sme{check reference} again.  As we have seen, the set of rational points contains $9$ elements:
$$
E_1(\F_5) = \{\Oinf, (0,1),(2,1),(3,1),(4,2),(4,3),(0,4),(2,4),(3,4)\}
$$
We know that this set defines a group, so we can add any two elements from $E_1(\F_5)$ to get a third element. 

To give an example, consider the elements $(0,1)$ and $(4,2)$. Neither of these elements is the neutral element $\Oinf$, and since, the $x$ coordinate of $(0,1)$ is different from the $x$ coordinate of $(4,2)$, we know that we have to use the chord rule, that is, rule number 4 from definition \ref{def:chord-tangent-algebra}\sme{check reference} to compute the sum $(0,1)\oplus (4,2)$:
%\begin{tabular}{lr}
\begin{align*}
x_3  & = \left(\frac{y_2-y_1}{x_2-x_1}\right)^2 -x_1-x_2 & \text{\# insert points}\\
     & = \left(\frac{2-1}{4-0}\right)^2 -0-4  & \text{\# simplify in } \F_5\\
     & = \left(\frac{1}{4}\right)^2 +1
       = 4^2 +1
       = 1 +1 
       = 2
\\
\\
y_3  & = \left(\frac{y_2-y_1}{x_2-x_1} \right)\left(x_1-x_3\right) - y_1  & \text{\# insert points}\\     
     & = \left(\frac{2-1}{4-1} \right)\left(0-2\right) - 1   & \text{\# simplify in } \F_5\\    
     & = \left(\frac{1}{4} \right)\cdot 3 + 4   
       = 4\cdot 3 + 4
       = 2 + 4
       = 1          
\end{align*} 
%\end{tabular}
So, in our elliptic curve $E_1(\F_5)$ we get $(0,1)\oplus (4,2) =(2,1)$, and, indeed, the pair $(2,1)$ is an element of $E_1(\F_5)$ as expected. On the other hand, $(0,1)\oplus (0,4) =\Oinf$, since both points have equal $x$ coordinates and inverse $y$ coordinates, rendering them inverses of each other. Adding the point $(4,2)$ to itself, we have to use the tangent rule, that is, rule 3 from definition \ref{def:chord-tangent-algebra}\sme{check reference}:
\begin{align*}
x'  & = \left(\frac{3x^2+a}{2y}\right)^2 -2x   & \text{\# insert points}\\
    & = \left(\frac{3\cdot 4^2+1}{2\cdot 2}\right)^2 -2\cdot 4 & \text{\# simplify in } \F_5 \\
    & = \left(\frac{3\cdot 1+1}{4}\right)^2 +3\cdot 4
      = \left(\frac{4}{4}\right)^2 +2
      = 1 +2 
      = 3
\\
\\
y'  & = \left(\frac{3x^2+a}{2y}\right)^2\left(x-x'\right) - y  & \text{\# insert points} \\
    & = \left(\frac{3\cdot 4^2+1}{2\cdot 2}\right)^2\left(4-3\right) - 2 & \text{\# simplify in } \F_5\\
    & = 1\cdot 1 + 3
      = 4
\end{align*}
So, in our elliptic curve $E_1(\F_5)$, we get the doubling  of $(4,2)$, that is, $(4,2)\oplus (4,2) =(3,4)$, and, indeed the pair $(3,4)$ is an element of $E_1(\F_5)$ as expected. The group $E_1(\F_5)$ has no self-inverse points other than the neutral element $\Oinf$, since no point has $0$ as its $y$ coordinate. We can invoke Sage to double-check the computations. 
\begin{sagecommandline}
sage: F5 = GF(5)
sage: E1 = EllipticCurve(F5,[1,1])
sage: INF = E1(0) # point at infinity
sage: P1 = E1(0,1)
sage: P2 = E1(4,2)
sage: P3 = E1(0,4)
sage: R1 = E1(2,1)
sage: R2 = E1(3,4)
sage: R1 == P1+P2
sage: INF == P1+P3
sage: R2 == P2+P2
sage: R2 == 2*P2
sage: P3 == P3 + INF
\end{sagecommandline}
\end{example}
\begin{example}[Pen-jubjub]\label{ex:PJJ13-self-inverse} Consider the $\mathit{PJJ\_13}$-curve from example \ref{PJJ13}\sme{check reference} again and recall that its group of rational points is given as follows:
\begin{multline*}
\mathit{PJJ\_13} = \{\Oinf, (1, 2), (1, 11), (4, 0), (5, 2), (5, 11), (6, 5), (6, 8), (7,2), (7, 11), \\ (8, 5), (8, 8), (9, 4), (9, 9), (10, 3), (10,
10), (11, 6), (11, 7), (12, 5), (12, 8)\}
\end{multline*}
In contrast to the group from the previous example, this group contains a self-inverse point, which is different from the neutral element, defined by $(4,0)$. To see what this means, observe that we cannot add $(4,0)$ to itself using the tangent rule 3 from definition \ref{def:chord-tangent-algebra}\sme{check reference}, as the $y$ coordinate is zero. Instead, we have to use rule 2, since $0=-0$. We therefore get $(4,0)\oplus (4,0)=\Oinf$ in $\mathit{PJJ\_13}$. The point $(4,0)$ is therefore the inverse of itself, as adding it to itself results in the neutral element. 
\begin{sagecommandline}
sage: F13 = GF(13)
sage: MJJ = EllipticCurve(F13,[8,8])
sage: P = MJJ(4,0)
sage: INF = MJJ(0) # Point at infinity
sage: INF == P+P
sage: INF == 2*P
\end{sagecommandline}
\end{example}
\begin{example}
Consider the Secp256k1 curve from example \ref{Secp256k1} \sme{check reference} again. The following code invokes Sage to generate a random affine curve point, then applies our compression method:
\begin{sagecommandline}
sage: P = Secp256k1.random_point()
sage: Q = Secp256k1.random_point()
sage: INF = Secp256k1(0)
sage: R1 = -P
sage: R2 = P + Q
sage: R3 = Secp256k1.order()*P
sage: P.xy()
sage: Q.xy()
sage: (ZZ(R1[0]).str(16), ZZ(R1[1]).str(16))
sage: R2.xy()
sage: R3 == INF
sage: P[1]+R1[1] == Fp(0) # -(x,y) = (x,-y)
\end{sagecommandline}
\end{example}
\begin{exercise}
Consider the $\mathit{PJJ\_13}$-curve from example \ref{PJJ13}\sme{check reference}. 
\begin{enumerate}
\item Compute the inverse of $(10,10)$, $\Oinf$, $(4,0)$ and $(1,2)$.
\item Compute the expression $3*(1,11) - (9,9)$.
\item Solve the equation $x + 2(9,4) = (5,2) $ for some $x\in \mathit{PJJ\_13}$
\item Solve the equation $x\cdot (7,11) = (8,5)$ for $x\in \Z$
\end{enumerate}
\end{exercise}
\paragraph{Scalar multiplication}
As we have seen in the previous section, elliptic curves $E(\F)$ have the structure of a commutative group associated to them. Moreover, It can moreover be shown that this group is finite and cyclic whenever the field is finite. 

To understand elliptic curve scalar multiplication, recall from page \pageref{cyclic-groups}\sme{check reference} that every finite cyclic group of order $q$ has a generator $g$ and an associated exponential map $g^{(\cdot)}: \Z_q \to \G$, where $g^n$ is the $n$-fold product of $g$ with itself.  

Elliptic curve scalar multiplication is the exponential map written in additive notation. To be more precise, let $\F$ be a finite field, $E(\F)$ an elliptic curve of order $r$, and $P$ a generator of $E(\F)$. Then the \term{elliptic curve scalar multiplication} with base $P$ is defined as follows (where $[0]P = \Oinf$ and $[m]P = P+P+\ldots + P$ is the $m$-fold sum of $P$ with itself):
$$
[\cdot]P: \Z_r \to E(\F); m \mapsto [m]P
$$

therefore, elliptic curve scalar multiplication is an instantiation of the general exponential map using additive instead of multiplicative notation. This map is a homomorph of groups, which means that $[n+m]P = [n]P \oplus [m]P$. 

As with all finite, cyclic groups, the inverse of the exponential map exists and is usually called the \term{elliptic curve discrete logarithm map}. However, elliptic curves are believed to be XXX\sme{add term}-groups, which means that we don't know of any efficient way to actually compute this map.

Scalar multiplication and its inverse, the elliptic curve discrete logarithm, define \term{the elliptic curve discrete logarithm problem}, which consists of finding solutions $m\in\Z_r$ such that the following equation holds:
\begin{equation}
P = [m]Q
\end{equation}

Any solution $m$ is usually called a \term{discrete logarithm relation} between $P$ and $Q$. If $Q$ is a generator of the curve, then there is a discrete logarithm relation between $Q$ and any other point, since $Q$ generates the group by repeatedly adding $Q$ to itself. Therefore, we know that some discrete logarithm relation exists for generator $Q$ and point $P$. However, since elliptic curves are believed to be XXX\sme{add term}-groups, finding actual relations $m$ is computationally hard, with runtimes being approximately the size of the order of the group. In practice, we often need the assumption that a discrete logarithm relation exists, while the relation itself is not known.

One useful property of the exponential map in regard to the examples in this book is that it can be used to greatly simplify pen-and-paper computations. As we have seen in example XXX\sme{add reference}, computing the elliptic curve addition law takes quite a bit of effort when done without a computer. However, when $g$ is a generator of a small pen-and-paper elliptic curve group of order $r$, we can use the exponential map to write the group using \uterm{cofactor clearing}, which implies that $[r]g=\Oinf$:
\begin{equation}\label{cofactor-clearing}
\G = \{[1]g\to [2]g \to [3]g\to\cdots\to [r-1]g\to \Oinf\}
\end{equation} 
``Logarithmic ordering'' like this greatly simplifies complicated elliptic curve addition to the much simpler case of modular $r$ addition. In order to add two curve points $P$ and $Q$, we only have to look up their discrete log relations with the generator, say $P=[n]g$ and $Q=[m]g$, and compute the sum as $P\oplus Q = [n+m]g$. This is, of course, only possible for small groups where we can keep a clear overview, such as XXX\sme{add reference}.

In the following example, we will look at some implications of the fact that elliptic curves are finite cyclic groups. We will apply the fundamental theorem of finite cyclic groups and look how it reflects on the curves in consideration.
\begin{example}\label{ex:G1G2-subgroups}Consider the elliptic curve group $E_1(\F_5)$ from example \ref{E1F5}\sme{check reference}. Since it is a finite cyclic group of order $9$, and the prime factorization of $9$ is $3\cdot 3$, we can use the fundamental theorem of finite cyclic groups to reason about all its subgroups. In fact, since the only prime factor of $9$ is $3$, we know that $E_1(\F_5)$ has the following subgroups:
\begin{itemize}
\item $\G_1 = E_1(\F_5)$ is a subgroup of order $9$. By definition, any group is a subgroup of itself.
\item $\G_2 = \{(2,1),(2,4),\Oinf\}$ is a subgroup of order $3$. This is the subgroup associated to the prime factor $3$.
\item $\G_3 = \{\Oinf\}$ is a subgroup of order $1$. This is the trivial subgroup.
\end{itemize}
Moreover, since $E_1(\F_5)$ and all its subgroups are cyclic, we know from page \pageref{cyclic-groups}\sme{check reference} that they must have generators. For example, the curve point $(2,1)$ is a generator of the order $3$ subgroup $\G_2$, since every element of $\G_2$ can be generated by repeatedly adding $(2,1)$ to itself: 
\begin{align*}
[1](2,1) & = (2,1) \\
[2](2,1) & = (2,4) \\
[3](2,1) & = \Oinf
\end{align*}
Since $(2,1)$ is a generator, we know from XXX\sme{add reference} that it gives rise to an exponential map from the finite field $\F_3$ onto $\G_2$ defined by scalar multiplication:
$$
[\cdot](2,1): \F_3 \to \G_2\; : \; x\mapsto [x](2,1) 
$$
To give an example of a generator that generates the entire group $E_1(\F_5)$, consider the point $(0,1)$. Applying the tangent rule repeatedly, we compute as follows:
$$
\begin{array}{lccr}
{}\begin{array}{lcl}
{}[0](0,1) &=& \Oinf \\
{}[2](0,1) &=& (4, 2) \\ 
{}[4](0,1) &=& (3, 4) \\ 
{}[6](0,1) &=& (2, 4) \\ 
{}[8](0,1) &=& (0, 4) \\ 
\end{array} & & &
{}\begin{array}{lcl}
{}[1](0,1) &=& (0, 1) \\
{}[3](0,1) &=& (2, 1) \\
{}[5](0,1) &=& (3, 1) \\
{}[7](0,1) &=& (4, 3) \\
{}[9](0,1) &=& \Oinf
\end{array}
\end{array}
$$
Again, since $(2,1)$ is a generator, we know from XXX\sme{add reference} that it gives rise to an exponential map. However, since the group order is not a prime number, the exponential map does not map from any field, but from the residue class ring $\Z_9$ only:
$$
[\cdot](0,1): \Z_9 \to \G_1\; : \; x\mapsto [x](0,1) 
$$
Using the generator $(0,1)$ and its associated exponential map, we can write $E(\F_1)$ i logarithmic order with respect to $(0,1)$ as explained in equation \ref{cofactor-clearing}\sme{check reference}. We get the following:
$$
E_1(\F_5) = \{(0, 1)\to (4, 2)\to (2, 1)\to (3, 4)\to (3, 1)\to (2, 4)\to (4, 3)\to (0, 4)\to \Oinf \}
$$
This indicates that the first element is a generator, and the $n$-th element is the scalar product of $n$ and the generator. To see how logarithmic orders like this simplify the computations in small elliptic curve groups, consider example \ref{ex:01+42}\sme{check reference} again. In that example, we use the chord-and-tangent rule to compute $(0,1)\oplus (4,2)$. Now, in the logarithmic order of $E_1(\F)$, we can compute that sum much easier, since we can directly see that $(0,1)=[1](0,1)$ and $(4,2)=[2](0,1)$. We can then deduce $(0,1)\oplus (4,2)= (2,1)$ immediately, since $[1](0,1)\oplus [2](0,1)= [3](0,1)=(2,1)$.

To give another example, we can immediately see that $(3,4)\oplus (4,3) = (4,2)$, without doing any expensive elliptic curve addition, since we know $(3,4)= [4](0,1)$ as well as $(4,3)= [7](0,1)$ from the logarithmic representation of $E_1(\F_5)$. Since $4+7 = 2$ in $\Z_9$, the result must be $[2](0,1)=(4,2)$.

Finally we can use $E_1(\F_5)$ as an example to understand the concept of cofactor clearing from \ref{cofactor-clearing}\sme{check reference}. Since the order of $E_1(\F_5)$ is $9$, we only have a single factor, which happen to be the cofactor as well. Cofactor clearing then implies that we can map any element from $E_1(\F_5)$ onto its prime factor group $\G_2$ by scalar multiplication with $3$. For example, taking the element $(3,4)$, which is not in $\G_2$, and multiplying it with $3$, we get $[3](3,4)= (2,1)$, which is an element of $\G_2$ as expected.
\end{example}
In the following example, we will look at the subgroups of our pen-jubjub curve, define generators, and compute the logarithmic order for pen-and-paper computations. Then we take another look at the principle of cofactor clearing.

\begin{example}\label{ex:PJJ13-cofactor-clearing} Consider the pen-jubjub curve $\mathit{PJJ\_13}$ from example \ref{PJJ13}\sme{check reference} again. Since the order of $\mathit{PJJ\_13}$ is $20$, and the prime factorization of $20$ is $2^2\cdot 5$, we know that the $\mathit{PJJ\_13}$ contains a ``large'' prime-order subgroup of size $5$ and a small prime oder subgroup of size $2$. 

To compute those groups, we can apply the technique of cofactor clearing in a try-and-repeat loop. We start the loop by arbitrarily choosing an element $P\in \mathit{PJJ\_13}$, then multiplying that element with the cofactor of the group that we want to compute. If the result is $\Oinf$, we try a different element and repeat the process until the result is different from the point at infinity $\Oinf$.  

To compute a generator for the small prime-order subgroup $(\mathit{PJJ\_13})_2$, first observe that the cofactor is $10$, since $20=2\cdot 10$. We then arbitrarily choose the curve point $(5,11)\in \mathit{PJJ\_13}$ and compute $[10](5,11)=\Oinf$. Since the result is the point at infinity, we have to try another curve point, say $(9,4)$. We get $[10](9,4)=(4,0)$ and we can deduce that $(4,0)$ is a generator of $(\mathit{PJJ\_13})_2$. Logarithmic order then gives
$(\mathit{PJJ\_13})_2 = \{(4,0)\to \Oinf\}$
as expected, since we know from example \ref{ex:PJJ13-self-inverse}\sme{check reference} that $(4,0)$ is self-inverse, with $(4,0)\oplus (4,0)=\Oinf$. We double check the computations using Sage: 
\begin{sagecommandline}
sage: F13 = GF(13)
sage: PJJ = EllipticCurve(F13,[8,8])
sage: P = PJJ(5,11)
sage: INF = PJJ(0)
sage: 10*P == INF
sage: Q = PJJ(9,4)
sage: R = PJJ(4,0)
sage: 10*Q == R
\end{sagecommandline}
We can apply the same reasoning to the ``large'' prime-order subgroup $(\mathit{PJJ\_13})_5$, which contains $5$ elements. To compute a generator for this group, first observe that the associated cofactor is $4$, since $20=5\cdot 4$. We choose the curve point $(9,4)\in \mathit{PJJ\_13}$ again, and compute $[4](9,4)=(7,11)$. \smelong{We can deduce that $(7,11)$ is a generator of $(\mathit{PJJ\_13})_5$.}\sme{Explain how} Using the generator $(7,11)$, we compute the exponential map $[\cdot](7,11): \F_5 \to \mathit{PJJ\_13}$ and get the following:
\begin{align*}
[0](7,11) &= \Oinf\\
[1](7,11) &= (7,11)\\
[2](7,11) &= (8,5)\\
[3](7,11) &= (8,8)\\
[4](7,11) &= (7,2)
\end{align*}
We can use this computation to write the large-order prime group $(\mathit{PJJ\_13})_5$ of the pen-jubjub curve in logarithmic order, which we will use quite frequently in what follows. We get the following:
\begin{equation}\label{eq:PJJ13-logarithmic-order}
(\mathit{PJJ\_13})_5 = \{(7,11)\to(8,5)\to(8,8)\to(7,2)\to \Oinf\}
\end{equation}
From this, we can immediately see, for example that  $(8,8)\oplus (7,2)= (8,5)$, since 
$3+4=2$ in $\F_5$.
\end{example}
From the previous two examples, the reader might get the impression that elliptic curve computation can be largely replaced by modular arithmetics. This however, is not true in general, but only an artifact of small groups, where it is possible to write the entire group in a logarithmic order. The following example gives some understanding of why this is not possible in cryptographically secure groups.
\begin{example}
\sme{write example}SEKTP BICOIN. DISCRETE LOG HARDNESS PROHIBITS ADDITION IN THE FIELD...
\end{example}
\paragraph{Projective short Weierstraß form}
% https://www.cosic.esat.kuleuven.be/bcrypt/lecture%20slides/wouter.pdf
As we have seen in the previous section, describing elliptic curves as pairs of points that satisfy a certain equation is relatively straight-forward. However, in order to define a group structure on the set of points, we had to add a special point at infinity to act as the neutral element. 

Recalling from the definition of projective planes (section \ref{sec:planes})\sme{check reference}, we know that points at infinity are handled as ordinary points in projective geometry. Therefore, it makes  sense to look at the definition of a short Weierstraß curve in projective geometry.

To see what a short Weierstraß curve in projective coordinates is, let $\F$ be a finite field of order $q$ and characteristic $>3$, let $a,b\in \F$ be two field elements such that $\Zmod{4a^3+ 27b^2}{q}\neq 0$ and let $\F\mathrm{P}^2$ be the projective plane over $\F$. Then a \term{short Weierstraß elliptic curve} over $\F$ in its projective representation is the set of all points $[X:Y:Z]\in \F\mathrm{P}^2$ from the projective plane that satisfy the \term{homogenous} cubic equation $Y^2\cdot Z = X^3+a\cdot X\cdot Z^2 + b\cdot Z^3$:

\begin{equation}
E(\F\mathrm{P}^2) = \{[X:Y:Z]\in \F\mathrm{P}^2\;|\; Y^2\cdot Z = X^3+a\cdot X\cdot Z^2 + b\cdot Z^3 \}
\end{equation}

To understand how the point at infinity is unified in this definition, recall from XXX\sme{add reference} that, in projective geometry, points at infinity are given by homogeneous coordinates $[X:Y:0]$. Inserting representatives $(x_1,y_1,0)\in [X:Y:0]$ from those classes into the defining homogenous cubic equations gives the following:
\begin{align*}
y_1^2\cdot 0 & = x_1^3+a\cdot x_1\cdot 0^2 + b\cdot 0^3 & \Leftrightarrow \\
0 & = x_1^3
\end{align*} 

This shows that the only point at infinity, that is also a point on a projective short Weierstraß curve is the class $[0,1,0] = \{(0,y,0)\;|\; y\in \F\}$.

This point is the projective representation of $\mathcal{O}$. The projective representation of a short Weierstraß curve, therefore, has the advantage that it does not need a special symbol to represent the point at infinity $\mathcal{O}$ from the affine definition.

\begin{example} To get an intuition of how an elliptic curve in projective geometry looks, consider curve $E_1(\F_5)$ from example (\ref{E1F5}\sme{check reference}). We know that, in its affine representation, the set of rational points is given as follows:

\begin{equation}\label{eq:E1F5-affine}
E_1(\F_5) = \{\Oinf, (0,1),(2,1),(3,1),(4,2),(4,3),(0,4),(2,4),(3,4)\}
\end{equation}

This is defined as the set of all pairs $(x,y)\in \F_5\times \F_5$ such that the affine short Weierstraß equation $y^2 = x^3 + ax +b$ with $a=1$ and $b=1$ is satisfied.

To find the projective representation of a short Weierstraß curve with the same parameters $a=1$ and $b=1$, we have to compute the set of projective points $[X:Y:Z]$ from the projective plane $\F_5\mathrm{P}^2$ that satisfy the following homogenous cubic equation for any representative $(x_1,y_1,z_1)\in [X:Y:Z]$:
\begin{equation}\label{eq:homogenous-cubic}
y_1^2z_1 = x_1^3 + 1\cdot x_1 z_1^2 + 1\cdot z_1^3
\end{equation}
 We know from XXX\sme{add reference} that the projective plane $\F_5\mathrm{P}^2$ contains $5^2+5+1= 31$ elements, so we can take the effort and insert all elements into equation \ref{eq:homogenous-cubic}\sme{check reference} and see if both sides match.

For example, consider the projective point $[0:4:1]$. We know from XXX\sme{add reference} that this point in the projective plane represents the following line in the three-dimensional space $\F^3$:
$$
[0:4:1] = \{(0,0,0),(0,4,1),(0,3,2),(0,2,3),(0,1,4)\}
$$  
To check whether or not $[0:4:1]$ satisfies \ref{eq:homogenous-cubic}\sme{check reference}, we can insert any representative, in other words, any element from XXX\sme{add reference}. Each element satisfies the equation if and only if all other elements satisfy the equation. So, we insert $(0,4,1)$ and get the following result:
$$
1^2\cdot 1 = 0^3 + 1\cdot 0\cdot 1^2 + 1\cdot 1^3
$$

This tells us that the affine point $[0:4:1]$ is indeed a solution to the equation \ref{eq:homogenous-cubic}, but we could just as well have inserted any other representative. For example, inserting $(0,3,2)$ also satisfies \ref{eq:homogenous-cubic}\sme{check reference}: 
$$
3^2\cdot 2 = 0^3 + 1\cdot 0\cdot 2^2 + 1\cdot 2^3
$$
To find the projective representation of $E_1$, we first observe that the projective line at infinity $[1:0:0]$ is not a curve point on any projective short Weierstraß curve, since it cannot satisfy XXX\sme{add reference} for any parameter $a$ and $b$. Therefore, we can exclude it from our consideration. 

Moreover, a point at infinity $[X:Y:0]$ can only satisfy equation XXX\sme{add reference} for any $a$ and $b$, if $X=0$, which implies that the only point at infinity relevant for short Weierstraß elliptic curves is $[0:1:0]$, since $[0:k:0]= [0:1:0]$ for all $k$ from the finite field. Therefore, we can exclude all points at infinity except the point $[0:1:0]$.

All points that remain are the affine points $[X:Y:1]$. Inserting all of them into XXX,\sme{add reference} we get the set of all projective curve points as follows:

\begin{multline*}
E_1(\F_5\mathrm{P}^2)=\{[0:1:0], [0:1:1], [2:1:1], [3:1:1], \\ [4:2:1], [4:3:1], [0:4:1], [2:4:1], [3:4:1]\}
\end{multline*}

If we compare this with the affine representation, we see that there is a 1:1 correspondence between the points in the affine representation in \ref{eq:E1F5-affine}\sme{check reference} and the affine points in projective geometry, and that the point $[0:1:0]$ represents the additional point $\Oinf$ in the projective representation.
\end{example} 

\begin{exercise}
Compute the projective representation of the pen-jubjub curve and the logarithmic order of its large prime-order subgroup with respect to the generator $(7,11)$.
\end{exercise}

\paragraph{Projective Group law}
As we have seen on page \pageref{chap:elliptic-curves}\sme{check reference}, one of the key properties of an elliptic curve is that it comes with a definition of a group law on the set of its rational points, described geometrically by the chord-and-tangent rule (definition \ref{def:chord-and-tangent}). This rule was kind of intuitive, with the exception of the distinguished point at infinity, which appeared whenever the chord or the tangent did not have a third intersection point with the curve.

One of the key features of projective coordinates is that, in projective space, it is guaranteed that any chord will always intersect the curve in three points, and any tangent will intersect it in two points including the tangent point. So, the geometric picture simplifies, as we don't need to consider external symbols and associated cases.

Again, it can be shown that the points of an elliptic curve in projective space form a commutative group with respect to the tangent-and-chord rule such that the projective point $[0:1:0]$ is the neutral element, and the additive inverse of a point $[X:Y:Z]$ is given by $[X:-Y:Z]$. The addition law is usually described by the following algorithm, minimizing the number of necessary additions and multiplications in the base field. \final{Check if following Alg is floated too far}
% https://www.hyperelliptic.org/EFD/precomp.pdf

\begin{algorithm}\caption{Projective Weierstraß Addition Law}
% https://en.wikibooks.org/wiki/Cryptography/Prime_Curve/Standard_Projective_Coordinates
\begin{algorithmic}[0]
\Require $[X_1:Y_1:Z_1],[X_2:Y_2:Z_2] \in E(\F\mathbb{P}^2)$
\Procedure{Add-Rule}{$[X_1:Y_1:Z_1],[X_2:Y_2:Z_2]$}
\If{$[X_1:Y_1:Z_1] == [0:1:0]$}
  \State $[X_3:Y_3:Z_3] \gets [X_2:Y_2:Z_2]$
\ElsIf{$[X_2:Y_2:Z_2] == [0:1:0]$}
  \State $[X_3:Y_3:Z_3] \gets [X_1:Y_1:Z_1]$
\Else
  \State $U_1 \gets Y_2\cdot Z_1$
  \State $U_2 \gets Y_1\cdot Z_2$
  \State $V_1 \gets X_2\cdot Z_1$
  \State $V_2 \gets X_1\cdot Z_2$
  \If{$V_1 == V_2$}
    \If{$U_1 \neq U_2$}
      $[X_3:Y_3:Z_3] \gets [0:1:0]$
    \Else
      \If{$Y_1 == 0$}
        $[X_3:Y_3:Z_3] \gets [0:1:0]$
      \Else
        \State $W \gets a\cdot Z_1^2 + 3\cdot X_1^2$
        \State $S \gets Y_1\cdot Z_1$
        \State $B \gets X_1\cdot Y_1\cdot S$
        \State $H \gets W^2 - 8\cdot B$
        \State $X' \gets 2\cdot H\cdot S$
        \State $Y' \gets W\cdot (4\cdot B - H) - 8\cdot Y_1^2\cdot S^2$
        \State $Z' \gets 8\cdot S^3$
        \State $[X_3:Y_3:Z_3] \gets [X':Y':Z']$
      \EndIf
    \EndIf
  \Else
    \State $U = U_1 - U_2$
    \State $V = V_1 - V_2$
    \State $W = Z_1\cdot Z_2$
    \State $A = U^2\cdot W - V^3 - 2\cdot V^2\cdot V_2$
    \State $X' = V\cdot A$
    \State $Y' = U\cdot(V^2\cdot V_2 - A) - V^3\cdot U_2$
    \State $Z' = V^3\cdot W$
    \State $[X_3:Y_3:Z_3]\gets [X':Y':Z']$
  \EndIf
\EndIf
\State \textbf{return} $[X_3:Y_3:Z_3]$
\EndProcedure
\Ensure $ [X_3:Y_3:Z_3] == [X_1:Y_1:Z_1] \oplus [X_2:Y_2:Z_2]$
\end{algorithmic}
\end{algorithm}
%\begin{example}[Polynomial evaluation on secret points]
%Since scalar multiplication is assumed to be a one way function, it can be used to encrypt computations. For example it can be used to proof identities of bounded degree polynomials (with some probability), without actually revealing the polynomials. To see what this means, consider the moon-jubjub curve $MJJ(\F_{13})$ from XXX\sme{add reference} and the set $\F_{13}[x]_{\leq 2}$ of all polynomials with coefficients in $\F_{13}$ and maximum degree $2$.

%Now assume that there are two parties $A$ and $B$ such that $A$ choose polynomial $P_A$ and $B$ chooses polynomial $P_B$ from $\F_{13}[x]_{\leq 2}$. The task is to check (with some probability) weather or not $P_A$ equals $P_B$ without actually revealing any information about the polynomials. 

%This task can be solved, by evaluating the polynomials at a secret point in the exponent of a (DFHM-PROPERTY) group and then compare the results.

%So we assume that there is some trusted third party $C$ that chooses a publicly known generator of a large prime-order subgroup of $MJJ(\F_{13})$, say  a secrete point $s\in\F_{13}$, say $s=2$. $C$ then c
%\end{example}


\begin{exercise}
Compare the affine addition law for short Weierstraß curves with the projective addition rule. Which branch in the projective rule corresponds to which case in the affine law? 
\end{exercise}

\paragraph{Coordinate Transformations} As we have seen in example XXX\sme{add reference}, there was a close relation between the affine and the projective representation of a short Weierstraß curve. This was not a coincidence.
In fact, from a mathematical point of view, projective and affine short Weierstraß curves describe the same thing, as there is a one-to-one correspondence (an isomorphism) between both representations for any arbitrary parameters $a$ and $b$. 

To specify the isomorphism, let $E(\F)$ and $E(\F\mathrm{P}^2)$ be an affine and a projective short Weierstraß curve defined for the same parameters $a$ and $b$. Then the map in \ref{eq:weierstrass-isomorphism-map} maps points from the affine representation to points from the projective representation of a short Weierstraß curve. In other words, if the pair of points $(x,y)$ satisfies the affine equation $y^2= x^3 + ax + b$, then all homogeneous coordinates $(x_1,y_1,z_1)\in [x:y:1]$ satisfy the projective equation $y_1^2\cdot z_1= x_1^3 + ay_1\cdot z_1^2 + b\cdot z_1^3$. 

\begin{equation}\label{eq:weierstrass-isomorphism-map}
\Phi : E(\F) \to E(\F\mathrm{P}^2)\;:\;
\begin{array}{lcl}
(x,y)       &\mapsto & [x:y:1]\\
\mathcal{O} &\mapsto & [0:1:0]
\end{array}
\end{equation}


The inverse is given by the following map:
\begin{equation}
\Phi^{-1} : E(\F\mathbb{P}^2)\to E(\F) \;:\; [X:Y:Z] \mapsto \begin{cases}
(\frac{X}{Z},\frac{Y}{Z}) & \text{ if } Z\neq 0\\
\mathcal{O} & \text{ if } Z=0
\end{cases}
\end{equation}

Note that the only projective point $[X:Y:Z]$ with $Z\neq 0$ that satisfies XXX\sme{add reference} is given by the class $[0:1:0]$. 

One key feature of $\Phi$ and its inverse is that it respects the group structure, which means that $\Phi((x_1,y_1)\oplus (x_2,y_2))$ is equal to $\Phi(x_1,y_1)\oplus \Phi(x_2,y_2)$. The same holds true for the inverse map $\Phi^{-1}$.

Maps with these properties are called \term{group isomorphisms}, and, from a mathematical point of view, the existence of $\Phi$ implies that these two definitions are equivalent, and implementations can choose freely between these representations. 


\subsection{Montgomery Curves}\label{sec:montgomery}
% https://eprint.iacr.org/2017/212.pdf
\smelong{History and use of them (optimized scalar multiplication)}\sme{write up this part}

\paragraph{Affine Montgomery Form}
To see what a Montgomery curve in affine coordinates is, let $\F$ be a finite field of characteristic $>2$, and let $A,B\in \F$ be two field elements such that $B\neq 0$ and $A^2 \neq 4$.  A \term{Montgomery elliptic curve} $M(\F)$ over $\F$ in its affine representation is the set of all pairs of field elements $(x,y)\in \F\times \F$ that satisfy the Montgomery cubic equation $B\cdot y^2 = x^3 + A\cdot x^2 + x$, together with a distinguished symbol $\Oinf$, called the \term{point at infinity}.

\begin{equation}\label{eq:montgomery-curve}M(\F) = \{(x,y)\in \F\times \F\;|\; B\cdot y^2 = x^3 + A\cdot x^2 + x  \} \bigcup \{\Oinf\}
\end{equation}\sme{is the label in \LaTeX{} correct here?}

Despite the fact that Montgomery curves look different from short Weierstraß curves, they are just a special way to describe certain short Weierstraß curves. In fact, every curve in affine Montgomery form can be transformed into an elliptic curve in Weierstraß form. To see that, assume that a curve is given in Montgomery form $B y^2 = x^3 + A x^2 + x$. The associated Weierstraß form is then as follows:

\begin{equation}\label{eq:montgomery-to-weierstrass}
y^2 = x^3 + \frac{3-A^2}{3B^2}\cdot x + \frac{2A^3-9A}{27B^3}
\end{equation}

On the other hand, an elliptic curve $E(\F)$ over base field $\F$ in Weierstraß form $y^2 = x^3 + a x + b$ can be converted to Montgomery form if and only if the following conditions hold:
\begin{definition}{\deftitle{Requirements for Montgomery curves}}\label{def:montgomery}
\begin{itemize}
\item The number of points on $E(F)$ is divisible by $4$
\item The polynomial $z^3 + a z + b \in \F[z]$ has at least one root $z_0\in\F$
\item $3z_0^2 + a$ is a quadratic residue in $\F$.
\end{itemize}
\end{definition}

When these conditions are satisfied, then for $s=({\sqrt{3z_0^{2}+a}})^{-1}$, the equivalent Montgomery curve is defined by the following equation:
\begin{equation}\label{eq:montgomery-form}
sy^{2}=x^{3}+(3z_0 s)x^{2}+x
\end{equation}

In the following example we will look at our pen-jubjub curve again, and show that it is actually a Montgomery curve.
\begin{example}\label{PJJ13-montgomery}
Consider the prime field $\F_{13}$ and the pen-jubjub curve $\mathit{PJJ\_13}$ from example \ref{PJJ13}\sme{check reference}. To see that it is a Montgomery curve, we have to check the requirements from \ref{def:montgomery}\sme{check reference}: 

Since the order  of $\mathit{PJJ\_13}$ is $20$, which is divisible by $4$, the first requirement is met. 

Next, since $a=8$ and $b=8$, we have to check if the polynomial $P(z) = z^3 + 8z + 8$ has a root in $\F_{13}$. We simply evaluate $P$ at all numbers $z\in \F_{13}$, and find that $P(4)=0$, so a root is given by $z_0=4$.

In the last step, we have to check that $3\cdot z_0^2 + a$ has a root in $\F_{13}$. We compute as follows:
\begin{align*}
3z_0^2 + a & = 3\cdot 4^2 + 8 \\
           & = 3 \cdot 3 + 8 \\
           & = 9 + 8 \\
           & = 4
\end{align*}

To see if $4$ is a quadratic residue, we can use Euler's criterion (\ref{eq: Euler_criterion})\sme{check reference} to compute the Legendre symbol of $4$. We get the following:

$$
\left(\frac{4}{13}\right) = 4^{\frac{13-1}{2}} = 4^6 = 1
$$ 
This means that $4$ does have a root in $\F_{13}$. In fact, computing a root of $4$ in $\F_{13}$ is easy, since the integer root $2$ of $4$ is also one of its roots in $\F_{13}$. The other root is given by $13-4=9$.

Since all requirements are meet, we have now shown that $\mathit{PJJ\_13}$ is indeed a Montgomery curve, and we can use \ref{eq:montgomery-form}\sme{check reference} to compute its associated Montgomery form. We compute as follows:
\begin{align*}
s & = \left(\sqrt{3\cdot z_0^2 +8}\right)^{-1} \\
  & = 2^{-1} & \text{\# Fermat's little theorem} \\
  & = 2^{13-2} & \text{\# }\Zmod{2048}{13} = 7\\
  & = 7
\end{align*}
The defining equation for the Montgomery form of our pen-jubjub curve is then given by the following equation:
\begin{align*}
sy^{2} & =x^{3}+(3z_0 s)x^{2}+x  & \Rightarrow\\
7\cdot y^{2} & =x^{3}+(3\cdot 4 \cdot 7)x^{2}+x &\Leftrightarrow\\
7\cdot y^{2} & =x^{3}+6x^{2}+x
\end{align*}
So, we get the defining parameters as $B= 7$ and $A=6$, and we can write the pen-jubjub curve in its affine Montgomery representation as follows:
\begin{equation}\label{eq:PJJ13-montgomery-representation}
\mathit{PJJ\_13} = \{(x,y)\in \F_{13}\times \F_{13}\;|\; 7\cdot y^{2} =x^{3}+6x^{2}+x \}\bigcup \{\Oinf\}
\end{equation}

Now that we have the abstract definition of our pen-jubjub curve in Montgomery form, we can compute the set of points by inserting all pairs $(x,y)\in\F_{13}\times \F_{13}$ similarly to how we computed the curve points in its Weierstraß representation. We get the following:
\begin{multline*}
\mathit{PJJ\_13} = \{\Oinf, (0, 0),(1, 4),(1, 9),(2, 4),(2, 9),(3, 5),(3, 8),(4, 4),(4, 9),\\ (5, 1),(5, 12),(7, 1),(7, 12),(8, 1),(8, 12),(9, 2),(9, 11),(10, 3),(10, 10)\}
\end{multline*}
\begin{sagecommandline}
sage: F13 = GF(13)
sage: L_MPJJ = []
....: for x in F13:
....:     for y in F13:
....:         if F13(7)*y^2 == x^3 + F13(6)*x^2 +x:
....:             L_MPJJ.append((x,y))
sage: MPJJ = Set(L_MPJJ)
sage: # does not compute the point at infinity
\end{sagecommandline}
\end{example}
\paragraph{Affine Montgomery coordinate transformation} Comparing the Montgomery representation of the previous example (equation \ref{eq:PJJ13-montgomery-representation}) with the Weierstraß representation of the same curve (equation \ref{eq:PJJ13-weierstrass}\sme{check reference}), we see that there is a 1:1 correspondence between the curve points in both examples. This is no accident. In fact, if $M_{A,B}$ is a Montgomery curve, and $E_{a,b}$ a Weierstraß curve with $a = \frac{3-A^2}{3B^2}$ and $b= \frac{2A^2 -9A}{27B^3}$ then the following function maps all points in Montgomery representation onto the points in Weierstraß representation:
\begin{equation}
\Phi: M_{A,B} \to E_{a,b}\; : \; (x,y) \mapsto \left(\frac{3x + A}{3B}, \frac{y}{B}\right)
\end{equation}
This map is a 1:1 correspondence (am isomorphism), and its inverse map is given by the following equation (where $z_0$ is a root of the polynomial $z^3 + a z + b \in \F[z]$ and $s=({\sqrt{3z_0^{2}+a}})^{-1}$).
\begin{equation}
\Phi^{-1}: E_{a,b} \to M_{A,B}\; : \; (x,y) \mapsto \left(s\cdot(x-z_0), s\cdot y\right)
\end{equation}
 Using this map, it is therefore possible for implementations of Montgomery curves to freely transit between the Weierstraß and the Montgomery representation. However, as we saw in definition \ref{def:montgomery}\sme{check reference}, not every Weierstraß curve is a Montgomery curve, as all criteria in \ref{def:montgomery}\sme{check reference} have to be satisfied. This means that the map $\Phi^{-1}$ does not always exist. 
 
\begin{example} Consider our pen-jubjub curve again. In equation \ref{eq:PJJ13-weierstrass}\sme{check reference} we derived its Weierstraß representation and in example \ref{eq:PJJ13-montgomery-representation}\sme{check reference}, we derived its Montgomery representation. 

To see how coordinate transformation $\Phi$ works in this example, let's map points from the Montgomery representation onto points from the Weierstraß representation. Inserting, for example, the point $(0,0)$ from the Montgomery representation \ref{eq:PJJ13-montgomery-representation}\sme{check reference} into $\Phi$ gives the following:
\begin{align*}
\Phi(0,0) & = \left(\frac{3\cdot 0 + A}{3B}, \frac{0}{B}\right) \\
          & = \left(\frac{3\cdot 0 + 6}{3\cdot 7}, \frac{0}{7}\right) \\
          & = \left(\frac{6}{8}, 0\right) \\
          & = \left(4, 0\right) \\
\end{align*}

As we can see, the Montgomery point $(0,0)$ maps to the self-inverse point $(4,0)$ of the Weierstraß representation. On the other hand, we can use our computations of $s=7$ and $z_0=4$ from XXX\sme{add reference} to compute the inverse map $\Phi^{-1}$, which maps points on the Weiertraß representation to points on the Mongomery form. Inserting, for example, $(4,0)$ we get the following:
\begin{align*}
\Phi^{-1}(4,0) & = \left(s\cdot(4-z_0), s\cdot 0\right)\\
               & = \left(7\cdot(4-4), 0\right)\\
               & = (0,0)
\end{align*}

As expected, the inverse map maps the Weierstraß point back to where it originated in the Montgomery form. We can invoke Sage to check that our computation of $\Phi$ is correct:
\begin{sagecommandline}
sage: # Compute PHI of Montgomery form:
sage: L_PHI_MPJJ = []
sage: for (x,y) in L_MPJJ: # LMJJ as defined previously                                   
....:     v = (F13(3)*x + F13(6))/(F13(3)*F13(7))
....:     w = y/F13(7)
....:     L_PHI_MPJJ.append((v,w))
sage: PHI_MPJJ = Set(L_PHI_MPJJ)
sage: # Computation Weierstrass form
sage: C_WPJJ = EllipticCurve(F13,[8,8]) 
sage: L_WPJJ = [P.xy() for P in C_WPJJ.points() if P.order() > 1]
sage: WPJJ = Set(L_WPJJ)
sage: # check PHI(Montgomery) == Weierstrass
sage: WPJJ == PHI_MPJJ
sage: # check the inverse map PHI^(-1)
sage: L_PHIINV_WPJJ = []
sage: for (v,w) in L_WPJJ:
....:     x = F13(7)*(v-F13(4))
....:     y = F13(7)*w
....:     L_PHIINV_WPJJ.append((x,y))
sage: PHIINV_WPJJ = Set(L_PHIINV_WPJJ)
sage: MPJJ == PHIINV_WPJJ
\end{sagecommandline}
\end{example}

\paragraph{Montgomery group law} We have seen that Montgomery curves special cases of short Weierstraß curves. As such, they have a group structure defined on the set of their points, which can also be derived from the chord-and-tangent rule. In accordance with short Weierstraß curves, it can be shown that the identity $x_1=x_2$ implies $y_2=\pm y_1$, meaning that the following rules are a complete description of the affine addition law.

\begin{definition}{\deftitle{Montgomery group law}}\label{def:montgomery-group-law}
\begin{itemize}
\item (Neutral element) Point at infinity $\Oinf$ is the neutral element.
\item (Additive inverse ) The additive inverse of $\mathcal{O}$ is $\mathcal{O}$. For any other curve point $(x,y) \in M(\F_q)\textbackslash \{\mathcal{O}\}$, the additive inverse is given by $(x,-y)$.
\item (Addition rule) For any two curve points $P, Q \in M(\F_q)$, addition is defined by one of the following cases:
\begin{enumerate}
\item (Adding the neutral element) If $Q=\Oinf$, then the sum is defined as $P+Q=P$.
\item (Adding inverse elements)  If $P=(x,y)$ and $Q=(x,-y)$, the sum is defined as $P+Q=\Oinf$.
\item (Adding non-self-inverse equal points) If $P=(x,y)$ and $Q=(x,y)$ with $y\neq 0$, the sum $2P=(x',y')$ is defined as follows:
$$
\begin{array}{llr}
x' = (\frac{3x_1^2 + 2A x_1 +1}{2By_1})^2\cdot B - (x_1 + x_2) - A &,&
y' = \frac{3x_1^2 + 2A x_1 +1}{2By_1}(x_1-x') - y_1
\end{array} 
$$
\item (Adding non-inverse different points) If $P=(x_1,y_1)$ and $Q=(x_2,y_2)$ such that $x_1 \neq x_2$, the sum $R=P+Q$ with $R=(x_3,y_3)$ is defined as follows:
$$
\begin{array}{llr}
x' = (\frac{y_2-y_1}{x_2-x_1})^2B - (x_1 + x_2) - A &, &
y' = \frac{y_2-y_1}{x_2-x_1}(x_1-x') - y_1
\end{array} 
$$
\end{enumerate}
\end{itemize} 
\end{definition}
%\paragraph{Projective Montgomery Form}
%As with more general curves in short Weierstraß form, we can look at Montgomery curves in projective space. To see how such a curve looks in projective coordinates is, let $\F_{q}$ be a finite field and $A,B\in \F_q$ two field elements such that $B\neq 0$ and $A^2\neq 4$. Then a \term{Montgomery elliptic curve} $E/\F_q$ over $\F_q$ in its projective representation is the set
%\begin{equation}
%\label{def_short_weierstrass_curve}
%E/\F_q\mathbb{P}^2 = \{[X:Y:Z]\in \F_q\mathbb{P}^2\;|\; B\cdot Y^2 \cdot Z = X^3 + A\cdot X^2\cdot Z + X\cdot Z^2  \}
%\end{equation}
%of all points $[X:Y:Z]\in \F_q\mathbb{P}^2$ from the projective plane that satisfy the \term{homogenous} cubic equation $B\cdot Y^2 \cdot Z = X^3 + A\cdot X^2\cdot Z + X\cdot Z^2$.

% TODO: https://maths-people.anu.edu.au/~brent/pd/Subramanya-thesis.pdf
% x-coordinate only aka differential arithmetics on Montgomery curves
\subsection{Twisted Edwards Curves}\label{sec:edwards}
As we have seen in \ref{def:montgomery-group-law}\sme{check reference} both Weierstraß and Montgomery curves have somewhat complicated addition and doubling laws, as many cases have to be distinguished. Those various cases translate to branches in computer programs.

In the context of SNARK development, two computational models for bounded computations are used, called \term{circuits} and \term{rank-1 constraint systems}. Program branches are undesirably costly when implemented in those models. It is therefore advantageous to look for curves with an addition/doubling rule that requires no branches and as few field operations as possible.

\term{Twisted Edwards curves} are particularly useful here, as a subclass of these curves has a compact and easily implementable addition law that works for all points including the point at infinity. Implementing this law needs no branching. 
\paragraph{Twisted Edwards Form}
% https://eprint.iacr.org/2008/013.pdf
To see what an affine \term{twisted Edwards curve} looks like, let $\F$ be a finite field of characteristic $>2$, and let $a,d\in \F\backslash\{0\}$ be two non-zero field elements with $a\neq d$.  A \term{twisted Edwards elliptic curve} in its affine representation is the set of all pairs $(x,y)$ from $\F\times \F$ that satisfy the twisted Edwards equation $a\cdot x^2+y^2= 1+d\cdot x^2y^2$, given below:
\begin{equation}
E(\F)=\{(x,y)\in\F\times\F\;|\; a\cdot x^2+y^2= 1+d\cdot x^2y^2\}
\end{equation} 
A twisted Edwards curve is called an \term{Edwards curve (non-twisted)}, if the parameter $a$ is equal to $1$, and it is called a \term{SNARK-friendly twisted Edwards curve} if the parameter $a$ is a quadratic residue and the parameter $d$ is a quadratic non-residue.

As we can see from the definition, affine twisted Edwards curves look somewhat different from Weierstraß curves, as their affine representation does not need a special symbol to represent the point at infinity. In fact, we we will see that the pair $(0,1)$ is always a point on any twisted Edwards curve, and that it takes the role of the point at infinity.

Despite their different appearances however, twisted Edwards curves are equivalent to Montgomery curves in the sense that, for every twisted Edwards curve, there is a Montgomery curve, and a way to map the points of one curve in a 1:1 correspondence onto the other and vice versa. To see that, assume that a curve in twisted Edwards form $a\cdot x^2+y^2= 1+d\cdot x^2y^2$ is given. The associated Montgomery curve is then defined by the Montgomery equation:
\begin{equation}
\frac{4}{a-d} y^2 = x^3 + \frac{2(a+d)}{a-d}\cdot x^2 + x 
\end{equation}

On the other hand, a Montgomery curve $By^{2}=x^{3}+Ax^{2}+x$ with $B\neq 0$ and $A^2\neq 4$ can give rise to a twisted Edwards curve defined by the following equation:
\begin{equation}\label{eq:montgomery-to-twisted-edwards}
(\frac{A+2}{B})x^2+y^2= 1+(\frac{A-2}{B})x^2y^2
\end{equation}

As we have seen in equation \ref{eq:montgomery-to-weierstrass} and the following discussion,\sme{check reference}  Montgomery curves are just a special class of Weierstraß curves. Furthermore we now know that twisted Edwards curves are special Weierstraß curves too. This means that the more general way to describe elliptic curves is as Weierstraß curves.

\begin{example}Consider the pen-jubjub curve from example \ref{PJJ13}\sme{check reference} again. We know from example \ref{PJJ13-montgomery}\sme{check reference} that it is a Montgomery curve, and, since Montgomery curves are equivalent to twisted Edwards curves, we want to write this curve in twisted Edwards form. We use equation \ref{eq:montgomery-to-twisted-edwards},\sme{check reference} and compute the parameters $a$ and $d$ as follows:
\begin{align*}
a & = \frac{A+2}{B} & \text{\# insert A=6 and B=7}\\
  & = \frac{8}{7} = 3 & \text{\# } 7^{-1}= 2 \\
  \\
d & = \frac{A-2}{B} \\
  & = \frac{4}{7} = 8 
\end{align*}

Thus, we get the defining parameters as $a= 3$ and $d=8$. Since our goal is to use this curve later on in implementations of pen-and-paper SNARKs, let us show that tiny-jubjub\sme{change ``tiny-jubjub'' to ``pen-jubjub'' throughout?} is also a \term{SNARK-friendly} twisted Edwards curve. To see that, we  have to show that $a$ is a quadratic residue and $d$ is a quadratic non-residue. We therefore compute the Legendre symbols of $a$ and $d$ using Euler's criterion. We get the following:
\begin{align*}
\left(\frac{3}{13}\right) &= 3^{\frac{13-1}{2}} \\
                          & = 3^6 
                            = 1\\
                          \\
\left(\frac{8}{13}\right) &= 8^{\frac{13-1}{2}} \\
                          & = 8^6 
                            = 12
                            = -1                     
\end{align*}

This proves that tiny-jubjub is SNARK-friendly. We can write the tiny-jubjub curve in its affine twisted Edwards representation as follows:
\begin{equation}\label{PJJ13-twisted-edwards}
\mathit{TJJ\_13} = \{(x,y)\in \F_{13}\times \F_{13}\;|\; 3\cdot x^{2} + y^2 =1+ 8\cdot x^{2}\cdot y^2 \}
\end{equation}

Now that we have the abstract definition of our pen-jubjub curve in twisted Edwards form, we can compute the set of points by inserting all pairs $(x,y)\in\F_{13}\times \F_{13}$, similarly to how we computed the curve points in its Weierstraß or Edwards representation. We get the following:
\begin{equation}
\begin{split}
\mathit{PJJ\_13} = \{(0, 1),(0, 12),(1, 2),(1, 11),(2, 6),(2, 7),(3, 0),(5, 5),(5, 8),(6, 4),\\
(6, 9),(7, 4),(7, 9),(8, 5),(8, 8),(10, 0),(11, 6),(11, 7),(12, 2),(12, 11)\}
\end{split}
\end{equation}
\begin{sagecommandline}
sage: F13 = GF(13)
sage: L_EPJJ = []
....: for x in F13:
....:     for y in F13:
....:         if F13(3)*x^2 + y^2 == 1+ F13(8)*x^2*y^2:
....:             L_EPJJ.append((x,y))
sage: EPJJ = Set(L_EPJJ)
\end{sagecommandline}


\end{example}
\paragraph{Twisted Edwards group law}As we have seen, twisted Edwards curves are equivalent to Montgomery curves, and, as such, also have a group law. However, in contrast to Montgomery and Weierstraß curves, the group law of SNARK-friendly twisted Edwards curves can be described by a single computation that works in all cases, no matter if we add the neutral element, the inverse, or if we have to double a point. To see what the group law looks like, first observe that the point $(0,1)$ is
a solution to $a\cdot x^{2} + y^2 =1+ d\cdot x^{2}\cdot y^2$ for any curve. The sum of any two points $(x_1, y_1)$, $(x_2, y_2)$ on an Edwards curve $E(\F)$ is then given by the following equation:

\begin{equation}\label{twisted-edwards-group-law}
(x_1, y_1) \oplus (x_2, y_2) =\left(\frac{x_1y_2+y_1x_2}{1 +dx_1x_2y_1y_2},\frac{y_1y_2-ax_1x_2}{1-dx_1x_2y_1y_2}\right)
\end{equation}

and it can be shown that the point $(0,1)$ serves as the neutral element and the inverse of a point $(x_1, y_1)$ is given by $(-x_1, y1)$.
\begin{example} Lets look at the tiny-jubjub curve in Edwards form from example \ref{PJJ13-twisted-edwards}\sme{check reference} again. As we have seen, this curve is given by
\begin{multline*}
\mathit{PJJ\_13} = \{(0, 1),(0, 12),(1, 2),(1, 11),(2, 6),(2, 7),(3, 0),(5, 5),(5, 8),(6, 4),\\
(6, 9),(7, 4),(7, 9),(8, 5),(8, 8),(10, 0),(11, 6),(11, 7),(12, 2),(12, 11)\}
\end{multline*}
To get an understanding of the twisted Edwards addition law, let's first add the neutral element $(0,1)$ to itself. We apply the group law \ref{twisted-edwards-group-law}\sme{check reference} and get the following:
\begin{align*}
(0, 1) \oplus (0, 1) &= \left(\frac{0\cdot 1+1 \cdot 0}{1 +8\cdot0\cdot 0\cdot 1\cdot 1},\frac{1\cdot 1-3\cdot 0\cdot 0}{1-8\cdot 0\cdot 0\cdot 1\cdot 1}\right)\\
                     & = (0,1)
\end{align*}
So, as expected, the neutral element added to itself gives the neutral element again. Now let's add the neutral element to some other curve point. We get the following:
\begin{align*}
(0, 1) \oplus (8, 5) &= \left(\frac{0\cdot 5+1 \cdot 8}{1 +8\cdot0\cdot 8\cdot 1\cdot 5},\frac{1\cdot 5 - 3\cdot 0\cdot 8}{1-8\cdot 0\cdot 8\cdot 1\cdot 5}\right)\\
                     & = (8,5)
\end{align*}

Again, as expected, adding the neutral element to any element will result in that element again. Given any curve point $(x,y)$, we know that its inverse is given by $(-x,y)$. To see how the addition of a point to its inverse works, we compute as follows:
\begin{align*}
(5, 5) \oplus (8, 5) &= \left(\frac{5\cdot 5+5 \cdot 8}{1 +8\cdot 5\cdot 8\cdot 5\cdot 5},\frac{5\cdot 5 - 3\cdot 5\cdot 8}{1-8\cdot 5\cdot 8\cdot 5\cdot 5}\right)\\
                     &= \left(\frac{12+1}{1 +5},\frac{12 - 3}{1-5}\right)\\
                     &= \left(\frac{0}{6},\frac{12 + 10}{1+8}\right)\\
                     &= \left(0,\frac{9}{9}\right)\\
                     &=  (0,1)
\end{align*}

Adding a curve point to its inverse gives the neutral element, as expected. As we have seen from these examples, the twisted Edwards addition law handles edge cases particularly well and in a unified way.
\end{example}

%\begin{example}[Non twisted Edwards curves have order 4 points]
%In this example, we will show that every Edwards curve has a point of order $4$. To see that let $E$ be an arbitrary Edwards curve ($a=1$). Then the point $(1,0)$ is on that curve, since $1^2+0^2= 1+d 1^2 0^2$. We compute 
%\begin{align*}
%[4](1,0) = \\
%[2]([2](1,0))=\\
%[2]\left(\frac{1 0 +1 0}{1 +d 1 1 0 0},\frac{00-1\cdot 1}{1-d1 1 00}\right)=\\
%[2](0,-1)=\\
%\left(\frac{0(-1)+(-1)0}{1 +d 0 0 (-1)(-1)},\frac{(-1)(-1)-00}{1-d00(-1)(-1)}\right) =\\
%(0,1)
%\end{align*} 

%Now having seen that every Edwards curve has point of order $4$, we can deduce that the order of every Edwards curve must contain $4$ as a factor. This is restrictive and in fact Edwards curves are rare. 
%\end{example}
\section{Elliptic Curve Pairings} As we have seen in equation \ref{pairing-map},\sme{check reference} some groups come with the notation of a so-called pairing map, which is a non-degenerate bilinear map from two groups into another group.

In this section, we discuss \term{pairings on elliptic curves}, which form the basis of several zk-SNARKs and other zero-knowledge proof schemes. The SNARKs derived from pairings have the advantage of constant proof sizes, which is crucial to blockchains. 

We start out by defining elliptic curve pairings and discussing a simple application which bears some resemblance to more advanced SNARKs. We then introduce the pairings arising from elliptic curves and describe Miller's algorithm, which makes these pairings practical rather than just theoretically interesting.

\smelong{Elliptic curves have a few structures, like the Weil or the Tate map that qualifies as pairing.}\sme{either expand on this or delete it}  

\paragraph{Embedding Degrees}As we will see in what follows, every elliptic curves gives rise to a pairing map. However, we will also see in example XXX\sme{add reference} that not every such pairing can be efficiently computed. In order to distinguish curves with efficiently computable pairings from the rest, we need to start with an introduction to the so-called \term{embedding degree} of a curve. 

\begin{definition}{\deftitle{Embedding degree}}\label{def:embedding-degree}\\
Let $\F$ be a finite field, let $E(\F)$ be an elliptic curve over $\F$, and let $n$ be a prime number that divides the order of $E(\F)$. The embedding degree of $E(\F)$ with respect to $n$ is then the smallest integer $k$ such that $n$ divides $q^k-1$. 
\end{definition}

Fermat's little theorem (page \pageref{fermats-little-theorem} ff.)\sme{check reference} implies that every curve has at least \term{some} embedding degree $k$, since at least $k=n-1$ is always a solution to the congruency 
$\kongru{q^k}{1}{n}$. This implies that the remainder of the integer division of $q^k-1$ by $n$ is $0$.

\begin{example} To get a better intuition of the embedding degree, let's consider the elliptic curve $E_1(\F_5)$ from example \ref{E1F5}\sme{check reference}. We know from \ref{E1F5}\sme{check reference} that the order of $E_1(\F_5)$ is $9$, and, since the only prime factor of $9$ is $3$, we compute the embedding degree of $E_1(\F_5)$ with respect to $3$. 

To find the embedding degree, we have to find the smallest integer $k$ such that $3$ divides $q^k-1= 5^k-1$. We try and increment until we find a proper $k$. 

\begin{align*}
k=1 &\text{: } 5^1-1 = 4 & \text{ not divisible by } 3\\ 
k=2 &\text{: } 5^2-1 = 24 & \text{ divisible by } 3
\end{align*} 

Now we know that the embedding degree of $E_1(\F_5)$ is $2$ relative to the the prime factor $3$.
\end{example}

\begin{example}\label{ex:PJJ13-embedding-degree} Let us consider the tiny jubjub curve $\mathit{TJJ\_13}$ from example \ref{PJJ13}\sme{check reference}. We know from \ref{PJJ13}\sme{check reference} that the order of $\mathit{TJJ\_13}$ is $20$, and that the order therefore has two prime factors. A ``large'' prime factor $5$ and a small prime factor $2$. 

We start by computing the embedding degree of $\mathit{TJJ\_13}$ with respect to the large prime factor $5$. To find that embedding degree, we have to find the smallest integer $k$ such that $5$ divides $q^k-1= 13^k-1$. We try and increment until we find a proper $k$. 
\begin{align*}
k=1 &\text{: } 13^1-1 = 12 & \text{ not divisible by } 5\\ 
k=2 &\text{: } 13^2-1 = 168 & \text{ not divisible by } 5\\ 
k=3 &\text{: } 13^3-1 = 2196 & \text{ not divisible by } 5\\ 
k=4 &\text{: } 13^4-1 = 28560 & \text{ divisible by } 5
\end{align*} 
Now we know that the embedding degree of $\mathit{TJJ\_13}$ is $4$ relative to the the prime factor $5$.

In real-world applications, like on pairing-friendly elliptic curves such as BLS\_12-381, usually only the embedding degree of the large prime factor is relevant, which in the case of our tiny-jubjub curve is represented by $5$. It should be noted, however that every prime factor of a curve's order has its own notation of embedding degree despite the fact that this is mostly irrelevant in applications.

To find the embedding degree of the small prime factor $2$, we have to find the smallest integer $k$ such that $2$ divides $q^k-1= 13^k-1$. We try and increment until we find a proper $k$. 
\begin{align*}
k=1 &\text{: } 13^1-1 = 12 & \text{ divisible by } 2
\end{align*} 

Now we know that the embedding degree of $\mathit{TJJ\_13}$ is $1$ relative to the the prime factor $2$. As we have seen, different prime factors can have different embedding degrees in general.

\begin{sagecommandline}
sage: p = 13
sage: # large prime factor
sage: n = 5
sage: for k in range(1,5): # Fermat's little theorem
....:     if (p^k-1)%n == 0:
....:         break
sage: k
sage: # small prime factor
sage: n = 2
sage: for k in range(1,2): # Fermat's little theorem
....:     if (p^k-1)%n == 0:
....:         break
sage: k
\end{sagecommandline}
\end{example}

\begin{example} To give an example of a cryptographically secure real-world elliptic curve that does not have a small embedding degree, let's look at curve Secp256k1 again. We know from \ref{Secp256k1}\sme{check reference} that the order of this curve is a prime number, so we only have a single embedding degree.

To test potential embedding degrees $k$, say, in the range $1\ldots 1000$, we can invoke Sage and compute as follows:
\begin{sagecommandline}
sage: p = 115792089237316195423570985008687907853269984665640564039457584007908834671663
sage: n = 115792089237316195423570985008687907852837564279074904382605163141518161494337
sage: for k in range(1,1000):
....:     if (p^k-1)%n == 0:
....:         break
sage: k
\end{sagecommandline}
We see that Secp256k1 has at least no embedding degree $k<1000$, which renders Secp256k1 a curve that has no small embedding degree. This property will be of importance later on.
\end{example}

\paragraph{Elliptic Curves over extension fields} Suppose that $p$ is a prime number, and $\F_p$ its associated prime field. We know from equation \ref{eq:prime-extension-field}\sme{check reference} that the fields $\F_{p^m}$ are extensions of $\F_p$ in the sense that $\F_{p}$ is a subfield of $\F_{p^m}$. This implies that we can extend the affine plane that an elliptic curve is defined on by changing the base field to any extension field. To be more precise, let 
$E(\F)=\{ (x,y)\in \F \times \F \;|\; y^2 = x^3 +a\cdot x +b \}$ be an affine short Weierstraß curve, with parameters $a$ and $b$ taken from $\F$. If $\F'$ is an extension field of $\F$, then we extend the domain of the curve by defining $E(\F')$ as follows:

\begin{equation}\label{elliptic-curve-extension}
E(\F')=\{ (x,y)\in \F' \times \F' \;|\; y^2 = x^3 +a\cdot x +b \}
\end{equation}   

While we did not change the defining parameters, we consider curve points from the affine plane over the extension field now. Since $\F\subset \F'$, it can be shown that the original elliptic curve $E(\F)$ is a sub-curve of the extension curve $E(\F')$.

\begin{example}\label{ex:EF52} Consider the prime field $\F_5$ from example \ref{prime-field-F5}\sme{check reference} and the elliptic curve $E_1(\F_5)$ from example \ref{E1F5}\sme{check reference}. Since we know from XXX\sme{add reference} that $\F_{5^2}$ is an extension field of $\F_5$, we can extend the definition of $E_1(\F_5)$ to define a curve over $\F_{5^2}$:
$$
E_1(\F_{5^2}) = \{ (x,y)\in \F\times \F \;|\; y^2 = x^3 + x +1 \}
$$
Since $\F_{5^2}$ contains $25$ points, in order to compute the set $E_1(\F_{5^2})$, we have to try $25\cdot 25 = 625$ pairs, which is probably a bit too much for the average motivated reader. Instead, we invoke Sage to compute the curve for us. To do, we so choose the representation of $\F_{5^2}$ from XXX\sme{add reference}. We get:
\begin{sagecommandline}
sage: F5= GF(5)
sage: F5t.<t> = F5[] 
sage: P = F5t(t^2+2)
sage: P.is_irreducible()
sage: F5_2.<t> = GF(5^2, name='t', modulus=P)
sage: E1F5_2 = EllipticCurve(F5_2,[1,1])
sage: E1F5_2.order()
\end{sagecommandline}
The curve $E_1(\F_{5^2})$ consist of $27$ points, in contrast to curve $E_1(\F_{5})$, which consists of $9$ points. Printing the points gives the following:
\begin{multline*}
E_1(\F_{5^2}) = \{\Oinf, (0, 4), (0, 1), (3, 4), (3, 1), (4, 3), (4, 2), (2, 4), (2, 1),\\ 
(4t + 3, 3t + 4), (4t + 3, 2t + 1),  (3t + 2, t), (3t + 2, 4t),\\ 
(2t + 2, t), (2t + 2, 4t), (2t + 1, 4t + 4), (2t + 1, t + 1),\\ 
(2t + 3, 3), (2t + 3, 2), (t + 3, 2t + 4), (t + 3, 3t + 1),\\ 
(3t + 1, t + 4), (3t + 1, 4t + 1), (3t + 3, 3), (3t + 3, 2), (1, 4t)
\}
\end{multline*}
As we can see, curve $E_1(\F_5)$ sits inside curve $E(\F_{5^2})$, which is implied by $\F_5$ being a subfield of $\F_{5^2}$.
\end{example}
\paragraph{Full torsion groups} The fundamental theorem of finite cyclic groups XXX\sme{This needs to be written (in Algebra)} implies that every prime factor $n$ of a cyclic group's order defines a subgroup of the size of the prime factor. Such a subgroup is called an $n$-torsion group. We have seen many of those subgroups in the examples XXX\sme{add reference} and XXX\sme{add reference}.

When we consider elliptic curve extensions as defined in \ref{elliptic-curve-extension}\sme{check reference}, we could ask what happens to the $n$-torsion groups in the extension. One might intuitively think that their extension just parallels the extension of the curve. For example, when $E(\F_p)$ is a curve over prime field $\F_p$, with some $n$-torsion group $\G$ and when we extend the curve to $E(\F_{p^m})$, then there is a bigger $n$-torsion group such that $\G$ is a subgroup. This might make intuitive sense, as $E(\F_p)$ is a sub-curve of $E(\F_{p^m})$. 

However, the actual situation is a bit more surprising than that. To see that, let $\F_p$ be a prime field and let $E(\F_p)$ be an elliptic curve of order $r$, with embedding degree $k$ and $n$-torsion group $E(\F_p)[n]$ for the same prime factor $n$ of $r$. Then it can be shown that the $n$-torsion group $E(\F_{p^m})[n]$ of a curve extension is equal to $E(\F_p)[n]$, as long as the power $m$ is less than the embedding degree $k$ of $E(\F_p)$. 

However, for the prime power $p^m$, for any $m\geq k$, $E(\F_{p^m})[n]$ is strictly larger than $E(\F_p)[n]$ and contains $E(\F_p)[n]$ as a subgroup. We call the $n$-torsion group $E(\F_{p^k})[n]$ of the extension of $E$ over $\F_{p^k}$ the \term{full $n$-torsion group} of that elliptic curve. It can be shown that it contains $n^2$ many elements and consists of $n+1$ subgroups, one of which is $E(\F_{p})[n]$.

So, roughly speaking, when we consider \uterm{towers of curve extensions} $E(\F_{p^m})$ ordered by the prime power $m$, then the $n$-torsion group stays constant for every level $m$, that is smaller than the embedding degree, while it suddenly blossoms into a larger group on level $k$ with $n+1$ subgroups, and then stays like that for any level $m$ larger than $k$. In other words, once the extension field is big enough to find one more point of order $n$ (that is not defined over the base field), then we actually find all of the points in the full torsion group.

\begin{example} Consider curve $E_1(\F_5)$ again. We know that it contains a $3$-torsion group and that the embedding degree of $3$ is $2$. From this we can deduce that we can find the full $3$-torsion group $E_1[3]$ in the curve extension $E_1(\F_{5^2})$, the latter of which we computed in example \ref{ex:EF52} \sme{check reference}. 

Since that curve is small, in order to find the full $3$-torsion, we can loop through all elements of $E_1(\F_{5^2})$ and check check the defining equation $[3]P= \Oinf$. Invoking Sage, we compute as follows:
\begin{sagecommandline}
sage: INF = E1F5_2(0) # Point at infinity
sage: L_E1_3 = []
sage: for p in E1F5_2:
....:     if 3*p == INF:
....:         L_E1_3.append(p)
sage: E1_3 = Set(L_E1_3) # Full 3-torsion set
\end{sagecommandline}
We get the following result:
$$
E_1[3] = \{\Oinf,(1,t),(1,4t),(2,1),(2,4), (2t + 1,t + 1),
 (2t + 1, 4t + 4), (3t + 1,t + 4), (3t + 1, 4t + 1) \}
$$
\end{example}
\begin{example}\label{ex:PJJ13-full-torsion} Consider the tiny jubjub curve from example \ref{PJJ13}\sme{check reference}. We know from example \ref{ex:PJJ13-embedding-degree}\sme{check reference} that it contains a $5$-torsion group and that the embedding degree of $5$ is $4$. This implies that we can find the full $5$-torsion group $\mathit{TJJ\_13}[5]$ in the curve extension $\mathit{TJJ\_13}(\F_{13^4})$. 

To compute the full torsion, first observe that, since $\F_{13^4}$ contains $28561$ elements, computing $\mathit{TJJ\_13}(\F_{13^4})$ means checking $28561^2=815730721$ elements. From each of these curve points $P$, we then have to check the equation $[5]P=\Oinf$. Doing this for $815730721$ is a bit too slow even on a computer.

Fortunately, Sage has a way to loop through points of a given order efficiently. The following Sage code  provides a way to compute the full torsion group:
\begin{sagecommandline}
sage: # define the extension field
sage: F13= GF(13) # prime field
sage: F13t.<t> = F13[] # polynomials over t
sage: P = F13t(t^4+2) # irreducible polynomial of degree 4
sage: P.is_irreducible()
sage: F13_4.<t> = GF(13^4, name='t', modulus=P) # F_{13^4}
sage: TJJF13_4 = EllipticCurve(F13_4,[8,8]) # tiny jubjub extension
sage: # compute the full 5-torsion
sage: L_TJJF13_4_5 = []
sage: INF = TJJF13_4(0)
sage: for P in INF.division_points(5): # [5]P == INF
....:     L_TJJF13_4_5.append(P)
sage: len(L_TJJF13_4_5)
sage: TJJF13_4_5 = Set(L_TJJF13_4_5)
\end{sagecommandline}
As expected, we get a group that contains $5^2=25$ elements. As it's rather tedious to write this group down, and as we don't need it in what follows, we forgo doing this. To see that the embedding degree $4$ is actually the smallest prime power to find the full $5$-torsion group, let's compute the $5$-torsion group over of the tiny-jubjub curve of the extension field $\F_{13^3}$. We get the following:
\begin{sagecommandline}
sage: # define the extension field
sage: P = F13t(t^3+2) # irreducible polynomial of degree 3
sage: P.is_irreducible()
sage: F13_3.<t> = GF(13^3, name='t', modulus=P) # F_{13^3}
sage: TJJF13_3 = EllipticCurve(F13_3,[8,8]) # tiny jubjub extension
sage: # compute the 5-torsion
sage: L_TJJF13_3_5 = []
sage: INF = TJJF13_3(0)
sage: for P in INF.division_points(5): # [5]P == INF
....:     L_TJJF13_3_5.append(P)
sage: len(L_TJJF13_3_5)
sage: TJJF13_3_5 = Set(L_TJJF13_3_5) # full $5$-torsion
\end{sagecommandline}

As we can see, the $5$-torsion group of tiny-jubjub over $\F_{13^3}$ is equal to the $5$-torsion group of tiny-jubjub over $\F_{13}$ itself. 
\end{example}

\begin{example} 
% https://www.sikoba.com/docs/SKOR_SV_Pairing_Based_Crypto.pdf
Let's look at the curve Secp256k1. We know from example \ref{Secp256k1}\sme{check reference} that the curve is of some prime order $r$. Because of this, the only $n$-torsion group to consider is the curve itself, so the curve group is the $r$-torsion. 

However, in order to find the full $r$-torsion of Secp256k1, we need to compute the embedding degree $k$. And as we have seen in XXX\sme{add reference} it is at least not small. However, we know from Fermat's little theorem (page \pageref{fermats-little-theorem} ff.) that a finite embedding degree must exist. It can be shown that it is given by the following 256-bit number:
$$
k = \scriptstyle 192986815395526992372618308347813175472927379845817397100860523586360249056 
$$
 This means that the embedding degree is \smelong{huge}\sme{is ``huge'' a technical term?}, which implies that the field extension $\F_{p^k}$ is huge too. To understand how big $\F_{p^k}$ is, recall that an element of $\F_{p^m}$ can be represented as a string $[x_0,\ldots,x_m]$ of $m$ elements, each containing a number from the prime field $\F_p$. Now, in the case of Secp256k1, such a representation has $k$-many entries, each of them $256$ bits in size. So, without any optimizations, representing such an element would need $k\cdot 256$ bits, which is too much to be represented in the observable universe.
\end{example}

\paragraph{Torsion subgroups}As we have stated above, any full $n$-torsion group contains $n+1$ cyclic subgroups, two of which are of particular interest in pairing-based elliptic curve cryptography. To characterize these groups, we need to consider the so-called \term{Frobenius endomorphism} of an elliptic curve $E(\F)$ over some finite field $\F$ of characteristic $p$:
\begin{equation}\label{eq:frobenius-enomorphism}
\pi : E(\F) \to E(\F): \;\; 
\begin{array}{lcl}
(x,y)       &\mapsto & (x^p,y^p)\\
\Oinf &\mapsto & \Oinf
\end{array} 
\end{equation}
It can be shown that $\pi$ maps curve points to curve points. The first thing to note is that, in case  $\F$ is a prime field, the Frobenius endomorphism acts trivially, since $(x^p,y^p) = (x,y)$ on prime fields due to Fermat's little theorem (page \pageref{fermats-little-theorem} ff.)\sme{check reference}. This means that the Frobenius map is more interesting over prime field extensions.

With the Frobenius map at hand, we can characterize two important subgroups of the full $n$-torsion. The first subgroup is the $n$-torsion group that already exists in the curve over the base field. In pairing-based cryptography, this group is usually written as $\G_1$, assuming that the prime factor $n$ in the definition is implicitly given. Since we know that the Frobenius map acts trivially on curves over the prime field, we can define $\G_1$ as follows:
\begin{equation}
\G_1[n] := \{(x,y)\in E[n]\;| \pi(x,y) = (x,y)\;\}
\end{equation}
In more mathematical terms, this definition means that $\G_1$ is the \term{Eigenspace} of the Frobenius map with respect to the \term{Eigenvalue} $1$.\sme{S: either add more explanation or move to a footnote}

It can be shown that there is another subgroup of the full $n$-torsion group that can be characterized by the Frobenius map. In the context of so-called \uterm{type $3$ pairing-based cryptography}, this subgroup is usually called $\G_2$ and it is defined as follows:
\begin{equation}
\G_2[n] := \{(x,y)\in E[n]\;| \pi(x,y) = [p](x,y)\;\}
\end{equation}

In mathematical terms, $\G_2$ is the \term{Eigenspace} of the Frobenius map with respect to the \term{Eigenvalue} $p$.

\begin{notation}
If the prime factor $n$ of a curve's order is clear from the context, we sometimes simply write $\G_1$ and $\G_2$ to mean $\G_1[n]$ and $\G_2[n]$, respectively.
\end{notation}

It should be noted, however that other definitions of $\G_2$ also exists in the literature.\sme{add references?} However, in the context of pairing-based cryptography, this is the most common one. It is particularly useful because we can define hash functions that map into $\G_2$, which is not possible for all subgroups of the full $n$-torsion.

\begin{example} Consider the curve $E_1(\F_5)$ from example \ref{E1F5} again. As we have seen, this curve has the embedding degree $k=2$, and a full $3$-torsion group is given as follows:
\begin{equation}
\begin{split}
E_1[3] = \{\Oinf,(2,1),(2,4), (1,t), (1,4t),(2t + 1,t + 1),\\ (2t + 1, 4t + 4),
(3t + 1,t + 4), (3t + 1, 4t + 1) \}
\end{split}
\end{equation}

According to the general theory, $E_1[3]$ contains $4$ subgroups, and we can characterize the subgroups $\G_1$ and $\G_2$ using the Frobenius endomorphism. Unfortunately, at the time of writing, Sage does not have a predefined Frobenius endomorphism for elliptic curves, so we have to use the Frobenius endomorphism of the underlying field as a temporary workaround. We compute as follows:
\begin{sagecommandline}
sage: L_G1 = []
sage: for P in E1_3: 
....:     PiP = E1F5_2([a.frobenius() for a in P]) # pi(P)
....:     if P == PiP:
....:         L_G1.append(P)
sage: G1 = Set(L_G1)
\end{sagecommandline}
As expected, the group $\G_1=\{\Oinf, (2,4), (2,1)\}$ is identical to the $3$-torsion group of the (unextended) curve over the prime field $E_1(\F_5)$. We can use almost the same algorithm to compute the group $\G_2$ and get the following:
\begin{sagecommandline}
sage: L_G2 = []
sage: for P in E1_3: 
....:     PiP = E1F5_2([a.frobenius() for a in P]) # pi(P)
....:     pP = 5*P # [5]P
....:     if pP == PiP:
....:         L_G2.append(P)
sage: G2 = Set(L_G2)
\end{sagecommandline}

Thus, we have computed the the second subgroup of the full $3$-torsion group of curve $E_1$ as the set $\G_2 = \{\Oinf, (1,t), (1,4t)\}$. 
\end{example}

\begin{example}
Consider the tiny-jubjub curve $\mathit{TJJ\_13}$ from example \ref{PJJ13}\sme{check reference}. In example \ref{ex:PJJ13-full-torsion},\sme{check reference} we computed its full $5$ torsion, which is a group that has $6$ subgroups. We compute $G1$ using Sage as follows:
\begin{sagecommandline}
sage: L_TJJ_G1 = []
sage: for P in TJJF13_4_5: 
....:     PiP = TJJF13_4([a.frobenius() for a in P]) # pi(P)
....:     if P == PiP:
....:         L_TJJ_G1.append(P)
sage: TJJ_G1 = Set(L_TJJ_G1)
\end{sagecommandline}
We get $\G1= \{\Oinf, (7,2), (8,8), (8,5), (7,11)\}$
\begin{sagecommandline}
sage: L_TJJ_G1 = []
sage: for P in TJJF13_4_5: 
....:     PiP = TJJF13_4([a.frobenius() for a in P]) # pi(P)
....:     pP = 13*P # [5]P
....:     if pP == PiP:
....:         L_TJJ_G1.append(P)
sage: TJJ_G1 = Set(L_TJJ_G1)
\end{sagecommandline}
$\G_2 = \{\Oinf, (9t^2 + 7,t^3 + 11t,), (9t^2 + 7, 12t^3 + 2t), (4t^2 + 7,5t^3 + 10t),(4t^2 + 7,8t^3 + 3t)\}$
\end{example}

\begin{example}Consider Bitcoin's curve Secp256k1 again. Since the group $\G_1$ is identical to the torsion group of the unextended curve, and since Secp256k1 has prime order, we know that, in this case, $\G_1$ is identical to Secp256k1. It is however, infeasible not to compute not only $\G_2$ itself, but to even compute an average element of $\G_2$, as elements need too much storage to be representable in this universe.
\end{example}

\paragraph{The Weil pairing} In this part, we consider a pairing function defined on
the subgroups $\G_1[r]$ and $\G_2[r]$ of the full $r$-torsion $E[r]$ of a short Weierstraß elliptic curve. To be more precise, let $E(\F_p)$ be an elliptic curve of embedding degree $k$ such that $r$ is a prime factor of its order. Then the \term{Weil pairing} is a bilinear, non-degenerate map:
\begin{equation}\label{eq:weil-pairing}
e(\cdot,\cdot) : \G_1[r] \times \G_2[r] \to \F_{p^k}\; ;\; 
(P,Q)\mapsto (-1)^r \cdot \frac{f_{r,P}(Q)}{f_{r,Q}(P)}
\end{equation} 

The extension field elements $f_{r,P}(Q), f_{r,Q}(P)\in \F_{p^k}$ are computed by \term{Miller's algorithm}:\final{check floating of algorithm}
\begin{algorithm}\caption{Miller's algorithm for short Weierstraß curves $y^2 = x^3 +ax +b$}
% https://www.math.u-bordeaux.fr/~damienrobert/csi2018/pairings.pdf
\begin{algorithmic}[0]
\Require $r>3$, $P \in E[r]$, $Q\in E[r]$ and
\State $b_0,\ldots, b_t\in \{0,1\}$ with $r= b_0\cdot 2^0 + b_1\cdot 2^1 + \ldots + b_t\cdot 2^t$ and $b_t=1$
\Procedure{Miller's Algorithm}{$P,Q$}
\If{$P = \Oinf$ or $Q = \Oinf$ or $P = Q$}
	\State \textbf{return} $f_{r,P}(Q) \gets (-1)^r$
\EndIf
\State $(x_T,y_T) \gets (x_P,y_P)$
\State $f_1\gets 1$
\State $f_2\gets 1$
\For{$j\gets t-1,\ldots, 0$}
	\State $m \gets \frac{3\cdot x_T^2+a}{2\cdot y_T}$	
    \State $f_1 \gets f_1^2\cdot (y_Q - y_T - m\cdot(x_Q-x_T))$
	\State $f_2 \gets f_2^2\cdot (x_Q + 2x_T -m^2)$
	\State $x_{2T} \gets m^2 - 2 x_T$
	\State $y_{2T} \gets -y_T - m\cdot (x_{2T}-x_T)$
	\State $(x_T,y_T)\gets (x_{2T},y_{2T})$ 
	\If{$b_j = 1$}
		\State $m \gets \frac{y_T -y_P}{x_T - x_P}$
		\State $f_1 \gets f_1\cdot (y_Q -y_T -m\cdot (x_Q - x_T))$
		\State $f_2 \gets f_2\cdot (x_Q + (x_P+x_T) - m^2)$
		\State $x_{T+P} \gets m^2 -x_T -x_P$
		\State $y_{T+P}\gets -y_T - m\cdot (x_{T+P}-x_T)$
		\State $(x_T,y_T)\gets (x_{T+P},y_{T+P})$
	\EndIf
\EndFor
\State $f_1 \gets f_1\cdot (x_Q - x_T)$
\State \textbf{return} $f_{r,P}(Q) \gets \frac{f_1}{f_2}$
\EndProcedure
\end{algorithmic}
\end{algorithm}

Understanding how the algorithm works in detail requires the concept of \term{divisors}, which is outside of the scope this book. The interested reader might look at XXX.\sme{add references}

In real-world applications of pairing-friendly elliptic curves, the embedding degree is usually a small number like $2$, $4$, $6$ or $12$, and the number $r$ is the largest prime factor of the curve's order.

\begin{example}Consider curve $E_1(\F_5)$ from example \ref{E1F5}\sme{check reference}. Since the only prime factor of the group's order is $3$, we cannot compute the Weil pairing on this group using our definition of Miller's algorithm. In fact, since $\G_1$ is of order $3$, executing the if statement on line XXX\sme{add reference} will lead to a ``division by zero'' error in the computation of the slope $m$.
\end{example}

\begin{example} Consider the tiny-jubjub curve $\mathit{TJJ\_13}(\F_{13})$ from example \ref{PJJ13}\sme{check reference} again. We want to instantiate the general definition of the Weil pairing for this example. To do so, recall that, as we have see in example \ref{ex:PJJ13-embedding-degree}\sme{check reference}, its embedding degree is $4$, and that we have the following type-3 pairing groups (where $\G_1$ and $\G_2$ are subgroups of the full $5$-torsion found in the curve $\mathit{TJJ\_13}(\F_{13^4})$):
\begin{align*}
\G_1 & = \{\Oinf, (7,2), (8,8), (8,5), (7,11)\}\\
\G_2 & = \{\Oinf, (9t^2 + 7,t^3 + 11t), (9t^2 + 7, 12t^3 + 2t), 
(4t^2 + 7, 5t^3 + 10t), (4t^2 + 7, 8t^33 + 3t)\}
\end{align*}

The type-3 Weil pairing is a map $e(\cdot,\cdot): \G_1 \times \G_2 \to \F_{13^4}$. From the first if-statement in Miller's algorithm, we can deduce that 
$e(\Oinf,Q)=1$ as well as $e(P,\Oinf)=1$ for all arguments $P\in\G_1$ and $Q\in \G_2$. In order to compute a non-trivial Weil pairing, we choose the arguments 
$P=(7,2)$ and $Q=(9t^2 + 7, 12t^3 + 2t)$. 

To compute the pairing $e((7,2),(9t^2 + 7, 12t^3 + 2t))$, we have to compute the extension field elements $f_{5,P}(Q)$ and $f_{5,Q}(P)$ by applying Miller's algorithm. Do do so, observe that we have $5 = 1\cdot 2^0 + 0 \cdot 2^1 + 1\cdot 2^2$, so we get $t=2$ as well as $b_0=1$, $b_1=0$ and $b_2=1$. The loop therefore needs to be executed two times. 

Computing $f_{5,P}(Q)$, we initiate $(x_T,y_T) = (7,2)$ as well as $f_1=1$ and $f_2=1$. Then we proceed as follows:
$$
\begin{array}{l|l|lllllll}
j & b_j & m & f_1 & f_2 & x_{2T} & y_{2T} & x_{T+P} & y_{T+P}\\
\hline\\
1 & \cdot & 
\end{array}
$$

\begin{align*}
m & = \frac{3\cdot x_T^2 +a}{2\cdot y_T}\\
  & = \frac{3\cdot 2^2 +1}{2\cdot 4}
    = \frac{3}{3}\\
  & = 1\\
  \\
f_1 & = f_1^2\cdot (y_Q - y_T - m\cdot(x_Q-x_T)) \\
    & = 1^2\cdot (t - 4 - 1\cdot(1-2))
      = t-4+1\\
    & = t+2\\
    \\
f_2 & =  f_2^2\cdot (x_Q + 2x_T -m^2)\\
    & = 1^2\cdot (1 + 2\cdot 2 -1^2)
      = (1 + 4 -1)\\
    & = 4\\
\\  
x_{2T} & =  m^2 - 2 x_T\\
       & =  1^2 - 2\cdot 2
         = -3\\
       & = 2 \\  
\\
y_{2T} & = -y_T - m\cdot (x_{2T}-x_T)\\
       & = -4 - 1\cdot (2-2)
         = -4\\
       & = 1
\end{align*}

We update $(x_T,y_T) =(2,1)$ and, since $b_0=1$, we have to execute the if statement on line XXX\sme{add reference}\sme{should all lines of all algorithms be numbered?} in the \hilight{for} loop. However, since we only loop a single time, we don't need to compute $y_{T+P}$, since we only need the updated $x_T$ in the final step. We get:
\begin{align*}
m & =  \frac{y_T -y_P}{x_T - x_P}\\
  & =  \frac{1 -4}{2 - x_P}\\
\\
f_1 & = f_1\cdot (y_Q -y_T -m\cdot (x_Q - x_T))\\
\\
f_2 & = f_2\cdot (x_Q + (x_P+x_T) - m^2)\\
\\
x_{T+P} & = m^2 -x_T -x_P\\
\end{align*}
\end{example}

% http://www.pdmi.ras.ru/~lowdimma/BSD/Silverman-Arithmetic_of_EC.pdf
% p. 396ff

\section{Hashing to Curves} Elliptic curve cryptography frequently requires the ability to hash data onto elliptic curves. If the order of the curve is not a prime number, hashing to prime number subgroups is also of importance. In the context of pairing-friendly curves, it is also sometimes necessary to hash specifically onto the group $\G_1$ or $\G_2$.

As we have seen in section \ref{sec:hashing-to-groups}\sme{check reference}, many general methods are known for hashing into groups in general, and finite cyclic groups in particular. As elliptic groups are cyclic, those methods can be utilized in this case, too. However, in what follows we want to describe some methods specific to elliptic curves that are frequently used in real-world applications. 

\paragraph{Try-and-increment hash functions}
One of the most straight-forward ways of hashing a bitstring onto an elliptic curve point in a  secure way is to use a cryptographic hash function together with one of the methods we described in section \ref{sec:hashing-to-groups}\sme{check reference} to hash to the modular arithmetics base field of the curve. Ideally, the hash function generates an image that is at least one bit longer than the bit representation of the base field modulus.

The image in the base field can then be interpreted as the $x$ coordinate of the curve point, and the two possible $y$ coordinates are derived from the curve equation, while one of the bits that exceeded the modulus determines which of the two $y$ coordinates to choose.

Such an approach would be deterministic and easy to implement, and it would conserve the cryptographic properties of the original hash function. However, not all $x$ coordinates generated in such a way will result in quadratic residues when inserted into the defining equation. It follows that not all field elements give rise to actual curve points. In fact,
% https://www.cs.umd.edu/users/gasarch/TOPICS/res/burgess.pdf
on a prime field, only half of the field elements are quadratic residues. Hence, assuming an even distribution of the hash values in the field, this method would fail to generate a curve point in about half of the attempts. 

One way to account for this problem is the so-called \term{try-and-increment} method. Its basic assumption is that, when hashing different values, the result will eventually lead to a valid curve point. 

Therefore, instead of simply hashing a string $s$ to the field, we hash the concatenation of $s$ with additional bytes to the field instead. In other words, we use a try-and-increment hash as described in \ref{alg_try_and_increment}\sme{check reference}. If the first try of hashing to the field does not result in a valid curve point, the counter is incremented, and the hashing is repeated again. This is done until a valid curve point is found.\final{check if the algorithm is floated properly}

\begin{algorithm}\caption{Hash-to-$E(\F_r)$}
\begin{algorithmic}[0]
\Require $r \in \Z$ with $r.nbits()=k$ and $s\in\{0,1\}^*$
\Require Curve equation $y^2 = x^3 + ax +b$ over $\F_r$
\Procedure{Try-and-Increment}{$r,k,s$}
\State $c \gets 0$
\Repeat
\State $s' \gets s||c\_bits()$
\State $z \gets H(s')_0\cdot 2^0 + H(s')_1\cdot 2^1 + \ldots + H(s')_{k}\cdot 2^{k}$
\State $x\gets z^3 + a\cdot z + b$
\State $c\gets c+1$
\Until{$z<r$ and $\Zmod{x^{\frac{r-1}{2}}}{r}=1$ }
\If {$H(s')_{k+1} == 0$}
\State $y \gets \sqrt{x}$ \#(root in $\F_r$)
\Else 
\State $y \gets r-\sqrt{x}$ \#(root in $\F_r$)
\EndIf
\State \textbf{return} $(x,y)$
\EndProcedure
\Ensure $(x,y)\in E(\F_r)$
\end{algorithmic}
\end{algorithm}

This method has a number of advantages: It is relatively easy to implement in code, and it maintains the cryptographic properties of the original hash function. However, it is not guaranteed to find a valid curve point, as there is a chance that all possible values in the chosen size of the counter will fail to generate a quadratic residue. Fortunately, it is possible to make the probability for this arbitrarily small by choosing large enough counters and relying on the (approximate) uniformity of the hash-to-field function. 

%One might think that another disadvantage of this method in the context of SNARKs is that it can not be implemented as a circuit effectively. This however, is not fully true, as a circuit/r1cs only needs to enforce the correctness of the computation. Hence for the circuit it is enough to check the hash of the string and the correct counter. It does not need to find that counter.

%Considering certain subgroups of the elliptic curve, the usefulness of this methods depends highly on the actual situation. For example if a hash to the $n$-torsion subgroup $\mathbb{G}_1$ is desired, there are two possibilities: 

If the curve is not of prime order, the result will be a general curve point that might not be in the ``large'' prime-order subgroup. In this case, a \term{cofactor clearing} step is then necessary to project the curve point onto the subgroup. This is done by scalar multiplication with the cofactor of prime order with respect to the curves order.

\begin{example} Consider the tiny jubjub curve from example \ref{PJJ13}\sme{check reference}. We want to construct a try-and-increment hash function that hashes a binary string $s$ of arbitrary length onto the large prime-order subgroup of size $5$. 

Since the curve, as well as our targeted subgroup, is defined over the field $\F_{13}$, and the binary representation of $13$ is $13.bits()=1101$, we apply SHA256 from Sage's hashlib library on the concatenation $s||c$ for some binary counter string, and use the first $4$ bits of the image to try to hash into $\F_{13}$. In case we are able to hash to a value $z$ such that $z^3 +8\cdot z + 8$ is a quadratic residue in $\F_{13}$, we use the $5$-th bit to decide which of the two possible roots of $z^3 + 8\cdot z + 8$ we will choose as the $y$ coordinate. The result is a curve point different from the point at infinity. To project it to a point of $\G_1$, we multiply it with the cofactor $4$. If the result is still not the point at infinity, it is the result of the hash.

To make this concrete, let $s='10011001111010110100000111'$ be our binary string that we want to hash onto $\G_1$. We use a $4$-bit binary counter starting at zero, that is, we choose $c=0000$. Invoking Sage, we define the try-hash function as follows:
\begin{sagecommandline}
sage: import hashlib
sage: def try_hash(s,c):
....:     s_1 = s+c
....:     hasher = hashlib.sha256(s_1.encode('utf-8'))
....:     digest = hasher.hexdigest()
....:     d = Integer(digest,base=16)
....:     sign = d.str(2)[-5:-4]
....:     d = d.str(2)[-4:]
....:     z = Integer(d,base=2)
....:     return (z,sign)
sage: try_hash('10011001111010110100000111','0000')
\end{sagecommandline}

As we can see, our first attempt to hash into $\F_{13}$ was not successful, as $15$ is not a number in $\F_{13}$, so we increment the binary counter by $1$ and try again: 
\begin{sagecommandline}
sage: try_hash('10011001111010110100000111','0001')
\end{sagecommandline}

With this try, we found a hash into $\F_{13}$. However, this point is not guaranteed to define a curve point. To see that, we insert $z=3$ into the right side of the Weierstraß equation of the tiny.jubjub curve, and compute $3^3 + 8*3 + 8 = 7$. However, $7$ is not a quadratic residue in $\F_{13}$, since $7^{\frac{13-1}{2}}=7^6=12=-1$. This means that $3$ is a not a suitable point, and we have to increment the counter two more times: 
\begin{sagecommandline}
sage: try_hash('10011001111010110100000111','0010')
sage: try_hash('10011001111010110100000111','0011')
\end{sagecommandline}
Since $6^3 + 8\cdot 6 + 8 = 12$, and we have $\sqrt{12}\in\{5, 8\}$, we finally found the valid $x$ coordinate $x=6$ for the curve point hash. Now, since the sign bit of this hash is $1$, we choose the larger root $y=8$ as the $y$ coordinate and get the following hash which is a valid curve point point on the tiny jubjub curve:
$$
H('10011001111010110100000111') = (6,8)
$$

In order to project this onto the ``large'' prime-order subgroup, we have to do cofactor clearing, that is, we have to multiply the point with the cofactor $4$. We get the following:
$$
[4](6,8) = \Oinf
$$ 

This means that the hash value is still not right. We therefore have to increment the counter two more times again, until we finally find a correct hash to $\G_1$:
\begin{sagecommandline}
sage: try_hash('10011001111010110100000111','0100')
sage: try_hash('10011001111010110100000111','0101')
\end{sagecommandline}

Since $12^3 + 8\cdot 12 + 8 = 12$, and we have $\sqrt{12}\in\{5, 8\}$, we found another valid $x$ coordinate $x=12$ for the curve point hash. Since the sign bit of this hash is $0$, we choose the smaller root $y=5$ as the $y$ coordinate, and get the following hash, which is a valid curve point point on the tiny jubjub curve:
$$
H('10011001111010110100000111') = (12,5)
$$
In order to project this onto the ``large'' prime-order subgroup we have to do cofactor clearing,\sme{again?} that is, we have to multiply the point with the cofactor $4$. We get the following:
$$
[4](12,5) = (8,5)
$$
So, hashing the binary string $'10011001111010110100000111'$ onto $\G_1$ gives the hash value $(8,5)$ as a result. 
\end{example}

\section{Constructing elliptic curves} Cryptographically secure elliptic curves like Secp256k1 from example \ref{Secp256k1}\sme{check reference} have been known for quite some time. Given the latest advancements of cryptography, however, it is often necessary to design and instantiate elliptic curves from scratch that satisfy certain very specific properties. 

For example, in the context of SNARK development, it was necessary to design a curve that can be efficiently implemented inside of a so-called \uterm{circuit} in order to enable primitives like elliptic curve \uterm{signature schemes} in a zero-knowledge proof. Such a curve is given by the Baby-jubjub curve\sme{this was called ``pen-jubjub''. Baby-JubJub is the "real word curve" from zCash. tiny-JubJub is our very small example curve with similar properties} (\ref{PJJ13}\sme{check reference} and we have paralleled its definition by introducing the tiny-jubjub curve from example XX\sme{add reference}. \smelong{Clarify difference between baby- pen- and tiny-jubjub.} As we have seen, those curves are instances of so-called twisted Edwards curves, and as such have easy to implement addition laws that work without branching. However, we introduced the tiny-jubjub curve out of thin air, as we just gave the curve parameters without explaining how we came up with them.

Another requirement in the context of many so-called \term{pairing-based zero-knowledge proofing systems} is the existence of a suitable, pairing-friendly curve with a specified security level and a low embedding degree as defined in \ref{def:embedding-degree}\sme{check reference}. Famous examples are the BLS\_12 and the NMT curves.\sme{add references}

The major goal of this section is to explain the most important method of designing elliptic curves with predefined properties from scratch, called the \term{complex multiplication method}. We will apply this method in section XXX\sme{add reference} to synthesize a particular BLS\_6 curve, which is one of the most insecure curves, but it will serve as the main curve to build our pen-and-paper SNARKs on. As we will see, this curve has a ``large'' prime factor subgroup of order $13$, which implies that we can use our tiny-jubjub curve to implement certain elliptic curve cryptographic primitives in circuits over that BLS\_6 curve. 
 
Before we introduce the complex multiplication method, we have to explain a few properties of elliptic curves that are of key importance in understanding the complex multiplication method. 

\paragraph{The Trace of Frobenius} To understand the complex multiplication method of elliptic curves, we have to define the so-called \term{trace} of an elliptic curve first.

We know from XXX\sme{reference text to be written in Algebra} that elliptic curves over finite fields are products of cyclic groups of finite order. Therefore, an interesting question is whether it is possible to estimate the number of elements that this curve contains. Since an affine short Weierstraß curve consists of pairs $(x,y)$ of elements from a finite field $\F_q$ plus the point at infinity, and the field $\F_q$ contains $q$ elements, the number of curve points cannot be arbitrarily large, since it can contain at most $q^2+1$ many elements. 

There is however, a more precise estimation, usually called the \term{Hasse bound}. To understand it, let $E(\F_q)$ be an affine short Weierstraß curve over a finite field $\F_w$ of order $q$, and let $|E(\F_q)|$ be the order of the curve. Then there is an integer $t\in \Z$, called the \term{trace of Frobenius} of the curve, such that $|t| \leq 2\sqrt{q}$ and the following equation holds:
\begin{equation}\label{hasse-bound}
|E(\F)| = q +1 -t
\end{equation}

A positive trace, therefore, implies that the curve contains less points than the underlying field, whereas a negative trace means that the curve contains more points. However, the estimation $|t| \leq 2\sqrt{q}$ implies that the difference is not very large in either direction, and the number of elements in an elliptic curve is always approximately in the same order of magnitude as the size of the curve's base field.

\begin{example}\label{ex:E1F5-frobenius} Consider the elliptic curve $E_1(\F_5)$ from example \ref{E1F5}\sme{check reference}. We know that it contains $9$ curve points. Since the order of $\F_5$ is $5$, we compute the trace of $E_1(\F)$ to be $t=-3$, since the Hasse bound is given by the following equation:
$$
9 = 5 + 1 - (-3)
$$
Indeed, we have $|t| \leq 2\sqrt{q}$, since $\sqrt{5}> 2.23$ and 
$|-3|= 3 \leq 4.46 = 2\cdot 2.23< 2\cdot \sqrt{5}$.
\end{example}

\begin{example}\label{ex:PJJ13-frobenius} To compute the trace of the tiny-jubjub curve, recall from example \ref{ex:PJJ13-cofactor-clearing}\sme{check reference} that the order of $\mathit{PJJ\_13}$ is $20$. Since the order of $\F_{13}$ is $13$, we can therefore use the Hasse bound and compute the trace as $t=-6$:
\begin{equation}
20 = 13 + 1 - (-6)
\end{equation}

Again, we have $|t| \leq 2\sqrt{q}$, since $\sqrt{13}> 3.60$ and 
$|-6|= 6 \leq 7.20 = 2\cdot 3.60< 2\cdot \sqrt{13}$.
\end{example}

\begin{example}\label{ex:Secp256k1-trace}To compute the trace of Secp256k1, recall from example \ref{Secp256k1}\sme{check reference} that this curve is defined over a prime field with $p$ elements, and that the order of that group is given by $r$:  
\begin{align*}
p &= \scriptstyle 115792089237316195423570985008687907853269984665640564039457584007908834671663\\
r &= \scriptstyle 115792089237316195423570985008687907852837564279074904382605163141518161494337
\end{align*}

Using the Hesse bound $r = p + 1 -t$, we therefore compute $t= p+1 -r$, which gives the trace of curve Secp256k1 as follows:
$$
t = \scriptstyle 432420386565659656852420866390673177327
$$

As we can see, Secp256k1 contains less elements than its underlying field. However,  the difference is tiny, since the order of Secp256k1 is in the same order of magnitude as the order of the underlying field. Compared to $p$ and $r$, $t$ is tiny.

\begin{sagecommandline}
sage: p = 115792089237316195423570985008687907853269984665640564039457584007908834671663
sage: r = 115792089237316195423570985008687907852837564279074904382605163141518161494337
sage: t = p + 1 -r
sage: t.nbits()
sage: abs(RR(t)) <= 2*sqrt(RR(p))
\end{sagecommandline}
\end{example} 

\paragraph{The $j$-invariant} As we have seen in XXX\sme{add reference}, two elliptic curves $E_1(\F)$ defined by $y^2 = x^3 + ax +b$ and $E_2(\F)$ defined by $y^2 + a'x + b'$ are strictly isomorphic if and only if there is a quadratic residue $d\in \F$ such that $a' = a d^2$ and $b' = b d^3$. 

There is, however, a more general way to classify elliptic curves over finite fields $\F_q$, based on the so-called \term{$j$-invariant} of an elliptic curve with $j(E(\F_q))\in\F_q$, as defined below:
\begin{equation}\label{eq:j-invariant1}
j(E(\F_q)) = \left(\Zmod{1728}{q}\right) \frac{4\cdot a^3}{4\cdot a^3+ (\Zmod{27}{q})\cdot b^2}
\end{equation}

A detailed description of the $j$-invariant is beyond the scope of this book. For our present purposes, it is sufficient to note that two elliptic curves $E_1(\F)$ and $E_2(\F')$ are isomorphic over the \uterm{algebraic closures} of $\F$ and $\F'$, if and only if $\overline{\F}=\overline{\F'}$ and $j(E_1)=j(E_2)$.

So, the $j$-invariant is an important tool to classify elliptic curves and it is needed in the complex multiplication method to decide on an actual curve instantiation that implements abstractly chosen properties.

\begin{example} Consider the elliptic curve $E_1(\F_5)$ from example \ref{E1F5}\sme{check reference}. We compute its $j$-invariant as follows:
\begin{align*}
j(E_1(\F_5)) &= \left(\Zmod{1728}{5}\right) \frac{4\cdot 1^3}{4\cdot 1^3+ (\Zmod{27}{5})\cdot 1^2}\\
             &= 3 \frac{4}{4+ 2}\\
             &= 3\cdot 4
             & = 2
\end{align*}
\end{example}
\begin{example} Consider the elliptic curve $\mathit{PJJ\_13}$ from example \ref{PJJ13}\sme{check reference}. We compute its $j$-invariant as follows:
\begin{align*}
j(E_1(\F_5)) &= \left(\Zmod{1728}{13}\right) \frac{4\cdot 8^3}{4\cdot 8^3+ (\Zmod{27}{13})\cdot 8^2}\\
             &= 12\cdot \frac{4\cdot 5}{4\cdot 5+ 1\cdot 12}\\
             &= 12\cdot \frac{7}{7+ 12}\\
             &= 12\cdot 7\cdot 6^{-1}\\
             &= 12\cdot 7\cdot 11\\
             &0 1 
\end{align*}
\end{example}
\begin{example}Consider Secp256k1 from example Secp256k1\sme{check reference}. We compute its $j$-invariant using Sage: 
\begin{sagecommandline}
sage: p = 115792089237316195423570985008687907853269984665640564039457584007908834671663
sage: F = GF(p)
sage: j = F(1728)*((F(4)*F(0)^3)/(F(4)*F(0)^3+F(27)*F(7)^2))
sage: j == F(0)
\end{sagecommandline}
\end{example} 
\paragraph{The Complex Multiplication Method}\label{complex-multiplication-method}
As we have seen in the previous sections, elliptic curves have various defining properties, like their order, their prime factors, the embedding degree, or the cardinality (number of elements) of the base field. The \term{complex multiplication} (CM) method provides a practical way of constructing elliptic curves with pre-defined restrictions on the order and the base field.

% the detailed method is here https://arxiv.org/pdf/1207.6983.pdf
% http://users.uoa.gr/~kontogar/files/ElisavetDaras.pdf
% https://hal.inria.fr/inria-00075302/PDF/RR-1256.pdf
% https://www.ams.org/journals/mcom/2007-76-260/S0025-5718-07-01980-1/S0025-5718-07-01980-1.pdf
% https://graui.de/code/elliptic2/ //draw curves
% https://hal.inria.fr/inria-00075302/PDF/RR-1256.pdf proposition 2.1

The method usually starts by choosing a base field $\F_{q}$ of the curve $E(\F_q)$ we want to construct such that $q = p^m$ for some prime number $p$, and  `` $m\in \N$ with $m\geq 1$. We assume $p>3$ to simplify things in what follows. 

Next, the trace of Frobenius $t\in \Z$ of the curve is chosen such that $p$ and $t$ are coprime, that is, $gcd(p,t)=0$ holds true. The choice of $t$ also defines the curve's order $r$, since $r=p+1-t$ by the Hasse bound (equation \ref{hasse-bound})\sme{check reference}, so choosing $t$ will define the large order subgroup as well as all small cofactors. $r$ has to be defined in such a way that the elliptic curve meets the security requirements of the application it is designed for. 

Note that the choice of $p$ and $t$ also determines the embedding degree $k$ of any prime-order subgroup of the curve, since $k$ is defined as the smallest number such that the prime order $n$ divides the number $q^k-1$.

\begin{equation}\label{eq:D-criteria}
\begin{split}
D<0\\
\Zmod{D}{4}=0 \text{ or } \Zmod{D}{4}=1\\
4q  = t^2 + |D|v^2 
\end{split}
\end{equation}

In order for the complex multiplication method to work, neither $q$ nor $t$ can be arbitrary, but must be chosen in such a way that two additional integers $D\in \Z$ and $v\in \Z$ exist and the following conditions hold:


If such numbers exist, we call $D$ the \term{CM-discriminant}, and we know that we can construct a curve $E(\F_q)$ over a finite field $\F_q$ such that the order of the curve is $|E(\F_q)|= q+1-t$. 

It is the content of the complex multiplication method to actually construct such a curve, that is finding the parameters $a$ and $b$ from $\F_q$ in the defining Weiertraß equation such that the curve has the desired order $r$. 

Finding solutions to equation \ref{hasse-bound}\sme{check reference},= can be achieved in different ways, but we will forego the fine detail here. In general, it can be said that there are well-known constraints for elliptic curve families (e.g. the BLS (ECT) families) that provides families of solutions. In what follows, we will look at one type curve in the BLS-family, which gives an entire range of solutions.\sme{disambiguate}\smelong{Are we looking at a subtype of BLS or is BLS the specific type we're referring to?}

Assuming that the proper parameters $q$, $t$, $D$ and $v$ are found, we have to compute the so-called \term{Hilbert class polynomial} $H_D\in \Z[x]$ of the CM-discriminant $D$, which is a polynomial with integer coefficients. To do so, we first have to compute the following set:
\begin{multline*}
ICG(D)=\{(A,B,C)\;|\; A,B,C\in\Z, D = B^2-4AC, gcd(A,B,C)=1, \\
|B|\leq A \leq \sqrt{\frac{|D|}{3}}, A\leq C, 
\text{ if } B< 0 \text{ then } |B| < A < C\}
\end{multline*} 
One way to compute this set is to first compute the integer $A_{max}= Floor(\sqrt{\frac{|D|}{3}})$, then loop through all the integers $A$ in the range $[0,\ldots,A_{max}]$, as well as through all the integers $B$ in the range $[-A_{max},\ldots,A_{max}]$, then see if there is an integer $C$ that satisfies $D = B^2-4AC$ and the rest of the requirements in XXX\sme{add reference}.

To compute the Hilbert class polynomial, the so-called \term{$j$-function} (or $j$-invariant) is needed, which is a complex function defined on the upper half $\mathbb{H}$ of the complex plane $\mathbb{C}$, usually written as follows:\smelong{is this the same as equation {eq:j-invariant1 No they just sound very similar. But they are very different}?}\sme{unify terminology}
\begin{equation}\label{eq:j-invariant2}
j: \mathbb{H} \to \mathbb{C}
\end{equation}

Roughly speaking, what this means is that the $j$-functions takes complex numbers 
$(x +i\cdot y)$ with a positive imaginary part $y>0$ as inputs and returns a complex number $j(x+i\cdot y)$ as a result.

For the purposes of this book, it is not important to understand the $j$-function in detail, and we can use Sage to compute it in a similar way that we would use Sage to compute any other well-known function. It should be noted, however, that the computation of the $j$-function in Sage is sometimes prone to precision errors. For example, the $j$-function has a root in $\frac{-1+i\sqrt{3}}{2}$, which Sage only approximates. Therefore, when using Sage to compute the $j$-function, we need to take precision loss into account and possibly round to the nearest integer.

\begin{sagecommandline}
sage: z = ComplexField(100)(0,1)
sage: z # (0+1i)
sage: elliptic_j(z)
sage: # j-function only defined for positive imaginary arguments
sage: z = ComplexField(100)(1,-1)
sage: try:
....:     elliptic_j(z)
....: except PariError:
....:     pass
sage: # root at (-1+i sqrt(3))/2
sage: z = ComplexField(100)(-1,sqrt(3))/2
sage: elliptic_j(z)
sage: elliptic_j(z).imag().round()
sage: elliptic_j(z).real().round()
\end{sagecommandline}

With a way to compute the $j$-function and the precomputed set $ICG(D)$ at hand, we can now compute the Hilbert class polynomial as follows:
\begin{equation}
H_D(x) = \Pi_{(A,B,C)\in ICG(D)} \left(x - j\left(\frac{-B + \sqrt{D}}{2A}\right)\right)
\end{equation}

In other words, we loop over all elements $(A,B,C)$ from the set $ICG(D)$ and compute the $j$-function at the point $\frac{-B + \sqrt{D}}{2A}$, where $D$ is the CM-discriminant that we chose in a previous step. The result defines a factor of the Hilbert class polynomial and all factors are multiplied together.

It can be shown that the Hilbert class polynomial is an integer polynomial, but actual computations need high-precision arithmetics to avoid approximation errors that usually occur in computer approximations of the $j$-function (as shown above). So, in case the calculated Hilbert class polynomial does not have integer coefficients, we need to round the result to the nearest integer. Given that the precision we used was high enough, the result will be correct.

In the next step, we use the Hilbert class polynomial $H_D\in \Z[x]$, and project it to a polynomial $H_{D,q}\in\F_q[x]$ with coefficients in the base field $\F_q$ as chosen in the first step. We do this by simply computing the new coefficients as the old coefficients modulus $p$, that is, if $H_D(x)= a_mx^m +a_{m-1}x^{m-1}+\ldots + a_1 x + a_0$, we compute the $q$-modulus of each coefficient
$\tilde{a}_j = \Zmod{a_j}{p}$, which defines the \term{projected Hilbert class polynomial} as follows:
$$
H_{D,p}(x)=\tilde{a}_mx^m +\tilde{a}_{m-1}x^{m-1}+\ldots + \tilde{a}_1 x + \tilde{a}_0
$$
We then search for roots of $H_{D,p}$, since every root $j_0$ of $H_{D,p}$ defines a family of elliptic curves over $\F_q$, which all have a $j$-invariant \ref{eq:j-invariant1} or \ref{eq:j-invariant2}\sme{check reference} equal to $j_0$. We can pick any root, since all of them will lead to proper curves eventually.

However, some of the curves with the correct $j$-invariant might have an order different from the one we initially decided on. Therefore, we need a way to decide on a curve with the correct order. 

To compute such a curve, we have to distinguish a few different cases based on our choice of the root $j_0$ and of the CM-discriminant $D$. If $j_0\neq 0$ or $j_0\neq \Zmod{1728}{q}$, we compute $c_1=\frac{j_0}{(\Zmod{1728}{q}) -j_0}$, then we chose some arbitrary quadratic non-residue $c_2\in \F_q$, and some arbitrary cubic non-residue $c_3\in \F_q$. 

The following table is guaranteed to define a curve with the correct order $r= q+1 -t$ for the trace of Frobenius $t$ we initially decided on:\sme{actually make this a table?}
\begin{definition}\label{def:curve-order-frobenius}
\begin{itemize}
\item Case $j_0 \neq 0 $ and $j_0\neq \Zmod{1728}{q}$. A curve with the correct order is defined by one of the following equations:
\begin{equation}
y^2 = x^3 + 3c_1x + 2c_1 \text{\;\; or \;\; } y^2 = x^3 + 3c_1c_2^2x + 2c_1c_2^3
\end{equation}
\item Case $j_0 = 0 $ and $D\neq -3$. A curve with the correct order is defined by one of the following equations:
\begin{equation}
y^2 = x^3 + 1 \text{\;\; or \;\; } y^2 = x^3 + c_2^3
\end{equation}
\item Case $j_0 = 0 $ and $D= -3$. A curve with the correct order is defined by one of the following equations:
\begin{align*}
y^2 = x^3 +1 & \text{\;\; or \;\; } y^2 = x^3 + c_2^3 \text{ \;\; or}\\  
y^2 = x^3 + c_3^2 & \text{\;\; or \;\; } y^2 = c_3^2 c_2^3 \text{\;\; or}\\
y^2 = x^3 + c_3^{-2} & \text{\;\; or \;\; }  y^2 = x^3 + c_3^{-2}c_2^3 
\end{align*}
\item Case $j_0 = \Zmod{1728}{q} $ and $D\neq -4$. A curve with the correct order is defined by one of the following equations:
\begin{equation}
y^2 = x^3 + x \text{\;\; or \;\; } y^2 = x^3 + c_2^2x
\end{equation}
\item Case $j_0 = \Zmod{1728}{q} $ and $D= -4$. A curve with the correct order is defined by one of the following equations:
\begin{align*}
y^2 = x^3 +x & \text{\;\; or \;\; } y^2 = x^3 + c_2x \text{ \;\; or}\\  
y^2 = x^3 + c_2^2x & \text{\;\; or \;\; } y^2 = x^3 + c_2^3x
\end{align*}
\end{itemize} 
\end{definition}
To decide the proper defining Weierstraß equation, we therefore have to compute the order of any of the potential curves above, and then choose the one that fits our initial requirements. Since it can be shown that the Hilbert class polynomials for the CM-discriminants $D=-3$ and $D=-4$ are given by  $H_{-3,q}(x)=x$ and $H_{-4,q}= x-(\Zmod{1728}{q})$ (EXERCISE), the previous cases are exhaustive.\sme{exercise still to be written?}

To summarize, using the complex multiplication method, it is possible to synthesize elliptic curves with predefined order over predefined base fields from scratch. However, the curves that are constructed this way are just some representatives of a larger class of curves, all of which have the same order. Therefore, in real-world applications, it is sometimes more advantageous to choose a different representative from that class. To do so recall from XXX\sme{add reference} that any curve defined by the Weierstraß equation $y^2 = x^3 + ax b$ is isomorphic to a curve of the form $y^2 = x^3 + ad^2 x + bd^3$ for some quadratic residue $d\in \F_q$. 

In order to find a suitable representative (e.g. with small parameters $a$ and $b$) in the last step, the curve designer might choose a quadratic residue $d$ such that the transformed curve has the properties they wanted.

\begin{example} Consider curve $E_1(\F_5)$ from example \ref{E1F5}\sme{check reference}. We want to use the complex multiplication method to derive that curve from scratch. Since $E_1(\F_5)$ is a curve of order $r=9$ over the prime field of order $q=5$, we know from example \ref{ex:E1F5-frobenius}\sme{check reference} that its trace of Frobenius is $t=-3$, which also implies that $q$ and $|t|$ are coprime. 

We then have to find parameters $D,v\in\Z$ such that the criteria in \ref{eq:D-criteria} hold. We get the following:
\begin{align*}
4q & = t^2+ |D|v^2 & \Rightarrow \\
20 & = (-3)^2 + |D|v^2 & \Leftrightarrow \\
11 & = |D|v^2
\end{align*}
Now, since $11$ is a prime number, the only solution is $|D|=11$ and $v=1$ here. With $D=-11$ and  the Euclidean division of $-11$ by $4$ being $-11 = -3\cdot 4 +1$, we have $\Zmod{-11}{4}=1$, which shows that $D=-11$ is a proper choice.

In the next step, we have to compute the Hilbert class polynomial $H_{-11}$. To do so, we first have to find the set $ICG(D)$. To compute that set, observe that, since $\sqrt{\frac{|D|}{3}}\approx 1.915<2$, we know from $A\leq \sqrt{\frac{|D|}{3}}$ and $A\in\Z$ that $A$ must be either $0$ or $1$. 

For $A=0$, we know $B=0$ from the constraint $|B|\leq A$. However, in this case, there could be no $C$ satisfying $-11= B^2 -4AC$. So we try $A=1$ and deduce $B\in\{-1,0,1\}$ from the constraint $|B|\leq A$. The case $B=-1$ can be excluded, since then $B<0$ has to imply $|B|<A$. The case $B=0$ can also be excluded, as there cannot be an integer $C$ with $-11 = -4C$, since $11$ is a prime number. 

This leaves the case $B=1$, and we compute $C=3$ from the equation $-11 = 1^2 -4C$, which gives the solution $(A,B,C)=(1,1,3)$:
$$
ICG(D)=\{(1,1,3)\}
$$

With the set $ICG(D)$ at hand, we can compute the Hilbert class polynomial of $D=-11$. To do so, we have to insert the term $\frac{-1+\sqrt{-11}}{2\cdot1}$ into the $j$-function. To do so, first observe that $\sqrt{-11}=i\sqrt{11}$, where $i$ is the imaginary unit, defined by $i^2=-1$. Using this, we can invoke Sage to compute the $j$-invariant and get the following:
$$
H_{-11}(x) = x - j\left(\frac{-1+i\sqrt{11}}{2}\right) = x + 32768
$$

As we can see, in this particular case, the Hilbert class polynomial is a linear function with a single integer coefficient. In the next step, we have to project it onto a polynomial from $\F_5[x]$ by computing the modular $5$ remainder of the coefficients $1$ and $32768$. We get $\Zmod{32768}{5}=3$, from which it follows that the projected Hilbert class polynomial is considered a polynomial from $\F_5[x]$:
$$
H_{-11,5}(x)=x+3
$$ 
 As we can see, the only root of this polynomial is $j=2$, since $H_{-11,5}(2)=2+3=0$. We therefore have a situation with $j\neq 0$ and $j\neq 1728$, which tells us that we have to compute the parameter $c_1$ in modular $5$ arithmetics:
$$
c_1=\frac{2}{1728-2}
$$
 Since $\Zmod{1728}{5}=3$, we get $c_1=2$. 
 
 Next, we have to check if the curve $E(\F_5)$ defined by the Weierstraß equation  $y^2 = x^3 + 3\cdot 2 x + 2\cdot 2$ has the correct order. We invoke Sage, and find that the order is indeed $9$, so it is a curve with the required parameters. Thus, we have successfully constructed the curve with the desired properties.

Note, however, that in real-world applications, it might be useful to choose parameters $a$ and $b$ that have certain properties, e.g. to be a small as possible. As we know from XXX\sme{add reference}, choosing any quadratic residue $d\in \F_5$ gives a curve of the same order defined by $y^2 = x^2 + a k^2 x + bk^3$. Since $4$ is a quadratic residue in $\F_4$, we can transform the curve defined by 
$y^2 = x^3 +x+4$ into the curve $y^2 = x^3 + 4^2 + 4\cdot 4^3$ which gives the following:
$$
y^2 = x^3 + x +1
$$

This is the curve $E_1(\F_5)$ that we used extensively throughout this book. Thus, using the complex multiplication method, we were able to derive a curve with specific properties from scratch.
\end{example}

\begin{example} Consider the tiny jubjub curve $\mathit{TJJ\_13}$ from example \ref{PJJ13}\sme{check reference}. We want to use the complex multiplication method to derive that curve from scratch. Since $\mathit{TJJ\_13}$ is a curve of order $r=20$ over the prime field of order $q=13$, we know from example \ref{ex:PJJ13-frobenius}\sme{check reference} that its trace of Frobenius is $t=-6$, which also implies that $q$ and $|t|$ are coprime. 

We then have to find parameters $D,v\in\Z$ such that \ref{eq:D-criteria} holds. We get the following:
\begin{align*}
4q & = t^2+ |D|v^2 & \Rightarrow \\
4\cdot 13 & = (-6)^2+ |D|v^2 & \Rightarrow \\
52 & = 36 + |D|v^2 & \Leftrightarrow \\
16 & = |D|v^2
\end{align*}

This equation has two solutions for $(D,v)$, namely $(-4,\pm 2)$ and $(-16,\pm 1)$. Looking at the first solution, we know that  $D=-4$ implies $j=1728$, and the constructed curve is defined by a Weierstraß equation \ref{def_short_weierstrass_curve}\sme{check reference} that has a vanishing parameter $b=0$. We can therefore conclude that choosing $D=-4$ will not help us reconstructing $\mathit{TJJ\_13}$. It will produce curves with order $20$, just not the one we are looking for.

So we choose the second solution $D=-16$. In the next step, we have to compute the Hilbert class polynomial $H_{-16}$. To do so, we first have to find the set $ICG(D)$. To compute that set, observe that since $\sqrt{\frac{|-16|}{3}}\approx 2.31<3$, we know from $A\leq \sqrt{\frac{|-16|}{3}}$ and $A\in\Z$ that $A$ must be in the range $0..2$. So we loop through all possible values of $A$ and through all possible values of $B$ under the constraints $|B|\leq A$, and if $B<0$ then $|B|<A$.
Then we compute potential $C$'s from $-16 = B^2 -4AC$. We get the following two solutions for $ICG(D)$:
% sage has precomputed Hilbert class polynomials 
% https://doc.sagemath.org/html/en/reference/databases/sage/databases/db_class_polynomials.html
we get
$$
ICG(D)=\{(1,0,4),(2,0,2)\}
$$
With the set $ICG(D)$ at hand, we can compute the Hilbert class polynomial of $D=-16$. We can invoke Sage to compute the $j$-invariant and get the following:
\begin{align*}
H_{-16}(x) &= \left(x - j\left(\frac{i\sqrt{16}}{2}\right)\right)
 \left(x - j\left(\frac{i\sqrt{16}}{4}\right)\right) \\
           &= (x- 287496)(x-1728)
\end{align*}

As we can see, in this particular case, the Hilbert class polynomial is a quadratic function with two integer coefficients. In the next step, we have to project it onto a polynomial from $\F_5[x]$ by computing the modular $5$ remainder of the coefficients $1$, $287496$ and $1728$. We get $\Zmod{287496}{13}=1$ and $\Zmod{1728}{13}=2$, which means that the projected Hilbert class polynomial is as follows:
$$
H_{-11,5}(x)=(x-1)(x-12)= (x+12)(x+1)
$$ 
This is considered a polynomial from $\F_5[x]$. Thus, we have two roots, namely $j=1$ and $j=12$. We already know that $j=12$ is the wrong root to construct the tiny jubjub curve, since $\Zmod{1728}{13}=2$, and that case is not compatible with a curve with $b\neq 0$. So we choose $j=1$.

Another way to decide the proper root is to compute the $j$-invariant of the tiny-jubjub curve. We get the following:
\begin{align*}
j(\mathit{TJJ\_13}) & = 12\frac{4\cdot 8^3}{4\cdot 8^3+ 1\cdot 8^2}\\
                    & = 12\frac{4\cdot 5}{4\cdot 5+ 12}\\
                    & = 12\frac{7}{7+ 12}\\
                    & = 12\frac{7}{7+ 12}\\
                    & = 1
\end{align*}

This is equal to the root $j=1$ of the Hilbert class polynomial $H_{-16,13}$ as expected. We therefore have a situation with $j\neq 0$ and $j\neq 1728$, which tells us that we have to compute the parameter $c_1$ in modular $5$ arithmetics:
$$
c_1=\frac{1}{12-1} = 6
$$
Since $\Zmod{1728}{13}=12$, we get $c_1=6$. Then we have to check if the curve $E(\F_5)$ defined by the Weierstraß  equation $y^2 = x^3 + 3\cdot 6 x + 2\cdot 6$, which is equivalent to
$
y^2 = x^3 + 5x +12
$, has the correct order. We invoke Sage and find that the order is $8$, which implies that the trace of this curve is $6$, not $-6$ as required. So we have to consider the second possibility, and choose some quadratic non-residue $c_2\in\F_{13}$. We choose $c_2=5$ and compute the Weierstraß equation $y^2 = x^3 + 5 c_2^2 + 12 c_2^3$ as follows:
$$
y^2 = x^3 + 8 x + 5
$$
We invoke Sage and find that the order is $20$, which is indeed the correct one. As we know from XXX\sme{add reference}, choosing any quadratic residue $d\in \F_5$ gives a curve of the same order defined by $y^2 = x^2 + a d^2 x + bd^3$. Since $12$ is a quadratic residue in $\F_{13}$, we can transform the curve defined by 
$y^2 = x^3 +8x+5$ into the curve $y^2 = x^3 + 12^2\cdot 8 + 5\cdot 12^3$ which gives the following:
$$
y^2 = x^3 + 8x +8
$$

This is the tiny jubjub curve that we used extensively throughout this book. So using the complex multiplication method, we were able to derive a curve with specific properties from scratch.
\end{example}

\begin{example} To consider a real-world example, we want to use the complex multiplication method in combination with Sage to compute Secp256k1 from scratch. So based on example \ref{Secp256k1}\sme{check reference}, we decided to compute an elliptic curve over a prime field $\F_p$ of order $r$ for the following security parameters:
\begin{align*}
p &= \scriptstyle 115792089237316195423570985008687907853269984665640564039457584007908834671663\\
r &= \scriptstyle 115792089237316195423570985008687907852837564279074904382605163141518161494337
\end{align*}
According to example \ref{ex:Secp256k1-trace}\sme{check reference}, this gives the following trace of Frobenius:
$$t = \scriptstyle 432420386565659656852420866390673177327$$ 

We also decided that we want a curve of the form $y^2 = x^3 + b$, that is, we want the parameter $a$ to be zero. This implies that the $j$-invariant of our curve must be zero.

In a first step, we have to find a CM-discriminant $D$ and some integer $v$ such that the equation 
$
4p = t^2 +|D|v^2
$
is satisfied. Since we aim for a vanishing $j$-invariant, the first thing to try is $D=-3$. In this case, we can compute $v^2 = (4p -t^2)$, and if $v^2$ happens to be an integer that has a square root $v$, we are done. Invoking Sage we compute as follows:
\begin{sagecommandline}
sage: D = -3
sage: p = 115792089237316195423570985008687907853269984665640564039457584007908834671663
sage: r = 115792089237316195423570985008687907852837564279074904382605163141518161494337
sage: t = p+1-r
sage: v_sqr = (4*p - t^2)/abs(D)
sage: v_sqr.is_integer()
sage: v = sqrt(v_sqr)
sage: v.is_integer()
sage: 4*p == t^2 + abs(D)*v^2
sage: v
\end{sagecommandline}
The pair $(D,v)=(-3, 303414439467246543595250775667605759171)$ does indeed solve the equation, which tells us that there is a curve of order $r$ over a prime field of order $p$, defined by a Weierstraß equation $y^2 = x^3 + b$ for some $b\in \F_p$. Now we need to compute $b$.

For $D=-3$, we already know that the associated Hilbert class polynomial is given by $H_{-3}(x)=x$, which gives the projected Hilbert class polynomial as 
$H_{-3,p}=x$ and the $j$-invariant of our curve is guaranteed to be $j=0$. Now, looking at \ref{def:curve-order-frobenius}\sme{check reference}, we see that there are $6$ possible cases to construct a curve with the correct order $r$. In order to construct the curves in question, we have to choose some arbitrary quadratic and cubic non-residue. So we loop through $\F_p$ to find them, invoking Sage:

\begin{sagecommandline}
sage: F = GF(p)
sage: for c2 in F:
....:     try: # quadratic residue
....:         _ = c2.nth_root(2)
....:     except ValueError: # quadratic non-residue
....:         break
sage: c2
sage: for c3 in F:
....:     try:
....:         _ = c3.nth_root(3)
....:     except ValueError:
....:         break
sage: c3
\end{sagecommandline}

We found the quadratic non-residue $c_2=3$ and the cubic non-residue $c_3=2$. Using those numbers, we check the six cases against the the expected order $r$ of the curve we want to synthesize:
\begin{sagecommandline}
sage: C1 = EllipticCurve(F,[0,1])
sage: C1.order() == r
sage: C2 = EllipticCurve(F,[0,c2^3])
sage: C2.order() == r
sage: C3 = EllipticCurve(F,[0,c3^2])
sage: C3.order() == r
sage: C4 = EllipticCurve(F,[0,c3^2*c2^3])
sage: C4.order() == r
sage: C5 = EllipticCurve(F,[0,c3^(-2)])
sage: C5.order() == r
sage: C6 = EllipticCurve(F,[0,c3^(-2)*c2^3])
sage: C6.order() == r
\end{sagecommandline}

As expected, we found an elliptic curve of the correct order $r$ over a prime field of size $p$. In principle. we are done, as we have found a curve with the same basic properties as Secp256k1. However, the curve is defined by the following equation, which uses a very large parameter $b_1$, and so it might perform too slowly in certain algorithms.
$$
\scriptstyle y^2 = x^3 + 86844066927987146567678238756515930889952488499230423029593188005931626003754
$$
 It is also not very elegant to be written down by hand.\sme{what does this mean? Maybe just delete it}  It might therefore be advantageous to find an isomorphic curve with the smallest possible parameter $b_2$. In order to find such a $b_2$, we have to choose a quadratic residue $d$ such that $b_2 = b_1\cdot d^3$ is as small as possible. To do so, we rewrite the last equation into the following form:
$$
d = \sqrt[3]{\frac{b_2}{b_1}}
$$ 

Then we invoke Sage to loop through values $b_2\in \F_p$ until it finds some number such that the quotient $\frac{b_2}{b_1}$ has a cube root $d$ and this cube root itself is a quadratic residue. 
\begin{sagecommandline}
sage: b1=86844066927987146567678238756515930889952488499230423029593188005931626003754
sage: for b2 in F:
....:     try:
....:         d = (b2/b1).nth_root(3)
....:         try:
....:             _ = d.nth_root(2)
....:             if d != 0:
....:                 break
....:         except ValueError:
....:             pass
....:     except ValueError:
....:         pass
sage: b2
\end{sagecommandline}
Indeed, the smallest possible value is $b_2=7$ and the defining Weierstraß equation of a curve over $\F_p$ with prime order $r$ is 
$
y^2 = x^3 + 7
$,
which we might call Secp256k1. As we have just seen, the complex multiplication method is powerful enough to derive cryptographically secure curves like Secp256k1 from scratch.
\end{example}
%\section{twists}
%I think this has to wait for Volume 2, due to timing constraints...

\paragraph{The $BLS6\_6$ pen-and-paper curve}\label{BLS6}

% https://arxiv.org/pdf/1207.6983.pdf
% construction 6.6 in https://eprint.iacr.org/2006/372.pdf
In this paragraph, we summarize our understanding of elliptic curves to derive our main pen-and-paper example for the rest of the book. To do so, we want to use the complex multiplication method to derive a pairing-friendly elliptic curve that has similar properties to curves that are used in actual cryptographic protocols. However, we design the curve specifically to be useful in pen-and-paper examples, which mostly means that the curve should contain only a few points so that we are able to derive exhaustive addition and pairing tables.

A well-understood family of pairing-friendly curves is the the group of BLS curves \smelong{(STUFF ABOUT THE HISTORY AND THE NAMING CONVENTION)}\sme{write up this part}, which are derived in [XXX\sme{add reference}]. BLS curves are particularly useful in our case if the embedding degree $k$ satisfies $\kongru{k}{6}{0}$. Of course, the smallest embedding degree $k$ that satisfies this congruency is $k=6$ and we therefore aim for a BLS6 curve as our main pen-and-paper example. 

To apply the complex multiplication method from page \pageref{complex-multiplication-method} ff.\sme{check reference}, recall that this method starts with a definition of the base field $\F_{p^m}$, as well as the trace of Frobenius $t$ and the order of the curve. If the order $p^m+1-t$ is not a prime number, then the order $r$ of the largest prime factor group needs to be controlled. 

In the case of BLS\_6 curves, the parameter $m$ is chosen to be $1$, which means that the curves are defined over prime fields. All relevant parameters $p$, $t$ and $r$ are then themselves parameterized by the following three polynomials:
\begin{equation}
\begin{split}
r(x) &= \Phi_6(x)\\
t(x) &= x+1\\
q(x) &= \frac{1}{3}(x-1)^2(x^{2}-x+1) +x
\end{split}
\end{equation}

In the equations above, $\Phi_6$ is the $6$-th \uterm{cyclotomic polynomial} and  $x\in\N$ is a parameter that the designer has to choose in such a way that the evaluation of $p$, $t$ and $r$ at the point $x$ gives integers that have the proper size to meet the security requirements of the curve that they want to design. It is then guaranteed that the complex multiplication method can be used in combination with those parameters to define an elliptic curve with CM-discriminant $D=-3$, embedding degree $k=6$, and curve equation $y^2 = x^3 +b$ for some $b\in\F_p$. 

For example, if the curve should target the $128$-bit security level, due to the \uterm{Pholaard-rho attack} (TODO)\sme{todo} the parameter $r$ should be prime number of at least $256$ bits.

In order to design the smallest BLS\_6 curve, we therefore have to find a parameter $x$ such that $r(x)$, $t(x)$ and $q(x)$ are the smallest natural numbers that satisfy $q(x)>3$ and $r(x)>3$.\footnote{The smallest BLS curve will also be the most insecure BLS curve. However, since our goal with this curve is ease of pen-and-paper computation rather than security, it fits the purposes of this book.}

We therefore initiate the design process of our $BLS6$ curve by looking up the $6$-th cyclotomic polynomial, which is $\Phi_{6}=x^2-x+1$, and then insert small values for $x$ into the defining polynomials $r,t,q$. We get the following results:
$$
\begin{array}{lcr}
x=1 & (r(x),t(x),q(x)) & (1,2,1)\\
x=2 & (r(x),t(x),q(x)) & (3,3,3)\\
x=3 & (r(x),t(x),q(x)) & (7,4,\frac{37}{3})\\
x=4 & (r(x),t(x),q(x)) & (13,5,43)\\
\end{array}
$$
Since $q(1)=1$ is not a prime number, the first $x$ that gives a proper curve is $x=2$. However, such a curve would be defined over a base field of characteristic $3$, and \smelong{we would rather like to avoid that}.\sme{why? Because in this book elliptic curves are only defined for fields of chracteristic > 3 } We therefore find $x=4$, which defines a curve over the prime field of characteristic $43$ that has a trace of Frobenius $t=5$ and a larger order prime group of size $r=13$. 

Since the prime field $\F_{43}$ has $43$ elements and $43$'s binary representation is $43_2= 101011$, which consists of $6$ digits, the name of our pen-and-paper curve should be $BLS6\_6$, since its is common to name a BLS curve by its embedding degree and the bit-length of the modulus in the base field. We call $BLS6\_6$ the \term{moon-math-curve}.

Based on \ref{hasse-bound}\sme{check reference}, we know that the Hasse bound implies that $BLS6\_6$ will contain exactly $39$ elements. Since the prime factorization of $39$ is $39=3\cdot 13$, we have a ``large'' prime factor group of size $13$, as expected, and a small cofactor group of size $3$. Fortunately, a subgroup of order $13$ is well suited for our purposes, as $13$ elements can be easily handled in the associated addition, scalar multiplication and pairing tables in a pen-and-paper style. 

We can check that the embedding degree is indeed $6$ as expected, since $k = 6$ is the smallest number $k$ such that $r=13$ divides $43^k-1$. 
\begin{sagecommandline}
sage: for k in range(1,42): # Fermat's little theorem
....:     if (43^k-1)%13 == 0:
....:         break
sage: k
\end{sagecommandline}

In order to compute the defining equation $y^2=x^3 + ax + b$ of BLS6-6, we use the complex multiplication method as described in \ref{complex-multiplication-method}\sme{check reference}. The goal is to find $a,b\in\F_{43}$ representations that are particularly \smelong{nice to work with}\sme{what does this mean?}. The authors of XXX\sme{add reference} showed that the CM-discriminant of every BLS curve is $D=-3$ and, indeed, the following equation has the four solutions $(D,v)\in\{(-3,-7),(-3,7),(-49,-1),(-49,1)\}$ if $D$ is required to be negative, as expected: 

\begin{align*}
4p & = t^2 + |D|v^2 & \Rightarrow \\ 
4\cdot 43 & = 5^2 + |D|v^2 & \Rightarrow \\ 
172 & = 25 + |D| v^2 & \Leftrightarrow \\ 
49 & = |D| v^2
\end{align*}

This means that $D=-3$ is indeed a proper CM-discriminant, and we can deduce that the parameter $a$ has to be $0$, and that the Hilbert class polynomial is given by
$
H_{-3,43}(x) = x
$.

This implies that the $j$-invariant of $BLS6\_6$ is given by $j(BLS6\_6)=0$. We therefore have to look at case XXX\sme{add reference} in table \ref{def:curve-order-frobenius}\sme{check reference} to derive a parameter $b$. To decide the proper case for $j_0=0$ and $D=-3$, we therefore have to choose some arbitrary quadratic non-residue $c_2$ and cubic non-residue $c_3$ in $\F_{43}$. We choose $c_2 =5$ and $c_3=36$. We check these with Sage:
\begin{sagecommandline}
sage: F43 = GF(43)
sage: c2 = F43(5)
....: try: # quadratic residue
....:     c2.nth_root(2)
....: except ValueError: # quadratic non-residue
....:     c2
sage: c3 =F43(36)
....: try:
....:     c3.nth_root(3)
....: except ValueError:
....:     c3
\end{sagecommandline} 

Using those numbers we check the six possible cases from \ref{def:curve-order-frobenius}\sme{check reference} against the the expected order $39$ of the curve we want to synthesize:

\begin{sagecommandline}
sage: BLS61 = EllipticCurve(F43,[0,1])
sage: BLS61.order() == 39
sage: BLS62 = EllipticCurve(F43,[0,c2^3])
sage: BLS62.order() == 39
sage: BLS63 = EllipticCurve(F43,[0,c3^2])
sage: BLS63.order() == 39
sage: BLS64 = EllipticCurve(F43,[0,c3^2*c2^3])
sage: BLS64.order() == 39
sage: BLS65 = EllipticCurve(F43,[0,c3^(-2)])
sage: BLS65.order() == 39
sage: BLS66 = EllipticCurve(F43,[0,c3^(-2)*c2^3])
sage: BLS66.order() == 39
sage: BLS6 = BLS63 # our BLS6 curve in the book
\end{sagecommandline}
As expected, we found an elliptic curve of the correct order $39$ over a prime field of size $43$, defined by the following equation:

\begin{equation}
BLS6\_6 := \{(x,y)\;|\; y^2 = x^3 + 6 \text{ for all } x,y \in \F_{43}\}
\end{equation}

There are other choices for $b$, such as $b=10$ or $b=23$, but all these curves are isomorphic, and hence represent the same curve in a different way. Since BLS6-6 only contains $39$ points ,it is possible to give a visual impression of the curve:

\begin{sagesilent}
BLS63p = BLS63.plot()
\end{sagesilent}
\begin{center} 
\sageplot[scale=.5]{BLS63p} 
\end{center}

As we can see, our curve has some desirable properties: it does not contain self-inverse points, that is, points with $y=0$. It follows that the addition law can be optimized, since the branch for those cases can be eliminated. 

Summarizing the previous procedure, we have used the method of Barreto, Lynn and Scott\sme{add reference} to construct a pairing-friendly elliptic curve of embedding degree $6$. However, in order to do elliptic curve cryptography on this curve, note that, since the order of $BLS6\_6$ is $39$, its group of rational points is not a finite cyclic group of prime order. We therefore have to find a suitable subgroup as our main target. Since $39=13\cdot 3$, we know that the curve must contain a ``large'' prime-order group of size $13$ and a small cofactor group of order $3$. 

The following step is to construct this group. One way to do so is to find a generator. We can achieve this by choosing an arbitrary element of the group that is not the point at infinity, and then multiply that point with the cofactor of the group's order. If the result is not the point at infinity, the result will be a generator. If it is the point at infinity we have to choose a different element. 

In order to find a generator for the large order subgroup of size $13$, we first notice that the cofactor of $13$ is $3$, since $39=3\cdot 13$. We then need to construct an arbitrary element from $BLS6\_6$. To do so in a pen-and-paper style, we can choose some $arbitrary x\in\F_{43}$ and see if there is some solution $y\in\F_{43}$ that satisfies the defining Weierstraß equation $y^2 = x^3 + 6$. We choose $x=9$, and check that $y=2$ is a proper solution:
\begin{align*}
y^2 & = x^3 + 6 & \Rightarrow \\
2^2 & = 9^3 + 6 & \Leftrightarrow \\
4 & = 4
\end{align*}   

This implies that $P=(9,2)$ is therefore a point on $BLS6\_6$. To see if we can project this point onto a generator of the large order prime group $BLS6\_6[13]$, we have to multiply $P$ with the cofactor, that is, we have to compute $[3](9,2)$. After some computation \smelong{(EXERCISE)}\sme{add exercise} we get $[3](9,2) = (13,15)$. Since this is not the point at infinity, we know that $(13,15)$ must be a generator of $BLS6\_6[13]$. The generator $g_{BLS6\_6[13]}$, which we will use in pairing computations in the remainder of this book, is given as follows:

\begin{equation}\label{gBLS6-6-13}
g_{BLS6\_6[13]} = (13,15)
\end{equation}

Since $g_{BLS6\_6[13]}$ is a generator, we can use it to construct the subgoup $BLS6\_6[13]$ by repeatedly adding the generator to itself. Using Sage, we get the following:
\begin{sagecommandline}
sage: P = BLS6(9,2)
sage: Q = 3*P
sage: Q.xy()
sage: BLS6_13 = []
sage: for x in range(0,13): # cyclic of order 13
....:     P = x*Q
....:     BLS6_13.append(P)
\end{sagecommandline}
Repeatedly adding a generator to itself, as we just did, will generate small groups in logarithmic order with respect to the generator as, explained on page \pageref{generators} ff\sme{check reference}. We therefore get the following description of the large prime-order subgroup of $BLS6\_6$:

\begin{multline}
BLS6\_6[13]=\\
\{(13,15) \rightarrow (33,34) \rightarrow  (38,15) \rightarrow  (35,28) \rightarrow (26,34) \rightarrow  (27,34) \rightarrow  \\ 
(27,9)  \rightarrow  (26,9) \rightarrow  (35,15) \rightarrow  (38,28) \rightarrow  (33,9) \rightarrow (13,28) \rightarrow  \mathcal{O}\}$$
\end{multline}
Having a logarithmic description of this group is tremendously helpful in pen-and-paper computations. To see that, observe that we know fromXXX\sme{add reference} that there is an exponential map from the scalar field $\F_{13}$ to $BLS6\_6[13]$ with respect to our generator, which generates the group in logarithmic order:
$$
[\cdot]_{(13,15)}: \F_{13} \to BLS6\_6[13]\;;\; x \mapsto [x](13,15)
$$
 So, for example, we have $[1]_{(13,15)}= (13,15)$, $[7]_{(13,15)}= (27,9)$ and $[0]_{(13,15)}= \mathcal{O}$ and so on. The relevant point here is that we can use this representation to do computations in $BLS6\_6[13]$ efficiently in our head using XXX\sme{add reference}, as in the following example:
\begin{align*}
(27,34)\oplus (33,9)  & = [6](13,15)\oplus [11](13,15)\\
                      & = [6+11](13,15)\\
                      & = [4](13,15)\\
                      & = (35,28)\\
\end{align*}
So XXX\sme{add reference} is really all we need to do computations in $BLS6\_6[13]$ in this book efficiently. However, out of convenience, the following picture lists the entire addition table of that group, as it might be useful in pen-and-paper computations:
\begingroup
    \fontsize{7pt}{7pt}\selectfont
$$
\begin{array}{c|ccccccccccccc}
\oplus & \mathcal{O}  & (13,15) & (33,34) & (38,15) & (35,28) & (26,34) & (27,34) & (27,9) & (26,9) & (35,15) & (38,28) & (33,9) & (13,28)\\
\hline
\\
\mathcal{O} & \mathcal{O}  & (13,15) & (33,34) & (38,15) & (35,28) & (26,34) & (27,34) & (27,9) & (26,9) & (35,15) & (38,28) & (33,9) & (13,28)\\
\\
(13,15) & (13,15) & (33,34) & (38,15) & (35,28) & (26,34) & (27,34) & (27,9) & (26,9) & (35,15) & (38,28) & (33,9) & (13,28) & \mathcal{O}\\
\\
(33,34) & (33,34) & (38,15) & (35,28) & (26,34) & (27,34) & (27,9) & (26,9) & (35,15) & (38,28) & (33,9) & (13,28) & \mathcal{O} & (13,15)\\
\\
(38,15) & (38,15) & (35,28) & (26,34) & (27,34) & (27,9) & (26,9) & (35,15) & (38,28) & (33,9) & (13,28) & \mathcal{O} & (13,15) & (33,34)\\
\\
(35,28) & (35,28) & (26,34) & (27,34) & (27,9) & (26,9) & (35,15) & (38,28) & (33,9) & (13,28) & \mathcal{O} & (13,15) & (33,34) & (38,15)\\
\\
(26,34) & (26,34) & (27,34) & (27,9) & (26,9) & (35,15) & (38,28) & (33,9) & (13,28) & \mathcal{O} & (13,15) & (33,34) & (38,15) & (35,28)\\
\\
(27,34) & (27,34) & (27,9) & (26,9) & (35,15) & (38,28) & (33,9) & (13,28) & \mathcal{O} & (13,15) & (33,34) & (38,15) & (35,28) & (26,34)\\
\\
(27,9) & (27,9) & (26,9) & (35,15) & (38,28) & (33,9) & (13,28) & \mathcal{O} & (13,15) & (33,34) & (38,15) & (35,28) & (26,34) & (27,34)\\
\\
(26,9) & (26,9) & (35,15) & (38,28) & (33,9) & (13,28) & \mathcal{O} & (13,15) & (33,34) & (38,15) & (35,28) & (26,34) & (27,34) & (27,9)\\
\\
(35,15) & (35,15) & (38,28) & (33,9) & (13,28) & \mathcal{O} & (13,15) & (33,34) & (38,15) & (35,28) & (26,34) & (27,34) & (27,9) & (26,9)\\
\\
(38,28) & (38,28) & (33,9) & (13,28) & \mathcal{O} & (13,15) & (33,34) & (38,15) & (35,28) & (26,34) & (27,34) & (27,9) & (26,9) & (35,15)\\
\\
(33,9) & (33,9) & (13,28) & \mathcal{O} & (13,15) & (33,34) & (38,15) & (35,28) & (26,34) & (27,34) & (27,9) & (26,9) & (35,15) & (38,28)\\
\\
(13,28) & (13,28) & \mathcal{O} & (13,15) & (33,34) & (38,15) & (35,28) & (26,34) & (27,34) & (27,9) & (26,9) & (35,15) & (38,28) & (33,9)\\
\end{array}
$$
\endgroup
%To see how the small prime-order group of $BLS6\_6$ looks like we can apply the same approach but for the cofactor $13$ instead (EXERCISE). We get 
%$$BLS6\_6[3]=\{\mathcal{O},(0,7),(0,36)\}$$
%Now that we ha

Now that we have constructed a ``large'' cyclic prime-order subgroup of $BLS6\_6$ suitable for many pen-and-paper computations in elliptic curve cryptography, we have to look at how to do pairings in this context. We know that $BLS6\_6$ is a pairing-friendly curve by design, since it has a small embedding degree $k=6$. It is therefore possible to compute Weil pairings efficiently. However, in order to do so, we have to decide the groups $\G_1$ and $\G_2$ as explained in exercise \ref{ex:G1G2-subgroups}\sme{check reference}. 

Since $BLS6\_6$ has two non-trivial subgroups, it would be possible to use any of them as the $n$-torsion group. However, in cryptography, the only secure choice is to use the large prime-order subgroup, which in our case is $BLS6\_6[13]$. We therefore decide to consider the $13$-torsion and define ${G}_1[13]$ as the first argument for the Weil pairing function:

\begin{multline*}
\mathbb{G}_1[13]=\{(13,15) \rightarrow (33,34) \rightarrow  (38,15) \rightarrow  (35,28) \rightarrow (26,34) \rightarrow  (27,34) \rightarrow  \\ 
(27,9)  \rightarrow  (26,9) \rightarrow  (35,15) \rightarrow  (38,28) \rightarrow  (33,9) \rightarrow (13,28) \rightarrow  \mathcal{O}\}$$
\end{multline*}

In order to construct the domain for the second argument, we need to construct $\G_2[13]$, which, according to the general theory, should be defined by those elements $P$ of the full $13$-torsion group $BLS6\_6[13]$ that are mapped to $43\cdot P$ under the Frobenius endomorphism (equation \ref{eq:frobenius-enomorphism})\sme{check reference}. 

To compute $\G_2[13]$, we therefore have to find the full $13$-torsion group first. To do so, we use the technique from XXX, which tells us that the full $13$-torsion can be found in the curve extension over the extension field $\F_{43^6}$, since the embedding degree of $BLS6\_6$ is $6$:

\begin{equation}
BLS6\_6 := \{(x,y)\;|\; y^2 = x^3 + 6 \text{ for all } x,y \in \F_{43^6}\}
\end{equation}

Thus, we have to construct $\F_{43^6}$, a field that contains $6321363049$ elements. In order to do so, we use the procedure of XXX\sme{add reference} and start by choosing a non-reducible polynomial of degree $6$ from the ring of polynomials $\F_{43}[t]$. We choose $p(t) = t^6+6$. Using Sage, we get the following:
\begin{sagecommandline}
sage: F43 = GF(43)
sage: F43t.<t> = F43[]
sage: p = F43t(t^6+6)
sage: p.is_irreducible()
sage: F43_6.<v> = GF(43^6, name='v', modulus=p)
\end{sagecommandline}

Recall from XXX\sme{add reference} that elements $x\in\F_{43^6}$ can be seen as polynomials $a_0+a_1v + a_2v^2+\ldots + a_5 v^5$ with the usual addition of polynomials and multiplication modulo $t^6+6$. 

In order to compute $\G_2[13]$, we first have to extend $BLS6\_6$ to $\F_{43^6}$, that is, we keep the defining equation, but expand the domain from $\F_{43}$ to $\F_{43^6}$. After that, we have to find at least one element $P$ from that curve that is not the point at infinity, is in the full $13$-torsion and  satisfies the identity $\pi(P) = [43]P$. We can then use this element as our generator of $\G_2[13]$ and construct all other elements by repeatedly adding the generator to itself.

Since $BLS6(\F_{43^6})$ contains $6321251664$ elements, it's not a good strategy to simply loop through all elements. Fortunately, Sage has a way to loop through elements from the torsion group directly:

\begin{sagecommandline}
sage: BLS6 = EllipticCurve (F43_6,[0 ,6]) # curve extension
sage: INF = BLS6(0) # point at infinity
sage: for P in INF.division_points(13): # full 13-torsion
....: # PI(P) == [q]P
....:     if P.order() == 13: # exclude point at infinity
....:         PiP = BLS6([a.frobenius() for a in P])
....:         qP = 43*P
....:         if PiP == qP:
....:             break
sage: P.xy()
\end{sagecommandline}

We found an element from the full $13$-torsion that is in the Eigenspace of the Eigenvalue $43$, which implies that it is an element of $\G_2[13]$. As $\G_2[13]$ is cyclic of prime order, this element must be a generator:
\begin{equation}
g_{\G_2[13]} = (7v^2, 16v^3)
\end{equation}

We can use this generator to compute $\G_2$ in logarithmic order with respect to $g_{\G_[13]}$. Using Sage we get the following:
\begin{sagecommandline}
sage: Q = BLS6(7*v^2,16*v^3)
sage: BLS6_13_2 = []
sage: for x in range(0,13):
....:     P = x*Q
....:     BLS6_13_2.append(P)
\end{sagecommandline}
\begin{multline*}
\mathbb{G}_2=\{
(7v^2, 16v^3) \to
(10v^2, 28v^3)\to
(42v^2, 16v^3)\to
(37v^2, 27v^3)\to\\
(16v^2, 28v^3)\to
(17v^2, 28v^3)\to
(17v^2, 15v^3)\to
(16v^2, 15v^3)\to\\
(37v^2, 16v^3)\to
(42v^2, 27v^3)\to
(10v^2, 15v^3)\to
(7v^2, 27v^3)\to
\mathcal{O}\}
\end{multline*}

Again, having a logarithmic description of $\G_2[13]$ is tremendously helpful in pen-and-paper computations, as it reduces complicated computation in the extended curves to modular $13$ arithmetics, as in the following example:
\begin{align*}
(17v^2,28v^3)\oplus (10v^2,15v^2)  & = [6](7v^2,16v^3)\oplus [11](7v^2,16v^3)\\
                      & = [6+11](7v^2,16v^3)\\
                      & = [4](7v^2,16v^3)\\
                      & = (37v^2,27v^3)\\
\end{align*}

So XXX\sme{add reference} is really all we need to do computations in $\G_2[13]$ in this book efficiently. 

To summarize the previous steps, we have found two subgroups, $\G_1[13]$ and $\G_2[13]$ suitable to do Weil pairings on $BLS6\_6$ as explained in \ref{eq:weil-pairing}\sme{check reference}. Using the logarithmic order XXX\sme{add reference} of $\G_1[13]$, the logarithmic order XXX\sme{add reference} of $\G_2[13]$ and the bilinearity in \ref{eq:bilinearityBLS6}, we can do Weil pairings on $BLS6\_6$ in a pen-and-paper style:

\begin{equation}\label{eq:bilinearityBLS6}
e([k_1]g_{BLS6\_6[13]},[k_2]g_{\G_2[13]}) = 
e(g_{BLS6\_6[13]},g_{\G_2[13]})^{k_1\cdot k_2}
\end{equation}

Observe that the Weil pairing between our two generators is given by the following identity:
\begin{equation}\label{BLS6-weil-pairing}
e(g_{BLS6\_6[13]},g_{\G_2[13]})= 5v^5 + 16v^4 + 16v^3 + 15v^2 + 3v + 41
\end{equation}

\begin{sagecommandline}
sage: g1 = BLS6([13,15])
sage: g2 = BLS6([7*v^2, 16*v^3])
sage: g1.weil_pairing(g2,13)
\end{sagecommandline}

\paragraph{Hashing to pairing groups}
We give various constructions to hash into $\mathbb{G}_1$ and $\mathbb{G}_2$. 

We start with hashing to the scalar field... \smelong{TO APPEAR}\sme{finish writing this up}

None of these techniques work for hashing into $\mathbb{G}_2$. We therefore implement Pederson's Hash for BLS6. 

We start with $\mathbb{G}_1$. Our goal is to define an $12$-bit bounded hash function:
$$
H_{1}: \{0,1\}^{12} \to \mathbb{G}_1 
$$
Since $12= 3\cdot 4$ we ``randomly'' select $4$ uniformly distributed generators $\{(38, 15), (35,28),\\ (27, 34), (38, 28)\}$ from $\mathbb{G}_1$ and use the pseudo-random function from XXX\sme{add reference}. 
Therefore, we have to choose a set of $4$ randomly generated invertible elements from $\F_{13}$ for every generator. We choose the following:
$$
\begin{array}{lcl}
(38,15) &:& \{2,7,5,9\}\\
(35,28) &:& \{11,4,7,7\}\\
(27,34) &:& \{5,3,7,12\}\\
(38,28) &:& \{6,5,1,8\}
\end{array}
$$
Our hash function is then computed as follows:

\begin{multline*}
H_1(x_{11},x_1,\ldots, x_{0})=
[2\cdot 7^{x_{11}}\cdot 5^{x_{10}}\cdot 9^{x_9}](38,15)+
[11\cdot 4^{x_8}\cdot 7^{x_7}\cdot 7^{x_6}](35,28)+\\
[5\cdot 3^{x_5}\cdot 7^{x_4}\cdot 12^{x_3}](27,34) +
[6\cdot 5^{x_2}\cdot 1^{x_{1}}\cdot 8^{x_{0}}](38,28)
\end{multline*}

Note that $a^x=1$ when $x=0$. Hence, those terms can be omitted in the computation. 
In particular, the hash of the $12$-bit zero string is given as follows:\sme{correct computations}

\begin{multline*}\smelong{WRONG-ORDERING-REDO}\\
H_1(0)= [2](38,15)+[11](35,28)+[5](27,34)+[6](38,28)= \\
(27,34)+(26,34)+(35,28)+(26,9)= (33,9) + (13,28) = (38,28)
\end{multline*}

The hash of $011010101100$ is given as follows:\sme{fill in missing parts}
\begin{multline*}
H_1(011010101100)=\smelong{WRONG-ORDERING-REDO}\\
[2\cdot 7^{0}\cdot 5^{1}\cdot 9^{1}](38,15)+
[11\cdot 4^{0}\cdot 7^{1}\cdot 7^{0}](35,28)+
[5\cdot 3^{1}\cdot 7^{0}\cdot 12^{1}](27,34) +
[6\cdot 5^{1}\cdot 1^{0}\cdot 8^{0}](38,28)=\\
[2\cdot 5\cdot 9](38,15)+
[11\cdot 7](35,28)+
[5\cdot 3\cdot 12](27,34) +
[6\cdot 5](38,28)=\\
[12](38,15)+
[12](35,28)+
[11](27,34) +
[4](38,28)=\\ 
\smelong{TO APPEAR}
\end{multline*}
We can use the same technique to define a $12$-bit bounded hash function in $\mathbb{G}_2$:  
$$
H_{2}: \{0,1\}^{12} \to \mathbb{G}_2 
$$
Again, we ``randomly'' select $4$ uniformly distributed generators $\{(7v^2 , 16v^3 ), (42v^2 , 16v^3 ), \\(17v^2 , 15v^3 ), (10v^2 , 15v^3 )\}$ from $\mathbb{G}_2$, and use the pseudo-random function from XXX\sme{add reference}. Therefore, we have to choose a set of $4$ randomly generated invertible elements from $\F_{13}$ for every generator:
$$
\begin{array}{lcl}
(7v^2 , 16v^3 ) &:& \{8,4,5,7\}\\
(42v^2 , 16v^3 ) &:& \{12,1,3,8\}\\
(17v^2 , 15v^3 ) &:& \{2,3,9,11\}\\
(10v^2 , 15v^3 ) &:& \{3,6,9,10\}
\end{array}
$$
Our hash function is then computed like this:
\begin{multline*}
H_1(x_{11},x_{10},\ldots, x_{0})=
[8\cdot 4^{x_{11}}\cdot 5^{x_{10}}\cdot 7^{x_9}](7v^2 , 16v^3)+
[12\cdot 1^{x_8}\cdot 3^{x_7}\cdot 8^{x_6}](42v^2 , 16v^3 )+\\
[2\cdot 3^{x_5}\cdot 9^{x_4}\cdot 11^{x_3}](17v^2 , 15v^3 ) +
[3\cdot 6^{x_2}\cdot 9^{x_{1}}\cdot 10^{x_{0}}](10v^2 , 15v^3 )
\end{multline*}

We extend this to a hash function that maps unbounded bitstrings to $\mathbb{G}_2$ by precomposing with an actual hash function like $MD5$, and feed the first 12 bits of its outcome into our previously defined hash function, with $TinyMD5_{\mathbb{G}_2}(s)= H_2(MD5(s)_0,\ldots MD5(s)_{11})$:
$$
TinyMD5_{\mathbb{G}_2}: \{0,1\}^* \to \mathbb{G}_2
$$
For example, since 
$MD5("")=\\ 0xd41d8cd98f00b204e9800998ecf8427e$, and the binary representation of the hexadecimal number $0x27e$ is $001001111110$, we compute $TinyMD5_{\mathbb{G}_2}$ of the empty string as follows:
$$TinyMD5_{\mathbb{G}_2}("")= H_2(MD5(s)_{11},\ldots MD5(s)_{0}) = H_2(001001111110)=$$\sme{check equation}





% FOR THE SECOND VERSION OF THE BOOK
%\subsection{MNT4 MNT6 Cycles}
% https://eprint.iacr.org/2006/372.pdf theorem 5.2
%\begin{theorem}
%Let $q$ be a prime and $E/\F_q$ be an ordinary elliptic curve such that $r= |E(Fq)|$ is a prime greater than $3$.  
%\begin{itemize}
%\item $E$ has embedding  degree $k= 4$ if and only if there  exists $x\in \mathbb{Z}$ such  that $t=-x$ or $t=x+1$, and $q=x^2+x+1$.\item $E$ has  embedding  degree $k= 6$ if and only if there  exists $x\in \Z$ such that $t= 1\pm 2x$ and $q=4x^2+1$.
%\item There is an elliptic curve $E/\F_q$ with embedding degree $6$, discriminant $D$, and $|E(Fq)| = r$ if and only if there is an elliptic curve $E'/\F_r$ with embedding degree $4$, discriminant $D$, and $|E'(\F_r)| =q$.
%\end{itemize}
%\end{theorem}

%We can use this theorem to find an MNT6-MNT4 cycle over very small prime fields with characteristics $>3$: 
%\paragraph{MNT4}
%For our MNT4 curve, we can choose $x=2$. Then $q=7$ and if we choose $t= x+1 $ then $r = q + 1 - t = 7 + 1 -3 = 5$. Therefore our MNT4 curve is a curve $y^2=x^3+ax+b$ defined over $\F_7$ that consists of $5$ points. 

%To construct the actual curve we could use the complex multiplication method again, but since the parameters $a$ and $b$ are from $\F_7$ there are only $48$ possibilities so we simply loop through all possible $a$'s and $b$'s and count the curve points until we find a curve that has $5$ rational points. We get
%$$
%y^2 = x^3 + 4x + 1
%$$
%defined over $\F_7$, with scalar field $\F_5$. Since $7= 2^2+2+1$, we know from theorem XXX that this curve has embedding degree $4$ and hence qualifies as a pen-and-paper pairing-friendly elliptic curve. Since the curve's order is a prime and therefore has no non trivial factors, it has no non trivial subgroups. The curve has the following set of elements
%$$MNT4=\{(0,1)\to (0,6)\to (4,2)\to (4,5) \to \mathcal{O}\}$$ 
%\begin{sagecommandline}
%sage: F7 = GF(7)
%sage: MNT4 = EllipticCurve (F7,[4 ,1])
%sage: [P.xy() for P in MNT4.points() if P.order() > 1]
%\end{sagecommandline}
%The multiplication table is
%\begingroup
%    \fontsize{10pt}{10pt}\selectfont
%$$
%\begin{array}{c|ccccc}
%\cdot & \mathcal{O} & (0,1) & (4,5) & (4,2) & (0,6)\\
%\hline
%\\
%\mathcal{O} & \mathcal{O} & (0,1) & (4,5) & (4,2) & (0,6)\\
%\\
%(0,1) & (0,1) & (4,5) & (4,2) & (0,6) & \mathcal{O}\\
%\\
%(4,5) & (4,5) & (4,2) & (0,6) & \mathcal{O} & (0,1)\\
%\\
%(4,2) & (4,2) & (0,6) & \mathcal{O} & (0,1) & (4,5)\\
%\\
%(0,6) & (0,6) & \mathcal{O} & (0,1) & (4,5) & (4,2)\\
%\end{array}
%$$
%\endgroup
%In what follows we choose our generator to be $g_{MNT4}=(0,1)$.

%In what follows we want to compute type 2 pairings on our MNT4 curve. We therefore need to extract the subgroup $\mathbb{G}_1$ as well as $\mathbb{G}_2$ from the full $5$-torsion group. Since the order of MNT4 is a prime number, we already know from XXX\sme{add reference} that $\mathbb{G}_1$ is given by  $$\mathbb{G}_1=\{(0,1)\to (0,6)\to (4,2)\to (4,5) \to \mathcal{O}\}$$ 

%In type 2 pairings, the group $\mathbb{G}_2$ is defined by those elements $P$ of the full $5$-torsion group that are mapped to $7\cdot P$ under the Frobenius endomorphism XXX\sme{add reference}. Since $MNT4/\F_{7^4}$ only contains $2475$ elements, we can  loop through all elements, to find the full $5$-torsion group and extract all elements from $\mathbb{G}_2$:
%\begin{sagecommandline}
%sage: F7t.<t> = F7[]
%sage: F7_4.<u> = GF(7^4, name='u', modulus=t^4+t+1) # embedding degree is 4
%sage: MNT4 = EllipticCurve (F7_4,[4 ,1])
%sage: INF = MNT4(0) # point at infinity
%sage: for P in INF.division_points(5): # PI(P) == [q]P
%....:     if P.order() == 5: # exclude point at infinity
%....:         PiP = MNT4([a.frobenius() for a in P])
%....:         qP = 7*P
%....:         if PiP == qP:
%....:             print(P.xy())
%\end{sagecommandline}

%Choose $g_2=(2u^3 + 5u^2 + 4u + 2, 2u^3 + 3u + 5)$ as generator of $\mathbb{G}_2$, we get
%\begin{multline*}
%\mathbb{G}_2=\{ 
%(2u^3 + 5u^2 + 4u + 2, 2u^3 + 3u + 5) \to
%(5u^3 + 2u^2 + 3u + 6, 2u^2 + 3u) \to \\
%(5u^3 + 2u^2 + 3u + 6, 5u^2 + 4u) \to
%(2u^3 + 5u^2 + 4u + 2, 5u^3 + 4u + 2)\to
%\mathcal{O}\}
%\end{multline*}
%e.g. $[3]g_2= (5u^3 + 2u^2 + 3u + 6, 5u^2 + 4u)$.

%Having those groups we can do pairings. We choose the Weil pairing and invoke Sagemath. For example the Weil pairing between our two generators is
%$$
%e(g_1,g_2)= 5u^3 + 2u^2 + 6u
%$$
%\begin{sagecommandline}
%sage: g1 = MNT4([0,1])
%sage: g2 = MNT4(2*u^3 + 5*u^2 + 4*u + 2, 2*u^3 + 3*u + 5)
%sage: g1.weil_pairing(g2,5)
%\end{sagecommandline}
%The full pairing table can the be written as
%\begingroup
%    \fontsize{10pt}{10pt}\selectfont
    
% generate the table entries as:
% sage: for i in range(5):
% ....:     for j in range(5):
% ....:         p = (i*g1).weil_pairing((j*g2),5)
% ....:         print('e([',i,']g1,[',j,']g2=',p)         
    
    
%$$
%\begin{array}{c|lllll}
%e(\cdot,\cdot)    & \mathcal{O} & g_1            & [2]g_1         & [3]g_1         %& [4]g_1\\
%\hline
%\\
%      \mathcal{O} & 1           & 1              & 1              & 1              %& 1\\
%\\
%              g_2 & 1           & 5u^3+2u^2+6u   & 6u^3+5u^2+6    & 2u^3+u^2+2u+3  %& u^3+6u^2+6u+4\\
%\\
%\left[2\right]g_2 & 1           & 6u^3+5u^2+6    & u^3+6u^2+6u+4  & 5u^3+2u^2+6u   %& 2u^3+u^2+2u+3\\
%\\
%\left[3\right]g_2 & 1           & 2u^3+u^2+2u+3  & 5u^3+2u^2+6u   & u^3+6u^2+6u+4  & 6u^3+5u^2+6\\
%\\
%\left[4\right]g_2 & 1           & u^3+6u^2+6u+4  & 2u^3+u^2+2u+3  & 6u^3+5u^2+6    & 5u^3+2u^2+6u\\
%\end{array}
%$$
%\endgroup

%\paragraph{MNT6}
%For our MNT6 curve, we can choose $x=1$. Then $q=5$ and if we choose $t= 1 + 2x $ then $r= q + 1 - t = 5 + 1 + 1 = 7$. Therefore our MNT6 curve is a curve $y^2=x^3+ax+b$ defined over $\F_5$ that consists of $7$ points. 

%To construct the actual curve we could use the complex multiplication method again, but since the parameters $a$ and $b$ are from $\F_5$ there are only $24$ possibilities, we simply loop through all possible $a$'s and $b$'s and count the curve points until we find a curve that has $7$ rational points. We get
%$$
%y^2 = x^3 + 2x + 1
%$$
%defined over $\F_5$. Since $5= 4\cdot 1 + 1$, we know from theorem XXX that this curve has embedding degree $6$ and hence qualifies as a pen-and-paper pairing-friendly elliptic curve. 

%The curve has the following set of elements
%$$MNT6=\{(1,2)\to (3,3)\to (0,1)\to (0,4)\to (3,2)\to (1,3)\to \mathcal{O}\}$$
%The multiplication table is
%\begingroup
%    \fontsize{10pt}{10pt}\selectfont
%$$
%\begin{array}{c|ccccccc}
%\cdot & \mathcal{O} & (1,2) & (3,3) & (0,1) & (0,4) & (3,2) & (1,3)\\
%\hline
%\\
%\mathcal{O} & \mathcal{O} & (1,2) & (3,3) & (0,1) & (0,4) & (3,2) & (1,3)\\
%\\
%(1,2) & (1,2) & (3,3) & (0,1) & (0,4) & (3,2) & (1,3) & \mathcal{O}\\
%\\
%(3,3) & (3,3) & (0,1) & (0,4) & (3,2) & (1,3) & \mathcal{O} & (1,2)\\
%\\
%(0,1) & (0,1) & (0,4) & (3,2) & (1,3) & \mathcal{O} & (1,2) & (3,3)\\
%\\
%(0,4) & (0,4) & (3,2) & (1,3) & \mathcal{O} & (1,2) & (3,3) & (0,1)\\
%\\
%(3,2) & (3,2) & (1,3) & \mathcal{O} & (1,2) & (3,3) & (0,1) & (0,4)\\
%\\
%(1,3) & (1,3) & \mathcal{O} & (1,2) & (3,3) & (0,1) & (0,4) & (3,2)\\
%\end{array}
%$$
%\endgroup

%In what follows we choose our generator to be $g_{MNT6}=(1,2)$.

%In what follows we want to compute type 2 pairings on our MNT6 curve. We therefore need to extract the subgroup $\mathbb{G}_1$ as well as $\mathbb{G}_2$ from the full $7$-torsion group. Since the order of MNT6 is a prime number, we already know from XXX that $\mathbb{G}_1$ is given by
%$$\mathbb{G}_1=\{(1,2)\to (3,3)\to (0,1)\to (0,4)\to (3,2)\to (1,3)\to \mathcal{O}\}$$
%In type 2 pairings, the group $\mathbb{G}_2$ is defined by those elements $P$ of the full $7$-torsion group that are mapped to $5\cdot P$ under the Frobenius endomorphism XXX. Since $MNT6/\F_{5^6}$ contains $15680$ elements, we can still loop through all elements, to find the full $7$-torsion group and extract all elements from $\mathbb{G}_2$

%\begin{sagecommandline}
%sage: G.<x> = GF(5^6) # embedding degree is 6
%sage: MNT6 = EllipticCurve (G,[2 ,1])
%sage: INF = MNT6(0) # point at infinity
%sage: for P in INF.division_points(7): # PI(P) == [q]P
%....:     if P.order() == 7: # exclude point at infinity
%....:         PiP = MNT6([a.frobenius() for a in P])
%....:         qP = 5*P
%....:         if PiP == qP:
%....:             print(P.xy())
%\end{sagecommandline}

%\begin{multline*}
%\mathbb{G}_2=\{ 
%(x^3+2x^2+4x,x^5+2x^4+4x^3+3x^2+3)\to
%(x^5+4x^4+2x^3+3x^2+x+2,3x^4+2x^3+x)\to\\
%(4x^5+x^4+2x^3,3x^5+x^4+x^3+4x+4)\to
%(4x^5+x^4+2x^3,2x^5+4x^4+4x^3+x+1) \to\\
%(x^5+4x^4+2x^3+3x^2+x+2,2x^4+3x^3+4x)\to
%(x^3+2x^2+4x,4x^5+3x^4+x^3+2x^2+2)\to
%\mathcal{O}\}
%\end{multline*}
%We choose the generator $g_2 = (x^3+2x^2+4x,x^5+2x^4+4x^3+3x^2+3)$

%\begin{remark}
%Note however that our MNT6 curve discriminant $D=-16(4a^3 + 27 b^2)= -16(4\cdot 2^3 + 27\cdot 1^2)=-944$, while our MNT4 curve has discriminate XXX. Hence our example curves are not those guaranteed by theorem XXX. Those curve are both given by $y^2= x^3 + 2x +1$ over $\F_5$ and $\F_7$, respectively. However as both curves have the same defining equation, we rather choose examples that are visually distinguishable by their defining equations.




%\end{remark}



\chapter{Zk-Proof Systems}

Some philosophical stuff about compuational models for snarks. Bounded computability...

% https://docs.zkproof.org/reference.pdf

\section{Computational Models}
Proofs are the evidence of correctness of the assertions, and people can verify the cor-rectness by reading the proof. However, we obtain much more than the correctness itself:After you read one proof of an assertion, you know not only the correctness, but also why itis correct. Is it possible to solely show the correctness of an assertion without revealing theknowledge of proofs? It turns out that it is indeed possible, and this is the topic of today’slecture: Zero Knowledge Systems.
% from http://resources.mpi-inf.mpg.de/departments/d1/teaching/ss14/gitcs/notes6.pdf

\begin{example}[Generalized factorization snark]
\label{main_example_2_1}
As one of our major running examples we want to derive a zk-SNARK for the following generalized factorization problem: 

Given two numbers $a,b\in \mathbb{F}_{13}$, find two additional numbers $x,y\in \mathbb{F}_{13}$, such that
$$
(x\cdot y) \cdot a = b 
$$
and proof knowledge of those numbers, without actually revealing them.

Of course this example reduces to the classic factorization problem (over $\F_{13}$ by setting $y=1$)

This zero knowledge system deals with the following situation: "Given two publicly known numbers $a,b \in \mathbb{F}_{13}$ a proofer can show that they know two additional numbers $x,y\in \mathbb{F}_{13}$, such that $(x\cdot y) \cdot a = b$, without actually revealing $x$ or $y$." 

Of course our choice of what information to hide and what to reveal was completely arbitrary. Every other split would also be possible, but eventually gives a different problem. 

For example the task could be to not hide any of the variables.  Such 
a system has no zero knowledge and deals with verifiable computations: "A worker can proof that they multiplied three publicly known numbers $a,b,x \in \mathbb{F}_{13}$ and that the result is $z \in \mathbb{F}_{13}$, in such a way that no verifier has to repeat the computation."
\end{example}

\subsection{Formal Languages}
Roughly speaking a formal language is nothing but a set of words, that are strings of letters taken from some alphabet and formed according to some defining rules of that language. 

In computer science, formal languages are used for defining the grammar of programming languages in which the words of the language represent concepts that are associated with particular meanings or semantics. In computational complexity theory, decision problems are typically defined as formal languages, and complexity classes are defined as the sets of the formal languages that can be parsed by machines with limited computational power. 

\begin{definition}[Formal Language]
\label{def_formal_language}
 Let $\Sigma$ be a set and $\Sigma^*$ the set of all finite strings of elements from $\Sigma$. Then a \textbf{formal language} $L$ is a subset of $\Sigma^*$. The set $\Sigma$ is called the \textbf{alphabet} of $L$ and elements from $L$ are called \textbf{words}. The rules that specify which strings from $\Sigma^*$ belong to $L$ are called the \textbf{grammar} of $L$. 

In the context of proofing systems we often call words \textbf{statements}.
\end{definition}

\begin{example}[Generalized factorization snark]
\label{main_example_2_2}
Consider example \ref{main_example_2_1} again. Definition \ref{def_formal_language} is not quite suitable yet to define the example, since there is not distinction between public input and private input.

However if we assume for the moment that the task in example \ref{main_example_2_1} is to simply find $a,b,x,y\in \F_{13}$ such that that $x\cdot y\cdot a\cdot =b$, then we can define the entire solution set as a language $L_{factor}$ over the alphabet $\Sigma = \F_{13}$. We then say that a string $w\in \Sigma^*$ is a statement in our language $L_{factor}$ if and only if $w$ consists of 4 letters $w_1,w_2,w_3,w_4$ that satisfy the equation $w_1\cdot w_2\cdot w_3 =w_4$.
\end{example}

\begin{example}[Binary strings] If we take the set $\{0,1\}$ as our alphabet $\Sigma$ and imply no rules at all to form words in this set. Then our language $L$ is the set $\{0,1\}^*$ of all finite binary strings. So for example $(0,0,1,0,1,0,1,1,0)$ is a word in this language.
\end{example}

\begin{example}[Programing Language]
\end{example}

\begin{example}[Compiler]
\end{example}



As we have seen in general not all strings from an alphabet are words in a language. So an important question is, weather a given string belongs to a language or not. 

% https://www.claymath.org/sites/default/files/pvsnp.pdf
\begin{definition}[Relation, Statement, Instance and Witness] Let $\Sigma_I$ and $\Sigma_W$ be two alphabets. Then the binary relation $R\subset \Sigma_I^* \times \Sigma_W^*$ is called a \textbf{checking relation} for the language 
$$
L_R := \{(i,w) \in \Sigma_I^* \times \Sigma_W^*\;| R(i,w)\; \}
$$ 
of all \textbf{instances} $i\in \Sigma_I^*$ and \textbf{witnesses} $i\in \Sigma_I^*$, such that the \textbf{statement} $(i,w)$ satisfies the checking relation.
\end{definition}
\begin{remark}
% https://docs.zkproof.org/reference.pdf
To summarize the definition, a statement is nothing but a membership claim of the form $x\in L$. So statements are really nothing but strings in an alphabet that adhere to the rules of a language. 

However in the context of checking relations, there is another interpretations in terms of a knowledge claim of the form "In the scope of relation R, I know a witness for instance x." This is of particular importance in the context of zero knowledge proofing systems, where the instance represents public knowledge, while the witness represents the data that is hidden (the zero-knowledge part). 

For some cases, the knowledge and membership types of statements can be informally considered interchangeable, but formally there are technical reasons to distinguish between the two notions (See for example XXX
% https://docs.zkproof.org/reference.pdf
) 
\end{remark}
\begin{example}[Generalized factorization snark]
\label{main_example_2_3}
Consider example \ref{main_example_2_1} and our associate formal language \ref{main_example_2_2}. We can define another language $L_{zk-factor}$ for that example by defining the alphabet $\Sigma_I \times \Sigma_W$ to be $\F_{13} \times \F_{13}$ and the checking relation $R_{zk-factor}$ such that
$R(i,w)$ holds if and only if instance $i$ is a two letter string $i=(a,b)$ and witness $w$ is a two letter string $w=(x,y)$, such that the equation $x\cdot y \cdot a = b$ holds. 

So to summarize four elements $x,y,a,b\in \F_{13}$ form a statement 
$((x,y),(a,b))$ in $L_{zk-factor}$ with instance $(a,b)$ and witness $x,y$, precisely if, given $a$ and $b$, the values $x$ and $y$ are a solution to the generalized factorization problem $x\cdot y \cdot a = b$.
\end{example}




\begin{example}[SHA256 relation]
ssss
\end{example}

As the following example shows checking relations and their languages are quite general and able to express in particular the class of all terminating computer programs:
\begin{example}[Computer Program] Let $A$ be a terminating algorithm that transforms a binary string of inputs in finite execution steps into a binary output string. We can then interpret $A$ as a map 
$$
A :\{0,1\}^* \to \{0,1\}^*
$$
Algorithm $A$ then defines a relation
$R\subset \{0,1\}^* \times \{0,1\}^*$ in the following way: instance string $i\in \{0,1\}^*$ and witness string $w\in \{0,1\}^*$ satisfy the relation $R$, that is $R(i,w)$, if and only if $w$ is the result of algorithm $A$ executed on input instance $i$.
\end{example}

\subsection{Circuits} 
\begin{definition}[Circuits] Let $\Sigma_I$ and $\Sigma_W$ be two alphabets. Then a directed, acyclic graph $C$ is called a \textbf{circuit} over $\Sigma_I \times \Sigma_W$, if the graph has an ordering and every node has a label in the following way:
\begin{itemize}
\item Every source node (called input) has a letter from $\Sigma_I \times \Sigma_W$ as label.
\item Every sink node (called output) has a letter from $\Sigma_I \times \Sigma_W$ as label.
\item Every other node (called gate) with $j$ incoming edges has a label that consist of a function $f: \left(\Sigma_I \times \Sigma_W\right)^j \to \Sigma_I \times \Sigma_W$.
\end{itemize}
\end{definition}
\begin{remark}[Circuit-SAT] Every circuit with $n$ input nodes and $m$ output nodes can be seen a function that transforms strings of size $n$ from $\Sigma_I \times \Sigma_W$ into strings of size $m$ over the same alphabet. The transformation is done by sending the strings from a node along the outgoing edges to other nodes. If those nodes are gates, then the string is transformed according to the label.

By executing the previous transformation, every node of a circuit has an associated letter from $\Sigma_I \times \Sigma_W$ and this defines a checking relation over $\Sigma_I^* \times \Sigma_W^*$. To be more precise, let $C$ be a circuit with $n$ nodes and $(i,w) \in \Sigma_I^j \times \Sigma_W^k$ a string. Then $R_C(i,w)$ iff THE CIRCUIT IS SATISFIED WHEN ALL LABELS ARE ASSOCIATED TO ALL NODES IN THE CIRCUIT.... BUT MORE PRECISE

MODULO ERRORS. TO BE CONTINUED.....

An Assignment associates field elements to all edges (indices) in an algebraic circuit. An Assignment is valid, if the field element arise from executing the circuit. Every other assignment is invalid.

The checking relation for circuit-SAT then is satidfied if valid asignment (TODO: THE WITNESS/INSTANCE SPLITTING)

Valid assignments are proofs for proper circuit execution.
\end{remark}



So to summarize, algebraic circuits (over a field $\mathbb{F}$) are directed acyclic graphs, that express arbitrary, but bounded computation. Vertices with only outgoing edges (leafs, sources) represent inputs to the computation, vertices with only ingoing edges (roots, sinks) represent outputs from the computation and internal vertices represent field operations (Either addition or multiplication). It should be noted however that there are many circuits that can represent the same laguage...

Circuits have a notion of execution, where input values are send from leafs along edges, through internal vertices to roots.

\begin{remark}
Algebraic circuits are usually derived by  Compilers, that transform  higher languages to circuits. An example of such a compiler is XXX. Note: Different Compiler give very different circuit representations and Compiler optimization is important.
\end{remark}


\begin{example}[Generalized factorization snark]
\label{main_example_2_4}
Consider our generalized factorization example \ref{main_example_2_1} with associated language \ref{main_example_2_3}.

To write this example in circuit-SAT, consider the following function 
\[
f:\mathbb{F}_{13}\times\mathbb{F}_{13}\times\mathbb{F}_{13}\to\mathbb{F}_{13};(x_{1},x_{2},x_{3})\mapsto(x_{1}\cdot x_{2})\cdot x_{3}
\]

A valid circuit for $f:\mathbb{F}_{11}\times\mathbb{F}_{11}\times\mathbb{F}_{11}\to\mathbb{F}_{11};(x_{1},x_{2},x_{3})\mapsto(x_{1}\cdot x_{2})\cdot x_{3}$ is given by:

\[
\xymatrix{\star\ar^{in_1}[dr] &  & \star\ar_{in_2}[dl]\\
 & \star_{m_1}\ar^{mid_1}[drr] &   & & \star\ar_{in_3}[dl]\\
  &  &  & \star_{m_2}\ar_{out_1}[d]\\
  &  &  & \star
}
\]
with edge-index set $I:=\{in_{1},in_{2},in_{3},mid_{1},out_{1}\}$.

To given a valid assignment, consider the set $I_{valid}:=\{in_{1},in_{2},in_{3},mid_{1},out_{1}\} = \{2,3,4,6,10\}$

\[
\xymatrix{\star\ar^{2}[dr] &  & \star\ar_{3}[dl]\\
 & \star_{m_1}\ar^{6}[drr] &   & & \star\ar_{4}[dl]\\
  &  &  & \star_{m_2}\ar_{10}[d]\\
  &  &  & \star
}
\]
Appears from multiplying the input values at $m_1$, $m_2$ in $\mathbb{F}_{13}$, hence by executing the circuit.

Non valid assignment: $I_{err}:=\{in_{1},in_{2},in_{3},mid_{1},out_{1}\} =\{2,3,4,7,8\}$
\[
\xymatrix{\star\ar^{2}[dr] &  & \star\ar_{3}[dl]\\
 & \star_{m_1}\ar^{7}[drr] &   & & \star\ar_{4}[dl]\\
  &  &  & \star_{m_2}\ar_{8}[d]\\
  &  &  & \star
}
\]
Can not appear from multiplying the input values at $m_1$, $m_2$ in $\mathbb{F}_{13}$

To match the requirements of the inital task \ref{main_example_2_1}, we have to split the statement into instance and witness. So given index set $I:=\{in_{1},in_{2},in_{3},mid_{1},out_{1}\}$, we assume that every step in the computation other then $in_3$ and $out_1$ are part of the witness. So we choose:
\begin{itemize}
\item Instance $S=\{in_3, out_1\}$. 
\item Witness $W=\{in_1, in_2, mid_{1}\}$.
\end{itemize}
\end{example}

\begin{example}[Baby JubJub for BLS6-6]

\end{example}

\begin{example}[ECDH as a circuit]
over BLS6
\end{example}

\begin{example}[BLS Signature]
example of one layer recursion over MNT4 and MNT6
\end{example}


\begin{example}[Boolean Circuits]

\end{example}

\begin{example}[Algebraic (Aithmetic) Circuits]

\end{example}

Any program  can be reduced to  an arithmetic circuit  (a circuit that contains only addition and multiplication gates). A particular reduction can be found for example in [BSCG+13]



\subsection{Rank-1 Constraint Systems}

\begin{definition}[Rank-1 Constraint system]
Let $\F$ be a Galois field, $i,j,k$ three numbers and $A$, $B$ and $C$ three $(i+j+1) \times k$ matrices with coefficients in $\F$. Then any vector $x= (1,\phi,w)\in \F^{1+i+j}$ that satisfies the \textbf{rank-1 constraint system} (R1CS)
$$
Ax \odot Bx = Cx
$$
(where $\odot$ is the Hadamard/Schur product) is called a \textbf{statement} of that system, with \textbf{instance} $\phi$ and \textbf{witness} $w$.

We call $k$ the \textbf{number of constraints}, $i$ the \textbf{instance} size and $j$ the \textbf{witness} size.
\end{definition}

\begin{remark} Any Rank-1 constraint system defines a formal language in the following way: Consider the alphabets $\Sigma_I:= \F$ and $\Sigma_W:\F$. Then a checking relation $R_{R1CS} \subset \Sigma_I^i \times \Sigma_W^j \subset \Sigma_I^* \times \Sigma_W^*$ is defined by 
$$
R_{R1CS}(i,w) \Leftrightarrow (i,w)\text{ satisfies the R1CS}
$$
As shown in XXX such a checking relation defines a formal language. We call this language \textbf{R1CS satisfiability}.
\end{remark}

\begin{example}[Generalized factorization snark]
\label{main_example_2_4}
Defining the 5-dimensional affine vector $w =(1,in_1,in_2,in_3,m_1,out_1)$ for $in_1,in_2,in_3,m_1,out_1 \in \F_{13}$ and the $6\times ?$-matrices
$$
\begin{array}{lcr}
A = \begin{pmatrix}
0 & 1 & 0 & 0 & 0 & 0 \\ 
0 & 0 & 0 & 0 & 1 & 0
\end{pmatrix}, &
B = \begin{pmatrix}
0 & 0 & 1 & 0 & 0 & 0 \\ 
0 & 0 & 0 & 1 & 0 & 0
\end{pmatrix}, &
C = \begin{pmatrix}
0 & 0 & 0 & 0 & 1 & 0 \\ 
0 & 0 & 0 & 0 & 0 & 1
\end{pmatrix} 
\end{array}
$$
We can instantiate the general R1CS equation $Aw \odot Bw = Cw$ as
$$
\begin{pmatrix}
0 & 1 & 0 & 0 & 0 & 0 \\ 
0 & 0 & 0 & 0 & 1 & 0
\end{pmatrix} 
\begin{pmatrix}
1\\ in_1 \\ in_2 \\ in_3 \\ m_1 \\ out_1 
\end{pmatrix}\odot 
\begin{pmatrix}
0 & 0 & 1 & 0 & 0 & 0 \\ 
0 & 0 & 0 & 1 & 0 & 0
\end{pmatrix} 
\begin{pmatrix}
1\\ in_1 \\ in_2 \\ in_3 \\ m_1 \\ out_1 
\end{pmatrix} =
\begin{pmatrix}
0 & 0 & 0 & 0 & 1 & 0 \\ 
0 & 0 & 0 & 0 & 0 & 1
\end{pmatrix} 
\begin{pmatrix}
1\\ in_1 \\ in_2 \\ in_3 \\ m_1 \\ out_1 
\end{pmatrix}
$$
So evaluating all three matrix products and the Hadarmat prodoct we get two constraint equations
$$
\begin{array}{rcl}
in_1 \cdot in_2  &= & m_1 \\
m_1 \cdot in_3  &= & out_1 \\
\end{array}
$$
\end{example}
So from the way this R1CS is constructed, we know that whatever the underlying field $\F$ is, the only solutions to this equations are
$$
\{(0,0,0), (0,1,0), (1,0,0), (1,1,1)\}
$$

\subsection{Quadratic Arithmetic Programs}
As shown by [Pinocchio] rank-1 constraint systems can be transformed into so called quadratic  arithmetic  programs  assuming $\F$.

taken from the pinocchio paper. For proving arithmetic circuit-sat.  Given a R1CS QAPs transform potential solution vectors into two polynomials $p$ and $t$, such that $p$ is divisible by $t$ if and only if the vector is a solution to the R1CS. 

They are major building blocks for \textbf{succinct} proofs, since with high probability, the divisibility check can be performed in a single point of those polynomials. So computationally expensive polynomial division check is reduced TO WHAT? (IN FIELDS THERE IS ALWAYS DIVISIBILITY) 
% https://courses.cs.ut.ee/MTAT.07.022/2013_fall/uploads/Main/alisa-report

\begin{definition}[Quadratic Arithmetic Program]
Assume we have a Galois field $\F$, three numbers $i,j,k$ as well as three $(i+j+1) \times k$ matrices $A$, $B$ and $C$  with coefficients in $\F$ that define the R1CS
$Ax \odot Bx = Cx $ for some statement $x=(1,i,w)$ and let $m_1,\ldots,m_k\in \F$ be arbitrary field elements. 

Then a \textbf{quadratic arithmetic program} of the R1CS is the following set of polynomials over $\F$
$$
QAP = \left\{t\in \F[x],\left\{a_h,b_h,c_h\in \F[x]\right\}_{h=1}^{i+j+1}\right\}
$$
where $t(x) := \Pi_{l=1}^k (x- m_l)$ is a polynomial f degree $k$, called the \textbf{target polynomial} of the QAP and $a_h(x)$, $b_h(x)$ as well as $c_h(x)$ are the unique degree $k-1$ polynomials that are defined by the equations
$$
\begin{array}{lllr}
a_h(m_l)=A_{h,l} & b_h(m_l)=B_{h,l} & c_h(m_l)=C_{h,l} & h= 1, \ldots , i+j+1, l=1,\ldots,k 
\end{array}
$$  
\end{definition}
The major point is that R1CS-sat can be reformulated into the divisibility of a polynomials defined by any QAP.
\begin{theorem}
Assume that an R1CS and an associated QAP as defined in XXX are given. Then the affine vector $y=(1,i,w)$ is a solution to the R1CS, if and only if the polynomial
$$
p(x) = \left(\sum y_h\cdot a_h(x)\right)\cdot \left(\sum y_h\cdot b_h(x)\right)  - \sum y_h\cdot c_h(x) 
$$
is divisible by the target polynomial $t$.
\end{theorem}

The polynomials $a_h$, $b_h$ and $c_h$ are uniquely defined by the equations in XXX. However to actually compute them we need some algorithm like the Langrange XXX from XXX.

\begin{example}[Generalized factorization snark]
In this example we want to transform the R1CS from example \ref{main_example_2_3} into an associated QAP.

We start by choosing an arbitrary field element for every constraint in the R1CS, since we have $2$ constraints we choose $m_{1}=5$ and $m_{2}=7$

With this choice we get the target polynomial $t(x)=(x-m_1)(x-m_2)= (x-5)(x-7)= (x+8)(x+6)= x^2 + x +9$.

Since our statement has structure $w=(1, in_1,in_2,in_3,m_1,out_1)$ we have to compute the following degree $1$ polynomials

$\{a_{c},a_{in_{1}},a_{in_{2}},a_{in_{3}},a_{mid_{1}},a_{out}\}$
$\{b_{c},b_{in_{1}},b_{in_{2}},b_{in_{3}},b_{mid_{1}},b_{out}\}$
$\{c_{c},c_{in_{1}},c_{in_{2}},c_{in_{3}},c_{mid_{1}},c_{out}\}$

\item Apply QAP rule XXX to the $a_{k\in I}$ polynomials gives
$$
\begin{array}{llllll}
a_{c}(5)=0, & a_{in_{1}}(5)=1, & a_{in_{2}}(5)=0, & a_{in_{3}}(5)=0, & a_{mid_{1}}(5)=0, & a_{out}(5)=0 \\
a_{c}(7)=0, & a_{in_{1}}(7)=0, & a_{in_{2}}(7)=0, & a_{in_{3}}(7)=0, & a_{mid_{1}}(7)=1, & a_{out}(7)=0\\
\\
b_{c}(5)=0, & b_{in_{1}}(5)=0, & b_{in_{2}}(5)=1, & b_{in_{3}}(5)=0, & b_{mid_{1}}(5)=0, & b_{out}(5)=0 \\
b_{c}(7)=0, & b_{in_{1}}(7)=0, & b_{in_{2}}(7)=0, & b_{in_{3}}(7)=1, & b_{mid_{1}}(7)=0, & b_{out}(7)=0\\
\\
c_{c}(5)=0, & c_{in_{1}}(5)=0, & c_{in_{2}}(5)=0, & c_{in_{3}}(5)=0, & c_{mid_{1}}(5)=1, & c_{out}(5)=0 \\
c_{c}(7)=0, & c_{in_{1}}(7)=0, & c_{in_{2}}(7)=0, & c_{in_{3}}(7)=0, & c_{mid_{1}}(7)=0, & c_{out}(7)=1
\end{array}
$$

Since our polynomials are of degree $1$ only we don't have to invoke Langrange method but can deduce the solutions right away. 

Polynomials are defined on the two values $5$ and $7$ here.
Linear Polynomial $f(x)=m\cdot x + b$ is fully determined by this. Derive the general equation:
\begin{itemize}                        
\item  $5m+b=f(5)$  and $7m+b=f(7)$  
\item  $b=f(5)-5m$ and  $b=f(7)-7m$   
\item  $b=f(5)+8m$ and  $b=f(7)+6m$  
\item  $f(5)+8m=f(7)+6m$              
\item  $8m-6m=f(7)-f(5)$               
\item  $2m=f(7)+ 12f(5)$              
\item  $7\cdot 2m=7(f(7)+12f(5))$              
\item  $m=7(f(7)+12f(5))$ 
\item             
\item  $b=f(5)+8m$                   
\item  $b=f(5)+8\cdot(7(f(7)+12f(5)))$
\item  $b=f(5)+4(f(7)+12f(5))$ 
\item  $b=f(5)+4f(7)+9f(5)$ 
\item  $b= 10f(5)+4f(7)$ 
\end{itemize}
Gives the general equation: $f(x)=7(f(7)+12f(5))x+10f(5)+4f(7)$

For $a_{in_1}$ the computation looks like this:
\begin{itemize}
\item $ a_{in_{1}}(x) = 7(a_{in_{1}}(7)+12a_{in_{1}}(5))x+ 
10a_{in_{1}}(5)+4a_{in_{1}}(7)=$
\item $7(0 + 12\cdot 1)x+ 
10\cdot 1 +4\cdot 0 =$
\item $7\cdot 12 x + 10=$
\item $6x+10$
\end{itemize}
\begin{itemize}
\item $ a_{mid_{1}}(x) = 7(a_{mid_{1}}(7)+12a_{mid_{1}}(5))x+ 
10a_{mid_{1}}(5)+4a_{mid_{1}}(7)=$
\item $7(1 + 12\cdot 0)x+ 10\cdot 0 +4\cdot 1=$
\item $7\cdot 1x +4=$
\item $7x+4 $
\end{itemize}


\begin{tabular}{|l|l|l|}\hline 
$a_{c}(x)=0 $ &$ b_{c}(x)=0   $ & $c_{c}(x)=0$ \tabularnewline\hline 
$a_{in_{1}}(x)=6x+10 $ &$ b_{in_{1}}(x)=0   $ & $c_{in_1}(x)=0$ \tabularnewline\hline 
$a_{in_{2}}(x)=0    $ &$ b_{in_{2}}(x)=6x+10$ & $c_{in_2}(x)=0$ \tabularnewline\hline 
$a_{in_{3}}(x)=0    $ &$ b_{in_{3}}(x)=7x+4$ & $c_{in_{3}}(x)=0$ \tabularnewline\hline 
$a_{mid_{1}}(x)=7x+4$ &$ b_{mid_{1}}(x)=0  $ & $c_{mid_{1}}(x)=6x+10$ \tabularnewline\hline 
$a_{out}(x)=0       $ &$ b_{out}(x)=0      $ & $c_{out}(x)=7x+4$ \tabularnewline\hline 
\end{tabular}
This gives the quadratic arithmetic program for our generalized factorization snark as
$$QAP=\{x^{2}+x+9,\{0,6x+10,0,0,7x+4,0\},\{0,0,6x+10,7x+4,0,0\},\{0,0,0,0,6x+10,7x+4\}\}$$

Now as we recall, the main point for using QAPs in snarks is the fact, that solutions to R1CS are in 1:1 correspondence to the divisibility of a polynomial $p$, constructed from a R1CS solution and the polynomials of the QAP and the target polynomial.

So lets see this in our example. We already know from example XXX, that 
$I=\{1,2,3,4,6,11\}$ is a solution to the R1CS XXX of our problem. To see how this translates to polyinomial divisibility we compute the polynomial $p_I$ by
\begin{align*}
p_I(x)& = (\sum_{h\in |I|} I_h\cdot a_h(x))\cdot 
(\sum_{h\in |I|} I_h\cdot b_h(x)) - 
(\sum_{h\in |I|} I_h\cdot c_h(x)) \\
= & (2(6x+10)+6(7x+4))\cdot(3(6x+10)+4(7x+4))-(6(6x+10)+11(7x+4)) \\
= & ((12x+7)+(3x+11))\cdot((5x+4)+(2x+3))-((10x+8)+(12x+5)) \\
= & (2x+5)\cdot(7x+7)-(9x) \\
= & (x^{2}+2\cdot7x+5\cdot7x+5\cdot7)-(9x) \\
= & (x^{2}+x+9x+9)-(9x) \\
= & x^{2}+x+9
\end{align*}
And as we can see in this particular example $p_I(x)$ is equal to the target polynomial $t(x)$ and hence it is divisible by $t$ with $p/t=1$.

To give a counter example we already know from XXX that $I=\{1,2,3,4,8, 2\}$ is not a solution to our R1CS. To see how this translates to polyinomial divisibility we compute the polynomial $p_I$ by
\begin{align*}
p_I(x)& = (\sum_{h\in |I|} I_h\cdot a_h(x))\cdot 
(\sum_{h\in |I|} I_h\cdot b_h(x)) - 
(\sum_{h\in |I|} I_h\cdot c_h(x)) \\
= & (2(6x+10)+6(7x+4))\cdot(3(6x+10)+4(7x+4))-(6(6x+10)+11(7x+4)) \\
= & 8x^{2}+11x+3
\end{align*}
This polynomial is not divisible by the target polynomial $t$ since
Not divisible by $t$: $(8x^{2}+11x+3)/(x^{2}+x+9) =8+\frac{3x+8}{x^{2}+x+9} $
\end{example}

\subsubsection{Boolean Algebra} 
% implementations can be found here: https://github.com/filecoin-project/zexe/tree/master/snark-gadgets/src/bits

Sometimes it is necessary to assume that a statement describes boolean variables. However by definition the alphabet of a statement is a finite field, which is often the scalar field of a large prime order cyclic group. So developers need a way to simulate boolean algebra inside other finite fields.

The most common way to do this, is to interpret the additive and multiplicate neutral element $\{0,1\}\subset F$ as boolean values. This is convinient because they are defined in any field. 

\paragraph{Boolean Constraint}
So when a developer needs boolean variables as part of their statement, a R1CS is required on those variables, that enforces the variable to be either $1$ or $0$. So to "constrain a field element $x\in \F$ to be $1$ or $0$ what we need is a system of equation $(A_ix)\cdot (B_ix) = C_ix$ for some $A_i,B_i,C_i\in \F$, such that the only possible solutions for $x$ are $0$ or $1$.
As it turns out such a system can be realized by a single equation
$x \cdot (1-x) =0$
We see that indeed $0$ and $1$ are the only solutions here, since for the right side to be zero, at least one factor on the left side needs to be zero and this only happens for $0$ and $1$. 

So now that we have found a correct equation for a boolean constrain, we have to translate it into the associated R1CS format, which is given by 
$$
\begin{pmatrix}0 & 1 \end{pmatrix} \begin{pmatrix} 1 \\ x \end{pmatrix}\odot
\begin{pmatrix}1 & -1 \end{pmatrix} \begin{pmatrix} 1 \\ x \end{pmatrix} =
\begin{pmatrix}0 & 0 \end{pmatrix} \begin{pmatrix} 1 \\ x \end{pmatrix}
$$
So we get $w = \begin{pmatrix} 1 \\ x\end{pmatrix}$ as well as
$A=\begin{pmatrix}0 & 1\end{pmatrix}$, $B=\begin{pmatrix}1 & -1\end{pmatrix}$ and $C=\begin{pmatrix}0 & 0\end{pmatrix}$.

Once field elements are boolean constraint, we need constraints that are able to enforce boolean algebra on them. We therefore give constraints for the functionally complete set of Boolean operators give by $AND$ and $NOT$. As all other boolean operations can be constructed from $AND$ and $NOT$, this sufficies. However in actual implementations it is of high importance to limit the number of constraints as much as possible. In reality it is therefor advantageous to implement all logic operators in constraints.   

\paragraph{AND-constraints} Given three field elements $x,y,z\in\F$ that represent boolean variables, we want to find a R1CS, such that $w=(1,x,y,z)$ satisfies the constraint system if and only if $x\; AND \; y =z$. 

So first we have to constrain $x$, $y$ and $z$ to be boolean as explained in XXX. The next thin is we need to find a R1CS that enforces the $AND$ logic. We can simply choose $x\cdot y =z$, since (for boolean constraint values) $x\cdot y$ equals $1$ if and only if both $x$ and $y$ are $1$.  

Now that we have found a correct equation for a boolean constrain, we have to translate it into the associated R1CS format, which is given by 
$$
\begin{pmatrix}0 & 1 & 0 & 0 \end{pmatrix} \begin{pmatrix} 1 \\ x \\ y \\ z \end{pmatrix}\odot
\begin{pmatrix}0 & 0 & 1 & 0 \end{pmatrix} \begin{pmatrix} 1 \\ x \\ y \\ z \end{pmatrix} =
\begin{pmatrix}0 & 0 & 0 & 1 \end{pmatrix}\begin{pmatrix} 1 \\ x \\ y \\ z \end{pmatrix}
$$
Combining this R1CS with the required fthree boolean constraints for $x$, $y$ and $z$ we get
$$
\begin{pmatrix}
0 & 1 & 0 & 0 \\
\hline
0 & 1 & 0 & 0 \\
0 & 0 & 1 & 0 \\
0 & 0 & 0 & 1 
\end{pmatrix} \begin{pmatrix} 1 \\ x \\ y \\ z \end{pmatrix}\odot
\begin{pmatrix}
0 & 0 & 1 & 0 \\
\hline
1 & -1 & 0 & 0 \\
1 & 0  & -1 & 0 \\
1 & 0 & 0 & -1 
\end{pmatrix} \begin{pmatrix} 1 \\ x \\ y \\ z \end{pmatrix} =
\begin{pmatrix}
0 & 0 & 0 & 1 \\
\hline
0 & 0 & 0 & 0 \\
0 & 0 & 0 & 0 \\
0 & 0 & 0 & 0 
\end{pmatrix}\begin{pmatrix} 1 \\ x \\ y \\ z \end{pmatrix}
$$
So from the way this R1CS is constructed, we know that whatever the underlying field $\F$ is, the only solutions to this equations are
$$
\{(0,0,0), (0,1,0), (1,0,0), (1,1,1)\}
$$
which is the set of all $(x,y,z)\in\{0,1\}^3$ such that $x\; AND\; y = z$.
\paragraph{NOT constraint}
Given two field elements $x,y\in\F$ that represent boolean variables, we want to find a R1CS, such that $w=(1,x,y)$ satisfies the constraint system if and only if $x=\lnot y$. 

So again we have to constrain $x$ and $y$ to be boolean as explained in XXX. The next think is we need to find a R1CS that enforces the $NOT$ logic. We can simply choose $(1-x) =y$, since (for boolean constraint values) this enforces that $y$ is always the boolean opposite of $x$. 

Now that we have found a correct equation for a boolean constrain, we have to translate it into the associated R1CS format, which is given by 
$$
\begin{pmatrix}1 & -1 & 0 \end{pmatrix} \begin{pmatrix} 1 \\ x \\ y \end{pmatrix}\odot
\begin{pmatrix}1 & 0 & 0 \end{pmatrix} \begin{pmatrix} 1 \\ x \\ y \end{pmatrix} =
\begin{pmatrix}0 & 0 & 1 \end{pmatrix}\begin{pmatrix} 1 \\ x \\ y \end{pmatrix}
$$
So actually we wrote the linear equation $1-x=y$ like $(1-x)\cdot 1 = y$ and translated that into the matrix equation.

Combining this R1CS with the required fthree boolean constraints for $x$, $y$ and $z$ we get
$$
\begin{pmatrix}
1 & -1 & 0 \\
\hline
0 & 1 & 0 \\
0 & 0 & 1
\end{pmatrix} \begin{pmatrix} 1 \\ x \\ y \end{pmatrix}\odot
\begin{pmatrix}
1 & 0 & 0 \\
\hline
1 & -1 & 0 \\
1 & 0  & -1 
\end{pmatrix} \begin{pmatrix} 1 \\ x \\ y \end{pmatrix} =
\begin{pmatrix}
0 & 0 & 1 \\
\hline
0 & 0 & 0 \\
0 & 0 & 0 \\
\end{pmatrix}\begin{pmatrix} 1 \\ x \\ y \end{pmatrix}
$$
So from the way this R1CS is constructed, we know that whatever the underlying field $\F$ is, the only solutions to this equations are
$$
\{(0,1), (1,0)\}
$$
which is the set of all $(x,y)\in\{0,1\}^2$ such that $x=\lnot y$.

EXERCISE: DO OR; XOR; NAND

More complicated logical constraints can then be optained by combining all sub-R1CS together. For example if the task is to enforce $(in_1\; AND \lnot in_2 ) AND in_3 = out_1$ we first apply the FLATTENING technique from XXX, which gives is
$$
\begin{array}{lcr}
\lnot in_2 &=& mid_1\\
in_1\; AND \; mid_1 &=& mid_2\\
mid_2 \; AND \; in_3 &=& out_1
\end{array}
$$
So we have the statement $w=(1,in_1,in_2,in_3, mid_1, mid_2,out_1)$, $6$ boolean constraints for the variables, $2$ constraints for the $2$ $AND$ operations and $1$ constraint for the $NOT$ operation.
\subsubsection{Conditional branching}

\subsubsection{UintX}
As we know circuits are not defined over integers but over finite fields instead. We therefore have no notation of integers in circuits. However on computers we also not use integers natively but Uint's instead.

\subsubsection{Twisted Edwards curves}
Sometimes it required to do elliptic curve cryptography "inside of a circuit". This means that we have to implement the algebraic operations (addition, scalar multiplication) of an elliptic curve as a R1CS. To do this efficiently the curve that we want to implement must be defined over the same base field as the field that is used in the R1CS. 

% implmentations https://github.com/iden3/circomlib/blob/master/circuits/babyjub.circom

\begin{example}
So for example when we consider an R1CS over the field $\F_{13}$ as we did in example XXX, then we need a curve that is also defined over $\F_{13}$. Moreover it is advantegous to use a (twisted) Edwards curve inside a circuit, as the addition law contains no branching (See XXX). As we have seen in XXX our Baby-Jubjub curve is an Edwards curve defined over $\F_{13}$. So it is well suited for elliptic curve cryptography in our pend and paper examples
\end{example}

\paragraph{Twisted Edwards curves constraints} As we have seen in XXX, an Edwards curve over a finite field $F$ is the set of all pairs of points $(x,y)\in \F\times \F$, such that $x$ and $y$ satisfy the equation $a\cdot x^2+y^2= 1+d\cdot x^2y^2$. 

We can interpret this equation as a constraint on $x$ and $y$ and rewrite it as a R1CS by applying the flattenin technique from XXX.
$$
\begin{array}{lcr}
x \cdot x &=& x\_sq\\
y \cdot y &=& y\_sq\\
x\_sq \cdot y\_sq &=& xy\_sq\\
(a\cdot x\_sq+y\_sq)\cdot 1 &=& 1+d\cdot xy\_sq
\end{array}
$$
So we have the statement $w=(1,x,y,x\_sq, y\_sq, xy\_sq)$ and we need 4 constraints to enforce that $x$ and $y$ are points on the Edwards curve $x^2+y^2= 1+d\cdot x^2y^2$. Writing the constraint system in matrix form, we get:
\begingroup
    \fontsize{9pt}{9pt}\selectfont
$$
\begin{pmatrix}
0 & 1 & 0 & 0 & 0 & 0 \\
0 & 0 & 1 & 0 & 0 & 0 \\
0 & 0 & 0 & 1 & 0 & 0 \\
0 & 0 & 0 & a & 1 & 0 
\end{pmatrix} \begin{pmatrix} 1 \\ x \\ y \\ x\_sq \\ y\_sq \\ xy\_sq \end{pmatrix}\odot
\begin{pmatrix}
0 & 1 & 0 & 0 & 0 & 0 \\
0 & 0 & 1 & 0 & 0 & 0 \\
0 & 0 & 0 & 0 & 1 & 0 \\
1 & 0 & 0 & 0 & 0 & 0 
\end{pmatrix}  \begin{pmatrix} 1 \\ x \\ y \\ x\_sq \\ y\_sq \\ xy\_sq \end{pmatrix} =
\begin{pmatrix}
0 & 0 & 0 & 1 & 0 & 0 \\
0 & 0 & 0 & 0 & 1 & 0 \\
0 & 0 & 0 & 0 & 0 & 1 \\
1 & 0 & 0 & 0 & 0 & d 
\end{pmatrix} \begin{pmatrix} 1 \\ x \\ y \\ x\_sq \\ y\_sq \\ xy\_sq \end{pmatrix}
$$
\endgroup
EXERCISE: WRITE THE R1CS FOR WEIERSTRASS CURVE POINTS 
\begin{example}[Baby-JubJub]
Considering our pen and paper Baby JubJub curve over from XXX, we know that the curve is defined over $\F_{13}$ and that $(11,9)$ is a curve point, while $(2,3)$ is not a curve point. 

Starting with $(11,9)$, we can compute the statement $w=(1,11,9,4,3,12)$. Substituting this into the constraints we get
$$
\begin{array}{lcr}
11 \cdot 11 &=& 4\\
9 \cdot 9 &=& 3\\
4 \cdot 3 &=& 12\\
(1\cdot 4+3)\cdot 1 &=& 1+7\cdot 12
\end{array}
$$
which is true in $\F_{13}$. So our statement is indeed a valid assignment to the twisted Edwards curve constraining system.

Now considering the non valid point $(2,3)$, we can still come up with some kind of statement $w$ that will satisfy some of the constraints. But fixing $x=2$ and $y=3$, we can never satisfy all constraints. For example $w=(1,2,3,4,9,10)$ will satisfy the first three constraints, but the last constrain can not be satisfied. Or $w=(1,2,3,4,3,12)$ will satisfy the first and the last constrain, but not the others.
\end{example}
\paragraph{Twisted Edwards curves addition} As we have seen in XXX one the major advantages of working with (twisted) Edwards curves is the existence of an addition law, that contains no branching and is valid for all curve points. Moreover the neutral element is not "at infinity" but the actual curve poin $(0,1)$.

As we know from XXX, give two points $(x_1,y_1)$ and $(x_2,y_2)$ on a twisted Edwards curve their sum is given by
$$
(x_3,y_3) = \left(\frac{x_1y_2+y_1x_2}{1+d\cdot x_1x_2y_1y_2}, \frac{y_1y_2-a\cdot x_1x_2}{1-d\cdot x_1x_2y_1y_2}\right)
$$
% https://z.cash/technology/jubjub/
We can realize this equation as a R1CS as follows: First not that we can rewrite the addition law as
$$
\begin{array}{lcl}
x_1\cdot x_2 &=& x_{12}\\
y_1\cdot y_2 &=& y_{12}\\
x_1\cdot y_2 &=& xy_{12}\\
y_1\cdot x_2 &=& yx_{12}\\
x_{12}\cdot y_{12} &=& xy_{1212}\\
x_3\cdot (1+d\cdot xy_{1212}) &=& xy_{12}+yx_{12}\\
y_3\cdot (1-d\cdot xy_{1212}) &=& y_{12}-a\cdot x_{12}
\end{array}
$$
So we have the statement $w=(1,x_1,y_1,x_2,y_2,x_3,y_3,x_{12},y_{12},xy_{12},yx_{12},xy_{1212})$ and we need 7 constraints to enforce that $(x_1,y_1)+(x_2,y_2)=(x_3,y_3)$ 
\begin{example}[Baby-JubJub]
Considering our pen and paper Baby JubJub curve over from XXX. We recall from XXX that $(11,9)$ is a generator for the large prime order subgroup. We therefor already know from XXX that
$(11,9) + (7,8) = (11,9) + [3](11,9) = [4](11,9) = (2,9)$. So we compute a valid statement as 
$w=(1,11,9,7,8,2,9,12,7,10,11,6)$. Indeed
$$
\begin{array}{lcl}
11\cdot 7 &=& 12\\
9\cdot 8 &=& 7\\
11\cdot 8 &=& 10\\
9\cdot 7 &=& 11\\
10\cdot 11 &=& 6\\
2\cdot (1+7\cdot 6) &=& 10 + 11\\
9\cdot (1-7\cdot 6) &=& 7 -1\cdot 12
\end{array}
$$
\end{example}

\paragraph{Twisted Edwards curves inversion} Similar to elliptic curves in Weierstrass form, inversion is cheap on Edwards curve as the negative of a curve point $-(x,y)$ is given by $(-x,y)$. So a curve point $(x_2,y_2)$ is the additive inverse of another curve point $(x_1,y_1)$ precisely if the equation $(x_1,y_1) = (-x_2,y_2)$ holds. We can write this as
$$
\begin{array}{lcl}
x_1 \cdot 1 &=& -x_2 \\
y_1 \cdot 1 &=& y_2
\end{array}
$$
We therefor have a statement of the form $w=(1,x_1,y_1,x_2,y_2)$ and can write the constraints into a matrix equation as
$$
\begin{pmatrix}
0 & 1 & 0 & 0 & 0 \\
0 & 0 & 1 & 0 & 0
\end{pmatrix} \begin{pmatrix} 1 \\ x_1 \\ y_1 \\ x_2 \\ y_2 \end{pmatrix}\odot
\begin{pmatrix}
1 & 0 & 0 & 0 & 0\\
1 & 0 & 0 & 0 & 0
\end{pmatrix} \begin{pmatrix} 1 \\ x_1 \\ y_1 \\ x_2 \\ y_2 \end{pmatrix} =
\begin{pmatrix}
0 & 0 & 0 & -1 & 0\\
0 & 0 & 0 & 0 & 1
\end{pmatrix} \begin{pmatrix} 1 \\ x_1 \\ y_1 \\ x_2 \\ y_2 \end{pmatrix}
$$

\paragraph{Twisted Edwards curves scalar multiplication} 

\paragraph{Curve Cycles} A particulary interesting case with far reaching implication is the situation when we have two curve $E_1$ and $E_2$, such that the scalar field of curve $E_1$ is the base field of curve $E_2$ and vice versa. In that case it is possible to implement the group laws of one curve in circuits defined over the scalar field of the other curve. 

\subsubsection{Generalizations}
many circuits can be found here:
% https://github.com/iden3/circomlib

\subsection{Quadratic span programs}

\section{proof system}
Now a \textit{proof system} is nothing but a game between two parties, where one parties task is to convince the other party, that a given string over some alphabet is a statement is some agreed on language. To be more precise. Such a system is more over \textit{zero knowledge} if this possible without revealing any information about the (parts of) that string.
\begin{definition}[(Interactive) Proofing System]
% https://link.springer.com/content/pdf/10.1007/BF00195207.pdf
Let $L$ be some formal language over an alphabet $\Sigma$. Then an \textbf{interactive proof system} for $L$ is a pair $(P,V)$ of two probabilistic interactive algorithms, where $P$ is called the \textbf{prover} and $V$ is called the \textbf{verifier}. 

Both algorithms are able to send messages to one another. Each algorithm only sees its own state, some shared initial state and the communication messages. 

The verifier is bounded to a number of steps which is polynomial in the size of the shared initial state, after which it stops in an accept state or in a reject state. We impose no restrictions on the local computation conducted by the prover. 

We require that, whenever the verifier is executed the following two conditions hold:
\begin{itemize}
\item (Completeness) If a string $x\in \Sigma^*$ is a member of language $L$, that is $x\in L$ and both prover and verifier follow the protocol; the verifier will accept.
\item (Soundness) If a string $x\in \Sigma^*$ is not a member of language $L$, that is $x\notin L$ and the verifier follows the protocol; the verifier will not be convinced.
\item (Zero-knowledge) If a string $x\in \Sigma^*$ is a member of language $L$, that is $x\in L$ and the prover follows the protocol; the verifier will not learn anything about $x$ but $x\in L$.
\end{itemize}
\end{definition}

In the context of zero knowledge proving systems definition XXX gets a slight adaptation:
\begin{itemize}
\item Instance: Input commonly known to both prover (P) and verifier (V), and used to support the statement of what needs to be proven. This common input may either be local to the prover-verifier interaction, or public in the sense of being known by external parties (Some scientific articles use "instance" and "statement" interchangeably, but we distinguish between the two.).
\item Witness: Private input to the prover. Others may or may not know something about the witness.
\item Relation: Specification of relationship between instances and witness. A relation can be viewed as a set of permissible pairs (instance, witness).
\item Language: Set of statements that appear as a permissible pair in the given relation.
\item Statement:Defined by instance and relation. Claims the instance has a witness in the relation(which is either true or false).
\end{itemize}

The following subsections define ways to describe checking relations that are particularly useful in the context of zero knowledge proofing systems

\subsection{Succinct NIZK}
Blum, Feldman and Micali
% Manuel  Blum,  Paul  Feldman,  and  Silvio  Micali.   Non-interactive  zero-knowledge  and  itsapplications.  InSTOC, pages 103–112, 1988.
 extended the notion tonon-interactivezero-knowledge(NIZK)  proofs in the  common  reference  string  model.  NIZK  proofs  are  useful  in  theconstruction of non-interactive cryptographic schemes, e.g., digital signatures and CCA-secure public key encryption.
 
\begin{definition} 
Let $\mathcal{R}$ be a relation generator that given a security parameter $\lambda$ in unary returns a polynomial time decidable binary relation $R$. For pairs $(i,w)\in R$ we call $i$ the instance\footnote{Note that in Groth16 this is called the statement. We think the term instance is more consistent with SOMETHING. } and $w$ the witness. We define $R_\lambda$ to be the set of possible relations $R$ the relation generator may output given $1^\lambda$. We will in the following for notational simplicity assume $\lambda$ can be deduced from the description of $R$. The relation generator may also output some side information, an auxiliary input $z$, which will be given to the adversary. An efficient prover publicly verifiable non-interactive argument for $R$ is a quadruple of probabilistic polynomial algorithms $(\textsc{Setup},\textsc{Prove},\textsc{Vfy},\textsc{Sim})$ such 
\begin{itemize}
\item Setup: $(CRS,\tau)\rightarrow Setup(R)$: The setup produces a common reference string $CRS$ and a simulation trapdoor $\tau$ for the relation $R$.
\item Proof: $\pi\rightarrow Prove(R,CRS,i,w)$: The prover algorithm takes as input a common reference string $CRS$ and a statement $(i,w)\in R$ and returns an argument $\pi$.
\item Verify: $0/1\rightarrow Vfy(R,CRS,i,\pi)$: The  verification algorithm  takes as input a common reference string $CRS$, an instance $i$ and an argument $\pi$ and returns 0 (reject) or 1 (accept).
\item $\pi\rightarrow Sim(R,\tau,i)$: The simulator takes as input a simulation trapdoor $\tau$ and instance $i$ and returns an argument $\pi$. 
\end{itemize}
\end{definition}

\subsubsection{Groth16}
Groth’s  constant  size  NIZK  argument  is  based  on  constructing  a  set  of  polynomial equations and using pairings to efficiently verify these equations. Gennaro, Gentry,Parno and Raykova [Pinocchio] found an insightful construction of polynomial equations based on Lagrange interpolation polynomials yielding a pairing-based NIZK argumentwith a common reference string size proportional to the size of the statement and wit-ness.

It constructs a snark  for arithmetic circuit satisfiability, where a proof consists of only 3 group elements. In addition to being small, the proof is also easy to verify. The verifier just needs to compute a number of exponentiations proportional to the instance size and check a single pairing product equation, which only  has  3  pairings.  

The  construction  can  be  instantiated  with  any  type  of  pairings including Type III pairings, which are the most efficient pairings. The argument has perfect completeness and perfect zero-knowledge. For soundness ?? 

In the common reference string model.

Setup: 
\begin{itemize}
\item random elements $\alpha,\beta,\gamma, \delta, s \in \mathbb{F}_{scalar}$ 
\item Common reference string $CRS_{QAP}$, specific to the $QAP$ and the choice of statement and witness $CRS_{QAP}= (CRS_{\mathbb{G}_1},CRS_{\mathbb{G}_2})$, with $n=deg(t)$: 
$$
CRS_{\mathbb{G}_{1}}=\left\{ \begin{array}{c}
[\alpha]g,[\beta]g,[\delta]g,\left\{ [s^{k}]g\right\} _{k=0}^{n-1},\left\{ [\frac{\beta a_{k}(s)+\alpha b_{k}(s)+c_{k}(s)}{\gamma}]g\right\} _{k\in I}\\
\left\{ [\frac{\beta a_{k}(s)+\alpha b_{k}(s)+c_{k}(s)}{\delta}]g\right\} _{k\in W},\left\{ [\frac{s^{k}t(s)}{\delta}]g\right\} _{k=0}^{n-2}
\end{array}\right\} 
$$
$$
CRS_{\mathbb{G}_{2}}=\left\{ [\beta]h ,[\gamma]h,[\delta]h,\left\{[s^k]h\right\} _{k=0}^{n-1}\right\} 
$$
\item Toxic waste: Must delete random elements after $CRS_{QAP}$ generation.
\end{itemize} 

\begin{example}[Generalized factorization snark]
\label{main_example_2_5}
In this example we want to compile our main example in Groth16. Input is the R1CS from example \ref{main_example_2_4}. We choose the following parameters

\begin{tabular}{ccccc}
\\
curve = BLS6-6 & $\mathbb{G}_1=$ BLS6-6(13) & $g = (13,15) $
& $\mathbb{G}_2=$ & $h=(7v^2,16v^3)$
\end{tabular} 

Setup phase: Recall the quadratic arithmetic program of example XXX. 

For our example we choose the following elements $\alpha=6$, $\beta=5$, $\gamma=4$, $\delta=3$, $s=2$ from $\mathbb{F}_{13}$
$$
CRS_{\mathbb{G}_{1}}=\left\{ \begin{array}{c}
[6](13,15),[5](13,15),[3](13,15),\left\{ [s^{k}](13,15)\right\} _{k=0}^{1},\left\{ [\frac{5 a_{k}(2)+6 b_{k}(2)+c_{k}(2)}{4}](13,15)\right\} _{k\in S}\\
\left\{ [\frac{5 a_{k}(2)+6 b_{k}(2)+c_{k}(2)}{3}](13,15)\right\} _{k\in W},\left\{ [\frac{s^{k}t(2)}{3}](13,15)\right\} _{k=0}^{0}
\end{array}\right\}
$$
Since we have instance indices $I=\{1, in_1,in_2\}$ and witness indices $W=\{in_3,mid_1,out_1\}$ we have 
The instance parts.
\begin{multline*}
\left[\frac{5 a_{c}(2)+6 b_{c}(2)+c_{c}(2)}{4}\right](13,15) = 
\left[\frac{5\cdot 0 +6\cdot 0 + 0 }{4}\right](13,15) =
\left[0\right](13,15) = \mathcal{O}
\end{multline*}
\begin{multline*}
\left[\frac{5 a_{in_3}(2)+6 b_{in_3}(2)+c_{in_3}(2)}{4}\right](13,15) =
\left[(5\cdot 0+6\cdot(7\cdot 2 +4)+0)\cdot 10\right](13,15) =\\
\left[(6\cdot 5 )\cdot 10\right](13,15) =
\left[1\right](13,15) =
(13,15)
\end{multline*}
\begin{multline*}
\left[\frac{5 a_{out}(2)+6 b_{out}(2)+c_{out}(2)}{4}\right](13,15) = 
\left[(5\cdot 0 +6\cdot 0 + (7\cdot 2 + 4))\cdot 10 \right](13,15) =\\
\left[5\cdot 10 \right](13,15) =
\left[11\right](13,15) = 
(33,9)
\end{multline*}

Witness part:
\begin{multline*}
\left[\frac{5 a_{in_1}(2)+6 b_{in_1}(2)+c_{in_1}(2)}{3}\right](13,15) = 
\left[(5\cdot (6\cdot 2 +10) +6\cdot 0 +0 )\cdot 9\right](13,15) = \\
\left[(5\cdot 9)\cdot 9\right](13,15) =
\left[2\right](13,15) = (33,34)
\end{multline*}
\begin{multline*}
\left[\frac{5 a_{in_2}(2)+6 b_{in_2}(2)+c_{in_2}(2)}{3}\right](13,15) = 
\left[(5\cdot 0 +6\cdot (6\cdot 2 + 10) + 0 )\cdot 9\right](13,15) = \\
\left[(6\cdot 9)\cdot 9\right](13,15) =
\left[5\right](13,15) =
(26,34)
\end{multline*}
\begin{multline*}
\left[\frac{5 a_{mid_1}(2)+6 b_{mid_1}(2)+c_{mid_1}(2)}{3}\right](13,15) = 
\left[(5\cdot (7\cdot 2 + 4) +6\cdot 0 + 0 )\cdot 9\right](13,15) = \\
\left[(5\cdot 5)\cdot 9\right](13,15) =
\left[4\right](13,15) =
(35,28)
\end{multline*}
For $\left\{\left[\frac{s^{k}t(2)}{3}\right](13,15)\right\} _{k=0}^{0}$ we get
\begin{multline*}
\left[\frac{2^{0}t(2)}{3}\right](13,15)=
[t(2)\cdot 9](13,15)= 
[(2^2+2+9)\cdot 9](13,15)= 
[5](13,15) =
(26,34)
\end{multline*}
All together, the $\mathbb{G}_1$ part of the CRS is:
$$
CRS_{\mathbb{G}_{1}}=\left\{ \begin{array}{c}
(27,34),(26,34),(38,15),\left\{(13,15),(33,34)\right\},
\left\{\mathcal{O}, (13,15), (33,9)\right\}\\
\left\{(33,34),(26,34),(35,28)\right\},
\left\{(26,34)\right\}
\end{array}\right\}
$$
To compute the $\mathbb{G}_2$ part 
$$
CRS_{\mathbb{G}_{2}}=\left\{ [5](7v^2,16v^3) ,[4](7v^2,16v^3),[3](7v^2,16v^3),\left\{[2^k](7v^2,16v^3)\right\} _{k=0}^{1}\right\} 
$$
$$
CRS_{\mathbb{G}_{2}}=\left\{ [5](7v^2,16v^3) ,[4](7v^2,16v^3),[3](7v^2,16v^3),\left\{[1](7v^2,16v^3), [2](7v^2,16v^3)\right\}\right\} 
$$
$$
CRS_{\mathbb{G}_{2}}=\left\{(16v^2,28v^3) ,(37v^2,27v^3),(42v^2,16v^3),\left\{(7v^2,16v^3), (10v^2,28v^3)\right\}\right\} 
$$

So alltogether our common reference string is 
$$
\begin{pmatrix}
\left\{ \begin{array}{c}
(27,34),(26,34),(38,15),\left\{(13,15),(33,34)\right\},
\left\{\mathcal{O}, (13,15), (33,9)\right\}\\
\left\{(33,34),(26,34),(35,28)\right\},
\left\{(26,34)\right\}
\end{array}\right\}\\
\left\{(16v^2,28v^3) ,(37v^2,27v^3),(42v^2,16v^3),\left\{(7v^2,16v^3), (10v^2,28v^3)\right\}\right\}
\end{pmatrix}
$$

The proofer phase: 

\end{example}


\chapter{Exercises and Solutions}

TODO: All exercises we provided should have a solution, which we give here in all detail. 

\bibliography{moonmath}%use `moonmath.bib` to look for references, and print bibliography at this point in the document

\end{document}
