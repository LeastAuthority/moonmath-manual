\chapter{Statements}\label{sec:statements}\label{chap:statements}
% Update the circuit / r1cs examples to describe their languages and how naive proofs look

% https://people.cs.georgetown.edu/jthaler/ProofsArgsAndZK.pdf
% https://docs.zkproof.org/reference.pdf
As we have seen in the informal introduction, a SNARK is a succinct non-interactive argument of knowledge, where the knowledge-proof attests to the correctness of statements like ``The prover knows the prime factorization of a given number'' or ``The prover knows the preimage to a given SHA2 digest value'' and similar things. However,  human-readable statements like these are imprecise and not very useful from a formal perspective. 

In this chapter, we therefore look more closely at ways to formalize statements in mathematically rigorous ways, useful for SNARK development. We start by introducing formal languages as a way to define statements properly (\secname{} \ref{sec:formal-languages}). For a detailed introduction of formal languages, see \cite{moll-2012}, for example. We then look at algebraic circuits and \concept{rank-1 constraint systems} [R1CS] as two particularly useful ways to define statements in certain formal languages (\secname{} \ref{sec:statement-representations}). \concept{rank-1 constraint systems} and algebraic circuits are introduced for example in appendix E of \cite{sasson-2013}.

Proper statement design should be of high priority in the development of SNARKs, since unintended true statements can lead to potentially severe and almost undetectable security vulnerabilities in the applications of SNARKs.

\section{Formal Languages}\label{sec:formal-languages}

Formal languages provide the theoretical background in which statements can be formulated in a logically rigorous way, and where proving the correctness of any given statement can be realized by computing words in that language.

One might argue that the understanding of formal languages is not very important in SNARK development and associated statement design, but terms from that field of research are standard jargon in many papers on zero-knowledge proofs. We therefore believe that at least some introduction to formal languages and how they fit into the picture of SNARK development is beneficial, mostly to give developers a better intuition about where all this is located in the bigger picture of the logic landscape. In addition, formal languages give a better understanding of what a formal proof for a statement actually is.

Roughly speaking, a formal language (or just language for short) is a set of words. Words, in turn, are strings of letters taken from some alphabet, and formed according to some defining rules of the language. 

To be more precise, let $\Sigma$ be any set and $\Sigma^*$ the set of all \term{strings} of finite length $<x_1,\ldots,x_n>$ of elements $x_j$ from $\Sigma$ including the empty string $<>\in \Sigma^*$. Then, a \term{language} $L$, in its most general definition, is a subset of the set of all finite strings $\Sigma^*$. In this context, the set $\Sigma$ is called the \term{alphabet} of the language $L$, elements from $\Sigma$ are called \term{letters}, and elements from $L$ are called \term{words}. 
If there are rules that specify which strings from $\Sigma^*$ belong to the language and which don't, those rules are called the \term{grammar} of the language. If $L_1$ and $L_2$ are two formal languages over the same alphabet, we call $L_1$ and $L_2$ \term{equivalent} if they consist of the same set of words.\footnote{A more detailed explanation of this definition can be found for example in \secname{} 1.2 of \cite{moll-2012}.}

\begin{comment}
While the term language might suggest a deeper relation to well-known \term{natural languages} like English, formal languages and natural languages differ in many ways. The following two examples  provide some intuition about formal languages and their differences to natural languages: 
\begin{example}[The English language] The English alphabet is given by the following set:
\begin{multline}
\Sigma_{English} = \{A, a, B, b, C, c, D, d, E, e, F, f, G, g, H, h, I, i, J, j, K, k, L, l, M, m, N, n, O,\\ o, P, p, Q, q, R, r, S, s, T, t, U, u, V, v, W, w, X, x, Y, y, Z, z\}
\end{multline}
\smecomb{The natural language English does not define words by abstract rules, but by something we might call historical human convention as a word in the language English is distinguished from any other string from the English alphabet by human convention not by formal rules. In that regard, strings like ``tea'' or ``eat'' are words in English, but ``aet'' and ``tae'' are not, because people agree on that. So while the set of all English words is a formal language $L_{English}\subset \Sigma_{English}$ there is no formal grammar that defines what a word in English is and what not.}{SB: none of this is correct, needs to be rewritten or cut}\sme{alternative example: blick vs. bnick as possible words}

It should also be recognized that in such a simple definition, the ''grammar'' of the natural language English is different from the grammar of the formal language English. The grammar of natural English defines how sentences are generated from words, while sentences have no meaning in our simple formal language English. 

In order to express natural English sentences in a formal English language, one has to extend the alphabet by a space character together with the set of all punctuation symbols,  and then include the grammar of the natural language English into the formal language English. It should be noted, however, that in such a formal language, sentences like "I drink tea, but I don't eat meat." from the natural language English are actually words in the formal language English.\sme{SB: since we're not using sentences, I'd cut this}
\end{example}
\end{comment}
\begin{example}[Alternating Binary strings]
\label{ex:alternating_strings_lang}
To consider a very basic formal language with an almost trivial grammar, consider the set  of two letters, $0$ and $1$, as our alphabet $\Sigma$:
$$
\Sigma = \{0,1\}
$$
In addition, we use a grammar stating that a proper word must consist of alternating binary letters of arbitrary length, including the empty string. The associated language $L_{alt}$ is the set of all finite binary strings where a $1$ must follow a $0$ and vice versa. So, for example, $<1,0,1,0,1,0,1,0,1>\in L_{alt}$ is a word in this language, as is $<0>\in L_{alt}$ or the empty word $<>\in L_{alt}$. However, the binary string $<1,0,1,0,1,0,1,1,1>\in \{0,1\}^*$ is not a proper word, as it violates the grammar of $L_{alt}$, since the last 3 letters are all $1$. Furthermore, the string $<0,A,0,A,0,A,0>$ is not a proper word, as not all its letters are from the alphabet $\Sigma$. 
\end{example}

\subsection{Decision Functions}
\label{sec:decision_function}
 Our previous definition of formal languages is very general, and does not cover many subclasses of languages known in the literature. However, in the context of SNARK development, languages are commonly defined as \term{decision problems} where a so-called \term{deciding relation} $R\subset \Sigma^*$ decides whether a given string $x\in \Sigma^*$ is a word in the language or not. If $x\in R$ then $x$ is a word in the associated language $L_R$, and if $x\notin R$ then it is not. The \href{https://en.wikipedia.org/wiki/Relation_(mathematics)}{relation} $R$ therefore summarizes the grammar of language $L_R$.

Unfortunately, in some literature on proof systems, $x\in R$ is often written as $R(x)$, which is  misleading since, in general, $R$ is not a function, but a relation in $\Sigma^*$. For the sake of clarity, we therefore adopt a different point of view and work with what we might call a \term{decision function} instead:
\begin{equation}\label{eq:decision-function}
R: \Sigma^* \to \{true, false\}
\end{equation}
Decision functions decide if a string $x\in \Sigma^*$ is an element of a language or not. In case a decision function is given, the associated language itself can be written as the set of all strings that are decided by $R$:
\begin{equation}
L_R := \{x\in \Sigma^*\;|\; R(x)=true\}
\end{equation}
In the context of formal languages and decision problems, a \term{statement} $S$ is the claim that language $L$ contains a word $x$, that is, a statement claims that there exists some $x\in L$. A constructive \term{proof} for statement $S$ is given by some string $P\in \Sigma^*$ and  such a proof is \term{verified} by checking if $R(P)=true$. In this case, $P$ is called an \term{instance} of the statement $S$.

\begin{example}[Alternating Binary strings] To consider a very basic formal language with a decision function, consider the language $L_{alt}$ from \examplename{} \ref{ex:alternating_strings_lang}. Attempting to write the grammar of this language in a more formal way, we can define the following decision function:
$$
R: \{0,1\}^* \to \{true,false\}\;;\; <x_0,x_1,\ldots,x_n> \mapsto 
\begin{cases}
true & x_{j-1} \neq x_{j} \text{ for all } 1\leq j \leq n \\
false & \text{ else}
\end{cases}
$$
We can use this function to decide if a given binary string is a word in $L_{alt}$ or not. Some examples are given below:
$$
\begin{array}{l}
R(<1,0,1>)=true\\ 
R(<0>)=true\\
R(<>)=true\\ 
R(<1,1>)=false
\end{array} 
$$
Given language $L_{alt}$, it makes sense to claim the following statement: ``There exists an alternating string.'' One way to prove this statement \href{https://en.wikipedia.org/wiki/Constructive_proof}{constructively} is by providing an actual instance, that is, providing an example of an alternating string like $x = <1,0,1>$. Constructing the string $<1,0,1>$ therefore proves the statement ``There exists an alternating string.", because one can verify that $R(<1,0,1>)=true$.
\end{example}
\begin{example}[Programming Language]Programming languages are a very important class of formal languages. For these languages, the alphabet is usually (a subset) of the ASCII table, and the grammar is defined by the rules of the programming language's compiler. Words, then, are  properly written computer programs that the compiler accepts. The compiler can therefore be interpreted as the decision function.

To give an unusual example strange enough to highlight the point, consider the programming language \href{https://en.wikipedia.org/wiki/Malbolge}{Malbolge}. This language was specifically designed to be almost impossible to use, and writing programs in this language is a difficult task. An interesting claim is therefore the statement: ``There exists a computer program in Malbolge". As it turned out, proving this statement constructively, that is, providing an example instance of such a program, is not an easy task: it took two years after the introduction of Malbolge to write a program that its compiler accepts. So, for two years, no one was able to prove the statement constructively.

To look at the high-level description of Malbolge more formally, we write $L_{Malbolge}$ for the language that uses the ASCII table as its alphabet, and its words are strings of ASCII letters that the Malbolge compiler accepts. Proving the statement ``There exists a computer program in Malbolge'' is equivalent to the task of finding some word $x\in L_{Malbolge}$. The string in \eqref{malbolge-string} below is an example of such a proof, as it is excepted by the Malbolge compiler, which compiles it to an executable binary that displays ``Hello, World.'' \sme{add reference}. In this example, the Malbolge compiler therefore serves as the verification process.

\begin{multline}\label{malbolge-string}
\scriptstyle (=<':9876Z4321UT.-Q+*)M'\&\%\$H"!~\}|Bzy?=|\{z]KwZY44Eq0/
\{mlk** \\ 
\scriptstyle hKs\_dG5[m\_BA\{?-Y;;Vb'rR5431M\}/.zHGwEDCBA@98\backslash 6543W10/.R,+O<
\end{multline}
\end{example}
\begin{example}[The Empty Language] To see that not every language has even a single word, consider the alphabet $\Sigma = \Z_6$, where $\Z_6$ is the ring of modular $6$ arithmetic as derived in \examplename{} \ref{def_residue_ring_z_6}. Distinguishing the set $\Z_6^*$ of all elements in modular $6$ arithmetic that have multiplicative inverses from the set $(\Z_6)^*$ of all finite strings over the alphabet $\Z_6$, we define the following decision function:\sme{SB: mention that n=1 refers to the nr. of strings?}
\begin{equation}
R_{\emptyset} : (\Z_{6})^* \to \{true, false\}\;;\;
<x_1,\ldots,x_n> \mapsto
\begin{cases}
true & n=1 \text{ and } x\cdot x = 2\\
false & else
\end{cases}
\end{equation}
We write $L_\emptyset$ for the associated language. As we can see from the multiplication table of $\Z_6$ in \examplename{} \ref{def_residue_ring_z_6}, the ring $\Z_6$ does not contain any element $x$ such that $x\cdot x =2$, which implies $R_{\emptyset}(<x_1,\ldots,x_n>)=false$ for all strings $<x_1,\ldots,x_n>\in \Sigma^*$. The language therefore does not contain any words. Proving the statement ``There exists a word in $L_\emptyset$'' constructively by providing an instance is therefore impossible: the verification will never check any string.
\end{example}
\begin{example}[3-Factorization]\label{ex:3-factorization} We will use the following simple example repeatedly throughout this book. The task is to develop a SNARK that proves knowledge of three factors of an element from the finite field $\F_{13}$. There is nothing particularly useful about this example from an application point of view, however, it is the most simple example that gives rise to a non-trivial SNARK in some of the most common zero-knowledge proof systems. 

Formalizing the high-level description, we use $\Sigma := \F_{13}$ as the underlying alphabet of this problem and define the language $L_{3.fac}$ to consists of those strings of field elements from $\F_{13}$ that contain exactly $4$ letters $x_1,x_2,x_3,x_4$ which satisfy the equation $x_1\cdot x_2\cdot x_3 =x_4$.

So, for example, the string $<2, 12, 4, 5>$ is a word in $L_{3.fac}$, while neither $<2, 12, 11>$, nor $<2, 12, 4, 7>$ nor $<2, 12, 7, UPS>$ are words in $L_{3.fac}$ as they don't satisfy the grammar or are not defined over the alphabet $\F_{13}$. 

Distinguishing the set $\F_{13}^*$ of all elements in the multiplicative group of $\F_{13}$ from the set $(\F_{13})^*$ of all finite strings over the alphabet $\F_{13}$, we can describe the language $L_{3.fac}$ more formally by introducing a decision function:
\begin{equation}
R_{3.fac} : (\F_{13})^* \to \{true, false\}\;;\;
<x_1,\ldots,x_n> \mapsto
\begin{cases}
true & n=4 \text{ and } x_1\cdot x_2 \cdot x_3 = x_4\\
false & else
\end{cases}
\end{equation}
Having defined the language $L_{3.fac}$, it then makes sense to claim the statement ``There is a word in $L_{3.fac}$". The way $L_{3.fac}$ is designed, this statement is equivalent to the statement ``There are four elements $x_1,x_2,x_3,x_4$ from the finite field $\F_{13}$ such that the equation $x_1\cdot x_2\cdot x_3 =x_4$ holds.''

Proving the correctness of this statement constructively means to actually find some concrete field elements that satisfy the decision function $R_{3.fac}$, like $x_1= 2$, $x_2 =12$, $x_3=4$ and $x_4 = 5$. The string $<2,12,4,5>$ is therefore a constructive proof for the statement that $L_{3.fac}$ contains words, and the computation $R_{3.fac}(<2,12,4,5>)=true$ is a verification of that proof. In contrast, the string $<2, 12, 4, 7>$ is not a proof of the statement, since the check $R_{3.fac}(<2,12,4,7>)=false$ does not verify the proof.
\end{example}
\begin{example}[\curvename{Tiny-jubjub} Membership]\label{ex:tiny-jubjub} In one of our main examples, we derive a SNARK that proves a pair $(x,y)$ of field elements from $\F_{13}$ to be a point on the tiny-{jubjub} curve in its twisted Edwards form as derived in \examplename{} \ref{TJJ13-twisted-edwards}.

In the first step, we define a language such that points on the \curvename{Tiny-jubjub} curve are in 1:1 correspondence with words in that language.

Since the \curvename{Tiny-jubjub} curve is an elliptic curve over the field $\F_{13}$, we choose the alphabet $\Sigma = \F_{13}$. In this case, the set $(\F_{13})^*$ consists of all finite strings of field elements from $\F_{13}$. To define the grammar, recall from \eqref{TJJ13-twisted-edwards} that a point on the \curvename{Tiny-jubjub} curve is a pair $(x,y)$ of field elements such that $3\cdot x^2 + y^2 = 1+ 8\cdot x^2\cdot y^2$. We can use this equation to derive the following decision function:\sme{SB: mention that $x=x_1$ and $y=x_2$?}
$$
R_{tiny.jj} : (\F_{13})^* \to \{true, false\}\;;\;
<x_1,\ldots,x_n> \mapsto
\begin{cases}
true & n=2 \text{ and } 3\cdot x_1^2 + x_2^2 = 1+ 8\cdot x_1^2\cdot x_2^2\\
false & else
\end{cases}
$$
The associated language $L_{tiny.jj}$ is then given as the set of all strings from $(\F_{13})^*$ that are mapped onto $true$ by $R_{tiny.jj}$:
$$
L_{tiny.jj} = \{<x_1,\ldots,x_n>\in (\F_{13})^*\;|\; R_{tiny.jj(<x_1,\ldots,x_n>)=true}\}
$$
We can claim the statement ``There is a word in $L_{tiny.jj}$''. Because $L_{tiny.jj}$ is defined by $R_{tiny.jj}$, this statement is equivalent to the statement ``The \curvename{Tiny-jubjub} curve in its twisted Edwards form has a curve point.'' 

A constructive proof for this statement is a string $<x,y>$ of field elements from $\F_{13}$ that satisfies the twisted Edwards equation. Example \eqref{TJJ13-twisted-edwards}, therefore, implies that the string $<11,6>$ is a constructive proof, and the computation $R_{tiny.jj}(<11,6>)=true$ verifies the proof. In contrast, the string $<1,1>$ is not a proof of the statement, since the computation $R_{tiny.jj}(<1,1>)=false$ does not verify the proof.
\end{example}
\begin{exercise}
\label{ex:decision_function_1} Define a decision function such that the associated language $L_{Exercise_1}$ consists of all solutions to the equation $5x + 4 = 28 + 2x$ over $\F_{13}$. Provide a constructive proof for the claim: ``There exists a word in $L_{Exercise_1}$, and verify the proof.  
\end{exercise}
\begin{exercise} Consider modular $6$ arithmetic ($\Z_6$) from  \examplename{} \ref{def_residue_ring_z_6}, the alphabet $\Sigma = \Z_6$ and the following decision function:\sme{SB: call this $R_{\Z_6}$/ $L_{\Z_6}$ instead?}
\begin{equation*}
R_{example\_\ref{def_residue_ring_z_6}} : \Sigma^*\to \{true, false\}\;;\;
<x_1,\ldots,x_n> \mapsto
\begin{cases}
true & n=1 \text{ and } 3\cdot x_1 + 3 = 0\\
false & else
\end{cases}
\end{equation*}
Compute all words in the associated language $L_{example\_\ref{def_residue_ring_z_6}}$, provide a constructive proof for the statement ``There exist a word in $L_{example\_\ref{def_residue_ring_z_6}}$'' and verify the proof.
\end{exercise}

\subsection{Instance and Witness}
% https://www.claymath.org/sites/default/files/pvsnp.pdf
As we have seen in the previous paragraph, statements provide membership claims in formal languages, and instances serve as constructive proofs for those claims. However, in the context of \term{zero-knowledge} proof systems, our notion of constructive proofs is refined in such a way that  it is possible to hide parts of the proof instance and still be able to prove the statement. In this context, it is therefore necessary to split a proof into an unhidden, public part called the \term{instance} and a hidden, private part called a \term{witness}.

To account for this separation of a proof instance into an instance and a witness part, our previous definition of formal languages needs a refinement. Instead of a single alphabet, the refined definition considers two alphabets $\Sigma_I$ and $\Sigma_W$, and a decision function defined as follows:
\begin{equation}
R: \Sigma_I^* \times \Sigma_W^* \to \{true, false\}\;;\; (i\,;w) \mapsto R(i\,;w)
\end{equation}
Words are therefore strings $(i\,;w)\in \Sigma_I^* \times \Sigma_W^*$ with $R(i\,;w)=true$. The refined definition differentiates between inputs $i\in \Sigma_I$ and inputs $w\in \Sigma_W$. The input $i$ is called an \term{instance} and the input $w$ is called a \term{witness} of $R$. 

If a decision function is given, the associated language is defined as the set of all strings from the underlying alphabets that are verified by the decision function:
\begin{equation}
\label{def:refined_language}
L_R := \{(i\,;w)\in \Sigma_I^* \times \Sigma_W^* \;|\; R(i\,;w)=true\}
\end{equation}
In this refined context, a \term{statement} $S$ is a claim that, given an instance $i\in\Sigma_I^*$, there is a witness $w\in \Sigma_W^*$ such that language $L$ contains a word $(i\,;w)$. A constructive \term{proof} for statement $S$ is given by some string $P=(i\,; w) \in \Sigma_I^* \times \Sigma_W^*$, and a proof is \term{verified} by $R(P)=true$. 

At this point, it is important to note that, while constructive proofs in languages that distinguish between instance and witness (as in \defname{} \ref{def:refined_language}) don't look very different from constructive proofs in languages we have seen in \secname{} \ref{sec:decision_function}, given some instance, there are proof systems able to prove the statement (at least with high probability) without revealing anything about the witness, as we will see in \chaptname{} \ref{chapter:zk-protocols}. In this sense, the witness is often called the \term{private input}, and the instance is called the \term{public input}.

It is worth understanding the difference between statements as defined in \secname{} \ref{sec:decision_function} and the refined notion of statements from this section. While statements in the sense of the previous section can be seen as membership claims, statements in the refined definition can be seen as knowledge-claims, where a prover claims knowledge of a witness for a given instance. 
%For a more detailed discussion on this topic see [XXX\sme{update reference} sec 1.4]

\begin{example}[SHA256 -- Knowlege of Preimage] One of the most common examples in the context of zero-knowledge proof systems is the \term{knowledge-of-a-preimage proof} for some cryptographic hash function like $SHA256$, where a publicly known $SHA256$ \href{https://en.wikipedia.org/wiki/Hash_function}{digest} value is given, and the task is to prove knowledge of a preimage for that digest under the $SHA256$ function, without revealing that preimage. 

To understand this problem in detail, we have to introduce a language able to describe the knowledge-of-preimage problem in such a way that the claim ``Given digest $i$, there is a preimage $w$ such that $SHA256(w)=i$'' becomes a statement in that language. Since $SHA256$ is a function that maps binary strings of arbitrary length onto binary strings of length $256$:
$$
SHA256: \{0,1\}^* \to \{0,1\}^{256}
$$
Since we want to prove knowledge of preimages, we have to consider binary strings of size $256$ as instances and binary strings of arbitrary length as witnesses. 

An appropriate alphabet $\Sigma_I$ for the set of all instances, and an appropriate alphabet $\Sigma_W$ for the set of all witnesses is therefore given by the set $\{0,1\}$. A proper decision function is given as follows:
\begin{multline*}
R_{SHA256} : \{0,1\}^* \times \{0,1\}^* \to \{true, false\}\;;\;\\
(i;w) \mapsto
\begin{cases}
true & |i|=256,\; i = SHA256(w)\\
false & else
\end{cases}
\end{multline*}
We write $L_{SHA256}$ for the associated language, and note that it consists of words that are strings $(i\,;w)$ such that the instance $i$ is the $SHA256$ image of the witness $w$. 

Given some instance $i\in \{0,1\}^{256}$, a statement in $L_{SHA256}$ is the claim ``Given digest $i$, there is a preimage $w$ such that $SHA256(w)=i$", which is exactly what the knowledge-of-preimage problem is about. A constructive proof for this statement is therefore given by a preimage $w$ to the digest $i$ and proof verification is achieved by verifying that $SHA256(w)=i$. 
\end{example}
\begin{example}[3-factorization]
\label{ex:L-3fac-zk}
 To give another intuition about the implication of refined languages, consider $L_{3.fac}$ from \examplename{} \ref{ex:3-factorization} again. As we have seen, a constructive proof in $L_{3.fac}$ is given by $4$ field elements $x_1$, $x_2$, $x_3$ and $x_4$ from $\F_{13}$ such that the product of the first three elements is equal to the $4^{th}$ element in modular $13$ arithmetic. 

Splitting words from $L_{3.fac}$ into instance and witness parts, we can reformulate the problem and introduce different levels of knowledge-claims. For example, we could reformulate the membership statement of $L_{3.fac}$ into a statement where all factors $x_1$, $x_2$, $x_3$ are witnesses, and only the product $x_4$ is the instance. A statement for this reformulation is then expressed by the claim: ``Given an instance field element $x_4$, there are three witness factors of $x_4$". Assuming some instance $x_4$, a constructive proof for the associated knowledge claim is provided by any string $(x_1,x_2,x_3)$ such that $x_1\cdot x_2\cdot x_3= x_4$. 

We can formalize this new language, which we might call $L_{3.fac\_zk}$, by defining the following decision function:
\begin{multline*}
\label{R3-fac-zk}
R_{3.fac\_zk} : (\F_{13})^* \times (\F_{13})^* \to \{true, false\}\;;\;\\
(<i_1,\ldots,i_n>;<w_1,\ldots, w_m>) \mapsto
\begin{cases}
true & n=1,\; m=3,\; i_1 = w_1\cdot w_2 \cdot w_3\\
false & else
\end{cases}
\end{multline*}
The associated language $L_{3.fac\_zk}$ is defined by all strings from $(\F_{13})^* \times (\F_{13})^*$ that are mapped onto $true$ under the decision function $R_{3.fac\_zk}$. 

Considering the distinction we made between the instance and the witness part in $L_{3.fac\_zk}$, one might ask why we chose the factors $x_1$, $x_2$ and $x_3$ to be the witness and the product $x_4$ to be the instance rather than another combination? This was an arbitrary choice in the example. Every other combination of instance and witness would be equally valid. For example, it would be possible to declare all variables as witness or to declare all variables as instance. Actual choices are determined by the application only.
\end{example}
\begin{example}[The \curvename{Tiny-jubjub} Curve]
\label{ex:tiny-jubjub-instance-witness}
 Consider the language $L_{tiny.jj}$ from \examplename{} \ref{ex:tiny-jubjub}. As we have seen, a constructive proof in $L_{tiny.jj}$ is given by a pair $(x_1,x_2)$ of field elements from $\F_{13}$ such that the pair is a point of the \curvename{Tiny-jubjub} curve in its Edwards representation.

We look at a reasonable splitting of words from $L_{tiny.jj}$ into instance and witness parts. The two obvious choices are to either choose both coordinates $x_1$ as $x_2$ as instance inputs, or to choose both coordinates $x_1$ as $x_2$ as witness inputs.\sme{why not 1 witness and 1 input?} 

In case both coordinates are instances, we define the grammar of the associated language by introducing the following decision function:\sme{why are we using capital I and W here?}
\begin{multline*}
R_{tiny.jj.1} : (\F_{13})^*\times (\F_{13})^* \to \{true, false\}\;;\;\\
(<I_1,\ldots,I_n>;<W_1,\ldots,W_m>) \mapsto
\begin{cases}
true & n=2,\;m=0 \text{ and } 3\cdot I_1^2 + I_2^2 = 1+ 8\cdot I_1^2\cdot I_2^2\\
false & else
\end{cases}
\end{multline*}
The language $L_{tiny.jj.1}$ is defined as the set of all strings from $(\F_{13})^*\times (\F_{13})^*$ that are mapped onto $true$ by $R_{tiny.jj.1}$. 

In case both coordinates are witness inputs, we define the grammar of the associated refined language by introducing the following decision function:
\begin{multline*}
R_{tiny.jj\_zk} : (\F_{13})^*\times (\F_{13})^* \to \{true, false\}\;;\;\\
(<I_1,\ldots,I_n>;<W_1,\ldots,W_m>) \mapsto
\begin{cases}
true & n=0,\;m=m \text{ and } 3\cdot W_1^2 + W_2^2 = 1+ 8\cdot W_1^2\cdot W_2^2\\
false & else
\end{cases}
\end{multline*}
The language $L_{tiny.jj\_zk}$ is defined as the set of all strings from $(\F_{13})^*\times (\F_{13})^*$ that are mapped onto $true$ by $R_{tiny.jj\_zk}$. 
\end{example}\sme{pros and cons of $L_{tiny.jj.1}$ vs. $L_{tiny.jj\_zk}$?}


\begin{exercise} Consider the modular $6$ arithmetic $\Z_6$ from \examplename{} \ref{def_residue_ring_z_6} as the alphabets $\Sigma_I$ and $\Sigma_W$, and the following decision function:
\begin{multline*}
R_{linear} : \Sigma^* \times \Sigma^* \to \{true, false\}\;;\;\\
(i;w) \mapsto
\begin{cases}
true & |i|=3 \text{ and } |w|=1 \text{ and } i_1\cdot w_1 + i_2 = i_3\\
false & else
\end{cases}
\end{multline*}
Which of the following instances $(i_1,i_2,i_3)$ has a proof of knowledge in $L_{linear}$?\tbdsm{make this look nicer}
$$ 
\begin{array}{ccc}
(3,3,0) , & (2,1,0), & (4,4,2)
\end{array}
$$
\end{exercise}
\begin{exercise}[Edwards Addition on \curvename{Tiny-jubjub}]
\label{ex:TJJ-addition-lang} Consider the \curvename{Tiny-jubjub} curve together with its twisted Edwards addition law from \examplename{} \ref{TJJ13}. Define an instance alphabet $\Sigma_I$, a witness alphabet $\Sigma_W$, and a decision function $R_{add}$ with associated language $L_{add}$ such that a string $(i\,;w)\in \Sigma_I^* \times \Sigma_W^*$ is a word in $L_{add}$ if and only if $i$ is a pair of curve points on the \curvename{Tiny-jubjub} curve in Edwards form, and $w$ is the sum of those curve points.

Choose some instance $i\in \Sigma_I^*$, provide a constructive proof for the statement ``There is a witness $w\in \Sigma_W^*$ such that $(i\,;w)$ is a word in $L_{add}$'', and verify that proof. Then find some instance $i\in \Sigma_I^*$ such that $i$ has no knowledge proof in $L_{add}$.
\end{exercise}

\subsection{Modularity}\label{modularity} 
From a developer's perspective, it is often useful to construct complex statements and their representing languages from simple ones. In the context of zero-knowledge proof systems, those simple building blocks are often called \term{gadgets}, and gadget libraries usually contain representations of atomic types like booleans, integers, various hash functions, elliptic curve cryptography and much more (see \chaptname{} \ref{chap:circuit-compilers}). In order to synthesize statements, developers then combine predefined gadgets into complex logic. We call the ability to combine statements into more complex statements \term{modularity}. 

To understand the concept of modularity on the level of formal languages defined by decision functions, we need to look at the \term{intersection} of two languages, which exists whenever both languages are defined over the same alphabet. In this case, the intersection is a language that consists of strings which are words in both languages. 

To be more precise, let $L_1$ and $L_2$ be two languages defined over the same instance and witness alphabets $\Sigma_I$ and $\Sigma_W$.  The intersection $L_1 \cap L_2$ of $L_1$ and $L_2$ is defined as follows:
\begin{equation}
L_1 \cap L_2 := \{x\;|\; x\in L_1 \text{ and } x\in L_2\}
\end{equation} 
If both languages are defined by decision functions $R_1$ and $R_2$, the following function is a decision function for the intersection language $L_1 \cap L_2$:
\begin{equation}
R_{L_1 \cap L_2}: \Sigma_I^* \times \Sigma_W^* \to \{true, false\}\;;\;
(i,w) \mapsto R_1(i,w) \text{ and } R_2(i,w)
\end{equation}

Thus, the intersection of two decision-function-based languages is also decision-function-based language. This is important from an implementation point of view: it allows us to construct complex decision functions, their languages and associated statements from simple building blocks. Given a publicly known instance $I\in \Sigma_I^*$, a statement in an intersection language claims knowledge of a witness that satisfies all relations simultaneously. 

\section{Statement Representations}\label{sec:statement-representations}
As we have seen in the previous section, formal languages and their definitions by decision functions are a powerful tool to describe statements in a formally rigorous manner. 

However, from the perspective of existing zero-knowledge proof systems, not all ways to actually represent decision functions are equally useful. Depending on the proof system, some are more suitable than others. In this section, we will describe two of the most common ways to represent decision functions and their statements.
\subsection{Rank-1 Quadratic Constraint Systems}
\label{sec:R1CS}
Although decision functions are expressible in various ways, many contemporary proving systems require the decision function to be expressed in terms of a system of quadratic equations over a finite field. This is true in particular for pairing-based proving systems like the ones we describe in \chaptname{}  \ref{chapter:zk-protocols}, because in these cases it is possible to separate instance and witness and then check solutions to those equations ``in the exponent'' of pairing-friendly cryptographic groups.

In this section, we will have a closer look at a particular type of quadratic equations called \term{Rank-1 (quadratic) Constraint Systems} (R1CS), which are a common standard in zero-knowledge proof systems (cf. appendix E of \cite{sasson-2013}). We will start with a general introduction to those constrain systems and then look at their relation to formal languages. Then we will look into a common way to compute solutions to those systems. 

\subsubsection{R1CS representation} To understand what \term{Rank-1 (quadratic) Constraint Systems} )(R1CS) are in detail, let $\F$ be a field, $n$, $m$ and $k\in\N$ three numbers and $a_j^i$, $b_j^i$ and $c_j^i\in\F$ constants from $\F$ for every index $0\leq j \leq n+m$ and $1\leq i \leq k$. Then a \concept{rank-1 constraint system} is defined as the following set of $k$ many equations:\tbdsm{font size too small} 

\begin{definition}[\deftitle{Rank-1 (quadratic) Constraint System}]\label{R1CS}
\begin{equation}\label{eq:R1CS}
\begin{split}
\scriptstyle\left(a^1_0 + \sum_{j=1}^n a^1_j \cdot I_j + \sum_{j=1}^m a^1_{n+j} \cdot W_j  \right) \cdot 
\left(b^1_0 + \sum_{j=1}^n b^1_j \cdot I_j + \sum_{j=1}^m b^1_{n+j} \cdot W_j  \right) &=\, 
\scriptstyle c^1_0 + \sum_{j=1}^n c^1_j \cdot I_j + \sum_{j=1}^m c^1_{n+j} \cdot W_j\\
       & \vdots\\
\scriptstyle\left(a^k_0 + \sum_{j=1}^n a^k_j \cdot I_j + \sum_{j=1}^m a^k_{n+j} \cdot W_j  \right) \cdot 
\left(b^k_0 + \sum_{j=1}^n b^k_j \cdot I_j + \sum_{j=1}^m b^k_{n+j} \cdot W_j  \right) &=\, 
\scriptstyle c^k_0 + \sum_{j=1}^n c^k_j \cdot I_j + \sum_{j=1}^m c^k_{n+j} \cdot W_j       
\end{split}
\end{equation}
\end{definition}

In a \concept{rank-1 constraint system}, the parameter $k$ is called the \term{number of constraints}, and each equation is called a \term{constraint}. If a pair of strings of field elements $(<I_1,\ldots, I_n>; <W_1,\ldots,W_m>)$  satisfies theses equations, $<I_1,\ldots, I_n>$ is called an \term{instance} and $<W_1,\ldots,W_m>$ is called a \term{witness} of the system.%
\footnote{The presentation of \concept{rank-1 constraint systems} can be simplified using the notation of vectors and matrices, which abstracts over the indices. In fact, if $x= (1,I,W)\in \F^{1+n+m}$ is a $(n+m+1)$-dimensional vector, $A$, $B$, $C$ are $(n+m+1)\times k$-dimensional matrices and $\odot$ is the Schur/Hadamard product, then a R1CS can be written as follows:
$$
Ax \odot Bx = Cx
$$
However,  since we did not introduce vector spaces and matrix calculus in the book, we use \ref{eq:R1CS} as the defining equation for \concept{rank-1 constraint systems}. We only highlighted the matrix notation because it is sometimes used in the literature.}

It can be shown that every bounded computation is expressible as a \concept{rank-1 constraint system}. R1CS is therefore a universal model for bounded computations. We will derive some insights into common approaches of how to compile bounded computation into \concept{rank-1 constraint systems} in \chaptname{} \ref{chap:circuit-compilers}. 

Generally speaking, the idea of a \concept{rank-1 constraint system} is to keep track of all the values that any variable can hold during a computation, and to bind the relationships among all those variables that are implied by the computation itself. Once relations between all steps of a computer program are constrained, program execution is then enforced to be computed in exactly in the expected way without any opportunity for deviations. In this sense, solutions to \concept{rank-1 constraint systems} are proofs of proper program execution.

\begin{example}[R1CS for 3-Factorization]\label{ex:3-factorization-r1cs} To provide a better intuition of \concept{rank-1 constraint systems}, consider the language $L_{3.fac\_zk}$ from \examplename{} \ref{ex:L-3fac-zk} again. As we have seen, $L_{3.fac\_zk}$ consists of words $(<I_1>;<W_1,W_2,W_3>)$ over the alphabet $\F_{13}$ such that $I_1 = W_1\cdot W_2\cdot W_3$. We show how to rewrite the problem as a \concept{rank-1 constraint system}.

Since R1CS are systems of quadratic equations, expressions like $W_1\cdot W_2\cdot W_3$, which contain products of more than two factors (which are therefore not quadratic) have to be rewritten in a process often called \term{flattening}. To flatten the defining equation $I_1 = W_1\cdot W_2\cdot W_3$ of $L_{3.fac\_zk}$, we introduce a new variable $W_4$, which captures two of the three multiplications in $W_1\cdot W_2\cdot W_3$. We get the following two constraints
\begin{align*}
W_1 \cdot W_2 & = W_4 & \text{constraint } 1 \\
W_4 \cdot W_3 & = I_1 & \text{constraint } 2
\end{align*}
Given some instance $I_1$, any solution $(W_1,W_2,W_3,W_4)$ to this system of equations provides a solution to the original equation $I_1 = W_1\cdot W_2\cdot W_3$ and vice versa. Both equations are therefore equivalent in the sense that solutions are in a 1:1 correspondence.

Looking at both equations from this constraint system, we see how each constraint enforces a step in the computation. Constraint $1$ forces any computation to multiply the witnesses $W_1$ and $W_2$; otherwise, it would not be possible to compute the witness $W_4$ first, which is needed to solve constraint $2$. Witness $W_4$ therefore expresses the constraining of an intermediate computational state.

At this point, one might ask why equation $1$ constrains the system to compute $W_1\cdot W_2$ first. In order to compute $W_1\cdot W_2 \cdot W_3$, calculating $W_2\cdot W_3$, or $W_1\cdot W_3$ in the beginning and then multiplying the result with the remaining factor gives the exact same result.  The reason is purely a matter of choice. For example, the following R1CS would define the exact same language:
\begin{align*}
W_2 \cdot W_3 & = W_4 & \text{constraint } 1 \\
W_4 \cdot W_1 & = I_1 & \text{constraint } 2 
\end{align*}
It follows that R1CS are generally \term{not unique} descriptions of any given situation: many different R1CS are able to describe the same problem.

To see that the two quadratic equations qualify as a \concept{rank-1 constraint system}, choose the parameter $n=1$, $m=4$ and $k=2$ as well as the following values:
$$
\begin{array}{llllll}
a_0^1 = 0 & a_1^1= 0 & a_2^1= 1 & a_3^1 = 0 & a_4^1= 0  & a_5^1= 0 \\ 
a_0^2 = 0 & a_1^2= 0 & a_2^2= 0 & a_3^2 = 0 & a_4^2= 0  & a_5^2= 1 \\ 
\\
b_0^1 = 0 & b_1^1= 0 & b_2^1= 0 & b_3^1 = 1 & b_4^1= 0  & b_5^1= 0 \\ 
b_0^2 = 0 & b_1^2= 0 & b_2^2= 0 & b_3^2 = 0 & b_4^2= 1  & b_5^2= 0 \\ 
\\
c_0^1 = 0 & c_1^1= 0 & c_2^1= 0 & c_3^1 = 0 & c_4^1= 0  & c_5^1= 1 \\ 
c_0^2 = 0 & c_1^2= 1 & c_2^2= 0 & c_3^2 = 0 & c_4^2= 0  & c_5^2= 0 
\end{array} 
$$
With this choice, the \concept{rank-1 constraint system} of our $3$-factorization problem can be written in its most general form as follows:
\begin{align*}
\scriptstyle
\left(a^1_0 + a_1^1 I_1 + a_2^1 W_1 + a_3^1 W_2 + a_4^1 W_3 + a_5^1 W_4\right)\cdot
\left(b^1_0 + b_1^1 I_1 + b_2^1 W_1 + b_3^1 W_2 + b_4^1 W_3 + b_5^1 W_4\right) &=
\scriptstyle
\left(c^1_0 + c_1^1 I_1 + c_2^1 W_1 + c_3^1 W_2 + c_4^1 W_3 + c_5^1 W_4\right)\\
\scriptstyle
\left(a^2_0 + a_1^2 I_1 + a_2^2 W_1 + a_3^2 W_2 + a_4^2 W_3 + a_5^2 W_4\right)\cdot
\left(b^2_0 + b_1^2 I_1 + b_2^2 W_1 + b_3^2 W_2 + b_4^2 W_3 + b_5^2 W_4\right) &=
\scriptstyle
\left(c^2_0 + c_1^2 I_1 + c_2^2 W_1 + c_3^2 W_2 + c_4^2 W_3 + c_5^2 W_4\right)
\end{align*}
\end{example}
\begin{example}[R1CS for the points of the \curvename{Tiny-jubjub} curve ]
\label{ex:TJJ-r1cs}
 Consider the languages $L_{tiny.jj.1}$ from \examplename{} \ref{ex:tiny-jubjub-instance-witness}, which consists of words $<I_1,I_2>$ over the alphabet $\F_{13}$ such that $3\cdot I_1^2 + I_2^2 = 1 + 8\cdot I_1^2\cdot I_2^2$. 

We derive a \concept{rank-1 constraint system} such that its solutions are in a 1:1 correspondence with words in $L_{tiny.jj.1}$.  To achieve this, we first rewrite the defining equation:
\begin{align*}
3\cdot I_1^2 + I_2^2  & = 1 + 8\cdot I_1^2\cdot I_2^2 & \Leftrightarrow \\
 0 & = 1 + 8\cdot I_1^2\cdot I_2^2 - 3\cdot I_1^2 - I_2^2  & \Leftrightarrow \\
 0 & = 1 + 8\cdot I_1^2\cdot I_2^2 + 10\cdot I_1^2 +12\cdot I_2^2
\end{align*}
Since R1CSs are systems of quadratic equations, we have to reformulate this expression into a system of quadratic equations. To do so, we have to introduce new variables that constrain intermediate steps in the computation, and we have to decide if those variables should be instance or witness variables. We decide to declare all new variables as witness variables, and get the following constraints:
\begin{align*}
I_1 \cdot I_1 & = W_1 & \text{constraint } 1\\
I_2 \cdot I_2 & = W_2 & \text{constraint } 2\\
(8 \cdot W_1) \cdot W_2 & = W_3 & \text{constraint } 3\\
(12\cdot W_2 + W_3 + 10\cdot W_1 + 1)\cdot 1 & = 0 & \text{constraint } 4
\end{align*}
To see that these four quadratic equations qualify as a \concept{rank-1 constraint system} according to \defname{} \ref{R1CS}, choose the parameter $n=2$, $m=3$,  $k=4$, and the following values:
$$
\begin{array}{llllll}
a_0^1 = 0 & a_1^1= 1 & a_2^1= 0 & a_3^1 = 0 & a_4^1= 0  & a_5^1= 0 \\ 
a_0^2 = 0 & a_1^2= 0 & a_2^2= 1 & a_3^2 = 0 & a_4^2= 0  & a_5^2= 0 \\ 
a_0^3 = 0 & a_1^3= 0 & a_2^3= 0 & a_3^3 = 8 & a_4^3= 0  & a_5^3= 0 \\ 
a_0^4 = 1 & a_1^4= 0 & a_2^4= 0 & a_3^4 = 10 & a_4^4= 12  & a_5^4= 1 \\ 
\\
b_0^1 = 0 & b_1^1= 1 & b_2^1= 0 & b_3^1 = 0 & b_4^1= 0  & b_5^1= 0 \\ 
b_0^2 = 0 & b_1^2= 0 & b_2^2= 1 & b_3^2 = 0 & b_4^2= 0  & b_5^2= 0 \\ 
b_0^3 = 0 & b_1^3= 0 & b_2^3= 0 & b_3^3 = 0 & b_4^3= 1  & b_5^3= 0 \\ 
b_0^4 = 1 & b_1^4= 0 & b_2^4= 0 & b_3^4 = 0 & b_4^4= 0  & b_5^4= 0 \\ 
\\
c_0^1 = 0 & c_1^1= 0 & c_2^1= 0 & c_3^1 = 1 & c_4^1= 0  & c_5^1= 0 \\ 
c_0^2 = 0 & c_1^2= 0 & c_2^2= 0 & c_3^2 = 0 & c_4^2= 1  & c_5^2= 0 \\
c_0^3 = 0 & c_1^3= 0 & c_2^3= 0 & c_3^3 = 0 & c_4^3= 0  & c_5^3= 1 \\ 
c_0^4 = 0 & c_1^4= 0 & c_2^4= 0 & c_3^4 = 0 & c_4^4= 0  & c_5^4= 0
\end{array} 
$$
With this choice, the \concept{rank-1 constraint system} of our \curvename{Tiny-jubjub} curve point problem can be written in its most general form as follows:\tbdsm{font too small}
\begin{align*}
\scriptstyle
\left(a_0^1 + a_1^1 I_1 + a_2^1 I_2 + a_3^1 W_1 + a_4^1 W_2 + a_5^1 W_3\right)\cdot
\left(b_0^1 + b_1^1 I_1 + b_2^1 I_2 + b_3^1 W_1 + b_4^1 W_2 + b_5^1 W_3\right) &=
\scriptstyle
\left(c_0^1 + c_1^1 I_1 + c_2^1 I_2 + c_3^1 W_1 + c_4^1 W_2 + c_5^1 W_3\right)\\
\scriptstyle
\left(a_0^2 + a_1^2 I_1 + a_2^2 I_2 + a_3^2 W_1 + a_4^2 W_2 + a_5^2 W_3\right)\cdot
\left(b_0^2 + b_1^2 I_1 + b_2^2 I_2 + b_3^2 W_1 + b_4^2 W_2 + b_5^2 W_3\right) &=
\scriptstyle
\left(c_0^2 + c_1^2 I_1 + c_2^2 I_2 + c_3^2 W_1 + c_4^2 W_2 + c_5^2 W_3\right)\\\scriptstyle
\left(a_0^3 + a_1^3 I_1 + a_2^3 I_2 + a_3^3 W_1 + a_4^3 W_2 + a_5^3 W_3\right)\cdot
\left(b_0^3 + b_1^3 I_1 + b_2^3 I_2 + b_3^3 W_1 + b_4^3 W_2 + b_5^3 W_3\right) &=
\scriptstyle
\left(c_0^3 + c_1^3 I_1 + c_2^3 I_2 + c_3^3 W_1 + c_4^3 W_2 + c_5^3 W_3\right)\\\scriptstyle
\left(a_0^4 + a_1^4 I_1 + a_2^4 I_2 + a_3^4 W_1 + a_4^4 W_2 + a_5^4 W_3\right)\cdot
\left(b_0^4 + b_1^4 I_1 + b_2^4 I_2 + b_3^4 W_1 + b_4^4 W_2 + b_5^4 W_3\right) &=
\scriptstyle
\left(c_0^4 + c_1^4 I_1 + c_2^4 I_2 + c_3^4 W_1 + c_4^4 W_2 + c_5^4 W_3\right)\\
\end{align*}

Solutions to this constraint system are in 1:1 correspondence with words in $L_{tiny.jj.1}$: if $(<I_1,I_2>; <W_1, W_2, W_3>)$ is a solution, then $<I_1,I_2>$ is a word in $L_{tiny.jj.1}$, since the defining R1CS implies that $I_1$ and $I_2$ satisfy the twisted Edwards equation of the \curvename{Tiny-jubjub} curve. On the other hand, if  $<I_1,I_2>$ is a word in $L_{tiny.jj.1}$, then $(<I_1,I_2>; <I_1^2, I_2^2, 8\cdot I_1^2\cdot I_2^2>)$ is a solution to our R1CS.
\end{example}
\begin{exercise} Consider the language $L_{add}$ from exercise \ref{ex:TJJ-addition-lang}. Define an R1CS such that words in $L_{add}$ are in 1:1 correspondence with solutions to this R1CS.
\end{exercise} 
\subsubsection{R1CS Satisfiability}
\label{r1cs-constructive-proofs}
To understand how \concept{rank-1 constraint systems} define formal languages, observe that every R1CS over a field $\F$ defines a decision function over the alphabet $\Sigma_I \times \Sigma_W = \F \times \F$ in the following way:
\begin{equation}
R_{R1CS} : (\F)^* \times (\F)^* \to \{true, false\}\;;\;
(I;W) \mapsto
\begin{cases}
true & (I;W) \text{ satisfies R1CS}\\
false & else
\end{cases}
\end{equation}
Every R1CS therefore defines a formal language. The grammar of this language is encoded in the constraints, words are solutions to the equations, and  a \term{statement} is a knowledge claim ``Given instance $I$, there is a witness $W$ such that $(I;W)$ is a solution to the \concept{rank-1 constraint system}". A constructive proof to this claim is therefore equivalent to assigning a field element to every witness variable, which is verified whenever the set of all instance and witness variables solves the R1CS. 

\begin{remark}[R1CS satisfiability]\label{r1cs-satisfiability}  \sme{SB: delete this or move to a footnote}It should be noted that, in our definition, every R1CS defines its own language. However,  in more theoretical approaches, another language usually called \term{R1CS satisfiability} is often considered, which is useful when it comes to more abstract problems like expressiveness or the computational complexity of the class of \term{all} R1CS. From our perspective, the R1CS satisfiability language is obtained by the union of all R1CS languages that are in our definition. To be more precise, let the alphabet $\Sigma=\F$ be a field. Then the language $L_{R1CS\_SAT(\F)}$ is defined as follows:
$$
L_{R1CS\_SAT(\F)} = \{(i;w)\in \Sigma^*\times \Sigma^*\;|\; \text{there is a R1CS $R$ such that } R(i;w)=true  \}
$$
\end{remark}
\begin{example}[3-Factorization]
\label{ex:3-fac-R1CS-constr-proof}
Consider the language $L_{3.fac\_zk}$ from \examplename{} \ref{ex:L-3fac-zk} and the R1CS defined in \examplename{} \ref{ex:3-factorization-r1cs}. As we have seen in \ref{ex:3-factorization-r1cs}, solutions to the R1CS are in 1:1 correspondence with solutions to the decision function of $L_{3.fac\_zk}$. Both languages are therefore equivalent in the sense that there is a 1:1 correspondence between words in both languages.

To give an intuition of what constructive R1CS-based proofs in $L_{3.fac\_zk}$ look like, consider the instance $I_1= 11$. To prove the statement ``There exists a witness $W$ such that $(I_1;W)$ is a word in $L_{3.fac\_zk}$'' constructively, a proof has to provide a solution to the R1CS from \examplename{} \ref{ex:3-factorization-r1cs}, that is, an assignments to all witness variables $W_1$, $W_2$, $W_3$ and $W_4$. Since the alphabet is $\F_{13}$, an example assignment is given by
$W=<2,3,4,6>$ since $(I_1;W)$ satisfies the R1CS:
\begin{align*}
W_1 \cdot W_2 &= W_4 & \text{\# } 2\cdot 3 = 6\\
W_4 \cdot W_3 &= I_1 & \text{\# } 6\cdot 4 = 11
\end{align*}
A proper constructive proof is therefore given by $\pi=<2,3,4,6>$. Of course, $\pi$ is not the only possible proof for this statement. Since factorization is not unique in a field in general, another constructive proof is given by $\pi'=<3,5,12,2>$. 

\begin{align*}
W_1 \cdot W_2 &= W_4 & \text{\# } 3\cdot 5 = 2\\
W_4 \cdot W_3 &= I_1 & \text{\# } 12\cdot 2 = 11
\end{align*}

\end{example}

\begin{example}[The \curvename{Tiny-jubjub} curve]\label{ex:tiny-jubjub-r1cs} Consider the language $L_{tiny.jj.1}$ from \examplename{} \ref{ex:tiny-jubjub-instance-witness}, and its associated R1CS from \examplename{} \ref{ex:TJJ-r1cs}. To see how constructive proofs in $L_{tiny.jj.1}$ using the R1CS from \examplename{} \ref{ex:TJJ-r1cs} look like, consider the instance $<I_1,I_2>= <11,6>$. To prove the statement ``There exists a witness $W$ such that $(<I_1,I_2>;W)$ is a word in $L_{tiny.jj.1}$'' constructively, a proof has to provide a solution to the R1CS \ref{ex:TJJ-r1cs} which is an assignment to all witness variables $W_1$, $W_2$ and $W_3$. Since the alphabet is $\F_{13}$, an example assignment is given by
$W=<4,10,8>$ since $(<I_1,I_2>;W)$ satisfies the R1CS:
\begin{align*}
I_1 \cdot I_1 & = W_1 & 11\cdot 11 = 4\\
I_2 \cdot I_2 & = W_2 & 6 \cdot 6 = 10 \\
(8 \cdot W_1) \cdot W_2 & = W_3 & (8\cdot 4)\cdot 10 = 8\\
(12\cdot W_2 + W_3 + 10\cdot W_1 + 1)\cdot 1 & = 0 & 12\cdot 10 + 8 + 10\cdot 4 + 1 = 0
\end{align*}
A proper constructive proof is therefore given by $\pi=<4,10,8>$, which shows that the instance $<11,6>$ is a point on the \curvename{Tiny-jubjub} curve. 
\end{example}
\subsubsection{Modularity}
\label{sec:R1CS_modularity} As we discussed in \ref{modularity}, it is often useful to construct complex statements and their representing languages from simple ones. \concept{rank-1 constraint systems} are particularly useful for this, as the intersection of two R1CS over the same alphabet results in a new R1CS over that same alphabet. 

To be more precise, let $S_1$ and $S_2$ be two R1CS over $\F$. A new R1CS $S_3$ is obtained by the intersection of $S_1$ and $S_2$, that is, $S_3 = S_1\cap S_2$ . In this context, intersection means that both the equations of $S_1$ \term{and} the equations of $S_2$ have to be satisfied in order to provide a solution for the system $S_3$.

As a consequence, developers are able to construct complex R1CS from simple ones. This modularity provides the theoretical foundation for many R1CS compilers, as we will see in \chaptname{} \ref{chap:circuit-compilers}.

\subsection{Algebraic Circuits}
\label{sec:circuits} As we have seen in the previous paragraphs, \concept{rank-1 constraint systems} are quadratic equations such that solutions are knowledge proofs for the existence of words in associated languages. From the perspective of a prover, it is therefore important to solve those equations efficiently. 

However,  in contrast to systems of linear equations, no general methods are known that solve systems of quadratic equations efficiently. \concept{rank-1 constraint systems} are therefore impractical from a provers perspective and auxiliary information is needed that helps to compute solutions efficiently.

Methods which compute R1CS solutions are sometimes called \term{witness generator functions}.  To provide a common example, we introduce another class of decision functions called \term{algebraic circuits}. As we will see, every algebraic circuit defines an associated R1CS and also provides an efficient way to compute solutions for that R1CS. This method is introduced, for example, in \cite{sasson-2013}.

It can be shown that every space- and time-bounded computation is expressible as an algebraic circuit. Transforming high-level computer programs into those circuits is a process often called \term{flattening}. We will look at those transformations in \chaptname{} \ref{chap:circuit-compilers}.

In this section we will introduce our model for algebraic circuits and look at the concept of circuit execution and valid assignments. After that, we will show how to derive \concept{rank-1 constraint systems} from circuits and how circuits are useful to compute solutions to associated R1CS efficiently.
\subsubsection{Algebraic circuit representation} To see what algebraic circuits are, let $\F$ be a field. An algebraic circuit is then a directed acyclic (multi)graph that computes a polynomial function over $\F$. Nodes with only outgoing edges (source nodes) represent the variables and constants of the function and nodes with only incoming edges (sink nodes) represent the outcome of the function. All other nodes have exactly two incoming edges and represent the field operations \term{addition} as well as \term{multiplication}. Graph edges are directed and represent the flow of the computation along the nodes.

To be more precise, in this book, we call a directed acyclic multi-graph $C(\F)$  an \term{algebraic circuit} over $\F$ if the following conditions hold:

\begin{itemize}
\label{def:algebraic-circuit}
\item The set of edges has a total order.  
\item Every source node has a label that represents either a variable or a constant from the field $\F$.
\item Every sink node has exactly one incoming edge and a label that represents either a variable or a constant from the field $\F$.
\item Every node that is neither a source nor a sink has exactly two incoming edges and a label from the set $\{+,*\}$ that represents either addition or multiplication in $\F$.
\item All outgoing edges from a node have the same label.
\item Outgoing edges from a node with a label that represents a variable have a label.
\item Outgoing edges from a node with a label that represents multiplication have a label, if there is at least one labeled edge in both input path.
\item All incoming edges to sink nodes have a label.
\item If an edge has two labels $S_i$ and $S_j$ it gets a new label $S_i = S_j$.
\item No other edge has a label.
\item Incoming edges to labeled sink nodes, where the label is a constant $c\in\F$ are labeled with the same constant. Every other edge label is taken from the set $\{W,I\}$ and indexed compatible with the order of the edge set. 
\end{itemize} 

It should be noted that the details in the definitions of algebraic circuits vary between different sources. We use this definition as it is conceptually straightforward and well-suited for pen-and-paper computations.

To get a better intuition of our definition, let $C(\F)$ be an algebraic circuit. Source nodes are the inputs to the circuit and either represent variables or constants. In a similar way, sink nodes represent termination points of the circuit and are either output variables or constants. Constant sink nodes enforce computational outputs to take on certain values.  

Nodes that are neither source nodes nor sink nodes are called \term{arithmetic gates}. Arithmetic gates that are decorated with the ``$+$"-label are called \term{addition-gates} and arithmetic gates that are decorated with the ``$\cdot$"-label are called \term{multiplication-gates}. Every arithmetic gate has exactly two inputs, represented by the two incoming edges.

Since the set of edges is ordered, we can write it as a string $<E_1,E_2,\ldots, E_n>$ for some $n\in \N$ and we use those indices to index the edge labels, too. Edge labels are therefore either constants or symbols like $I_j$, $W_j$, where $j$ is an index compatible with the edge order. Labels $I_j$ represent instance variables, labels $W_j$ witness variables. Labels on the outgoing edges of input variables constrain the associated variable to that edge. 

\begin{notation}
In synthesizing algebraic circuits, assigning instance $I_j$ or witness $W_j$ labels to appropriate edges is often the final step. It is therefore convenient to not distinguish these two types of edges in previous steps. To account for that, we often simply write $S_j$ for an edge label, indicating that the instance/witness property of the label is unspecified and it might represent both an instance or a witness label. 
\end{notation}
\begin{example}[Generalized factorization SNARK]
\label{ex:3-fac-zk-circuit} To give a simple example of an algebraic circuit, consider our $3$-factorization problem from \examplename{} \ref{ex:L-3fac-zk} again.  To express the problem in the algebraic circuit model, consider the following function 
\[
f_{3.fac}:\mathbb{F}_{13}\times\mathbb{F}_{13}\times\mathbb{F}_{13}\to\mathbb{F}_{13};(x_{1},x_{2},x_{3})\mapsto x_{1}\cdot x_{2}\cdot x_{3}
\]
Using this function, we can describe the zero-knowledge $3$-factorization problem from \ref{ex:L-3fac-zk}, in the following way: Given instance $I_1\in \F_{13}$, a valid witness is a preimage of $f_{3.fac}$ at the point $I_1$, i.e., a valid witness consists of three values $W_1$, $W_2$ and $W_3$ from $\F_{13}$ such that $f_{3.fac}(W_1,W_2,W_3)=I_1$. 

To see how this function can be transformed into an algebraic circuit over $\F_{13}$, it is a common first step to introduce brackets into the function's definition and then write the operations as binary operators, in order to highlight how exactly every field operation acts on its two inputs. Due to the associativity laws in a field, we have several choices. We choose
\begin{align*}
f_{3.fac}(x_1,x_2,x_3) & = x_1\cdot x_2 \cdot x_3  & \text{\# bracket choice} \\
                       & = (x_1\cdot x_2 ) \cdot x_3  & \text{\# operator notation} \\
                       & = MUL(MUL(x_1,x_2),x_3)
\end{align*}
Using this expression, we can write an associated algebraic circuit by first constraining the variables to edge labels $W_1=x_1$, $W_2=x_2$ and $W_3=x_3$ as well as $I_1=f_{3.fac}(x_1,x_2,x_3)$, taking the distinction between witness and instance inputs into account. We then rewrite the operator representation of $f_{3.fac}$ into circuit nodes and get the following: 
\begin{center}
% https://www.graphviz.org/pdf/dotguide.pdf
\digraph[scale=0.5]{G1}{
	forcelabels=true;
	//center=true;
	splines=ortho;
	nodesep= 2.0;
	n1 -> n3 [xlabel="W_2  "];
	n2 -> n3 [xlabel="  W_1"];
	n3 -> n5 [label="W_4"];
	n4 -> n5 [label=" W_3"];
	n5 -> n6 [label=" I_1"];
	n1 [shape=box, label="x_2"];
	n2 [shape=box, label="x_1"];
	n3 [label="*"];
	n4 [shape=box, label="x_3"];
	n5 [label="*"];
	n6 [shape=box, label="f_(3.fac_zk)"];
}
\end{center}
%\[
%\xymatrix{x_1\ar^{W_1}[dr] &  & x_2\ar_{W_2}[dl]\\
% & \cdot \ar^{W_4}[drr] &   & & x_3\ar_{W_3}[dl]\\
%  &  &  & \cdot\ar_{I_1}[d]\\
%  &  &  & f(x_1,x_2,x_3)
%}
%\]


In this case, the directed acyclic multi-graph is a binary tree with three leaves (the source nodes) labeled by $x_1$, $x_2$ and $x_3$, one root (the single sink node) labeled by $f_{3.fac}(x_1,x_2,x_3)$ and two internal nodes, which are labeled as multiplication gates. 

The order we use to label the edges is chosen to make the edge labeling consistent with the choice of $W_4$ as defined in definition \ref{def:algebraic-circuit}. This order can be obtained by a depth-first right-to-left-first traversal algorithm.
\end{example}
\begin{example}
\label{ex:TJJ-circuit_1} To give a more realistic example of an algebraic circuit, look at the defining equation  of the \curvename{Tiny-jubjub} curve \ref{TJJ13} again. A pair of field elements 
$(x,y)\in \F_{13}^2$ is a curve point, precisely if the following equation holds:
$$
3\cdot x^2 + y^2 = 1+ 8\cdot x^2\cdot y^2
$$ 
To understand how one might transform this identity into an algebraic circuit, we first rewrite this equation by shifting all terms to the right. We get the following:
\begin{align*}
3\cdot x^2 + y^2 & = 1+ 8\cdot x^2\cdot y^2 & \Leftrightarrow\\
0 & = 1+ 8\cdot x^2\cdot y^2 - 3\cdot x^2 - y^2 & \Leftrightarrow\\
0 & = 1+ 8\cdot x^2\cdot y^2 + 10\cdot x^2 +12\cdot y^2
\end{align*}
Then we use this expression to define a function such that all points of the \curvename{Tiny-jubjub} curve are characterized as the function preimages at $0$.
$$
f_{tiny-jj}: \F_{13}\times \F_{13}\to \F_{13}\; ; \;
(x,y)\mapsto 1+ 8\cdot x^2\cdot y^2 + 10\cdot x^2 +12\cdot y^2
$$
Every pair of field elements $(x,y)\in \F_{13}^2$ with $f_{tiny-jj}(x,y)=0$ is a point on the \curvename{Tiny-jubjub} curve, and there are no other curve points. The preimage $f_{tiny-jj}^{-1}(0)$ is therefore a complete description of the \curvename{Tiny-jubjub} curve.

We can transform this function into an algebraic circuit over $\F_{13}$. We first introduce brackets into potentially ambiguous expressions and then rewrite the function in terms of binary operators. We get the following:
\begin{align*}
\label{ex:tiny_circuit_brackets}
f_{tiny-jj}(x,y) & = 1+ 8\cdot x^2\cdot y^2 + 10\cdot x^2 +12 y^2  & \Leftrightarrow\\
 & = ((8\cdot ((x\cdot x)\cdot (y\cdot y))) + (1+ 10\cdot (x\cdot x))) + (12\cdot(y\cdot y))  & \Leftrightarrow\\
  & = \scriptstyle ADD(ADD(MUL(8,MUL(MUL(x,x),MUL(y,y))), ADD(1,MUL(10,MUL(x,x)))),MUL(12,MUL(y,y)))
\end{align*}\tbdsm{this needs to be numbered}
Since we haven't decided which part of the computation should be instance and which part should be witness, we use the unspecified symbol $S$ to represent edge labels. Constraining all variables to edge labels $S_1=x$, $S_2=y$ and $S_6=f_{tiny-jj}$, we get the following circuit, representing the function $f_{tiny-jj}$, by inductively replacing binary operators with their associated arithmetic gates:
\begin{center}
% https://www.graphviz.org/pdf/dotguide.pdf
\digraph[scale=0.4]{G2}{
	forcelabels=true;
	//center=true;
	splines=ortho;
	nodesep= 2.0;
	n1 -> n4 [xlabel="S_1" /*, color=lightgray */];
        n1 -> n4 /*[ color=lightgray ]*/
	n2 -> n5 [xlabel="S_2" /*, color=lightgray */];
	n2 -> n5 /*[ color=lightgray ]*/ ;
	n3 -> n8 /*[ color=lightgray ]*/ ;
	n4 -> n8 [taillabel="S_3", labeldistance="2" /*, color=lightgray */];
	n4 -> n10 [taillabel="S_3", labeldistance="4" /*, color=lightgray */];
	n5 -> n10 [xlabel="S_4" /*, color=lightgray */];
	n5 -> n13 [xlabel="S_4" /*, color=lightgray */];
	n6 -> n13 /*[ color=lightgray ]*/ ;
	n7 -> n11 /*[ color=lightgray ]*/ ;
	n8 -> n11 /*[ color=lightgray ]*/ ;
	n9 -> n12 /*[ color=lightgray ]*/ ;
	n10 -> n12 [taillabel="S_5", labeldistance="4" /*, color=lightgray */];
	n11 -> n14 /*[ color=lightgray ]*/ ;
	n12 -> n14 /*[ color=lightgray ]*/ ;	
	n13 -> n15 /*[ color=lightgray ]*/ ;
	n14 -> n15 /*[ color=lightgray ]*/ ;
	n15 -> n16 [label="  S_6", labeldistance="2" /*, color=lightgray */];
	n1 [shape=box, label="x" /*, color=lightgray */];
	n2 [shape=box, label="y" /*, color=lightgray */];
	n3 [shape=box, label="10" /*, color=lightgray */];
	n4 [label="*" /*, color=lightgray */];
	n5 [label="*" /*, color=lightgray */];
	n6 [shape=box, label="12" /*, color=lightgray */];
	n7 [shape=box, label="1" /*, color=lightgray */];
	n8 [label="*" /*, color=lightgray */];
	n9 [shape=box, label="8" /*, color=lightgray */];
	n10 [label="*" /*, color=lightgray */];
	n11 [label="+" /*, color=lightgray */];
	n12 [label="*" /*, color=lightgray */];	
	n13 [label="*" /*, color=lightgray */];
	n14 [label="+" /*, color=lightgray */];
	n15 [label="+" /*, color=lightgray */];
	n16 [shape=box, label="f_tiny-jj" /*, color=lightgray */];		
}
\end{center}
This circuit is not a graph, but a multigraph, since there is more than one edge between some of the nodes. 

In the process of designing  of circuits from functions, it should be noted that circuit representations are not unique in general. In case of the function $f_{tiny-jj}$, the circuit shape is dependent on our choice of bracketing above.%in \ref{ex:tiny_circuit_brackets}. 
An alternative design is for example, given by the following circuit, which occurs when the bracketed expression $8\cdot ( (x\cdot x) \cdot (y\cdot y) )$ is replaced by the expression $(x\cdot x) \cdot ( 8 \cdot (y\cdot y) )$. 
\begin{center}
\digraph[scale=0.4]{G2A}{
	forcelabels=true;
	center=true;
	splines=ortho;
	nodesep= 2.0;
	n1 -> n4 [label="S_1" /* comment*/ labeldistance="4" /*, color=lightgray */];
        n1 -> n4 /*[ color=lightgray ]*/
	n2 -> n6 [label="S_2" /*, color=lightgray */];
	n2 -> n6 /*[ color=lightgray ]*/ ;
	n3 -> n9 /*[ color=lightgray ]*/ ;
	n4 -> n9 [taillabel="S_3", labeldistance="2" /*, color=lightgray */];
	n4 -> n12 [taillabel="S_3", labeldistance="4" /*, color=lightgray */];
	n5 -> n10 /*[ color=lightgray ]*/ ;
	n6 -> n10 [headlabel="S_4", labeldistance="4" /*, color=lightgray */];
	n6 -> n13 [taillabel="S_4", labeldistance="4" /*, color=lightgray */];
	n7 -> n13 /*[ color=lightgray ]*/ ;
	n8 -> n11 /*[ color=lightgray ]*/ ;
	n9 -> n11 /*[ color=lightgray ]*/ ;
	n10 -> n12 /*[ color=lightgray ]*/ ; 
	n11 -> n14 /*[ color=lightgray ]*/ ;
	n12 -> n14 [xlabel="S_5  "  /*, color=lightgray */];	
	n13 -> n15 /*[ color=lightgray ]*/ ;
	n14 -> n15 /*[ color=lightgray ]*/ ;
	n15 -> n16 [label="  S_6", labeldistance="2" /*, color=lightgray */];
	n1 [shape=box, label="x" /*, color=lightgray */];
	n2 [shape=box, label="y" /*, color=lightgray */];
	n3 [shape=box, label="10" /*, color=lightgray */];
	n4 [label="*" /*, color=lightgray */];
	n5 [shape=box, label="8" /*, color=lightgray */];
	n6 [label="*" /*, color=lightgray */];
	n7 [shape=box, label="12" /*, color=lightgray */];
	n8 [shape=box, label="1" /*, color=lightgray */];
	n9 [label="*" /*, color=lightgray */];
	n10 [label="*" /*, color=lightgray */];
	n11 [label="+" /*, color=lightgray */];	
	n12 [label="*" /*, color=lightgray */];	
	n13 [label="*" /*, color=lightgray */];
	n14 [label="+" /*, color=lightgray */];
	n15 [label="+" /*, color=lightgray */];
	n16 [shape=box, label="f_tiny-jj" /*, color=lightgray */];		
}
\end{center}
Of course, both circuits represent the same function, due to the associativity and commutativity laws that hold true in any field.

With a circuit that represents the function $f_{tiny-jj}$, we can now proceed to derive a circuit that constrains arbitrary pairs $(x,y)$ of field elements to be points on the \curvename{Tiny-jubjub} curve. To do so, we have to constrain the output to be zero, that is, we have to constrain $S_6=0$. To indicate this in the circuit, we replace the output variable by the constant $0$ and constrain the related edge label accordingly. We get the following:
\begin{center}
\digraph[scale=0.4]{G2AJA}{
	forcelabels=true;
	center=true;
	splines=ortho;
	nodesep= 2.0;
	n1 -> n4 [label="S_1" /* comment*/ labeldistance="4" /*, color=lightgray */];
        n1 -> n4 /*[ color=lightgray ]*/
	n2 -> n6 [label="S_2" /*, color=lightgray */];
	n2 -> n6 /*[ color=lightgray ]*/ ;
	n3 -> n9 /*[ color=lightgray ]*/ ;
	n4 -> n9 [taillabel="S_3", labeldistance="2" /*, color=lightgray */];
	n4 -> n12 [taillabel="S_3", labeldistance="4" /*, color=lightgray */];
	n5 -> n10 /*[ color=lightgray ]*/ ;
	n6 -> n10 [headlabel="S_4", labeldistance="4" /*, color=lightgray */];
	n6 -> n13 [taillabel="S_4", labeldistance="4" /*, color=lightgray */];
	n7 -> n13 /*[ color=lightgray ]*/ ;
	n8 -> n11 /*[ color=lightgray ]*/ ;
	n9 -> n11 /*[ color=lightgray ]*/ ;
	n10 -> n12 /*[ color=lightgray ]*/ ; 
	n11 -> n14 /*[ color=lightgray ]*/ ;
	n12 -> n14 [xlabel="S_5  "  /*, color=lightgray */];	
	n13 -> n15 /*[ color=lightgray ]*/ ;
	n14 -> n15 /*[ color=lightgray ]*/ ;
	n15 -> n16 [label="  S_6=0", labeldistance="2" /*, color=lightgray */];
	n1 [shape=box, label="x" /*, color=lightgray */];
	n2 [shape=box, label="y" /*, color=lightgray */];
	n3 [shape=box, label="10" /*, color=lightgray */];
	n4 [label="*" /*, color=lightgray */];
	n5 [shape=box, label="8" /*, color=lightgray */];
	n6 [label="*" /*, color=lightgray */];
	n7 [shape=box, label="12" /*, color=lightgray */];
	n8 [shape=box, label="1" /*, color=lightgray */];
	n9 [label="*" /*, color=lightgray */];
	n10 [label="*" /*, color=lightgray */];
	n11 [label="+" /*, color=lightgray */];	
	n12 [label="*" /*, color=lightgray */];	
	n13 [label="*" /*, color=lightgray */];
	n14 [label="+" /*, color=lightgray */];
	n15 [label="+" /*, color=lightgray */];
	n16 [shape=box, label="0" /*, color=lightgray */];		
}
\end{center}
The previous circuit enforces input values assigned to the labels $S_1$ and $S_2$ to be points on the \curvename{Tiny-jubjub} curve. However,  it does not specify which labels are considered instance and which are considered witness. The following circuit defines the inputs to be instances, while all other labels represent witnesses:
\begin{center}
\digraph[scale=0.4]{G2AJA2}{
	forcelabels=true;
	center=true;
	splines=ortho;
	nodesep= 2.0;
	n1 -> n4 [label="I_1" /* comment*/ labeldistance="4" /*, color=lightgray */];
        n1 -> n4 /*[ color=lightgray ]*/
	n2 -> n6 [label="I_2" /*, color=lightgray */];
	n2 -> n6 /*[ color=lightgray ]*/ ;
	n3 -> n9 /*[ color=lightgray ]*/ ;
	n4 -> n9 [taillabel="W_1", labeldistance="2" /*, color=lightgray */];
	n4 -> n12 [taillabel="W_1", labeldistance="4" /*, color=lightgray */];
	n5 -> n10 /*[ color=lightgray ]*/ ;
	n6 -> n10 [headlabel="W_2", labeldistance="4" /*, color=lightgray */];
	n6 -> n13 [taillabel="W_2", labeldistance="4" /*, color=lightgray */];
	n7 -> n13 /*[ color=lightgray ]*/ ;
	n8 -> n11 /*[ color=lightgray ]*/ ;
	n9 -> n11 /*[ color=lightgray ]*/ ;
	n10 -> n12 /*[ color=lightgray ]*/ ; 
	n11 -> n14 /*[ color=lightgray ]*/ ;
	n12 -> n14 [xlabel="W_3  "  /*, color=lightgray */];	
	n13 -> n15 /*[ color=lightgray ]*/ ;
	n14 -> n15 /*[ color=lightgray ]*/ ;
	n15 -> n16 [label="  0", labeldistance="2" /*, color=lightgray */];
	n1 [shape=box, label="x" /*, color=lightgray */];
	n2 [shape=box, label="y" /*, color=lightgray */];
	n3 [shape=box, label="10" /*, color=lightgray */];
	n4 [label="*" /*, color=lightgray */];
	n5 [shape=box, label="8" /*, color=lightgray */];
	n6 [label="*" /*, color=lightgray */];
	n7 [shape=box, label="12" /*, color=lightgray */];
	n8 [shape=box, label="1" /*, color=lightgray */];
	n9 [label="*" /*, color=lightgray */];
	n10 [label="*" /*, color=lightgray */];
	n11 [label="+" /*, color=lightgray */];	
	n12 [label="*" /*, color=lightgray */];	
	n13 [label="*" /*, color=lightgray */];
	n14 [label="+" /*, color=lightgray */];
	n15 [label="+" /*, color=lightgray */];
	n16 [shape=box, label="0" /*, color=lightgray */];		
}
\end{center} 
\end{example}
It can be shown that every space- and time-bounded computation can be transformed into an algebraic circuit. We call any process that transforms a bounded computation into a circuit \term{flattening}.

\subsubsection{Circuit Execution} Algebraic circuits are directed, acyclic multi-graphs, where nodes represent variables, constants, or addition and multiplication gates. In particular, every algebraic circuit with $n$ input nodes decorated with variable symbols and $m$ output nodes decorated with variables can be seen as a function that transforms an input string $(x_1,\ldots, x_n)$ from $\F^n$ into an output string $(f_1,\ldots,f_m)$ from $\F^m$. The transformation is done by sending values associated to nodes along their outgoing edges to other nodes. If those nodes are gates, then the values are transformed according to the gate label and the process is repeated along all edges until a sink node is reached. We call this computation \term{circuit execution}.

When executing a circuit, it is possible to not only compute the output values of the circuit but to derive field elements for all edges, and, in particular, for all edge labels in the circuit. The result is a string $<S_1,S_2,\ldots, S_n>$ of field elements associated to all labeled edges, which we call a \term{valid assignment} to the circuit. In contrast, any assignment $<S'_1,S'_2,\ldots, S'_n>$ of field elements to edge labels that can not arise from circuit execution is called an \term{invalid assignment}.

Valid assignments can be interpreted as \term{proofs for proper circuit execution} because they keep a record of the computational result as well as intermediate computational steps. 
\begin{example}[3-factorization]
\label{ex:3-fac-zk-circuit_2}
Consider the $3$-factorization problem from \examplename{} \ref{ex:L-3fac-zk} and its representation as an algebraic circuit from \examplename{} \ref{ex:3-fac-zk-circuit}. We know that the string of edge labels is given by $S:=<I_{1};W_{1},W_{2},W_{3}, W_{4}>$. 

To understand how this circuit is executed, consider the variables $x_1=2$, $x_2=3$ as well as $x_3=4$. Following all edges in the graph, we get the assignments $W_1=2$, $W_2=3$ and $W_3=4$. Then the assignments of $W_1$ and $W_2$ enter a multiplication gate and the output of the gate is $2\cdot 3 = 6$, which we assign to $W_4$, i.e. $W_4=6$. The values $W_4$ and $W_3$ then enter the second multiplication gate and the output of the gate is $6\cdot 4 = 11$, which we assign to $I_1$, i.e. $I_1=11$. 

A valid assignment to the 3-factorization circuit $C_{3.fac}(\F_{13})$ is therefore given by the following string of field elements from $\F_{13}$:

\begin{equation}\label{C3fac}
S_{valid}:=<11;2,3,4,6>
\end{equation}

We can visualise this assignment by assigning every computed value to its associated label in the circuit as follows:
\begin{center}
% https://www.graphviz.org/pdf/dotguide.pdf
\digraph[scale=0.5]{G3}{
	forcelabels=true;
	//center=true;
	splines=ortho;
	nodesep= 2.0;
	n1 -> n3 [xlabel="W_2=3  "];
	n2 -> n3 [xlabel="W_1=2 "];
	n3 -> n5 [label="W_4=6"];
	n4 -> n5 [label=" W_3=4"];
	n5 -> n6 [label=" I_1=11"];
	n1 [shape=box, label="x_2"];
	n2 [shape=box, label="x_1"];
	n3 [label="*"];
	n4 [shape=box, label="x_3"];
	n5 [label="*"];
	n6 [shape=box, label="f_(3.fac_zk)"];
}
\end{center}
To see what an invalid assignment looks like, consider the assignment $S_{err}:=<8;2,3,4,7>$. In this assignment, the input values are the same as in the previous case. The associated circuit is:
\begin{center}
% https://www.graphviz.org/pdf/dotguide.pdf
\digraph[scale=0.5]{G4}{
	forcelabels=true;
	//center=true;
	splines=ortho;
	nodesep= 2.0;
	n1 -> n3 [xlabel="W_2=3  "];
	n2 -> n3 [xlabel="W_1=2 "];
	n3 -> n5 [label="W_4=7"];
	n4 -> n5 [label=" W_3=4"];
	n5 -> n6 [label=" I_1=8"];
	n1 [shape=box, label="x_2"];
	n2 [shape=box, label="x_1"];
	n3 [label="*"];
	n4 [shape=box, label="x_3"];
	n5 [label="*"];
	n6 [shape=box, label="f_(3.fac_zk)"];
}
\end{center}
This assignment is invalid, as the assignments of $I_1$ and $W_4$ cannot be obtained by executing the circuit.
\end{example}
\begin{example}
\label{ex:TJJ-circuit_2} 
 To compute a more realistic algebraic circuit execution, consider the defining circuit $C_{tiny-jj}(\F_{13})$ from \examplename{} \ref{ex:TJJ-circuit_1} again. We already know from the way this circuit is constructed that any valid assignment with $S_1=x$, $S_2=y$ and $S_6=0$ will ensure that the pair $(x,y)$ is a point on the \curvename{Tiny-jubjub} curve  in its Edwards representation (equation \ref{TJJ13-twisted-edwards}. 

From \examplename{} \ref{TJJ13-twisted-edwards}, we know that the pair $(11,6)$ is a proper point on the \curvename{Tiny-jubjub} curve and we use this point as input to a circuit execution. We get the following:
\begin{center}
% https://www.graphviz.org/pdf/dotguide.pdf
\digraph[scale=0.4]{G2C}{
	forcelabels=true;
	//center=true;
	splines=ortho;
	nodesep= 2.0;
	n1 -> n4 [label="S_1=11" labeldistance="4" /*, color=lightgray */];
        n1 -> n4 /*[ color=lightgray ]*/
	n2 -> n5 [label="S_2=6" /*, color=lightgray */];
	n2 -> n5 /*[ color=lightgray ]*/ ;
	n3 -> n8 /*[ color=lightgray ]*/ ;
	n4 -> n8 [taillabel="S_3=4  " /*, color=lightgray */];
	n4 -> n10 [taillabel="S_3=4", labeldistance="4" /*, color=lightgray */];
	n5 -> n10 [xlabel="S_4=10 " /*, color=lightgray */];
	n5 -> n13 [headlabel="S_4=10", labeldistance="4" /*, color=lightgray */];
	n6 -> n13 /*[ color=lightgray ]*/ ;
	n7 -> n11 /*[ color=lightgray ]*/ ;
	n8 -> n11 [headlabel="[10*4=1]    "];
	n9 -> n12 /*[ color=lightgray ]*/ ;
	n10 -> n12 [taillabel="S_5=1  " /*, color=lightgray */];
	n11 -> n14 [headlabel="[1+1=2]    "];
	n12 -> n14 [label="  [8*1=8]"];	
	n13 -> n15 [headlabel="    [10*12=3]"];
	n14 -> n15 [taillabel="   [2+8=10]"];
	n15 -> n16 [label="  S_6=0", labeldistance="2" /*, color=lightgray */];
	n1 [shape=box, label="x" /*, color=lightgray */];
	n2 [shape=box, label="y" /*, color=lightgray */];
	n3 [shape=box, label="10" /*, color=lightgray */];
	n4 [label="*" /*, color=lightgray */];
	n5 [label="*" /*, color=lightgray */];
	n6 [shape=box, label="12" /*, color=lightgray */];
	n7 [shape=box, label="1" /*, color=lightgray */];
	n8 [label="*" /*, color=lightgray */];
	n9 [shape=box, label="8" /*, color=lightgray */];
	n10 [label="*" /*, color=lightgray */];
	n11 [label="+" /*, color=lightgray */];
	n12 [label="*" /*, color=lightgray */];	
	n13 [label="*" /*, color=lightgray */];
	n14 [label="+" /*, color=lightgray */];
	n15 [label="+" /*, color=lightgray */];
	n16 [shape=box, label="0" /*, color=lightgray */];		
}
\end{center}
Executing the circuit, we indeed compute $S_6=0$ as expected, which proves that $(11,6)$ is a point on the \curvename{Tiny-jubjub} curve in its Edwards representation. A valid assignment of $C_{tiny-jj}(\F_{13})$ is therefore given by the following string:
$$
S_{tiny-jj} = <S_1, S_2, S_3, S_4, S_5, S_6> = <11, 6, 4, 10, 1, 0>
$$
\end{example}
\subsubsection{Circuit Satisfiability}
\label{circuit-satisfiability} To understand how algebraic circuits give rise to formal languages, observe that every algebraic circuit $C(\F)$ over a fields $\F$ defines a decision function over the alphabet $\Sigma_I \times \Sigma_W = \F \times \F$ in the following way:
\begin{equation}
R_{C(\F)} : \F^* \times \F^* \to \{true, false\}\;;\;
(I;W) \mapsto
\begin{cases}
true & (I;W) \text{is valid assignment to } C(\F)\\
false & else
\end{cases}
\end{equation}
Every algebraic circuit therefore defines a formal language. The grammar of this language is encoded in the shape of the circuit, words are assignments to edge labels that are derived from circuit execution, and \term{statements} are knowledge claims ``Given instance $I$, there is a witness $W$ such that $(I;W)$ is a valid assignment to the circuit". A constructive proof to this claim is therefore an assignment of a field element to every witness variable, which is verified by executing the circuit to see if the assignment of the execution meets the assignment of the proof. 

In the context of zero-knowledge proof systems, executing circuits is also often called \term{witness generation}, since in applications the instance part is usually public, while its the task of a prover to compute the witness part.

\begin{remark}[Circuit satisfiability] Similar to \ref{r1cs-satisfiability}, it should be noted that, in our definition, every circuit defines its own language. However, in more theoretical approaches another language usually called \term{circuit satisfiability} is often considered, which is useful when it comes to more abstract problems like expressiveness, or computational complexity of the class of \term{all} algebraic circuits over a given field. From our perspective, the circuit satisfiability language is obtained by union of all circuit languages that are in our definition. To be more precise, let the alphabet $\Sigma=\F$ be a field. Then 
$$
L_{CIRCUIT\_SAT(\F)} = \{(i;w)\in \Sigma^*\times \Sigma^*\;|\; \text{there is a circuit } C(\F) \text{ such that } (i;w) \text{ is valid assignment}\}
$$
\end{remark}
\begin{example}[3-Factorization]Consider the circuit $C_{3.fac}$ from \examplename{} \ref{ex:3-fac-zk-circuit} again. We call the associated language $L_{3.fac\_circ}$.

To understand how a constructive proof of a statement in $L_{3.fac\_circ}$ looks like, consider the instance $I_1= 11$. To provide a proof for the statement ``There exist a witness $W$ such that $(I_1;W)$ is a word in $L_{3.fac\_circ}$'' a proof therefore has to consists of proper values for the variables $W_1$, $W_2$, $W_3$ and $W_4$. Any prover therefore has to find input values for $W_1$, $W_2$ and $W_3$ and then execute the circuit to compute $W_4$ under the assumption $I_1=11$. 

Example \ref{ex:3-fac-zk-circuit_2}implies that $<2,3,4,6>$ is a proper constructive proof and in order to verify the proof a verifier needs to execute the circuit with instance $I_1=11$ and inputs $W_1=2$, $W_2=3$ and $W_3=4$ to decide whether the proof is a valid assignment or not. 
\end{example}
\begin{exercise}
Consider the circuit $C_{tiny-jj}(\F_{13})$ from \examplename{} \ref{ex:TJJ-circuit_1}, with its associated language $L_{tiny-jj}$. Construct a proof $\pi$ for the instance $<11,6>$ and verify the proof.
\end{exercise}

\subsubsection{Associated Constraint Systems}
\label{sec:circuits_associated_R1CS} As we have seen in \ref{sec:R1CS}, \concept{rank-1 constraint systems} define a way to represent statements in terms of a system of quadratic equations over finite fields, suitable for pairing-based zero-knowledge proof systems. However,  those equations provide no practical way for a prover to actually compute a solution. On the other hand, algebraic circuits can be executed in order to derive valid assignments efficiently. 

In this paragraph, we show how to transform any algebraic circuit into a \concept{rank-1 constraint system} such that valid circuit assignments are in 1:1 correspondence with solutions to the associated R1CS. 

To see this, let $C(\F)$ be an algebraic circuit over a finite field $\F$, with a string of edge labels $<S_1,S_2,\ldots, S_n>$. Then we start with an empty R1CS and one of the following steps is executed for every edge label $S_j$ from that set:
\begin{itemize}
\label{algebraic-gates}
\item If the edge label $S_j$ is an outgoing edge of a multiplication gate, the R1CS gets a new quadratic constraint
\begin{equation}
(\text{left input})\cdot (\text{right input}) = S_j
\end{equation} 
In this expression $(\text{left input})$ is the output from the symbolic execution of the subgraph that consists of the left input edge of this gate and all edges and nodes that have  this edge in their path, starting with constant inputs or labeled outgoing edges of other nodes. 

In the same way $(\text{right input})$ is the output from the symbolic execution of the subgraph that consists of the right input edge of this gate and all edges and nodes that have this edge in their path, starting with constant inputs or labeled outgoing edges of other nodes.
\item If the edge label $S_j$ is an outgoing edge of an addition gate, the R1CS gets a new quadratic constraint
\begin{equation}
(\text{left input} + \text{right input})\cdot 1 = S_j
\end{equation}  
In this expression $(\text{left input})$ is the output from the symbolic execution of the subgraph that consists of the left input edge of this gate and all edges and nodes that have  this edge in their path, starting with constant inputs or labeled outgoing edges of other nodes. 

In the same way $(\text{right input})$ is the output from the symbolic execution of the subgraph that consists of the right input edge of this gate and all edges and nodes that have this edge in their path, starting with constant inputs or labeled outgoing edges of other nodes.
\item No other edge label adds a constraint to the system.
\end{itemize}

If an algebraic circuit $C(\F)$ is constructed according to the rules from \ref{def:algebraic-circuit}, the result of this method is a \concept{rank-1 constraint system}, and, in this sense, every algebraic circuit $C(\F)$ generates a R1CS $R$, which we call the \term{associated R1CS} of the circuit. It can be shown that a string of field elements $<S_1,S_2,\ldots, S_n>$ is a valid assignment to a circuit if and only if the same string is a solution to the associated R1CS. Circuit executions therefore compute solutions to \concept{rank-1 constraint systems} efficiently. 

To understand the contribution of algebraic gates to the number of constraints, note that,
according to construction \ref{def:algebraic-circuit}, multiplication gates have labels on their outgoing edges if and only if there is at least one labeled edge in both input paths, or if the outgoing edge is an input to a sink node. This implies that multiplication with a constant is essentially free in the sense that it doesn't add a new constraint to the system, as long as that multiplication gate is not am input to an output node. 

Moreover, addition gates have labels on their outgoing edges if and only if they are inputs to sink nodes. This implies that addition is essentially free in the sense that it doesn't add a new constraint to the system, as long as that addition gate is not an input to an output node. 

\begin{example}[$3$-factorization] Consider our $3$-factorization problem from \examplename{} \ref{ex:L-3fac-zk} and the associated circuit $C_{3.fac}(\F_{13})$ from \examplename{} \ref{ex:3-fac-zk-circuit}. Our task is to transform this circuit into an equivalent \concept{rank-1 constraint system}.
\begin{center}
\digraph[scale=0.5]{G1}{
	forcelabels=true;
	//center=true;
	splines=ortho;
	nodesep= 2.0;
	n1 -> n3 [xlabel="W_2  "];
	n2 -> n3 [xlabel="  W_1"];
	n3 -> n5 [label="W_4"];
	n4 -> n5 [label=" W_3"];
	n5 -> n6 [label=" I_1"];
	n1 [shape=box, label="x_2"];
	n2 [shape=box, label="x_1"];
	n3 [label="*"];
	n4 [shape=box, label="x_3"];
	n5 [label="*"];
	n6 [shape=box, label="f_(3.fac_zk)"];
}
\end{center}
We start with an empty R1CS, and, in order to generate all constraints, we have to iterate over the set of edge labels $<I_1;W_1,W_2,W_3,W_4>$.  

Starting with the edge label $I_1$, we see that it is an outgoing edge of a multiplication gate, and, since both input edges are labeled, we  have to add the following constraint to the system:
\begin{align*}
(\text{left input})\cdot (\text{right input})  &= I_1 & \Leftrightarrow\\
W_4\cdot W_3  &= I_1
\end{align*}
Next, we consider the edge label $W_1$ and, since, it's not an outgoing edge of a multiplication or addition gate, we don't add a constraint to the system. The same holds true for the labels $W_2$ and $W_3$. 

For edge label $W_4$ , we see that it is an outgoing edge of a multiplication gate, and, since both input edges are labeled, we have to add the following constraint to the system:
\begin{align*}
(\text{left input})\cdot (\text{right input})  &= W_4 & \Leftrightarrow\\
W_2\cdot W_1  &= W_4
\end{align*} 
Since there are no more labeled edges, all constraints are generated, and we have to combine them to get the associated R1CS of $C_{3.fac}(\F_{13})$: 
\begin{align*}
 W_4\cdot W_3 & = I_1\\
 W_2\cdot W_1 & = W_4
\end{align*}
This system is equivalent to the R1CS we derived in \examplename{} \ref{ex:3-factorization-r1cs}. The languages $L_{3.fac\_zk}$ and $L_{3.fac\_circ}$ are therefore equivalent and both the circuit as well as the R1CS are just two different ways of expressing the same language.
\end{example}
\begin{example}
\label{ex:tjj-circuit-2-tjj-R1CS} To consider a more general transformation, we consider the \curvename{Tiny-jubjub} circuit from \examplename{} \ref{ex:TJJ-circuit_2} again. A proper circuit is given by the following graph, where we highlighted all nodes that contribute a constraint to the R1CS:
\begin{center}
\digraph[scale=0.4]{G2D}{
	forcelabels=true;
	center=true;
	splines=ortho;
	nodesep= 2.0;
	n1 -> n4 [label="S_1" labeldistance="4" /*, color=lightgray */];
        n1 -> n4 /*[ color=lightgray ]*/
	n2 -> n6 [label="S_2" /*, color=lightgray */];
	n2 -> n6 /*[ color=lightgray ]*/ ;
	n3 -> n9 /*[ color=lightgray ]*/ ;
	n4 -> n9 [taillabel="S_3", labeldistance="2" /*, color=lightgray */];
	n4 -> n13 [taillabel="S_3", labeldistance="4" /*, color=lightgray */];
	n5 -> n10 /*[ color=lightgray ]*/ ;
	n6 -> n10 [headlabel="S_4", labeldistance="4" /*, color=lightgray */];
	n6 -> n11 [taillabel="S_4", labeldistance="4" /*, color=lightgray */];
	n7 -> n11 /*[ color=lightgray ]*/ ;
	n8 -> n12 /*[ color=lightgray ]*/ ;
	n9 -> n12 /*[ color=lightgray ]*/ ;
	n10 -> n13 /*[ color=lightgray ]*/ ; 
	n11 -> n15 /*[ color=lightgray ]*/ ;
	n12 -> n14 /*[ color=lightgray ]*/ ;	
	n13 -> n14 [xlabel="S_5  "  /*, color=lightgray */];
	n14 -> n15 /*[ color=lightgray ]*/ ;
	n15 -> n16 [label="  S_6=0", labeldistance="2" /*, color=lightgray */];
	n1 [shape=box, label="x" /*, color=lightgray */];
	n2 [shape=box, label="y" /*, color=lightgray */];
	n3 [shape=box, label="10" /*, color=lightgray */];
	n4 [label="*", style=filled /*, color=lightgray */];
	n5 [shape=box, label="8" /*, color=lightgray */];
	n6 [label="*", style=filled /*, color=lightgray */];
	n7 [shape=box, label="12" /*, color=lightgray */];
	n8 [shape=box, label="1" /*, color=lightgray */];
	n9 [label="*" /*, color=lightgray */];
	n10 [label="*" /*, color=lightgray */];
	n11 [label="*" /*, color=lightgray */];	
	n12 [label="+" /*, color=lightgray */];	
	n13 [label="*", style=filled /*, color=lightgray */];
	n14 [label="+" /*, color=lightgray */];
	n15 [label="+", style=filled /*, color=lightgray */];
	n16 [shape=box, label="0" /*, color=lightgray */];		
}
\end{center}
To compute the number of constraints, observe that we have $3$ multiplication gates that have labels on their outgoing edges and $1$ addition gate that has a label on its outgoing edge. We therefore have to compute $4$ quadratic constraints. 

In order to derive the associated R1CS, we have start with an empty R1CS and then iterate over the set $<S_1,S_2,S_3,S_4,S_5,S_6=0>$ of all edge labels, in order to generate the constraints. 

Considering edge label $S_1$, we see that the associated edges are not outgoing edges of any algebraic gate, and we therefore have to add no new constraint to the system. The same holds true for edge label $S_2$. Looking at edge label $S_3$, we see that the associated edges are outgoing edges of a multiplication gate and that the associated subgraph is given by:
\begin{center}
\digraph[scale=0.4]{G2E}{
	forcelabels=true;
	center=true;
	splines=ortho;
	nodesep= 2.0;
	n1 -> n4 [label="S_1" labeldistance="4" /*, color=lightgray */];
    n1 -> n4 /*[ color=lightgray ]*/
	n2 -> n6 [label="S_2", color=lightgray];
	n2 -> n6 [ color=lightgray ] ;
	n3 -> n9 [ color=lightgray ] ;
	n4 -> n9 [taillabel="S_3", labeldistance="2" /*, color=lightgray */];
	n4 -> n13 [taillabel="S_3", labeldistance="4" /*, color=lightgray */];
	n5 -> n10 [ color=lightgray ] ;
	n6 -> n10 [headlabel="S_4", labeldistance="4", color=lightgray];
	n6 -> n11 [taillabel="S_4", labeldistance="4", color=lightgray];
	n7 -> n11 [ color=lightgray ] ;
	n8 -> n12 [ color=lightgray ] ;
	n9 -> n12 [ color=lightgray ] ;
	n10 -> n13 [ color=lightgray ] ; 
	n11 -> n15 [ color=lightgray ] ;
	n12 -> n14 [ color=lightgray ] ;	
	n13 -> n14 [xlabel="S_5  ", color=lightgray];
	n14 -> n15 [ color=lightgray ] ;
	n15 -> n16 [label="  S_6=0", labeldistance="2", color=lightgray];
	n1 [shape=box, label="x" /*, color=lightgray */];
	n2 [shape=box, label="y", color=lightgray];
	n3 [shape=box, label="10", color=lightgray ];
	n4 [label="*", style=filled /*, color=lightgray */];
	n5 [shape=box, label="8", color=lightgray];
	n6 [label="*", style=filled, color=lightgray];
	n7 [shape=box, label="12", color=lightgray];
	n8 [shape=box, label="1", color=lightgray];
	n9 [label="*", color=lightgray];
	n10 [label="*", color=lightgray];
	n11 [label="*", color=lightgray];	
	n12 [label="+", color=lightgray];	
	n13 [label="*", style=filled, color=lightgray];
	n14 [label="+", color=lightgray];
	n15 [label="+", style=filled, color=lightgray];
	n16 [shape=box, label="0", color=lightgray];		
}
\end{center}
Both the left and the right input to this multiplication gate are labeled by $S_1$. We therefore have to add the following constraint to the system:
$$
S_1 \cdot S_1 = S_3
$$
Looking at edge label $S_4$, we see that the associated edges are outgoing edges of a multiplication gate and that the associated subgraph is given by:
\begin{center}
\digraph[scale=0.4]{G2F}{
	forcelabels=true;
	center=true;
	splines=ortho;
	nodesep= 2.0;
	n1 -> n4 [label="S_1" labeldistance="4", color=lightgray];
        n1 -> n4 [ color=lightgray ]
	n2 -> n6 [label="S_2" /*, color=lightgray */];
	n2 -> n6 /* [ color=lightgray ] */ ;
	n3 -> n9 [ color=lightgray ] ;
	n4 -> n9 [taillabel="S_3", labeldistance="2", color=lightgray];
	n4 -> n13 [taillabel="S_3", labeldistance="4", color=lightgray];
	n5 -> n10 [ color=lightgray ] ;
	n6 -> n10 [headlabel="S_4", labeldistance="4" /*, color=lightgray */];
	n6 -> n11 [taillabel="S_4", labeldistance="4" /*, color=lightgray */];
	n7 -> n11 [ color=lightgray ] ;
	n8 -> n12 [ color=lightgray ] ;
	n9 -> n12 [ color=lightgray ] ;
	n10 -> n13 [ color=lightgray ] ; 
	n11 -> n15 [ color=lightgray ] ;
	n12 -> n14 [ color=lightgray ] ;	
	n13 -> n14 [xlabel="S_5  ", color=lightgray];
	n14 -> n15 [ color=lightgray ] ;
	n15 -> n16 [label="  S_6=0", labeldistance="2", color=lightgray];
	n1 [shape=box, label="x", color=lightgray];
	n2 [shape=box, label="y", /* color=lightgray */];
	n3 [shape=box, label="10", color=lightgray];
	n4 [label="*", style=filled, color=lightgray];
	n5 [shape=box, label="8", color=lightgray];
	n6 [label="*", style=filled /*, color=lightgray */];
	n7 [shape=box, label="12", color=lightgray];
	n8 [shape=box, label="1", color=lightgray];
	n9 [label="*", color=lightgray];
	n10 [label="*", color=lightgray];
	n11 [label="*", color=lightgray];	
	n12 [label="+", color=lightgray];	
	n13 [label="*", style=filled, color=lightgray];
	n14 [label="+", color=lightgray];
	n15 [label="+", style=filled, color=lightgray];
	n16 [shape=box, label="0", color=lightgray];		
}
\end{center}
Both the left and the right input to this multiplication gate are labeled by $S_2$ and we therefore have to add the following constraint to the system:
$$
S_2 \cdot S_2 = S_4
$$
Edge label $S_5$ is more interesting. To see how it implies a constraint, we have to construct the associated subgraph first, which consists of all edges, nodes, and paths, starting either at a constant input or a labeled edge. We get  
\begin{center}
\digraph[scale=0.4]{G2G}{
	forcelabels=true;
	center=true;
	splines=ortho;
	nodesep= 2.0;
	n1 -> n4 [label="S_1" labeldistance="4", color=lightgray];
        n1 -> n4 [ color=lightgray ]
	n2 -> n6 [label="S_2", color=lightgray];
	n2 -> n6 [ color=lightgray ] ;
	n3 -> n9 [ color=lightgray ] ;
	n4 -> n9 [taillabel="S_3", labeldistance="2", color=lightgray];
	n4 -> n13 [taillabel="S_3", labeldistance="4" /*, color=lightgray */];
	n5 -> n10 /*[ color=lightgray ]*/ ;
	n6 -> n10 [headlabel="S_4", labeldistance="4" /*, color=lightgray */];
	n6 -> n11 [taillabel="S_4", labeldistance="4", color=lightgray ];
	n7 -> n11 [ color=lightgray ] ;
	n8 -> n12 [ color=lightgray ] ;
	n9 -> n12 [ color=lightgray ] ;
	n10 -> n13 /*[ color=lightgray ]*/ ; 
	n11 -> n15 [ color=lightgray ] ;
	n12 -> n14 [ color=lightgray ] ;	
	n13 -> n14 [xlabel="S_5  "  /*, color=lightgray */];
	n14 -> n15 [ color=lightgray ] ;
	n15 -> n16 [label="  S_6=0", labeldistance="2" , color=lightgray ];
	n1 [shape=box, label="x", color=lightgray];
	n2 [shape=box, label="y", color=lightgray];
	n3 [shape=box, label="10", color=lightgray];
	n4 [label="*", style=filled /*, color=lightgray*/];
	n5 [shape=box, label="8" /*, color=lightgray */];
	n6 [label="*", style=filled /*, color=lightgray */];
	n7 [shape=box, label="12", color=lightgray];
	n8 [shape=box, label="1", color=lightgray];
	n9 [label="*", color=lightgray];
	n10 [label="*" /*, color=lightgray */];
	n11 [label="*", color=lightgray];	
	n12 [label="+", color=lightgray];	
	n13 [label="*", style=filled /*, color=lightgray */];
	n14 [label="+", color=lightgray];
	n15 [label="+", style=filled, color=lightgray];
	n16 [shape=box, label="0", color=lightgray];		
}
\end{center}
The right input to the associated multiplication gate is given by the labeled edge $S_3$. However,  the left input is not a labeled edge, but has a labeled edge in one of its path. To compute the left factor of that constraint, we have to compute the output of the subgraph associated to the left edge, which is $S_4\cdot 8$. This gives the constraint
$$
(S_4 \cdot 8) \cdot S_3 = S_5
$$ 
The last edge label is the constant $S_6=0$. To see how it implies a constraint, we have to construct the associated subgraph, which consists of all edges, nodes, and paths, starting either at a constant input or a labeled edge. We get
\begin{center}
\digraph[scale=0.4]{G2H}{
	forcelabels=true;
	center=true;
	splines=ortho;
	nodesep= 2.0;
	n1 -> n4 [label="S_1" /* comment*/ labeldistance="4", color=lightgray];
    n1 -> n4 [ color=lightgray ]
	n2 -> n6 [label="S_2", color=lightgray];
	n2 -> n6 [ color=lightgray ] ;
	n3 -> n9 /*[ color=lightgray ]*/ ;
	n4 -> n9 [taillabel="S_3", labeldistance="2" /*, color=lightgray */];
	n4 -> n13 [taillabel="S_3", labeldistance="4" , color=lightgray ];
	n5 -> n10 [ color=lightgray ] ;
	n6 -> n10 [headlabel="S_4", labeldistance="4" , color=lightgray ];
	n6 -> n11 [taillabel="S_4", labeldistance="4" /*, color=lightgray */];
	n7 -> n11 /*[ color=lightgray ]*/ ;
	n8 -> n12 /*[ color=lightgray ]*/ ;
	n9 -> n12 /*[ color=lightgray ]*/ ;
	n10 -> n13 [ color=lightgray ] ; 
	n11 -> n15 /*[ color=lightgray ]*/ ;
	n12 -> n14 /*[ color=lightgray ]*/ ;	
	n13 -> n14 [xlabel="S_5  "  /*, color=lightgray */];
	n14 -> n15 /*[ color=lightgray ]*/ ;
	n15 -> n16 [label="  S_6=0", labeldistance="2" /*, color=lightgray */];
	n1 [shape=box, label="x", color=lightgray];
	n2 [shape=box, label="y", color=lightgray];
	n3 [shape=box, label="10" /*, color=lightgray */];
	n4 [label="*", style=filled /*, color=lightgray */];
	n5 [shape=box, label="8", color=lightgray];
	n6 [label="*", style=filled /*, color=lightgray */];
	n7 [shape=box, label="12" /*, color=lightgray */];
	n8 [shape=box, label="1" /*, color=lightgray */];
	n9 [label="*" /*, color=lightgray */];
	n10 [label="*" , color=lightgray];
	n11 [label="*" /*, color=lightgray */];	
	n12 [label="+" /*, color=lightgray */];	
	n13 [label="*", style=filled /*, color=lightgray */];
	n14 [label="+" /*, color=lightgray */];
	n15 [label="+", style=filled /*, color=lightgray */];
	n16 [shape=box, label="0" /*, color=lightgray */];		
}
\end{center}
Both the left and the right input are  unlabeled, but have a labeled edges in their path. Since the gate is an addition gate, the right factor in the quadratic constraint is always $1$ and the left factor is computed by symbolically executing all inputs to all gates in the sub-circuit. We get
$$
(12\cdot S_4 + S_5 + 10\cdot S_3 + 1)\cdot 1 = 0
$$

Since there are no more labeled outgoing edges, we are done deriving the constraints. Combining all constraints together, we get the following R1CS:
\begin{align*}
 S_1 \cdot S_1 &= S_3\\
 S_2 \cdot S_2 &= S_4\\
 (S_4\cdot 8)\cdot S_3 &= S_5\\
 (12\cdot S_4 + S_5 + 10\cdot S_3 + 1)\cdot 1 &= 0
\end{align*}
which is equivalent to the R1CS we derived in \examplename{} \ref{ex:TJJ-r1cs} both the circuit as well as the R1CS are just two different ways to express the same language.
\end{example}
\subsection{Quadratic Arithmetic Programs}
\label{sec:QAP}
 We have introduced algebraic circuits and their associated \concept{rank-1 constraint systems} as two particular models able to represent bounded computation. Both models define formal languages, and associated membership as well as knowledge claims can be proofed in a constructive way by executing the circuit in order to compute solutions to its associated R1CS. 

One reason why those systems are useful in the context of succinct zero-knowledge proof systems is because any R1CS can be transformed into another computational model called a \term{Quadratic Arithmetic Program} [QAP], which serves as the basis for some of the most efficient succinct non-interactive zero-knowledge proof generators that currently exist. 

As we will see, proving statements for languages that have decision functions defined by Quadratic Arithmetic Programs can be achieved by providing certain polynomials, and those proofs can be verified by checking a particular divisibility property of those polynomials.
 
\subsubsection{QAP representation} To understand what Quadratic Arithmetic Programs are in detail, let $\F$ be a field and $R$ a \concept{rank-1 constraint system} over $\F$ such that the number of non-zero elements in $\F$ is strictly larger than the number $k$ of constraints in $R$. Moreover, let $a_j^i$, $b_j^i$ and $c_j^i\in\F$ for every index $0\leq j \leq n+m$ and $1\leq i \leq k$, be the defining constants of the R1CS and $m_1$, $\ldots$, $m_k$ be arbitrary, invertible and distinct elements from $\F$.
  
Then a \term{Quadratic Arithmetic Program} associated to the R1CS $R$ is the following set of polynomials over $\F$:
\begin{equation}
\label{def:QAP-target-poly}
QAP(R) = \left\{T\in \F[x],\left\{A_j,B_j,C_j\in \F[x]\right\}_{j=0}^{n+m}\right\}
\end{equation}
Here $T(x) := \Pi_{l=1}^k (x- m_l)$ is a polynomial of degree $k$, called the \term{target polynomial} of the QAP and $A_j$, $B_j$ as well as $C_j$ are the unique degree $k-1$ polynomials defined by the following equation:
\begin{equation}
\label{def:QAP-polynomials}
\begin{array}{lllr}
A_j(m_i)=a_j^i, & B_j(m_i)=b_j^i, & C_j(m_i)=c_j^i & \text{ for all } j= 1, \ldots , n+m+1, i=1,\ldots,k 
\end{array}
\end{equation}
Given some \concept{rank-1 constraint system}, an associated Quadratic Arithmetic Program is therefore a set of polynomials, computed from the constants in the R1CS. To see that the polynomials $A_j$, $B_j$ and $C_j$ are uniquely defined by the equations \ref{def:QAP-polynomials}, recall that a polynomial of degree $k-1$ is completely determined by $k$ evaluation points and it can be computed for example by Lagrange interpolation \ref{alg_lagrange_interplation}.

Computing a QAP from any given R1CS can be achieved in the following three steps. If the R1CS consists of $k$ constraints, first choose $k$ different, invertible element from the field $\F$. Every choice defines a different QAP for the same R1CS. Then compute the target polynomial $T$ according to its definition \ref{def:QAP-target-poly}. After that use Lagrange's method \ref{alg_lagrange_interplation} to compute the polynomials $A_j$ for every $1\leq j \leq k$ from the set 
\begin{equation}
S_{A_j} = \{(m_1,a^1_j),\ldots,(m_k,a^k_j)\}
\end{equation}
After that is done, execute the analog computation for the polynomials $B_j$ and $C_j$ for every $1\leq j \leq k$. 
\begin{example}[3-factorization]
\label{ex:3-fac-QAP}  To provide a better intuition of Quadratic Arithmetic Programs and how they are computed from their associated \concept{rank-1 constraint systems}, consider the language $L_{3.fac\_zk}$ from \examplename{} \ref{ex:L-3fac-zk} and its associated R1CS from \examplename{} \ref{ex:3-factorization-r1cs}:
\begin{align*}
W_1 \cdot W_2 & = W_4 & \text{constraint } 1\\
W_4 \cdot W_3 & = I_1 & \text{constraint } 2
\end{align*}
In this example, we want to transform this R1CS into an associated QAP. According to \examplename{} \ref{ex:3-factorization-r1cs} the defining constants $a_j^i$, $b_j^i$ and $c_j^i$ of the R1CS are given as follows:
$$
\begin{array}{llllll}
a_0^1 = 0 & a_1^1= 0 & a_2^1= 1 & a_3^1 = 0 & a_4^1= 0  & a_5^1= 0 \\ 
a_0^2 = 0 & a_1^2= 0 & a_2^2= 0 & a_3^2 = 0 & a_4^2= 0  & a_5^2= 1 \\ 
\\
b_0^1 = 0 & b_1^1= 0 & b_2^1= 0 & b_3^1 = 1 & b_4^1= 0  & b_5^1= 0 \\ 
b_0^2 = 0 & b_1^2= 0 & b_2^2= 0 & b_3^2 = 0 & b_4^2= 1  & b_5^2= 0 \\ 
\\
c_0^1 = 0 & c_1^1= 0 & c_2^1= 0 & c_3^1 = 0 & c_4^1= 0  & c_5^1= 1 \\ 
c_0^2 = 0 & c_1^2= 1 & c_2^2= 0 & c_3^2 = 0 & c_4^2= 0  & c_5^2= 0 
\end{array} 
$$
Since the R1CS is defined over the field $\F_{13}$ and since it has two constraints, we need to choose two arbitrary but invertible and distinct elements $m_1$ and $m_2$ from $\F_{13}$. We choose $m_{1}=5$, and $m_{2}=7$ and with this choice we get the target polynomial
\begin{align*}
T(x) & = (x-m_1)(x-m_2) & \text{\# Definition of T}\\
     & = (x-5)(x-7)  & \text{\# Insert our choice}\\
     & = (x+8)(x+6)  & \text{\# Negatives in } \F_{13}\\
     & = x^2 + x +9 & \text{\# expand}
\end{align*}
Then we have to compute the polynomials $A_j$, $B_j$ and $C_j$ by their defining equation from the R1CS coefficients. Since the R1CS has two constraining equations, those polynomials are of degree $1$ and they are defined by their evaluation at the point $m_1=5$ and the point $m_2=7$. 

At point $m_1$, each polynomial $A_j$ is defined to be $a_j^1$ and at point $m_2$, each polynomial $A_j$ is defined to be $a_j^2$. The same holds true for the polynomials $B_j$ as well as $C_j$. Writing all these equations down, we get:
$$
\begin{array}{llllll}
A_0(5)=0, & A_1(5)=0, & A_2(5)=1, & A_3(5)=0, & A_4(5)=0, & A_5(5)=0 \\
A_0(7)=0, & A_1(7)=0, & A_2(7)=0, & A_3(7)=0, & A_4(7)=0, & A_5(7)=1\\
\\
B_0(5)=0, & B_1(5)=0, & B_2(5)=0, & B_3(5)=1, & B_4(5)=0, & B_5(5)=0 \\
B_0(7)=0, & B_1(7)=0, & B_2(7)=0, & B_3(7)=0, & B_4(7)=1, & B_5(7)=0\\
\\
C_0(5)=0, & C_1(5)=0, & C_2(5)=0, & C_3(5)=0, & C_4(5)=0, & C_5(5)=1 \\
C_0(7)=0, & C_1(7)=1, & C_2(7)=0, & C_3(7)=0, & C_4(7)=0, & C_5(7)=0
\end{array}
$$
Lagrange's interpolation implies that a polynomial of degree $k$, that is zero on $k+1$ points has to be the zero polynomial. Since our polynomials are of degree $1$ and determined on $2$ points, we therefore know that the only non-zero polynomials in our QAP are $A_2$, $A_5$, $B_3$, $B_4$, $C_1$ and $C_5$, and that we can use Lagrange's interpolation to compute them. 

To compute $A_2$ we note that the set $S_{A_2}$ in our version of Lagrange's interpolation  is given by $S_{A_2}=\{(m_1,a^1_2), (m_2,a_2^2)\} = \{(5,1), (7,0)\}$. Using this set we get:
\begin{align*}
A_2(x) & = a^1_2\cdot(\frac{x-m_2}{m_1-m_2}) + a^2_2\cdot(\frac{x-m_1}{m_2-m_1})
      = 1\cdot(\frac{x-7}{5-7}) + 0\cdot(\frac{x-5}{7-5}) \\
    & = \frac{x-7}{-2}
      = \frac{x-7}{11} & \text{\# } 11^{-1}=6 \\
    & = 6(x-7) 
      = 6x + 10 & \text{\# } -7 = 6 \text{ and } 6\cdot 6 = 10
\end{align*}
To compute $A_5$, we note that the set $S_{A_5}$ in our version of Lagrange's method  is given by $S_{A_5}=\{(m_1,a^1_5), (m_2,a^2_5)\} = \{(5,0), (7,1)\}$. Using this set we get:
\begin{align*}
A_5(x) & = a^1_5\cdot(\frac{x-m_2}{m_1-m_2}) + a^2_5\cdot(\frac{x-m_1}{m_2-m_1})
      = 0\cdot(\frac{x-7}{5-7}) + 1\cdot(\frac{x-5}{7-5}) \\
    & = \frac{x-5}{2} & \text{\# } 2^{-1}=7 \\
    & = 7(x-5) 
      = 7x + 4 & \text{\# } -5 = 8 \text{ and } 7\cdot 8 = 4
\end{align*}
Using Lagrange's interpolation, we can deduce that $A_2=B_3=C_5$ as well as $A_5=B_4=C_1$, since they are polynomials of degree $1$ that evaluate to same values on $2$ points. Using this, we get the following set of polynomials
\begin{center}
\begin{tabular}{|l|l|l|}\hline 
$A_{0}(x)=0 $ &$ B_{0}(x)=0   $ & $C_{0}(x)=0$ \tabularnewline\hline 
$A_1(x)=0 $ &$ B_1(x)=0   $ & $C_1(x)=7x+4$ \tabularnewline\hline 
$A_2(x)=6x+10$ &$ B_2(x)=0$ & $C_2(x)=0$ \tabularnewline\hline 
$A_3(x)=0    $ &$ B_3(x)=6x+10$ & $C_3(x)=0$ \tabularnewline\hline 
$A_4(x)=0$ &$ B_4(x)=7x+4  $ & $C_4(x)=0$ \tabularnewline\hline 
$A_5(x)=7x+4$ &$ B_5(x)=0      $ & $C_5(x)=6x+10$ \tabularnewline\hline 
\end{tabular}
\end{center}
We can use Sage to verify our computation. In Sage, every polynomial ring has a function \code{lagrange\_polynomial} that takes the defining points as inputs and the associated Lagrange polynomial as output.
\begin{sagecommandline}
sage: F13 = GF(13)
sage: F13t.<t> = F13[]
sage: T = F13t((t-5)*(t-7))
sage: A2 = F13t.lagrange_polynomial([(5,1),(7,0)])
sage: A5 = F13t.lagrange_polynomial([(5,0),(7,1)])
sage: T == F13t(t^2 + t + 9)
sage: A2 == F13t(6*t + 10)
sage: A5 == F13t(7*t + 4)
\end{sagecommandline}

Combining this computation with the target polynomial we derived earlier, a Quadratic Arithmetic Program associated to the \concept{rank-1 constraint system} $R_{3.fac\_zk}$ is given as follows:

\begin{multline}
\label{QAP-R3-fac-zk}
QAP(R_{3.fac\_zk}) =\{x^{2}+x+9,\notag\\
 \{0,0,6x+10,0,0,7x+4\},\{0,0,0,6x+10,7x+4,0\},\{0,7x+4,0,0,0,6x+10\}\}
\end{multline}
\end{example}
\begin{exercise}
Consider the \concept{rank-1 constraint system} for points on the \curvename{Tiny-jubjub} curve from \examplename{} \ref{ex:TJJ-r1cs}. Compute an associated QAP for this R1CS and double check your computation using sage.
\end{exercise}
\subsubsection{QAP Satisfiability} One of the major points of Quadratic Arithmetic Programs in proving systems is that solutions of their associated \concept{rank-1 constraint systems} are in 1:1 correspondence with certain polynomials $P$ divisible by the target polynomial $T$ of the QAP. Verifying solutions to the R1CS and hence, checking proper circuit execution is then achievable by polynomial division of $P$ by $T$.

To be more specific, let $R$ be some \concept{rank-1 constraint system} with associated variables $(<I_1,\ldots,I_n>; <W_1,\ldots, W_m>)$ and let $QAP(R)$ be a Quadratic Arithmetic Program of $R$. Then the string $(<I_1,\ldots,I_n>; <W_1,\ldots, W_m>)$ is a solution to the R1CS if and only if the following polynomial is divisible by the target polynomial $T$:
\begin{equation}\label{polynomial-P-IW}
P_{(I;W)} = \scriptstyle \left(A_0 + \sum_{j}^n I_j\cdot A_j + \sum_{j}^m W_j\cdot A_{n+j} \right) \cdot \left(B_0 + \sum_{j}^n I_j\cdot B_j + \sum_{j}^m W_j\cdot B_{n+j} \right) 
-\left(C_0 + \sum_{j}^n I_j\cdot C_j + \sum_{j}^m W_j\cdot C_{n+j} \right)
\end{equation}

To understand how Quadratic Arithmetic Programs define formal languages, observe that every QAP over a field $\F$ defines a decision function over the alphabet $\Sigma_I \times \Sigma_W = \F \times \F$ in the following way:
\begin{equation}
R_{QAP} : (\F)^* \times (\F)^* \to \{true, false\}\;;\;
(I;W) \mapsto
\begin{cases}
true & P_{(I;W)} \text{ is divisible by } T\\
false & else
\end{cases}
\end{equation}
This means that every QAP defines a formal language $L_{QAP}$, and, if the QAP is associated to an R1CS, the language generated by the QAP and the language generated by the R1CS are equivalent. In the context of languages generated by Quadratic Arithmetic Programs, a \term{statement} is then a membership claim ``There is a word $(I;W)$ in $L_{QAP}$''. A proof to this claim is therefore given by a polynomial $P_{(I;W)}$, which is verified by dividing $P_{(I;W)}$ by $T$.

Note the structural similarities and differences in the definition of an R1CS and its associated language in \ref{R1CS}, of circuits and their associated languages in \ref{sec:circuits} and of QAPs and their associated languages as explained in this part. For circuits and their associated \concept{rank-1 constraint systems}, a constructive proof consists of a valid assignment of field elements to the edges of the circuit, or the variables in the R1CS. However, in the case of QAPs, a valid proof consists of a polynomial $P_{(I;W)}$.

To compute a constructive proof for a statement in $L_{QAP}$ given some instance $I$, a prover first needs to compute a constructive proof $W$ of the associated R1CS, e.g. by executing the circuit of the R1CS. With $(I;W)$ at hand, the prover can then compute the polynomial $P_{(I;W)}$ and publish the polynomial as proof.

Verifying a constructive proof in the case of a circuit is achieved by executing the circuit and then by comparing the result against the given proof. Verifying the same proof in the R1CS picture means checking if the elements of the proof satisfy the R1CS equations. In contrast, verifying a proof in the QAP picture is done by polynomial division of the proof $P$ by the target polynomial $T$. The proof is verified if and only if $P$ is divisible by $T$.

\begin{example} Consider the Quadratic Arithmetic Program $QAP(R_{3.fac\_zk})$ from \examplename{} \ref{ex:3-fac-QAP} and its associated R1CS from equation \ref{ex:3-factorization-r1cs}. To give an intuition of how proofs in the language $L_{QAP(R_{3.fac\_zk})}$ look like, lets consider the instance $I_1=11$. As we know from \examplename{} \ref{ex:3-fac-zk-circuit_2}, $(W_1,W_2,W_3,W_5)=(2,3,4,6)$ is a proper witness, since 
$(<I_1>;<W_1,W_2,W_3,W_5>)=(<11>;<2,3,4,6>)$ is a valid circuit assignment and hence, a solution to $R_{3.fac\_zk}$ and a constructive proof for language $L_{R_{3.fac\_zk}}$. 

In order to transform this constructive proof into a knowledge proof in language $L_{QAP(R_{3.fac\_zk})}$, a prover has to use the elements of the constructive proof, to compute the polynomial $P_{(I;W)}$. 

In the case of $(<I_1>;<W_1,W_2,W_3,W_5>)=(<11>;<2,3,4,6>)$,  the associated proof is computed as follows: 
\begin{align*}
P_{(I;W)}  = & \scriptstyle \left(A_0 + \sum_{j}^n I_j\cdot A_j + \sum_{j}^m W_j\cdot A_{n+j} \right) \cdot \left(B_0 + \sum_{j}^n I_j\cdot B_j + \sum_{j}^m W_j\cdot B_{n+j} \right) 
-\left(C_0 + \sum_{j}^n I_j\cdot C_j + \sum_{j}^m W_j\cdot C_{n+j} \right)\\
= & (2(6x+10)+6(7x+4))\cdot(3(6x+10)+4(7x+4))-(11(7x+4)+6(6x+10)) \\
= & ((12x+7)+(3x+11))\cdot((5x+4)+(2x+3))-((12x+5)+(10x+8)) \\
= & (2x+5)\cdot(7x+7)-(9x) \\
= & (x^{2}+2\cdot7x+5\cdot7x+5\cdot7)-(9x) \\
= & (x^{2}+x+9x+9)-(9x) \\
= & x^{2}+x+9
\end{align*}
Given instance $I_1=11$ a prover therefore provides the polynomial $x^2+x+9$ as proof. To verify this proof, any verifier can then look up the target polynomial $T$ from the QAP and divide $P_{(I;W)}$ by $T$. In this particular example, $P_{(I;W)}$ is equal to the target polynomial $T$, and hence, it is divisible by $T$ with $P/T=1$. The verifer therefore verifies the proof.
\begin{sagecommandline}
sage: F13 = GF(13)
sage: F13t.<t> = F13[]
sage: T = F13t(t^2 + t + 9)
sage: P = F13t((2*(6*t+10)+6*(7*t+4))*(3*(6*t+10)+4*(7*t +4))-(11*(7*t+4)+6*(6*t+10)))
sage: P == T
sage: P % T # remainder
\end{sagecommandline}

To give an example of a false proof, consider the string $(<I_1>;<W_1,W_2,W_3,W_4>)=(<11>, <2, 3, 4, 8>)$. Executing the circuit, we can see that this is not a valid assignment and not a solution to the R1CS, and hence, not a constructive knowledge proof in $L_{3.fac\_zk}$. However,  a prover might use these values to construct a false proof $P_{(I;W)}$:
\begin{align*}
P'_{(I;W)}  = & \scriptstyle \left(A_0 + \sum_{j}^n I_j\cdot A_j + \sum_{j}^m W_j\cdot A_{n+j} \right) \cdot \left(B_0 + \sum_{j}^n I_j\cdot B_j + \sum_{j}^m W_j\cdot B_{n+j} \right) 
-\left(C_0 + \sum_{j}^n I_j\cdot C_j + \sum_{j}^m W_j\cdot C_{n+j} \right)\\
= & (2(6x+10)+8(7x+4))\cdot(3(6x+10)+4(7x+4))-(8(6x+10)+11(7x+4)) \\
= & 8x^{2}+6
\end{align*}
Given instance $I_1=11$, a prover therefore provides the polynomial $8x^2+6$ as proof. To verify this proof, any verifier can look up the target polynomial $T$ from the QAP and divide $P_{(I;W)}$ by $T$. However, polynomial division has the following remainder:
$$
(8x^{2}+6)/(x^{2}+x+9) =8+\frac{5x+12}{x^{2}+x+9} 
$$
This implies that $P_{(I;W)}$ is not divisible by $T$, and hence, the verifier does not verify the proof. Any verifier can therefore show that the proof is false.
\begin{sagecommandline}
sage: F13 = GF(13)
sage: F13t.<t> = F13[]
sage: T = F13t(t^2 + t + 9)
sage: P = F13t((2*(6*t+10)+8*(7*t+4))*(3*(6*t+10)+4*(7*t+4))-(8*(6*t+10)+11*(7*t+4)))
sage: P == F13t(8*t^2 + 6)
sage: P % T # remainder
\end{sagecommandline}

\end{example}
