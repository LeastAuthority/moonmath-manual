\chapter{Software Used in This Book}

\section{Sagemath}
\label{sagemath_setup}
SageMath, also known as Sage, is a free and open-source software system that provides a comprehensive collection of mathematical tools and features. It offers a unified interface to various mathematical libraries and tools, including finite field arithmetic, elliptic curves, and cryptographic primitives, among others. With SageMath, users have access to a powerful and efficient platform for performing computations, analyzing data, and visualizing results in a wide range of mathematical domains. It order to provide an interactive learning experience, and to allow getting hands-on with the concepts described in this book, we give examples for how to program them in \href{https://www.sagemath.org/}{Sage}. Sage is based on the learning-friendly programming language \href{https://www.python.org/}{Python},  extended and optimized for computations involving algebraic objects. Therefore, we recommend installing Sage before diving into the following chapters.

The installation steps for various system configurations are described on the \href{https://doc.sagemath.org/html/en/installation/index.html}{Sage website}. Note that we use Sage version 9, so if you are using Linux and your package manager only contains version 8, you may need to choose a different installation path, such as using prebuilt binaries. If you are not familiar with SageMath, we recommend you consult the \href{https://doc.sagemath.org/html/en/tutorial/index.html}{Sage Tutorial}.

\section{Circom}
\label{circom_setup}
Circom is a programming language and compiler for designing arithmetic circuits. It provides a platform for programmers to create their own circuits, which can then be compiled to Rank-1 Constraint Systems (R1CS) and outputted as WebAssembly and C++ programs for efficient evaluation. The open-source library, CIRCOMLIB, offers a collection of pre-existing circuit templates that can be utilized by Circom users.

In order to compile our pen-and-paper calculations into real-world zk-SNARKs, we provide examples for how to implement them using Circom. Therefore, it is recommended to install Circom before exploring the following chapters. To generate and verify zk-SNARKS for our Circom circuits, we utilize the SNARK.js library, which is a JavaScript and Pure Web Assembly implementation of ZK-SNARK schemes. It employs the Groth16 Protocol and PLONK. The installation steps can be found at the Circom installation page \href{https://docs.circom.io/getting-started/installation/#installing-circom}{Circom installation}.

