\chapter{Introduction}
% Note: I want to avoid using links or \term{}'s or anything like this here. This is just an introduction. All the terms are defined later. Lets be as little formal as possible here
In cryptography so called \textit{zero-knowledge proofs} or \textit{zero-knowledge protocols} are a class of protocols by which one party called the prover can prove to other parties called the verifiers that a given statement is true without revealing any additional information apart from the fact that the statement is indeed true. It is the purpose of this book to introduce the mathematics behind those proving systems and their implementations to an audience with little to no knowledge in this field of research.

In this context, so called \textit{zero knowledge succinct, non interactive arguments of knowledge} (zk-SNARKs) are of particular interest, since the size of a zk-SNARK is much smaller than the size of the original data necessary to know that the statement is true and the prover can send a single message to the verifiers in order to convince them.

From a practical point of view this is interesting, because zk-SNARKs are able to prove honest computation to the public without revealing the inputs to that computation, by sending a single short transaction to a verifier that is implemented as a smart contract on a public blockchain. In this case it is possible to outsource heavy computation off-chain and then to publically verify the correctness of that computation on-chain. This enables publicly verifiable computations, blockchain scaling and increases transaction privacy.  

Because of this connection between blockchains and zk-SNARKs, increased interest in blockchain technology also increases the need for better and detailed understanding of zero knowledge protocols, their real world implementations and applications as well as their standards. Only then are developers able to implement secure and high quality code.

However the details of zero knowledge proofs are complex and a deeper understanding requires insight into various mathematical and computer theoretical disciplines, including alternatives to well established computational models and programming paradigms. Unfortunately resources are often scattered across blog posts, github libraries and mathematical papers and as a result zk-SNARKs remain somewhat elusive or ''magical'' and are therefore sometimes coined as ''moon math''. This increases the barrier of entry and deters developers from exploring or utilizing them in projects, thereby slowing the widespread adoption of the technology and the transition into web3.

The MoonMath Manual to zk-SNARKs is an attempt to change this as the book is specifically designed for an audience with only minimal experience in cryptography. The goal is to close knowledge gaps and to explain abstract concepts to developers in a hands-on way using simple pen-and-paper computations. As users go through the manual and calculate examples, they will grasp concepts necessary to understand the mathematics behind zk-SNARKs.

\section{Target audience}
This book is targeted primarily to software and smart contract developers who want to understand the internals of zk-SNARKs in order to be able to develop high quality, high security code, or who want to close some knowledge gaps. It is accessible for both beginners and experienced readers alike as concepts are gradually introduced in a logical and steady pace. It is assumed though, that the reader has a basic understanding of programming and enthusiasm as well as affinity for logical thinking and strategic problem solving.


\section{About the book}
How much mathematics do you need to understand and implement zero-knowledge proofs? The answer, of course, depends on the level of understanding you aim for and the level of security your application requires. It is possible to implement zero-knowledge proofs without any understanding of its underlying mathematics; however, to read a foundational paper, to understand the internals of a proof system, or to implement a secure and high quality zk-SNARK, some knowledge of mathematics is needed. 

Without a solid grounding in mathematics, someone who is interested in learning the concepts of zero-knowledge proofs, but who has never seen or dealt with, say, a prime field, or an elliptic curve, may quickly become overwhelmed. This is not so much due to the complexity of the mathematics needed, but rather because of the vast amount of technical jargon, unknown terms, and obscure symbols that quickly makes a text unreadable, even though the concepts themselves are not actually that complicated. As a result, the reader might either lose interest, or pick up some incoherent bits and pieces of knowledge that, in the worst case scenario, result in immature and non-secure implementations. 

This is why the book is dedicated large parts to explaining the mathematical foundations needed to understand the basic concepts underlying zk-SNARK development. We encourage the reader who is not familiar with basic number theory and elliptic curves to take the time and read the appropriate chapters in this book until they are able to solve at least a few exercises in each chapter. A great emphasis should be put on going through the examples in all details.

The book starts at a very basic level, and only assume pre-existing knowledge of fundamental concepts like high school integer arithmetic. Then it will show you that there are numbers and mathematical structures that appear at first to be very different from what you learned about in high school, but - on a deeper level - are actually quite similar. This will be illustrated in various examples throughout the book. 

It is worth to stress though, that the mathematics in this book is informal, incomplete and optimized to enable the reader to understand zero-knowledge concepts as efficiently as possible. The design choice is to include as little theory as necessary, focusing on a wealth of numerical examples. We believe that such an informal, example-driven approach makes it easier for beginners to digest the material in the initial stages. 

As a beginner, you would probably find it more beneficial to first compute a simple toy zk-SNARK with pen and paper all the way through, before you actually develop high security real-world zk-SNARKs. 

However, in order to be able to derive these toy examples, some mathematical groundwork is needed. This book therefore will help the inexperienced reader to focus on what we believe is important, accompanied by exercises that you are encouraged to recompute yourself. Every section contains a list of  exercises in increasing order of difficulty, to help you memorize and apply the concepts. 


\section{How to read this book}
Since the MoonMath Manual aims to become a complete introduction to all topics relevant for beginners, there are ways of reading it. The first and most obvious one is linear, following the order of the chapters. We recommend this if you have little pre-existing knowledge of mathematics and cryptography. We start with basic concepts like natural numbers, prime numbers, and how we can perform operations on such sets in different kinds of arithmetic. Then we move on to algebraic constructs like groups, rings, prime fields and elliptic curves.

Early in the book, we develop examples that we gradually extend with the things we learn in each chapter. This way, we incrementally build a few real-world SNARKs over full-fledged cryptographic systems that are nevertheless simple enough to be computed by pen and paper to illustrate all steps in great detail.

Readers who mainly want to get a better understanding of elliptic curves as they arise in zero-knowledge proving systems might want to start with our introduction of the $BLS6\_6$ curve in \secname{} \ref{BLS6}, which is a pen-and-paper, pairing-friendly elliptic curve. All concepts needed to understand this curve are introduced in \chaptname{} \ref{chap:elliptic-curves}. Programmers who develop zk-SNARKs frequently face the situation where it would be useful do some pen-and-paper computations before implementing the real thing, however those calculations are often prohibited by the size of cryptographic elliptic curves. In this book, we specifically designed the $BLS6\_6$ curve for this purpose. It is a pairing friendly curve that has all the properties needed to do pairing-based computations without any computer assistance, which often helps to understand delicate details of your system. In addition we derive the \curvename{Tiny-jubjub} curve \ref{TJJ13-twisted-edwards}, which can be used for pen-and-paper EdDSA in circuits over $BLS6\_6$. 


Readers who are primarily interested in building a simple pen-and-paper zk-SNARK from scratch in order to close their knowledge gaps might want to go through all the examples regarding our 3-factorization problem. In \examplename{} \ref{ex:3-factorization}, we introduce the 3-factorization problem as a statement in a formal language. If this is too abstract, the reader might start in \examplename{} \ref{ex:3-fac-zk-circuit}, where we describe the 3-factorization problem as an algebraic circuit. In \examplename{} \ref{ex:3-fac-zk-circuit_2} we execute the circuit. In \examplename{} \ref{ex:L-3fac-zk}, we introduce the concept of instance and witness into the problem in order to achieve various levels of zero knowledge later on. In \examplename{} \ref{ex:3-factorization-r1cs} we transform the circuit into an associated Rank-1 Constraint System and in \examplename{} \ref{ex:3-fac-R1CS-constr-proof} we compute a constructive proof for the problem. In \examplename{} \ref{ex:3-fac-QAP} we transform that constraint system into a Quadratic Arithmetic Program in and show how constructive proofs are transformed into polynomial divisibility problems. In \examplename{} \ref{ex:3-fac-groth-16-params}, we  use the result of those examples to derive a Groth16 zk-SNARK for the 3-factorization problem. In \ref{def:groth16-crs}, we compute the prover and the verifier key. In \examplename{} \ref{3-fac-snark-compute}, we compute a zk-SNARK and in \examplename{} \ref{3-fac-snark-verifier}, we verify that zk-SNARK. In \examplename{}, \ref{3-fac-snark-simulator} we show how to simulate proofs. 

Reader who want to get a better understanding of how high level programs are compiled into representations which zero knowledge proof systems accept, might want to start in chapter \ref{chap:circuit-compilers}, where we derive a toy language with a ''brain-compiler'' that compiles high level code into graphical circuit representations. Important representations like the R1CS are explained in section \ref{sec:R1CS} and core concepts of constructive proofs, witness and instance are explained in \ref{sec:formal-languages}.   

\section{Contributions} As zero knowledge proofing systems are a vital and fast developing field of research, getting the moonmath manual to a point where it includes most relevant topics is a huge task and hence community contributions are very welcome.

If you wish to participate in the development of the moonmath manual, we encourage you to submit adapted material or material created independently to Least Authority. If you are interested in adding a major contribution to the moonmath manual, please contact Least Authority directly under mmm@leastauthority.com. See our license for more details on this.

%rest is commented out

\begin{comment}
\section{The Zoo of Zero-Knowledge Proofs}

{First, a list of zero-knowledge proof systems:

\begin{enumerate}
	\item Pinocchio (2013): \href{https://eprint.iacr.org/2013/279.pdf}{{Paper}}
	\begin{itemize}[label={--}]
		\item Notes: trusted setup
	\end{itemize}

	\item BCGTV (2013): \href{https://eprint.iacr.org/2013/507.pdf}{{Paper}}
	\begin{itemize}[label={--}]
		\item Notes: trusted setup, implementation
	\end{itemize}

	\item BCTV (2013): \href{https://eprint.iacr.org/2013/879.pdf}{{Paper}}
	\begin{itemize}[label={--}]
		\item Notes: trusted setup, implementation
	\end{itemize}

	\item Groth16 (2016): 	\href{https://eprint.iacr.org/2016/260.pdf}{Paper}
	\begin{itemize}[label={--}]
		\item Notes: trusted setup
		\item Other resources: \href{https://www.gakonst.com/zksummit2019.pdf}{Talk in 2019 by Georgios Konstantopoulos}
	\end{itemize}

	\item GM17 (207): 	\href{https://eprint.iacr.org/2017/540.pdf}{Paper}
	\begin{itemize}[label={--}]
		\item Notes: trusted setup
		\item Other resources: later \href{https://eprint.iacr.org/2018/187}{Simulation extractability in ROM, 2018}
	\end{itemize}

	\item Bulletproofs (2017): \href{https://eprint.iacr.org/2017/1066.pdf}{Paper}
	\begin{itemize}[label={--}]
		\item Notes: no trusted setup
		\item Other resources: \href{https://eprint.iacr.org/2016/263.pdf}{Polynomial Commitment Scheme on DL, 2016} and \href{https://www.iacr.org/archive/asiacrypt2010/6477178/6477178.pdf}{KZG10, Polynomial Commitment Scheme on Pairings, 2010}
	\end{itemize}

	\item Ligero (2017): \href{https://acmccs.github.io/papers/p2087-amesA.pdf}{Paper}
	\begin{itemize}[label={--}]
		\item Notes: no trusted setup
		\item Other resources: 
	\end{itemize}

	\item Hyrax (2017): \href{https://eprint.iacr.org/2017/1132.pdf}{Paper}
	\begin{itemize}[label={--}]
		\item Notes: no trusted setup
		\item Other resources: 
	\end{itemize}

	\item STARKs (2018): \href{https://eprint.iacr.org/2018/046.pdf}{Paper}
	\begin{itemize}[label={--}]
		\item Notes: no trusted setup 
		\item Other resources: 
	\end{itemize}

	\item Aurora (2018): \href{https://eprint.iacr.org/2018/828.pdf}{Paper}
	\begin{itemize}[label={--}]
		\item Notes: transparent SNARK
		\item Other resources:
	\end{itemize}

	\item Sonic (2019): \href{https://eprint.iacr.org/2019/099.pdf}{Paper}
	\begin{itemize}[label={--}]
		\item Notes: SNORK - SNARK with universal and updateable trusted setup, PCS-based
		\item Other resources: \href{https://www.benthamsgaze.org/2019/02/07/introducing-sonic-a-practical-zk-snark-with-a-nearly-trustless-setup/}{Blog post by Mary Maller from 2019} and \href{https://eprint.iacr.org/2018/280}{work on updateable and universal setup from 2018}
	\end{itemize}

	\item Libra (2019): \href{https://eprint.iacr.org/2019/317}{Paper}
	\begin{itemize}[label={--}]
		\item Notes: trusted setup
		\item Other resources:
	\end{itemize}

	\item Spartan (2019): \href{https://eprint.iacr.org/2019/550.pdf}{Paper}
	\begin{itemize}[label={--}]
		\item Notes: transparent SNARK
		\item Other resources:
	\end{itemize}

	\item PLONK (2019): \href{https://eprint.iacr.org/2019/953.pdf}{Paper}
	\begin{itemize}[label={--}]
		\item Notes: SNORK, PCS-based
		\item Other resources: \href{https://www.plonk.cafe/t/welcome-to-discussion-of-plonk-related-research/24}{Discussion on Plonk systems} and \href{https://github.com/Fluidex/awesome-plonk}{Awesome Plonk list}
	\end{itemize}

	\item Halo (2019): \href{https://eprint.iacr.org/2019/1021}{Paper}
	\begin{itemize}[label={--}]
		\item Notes: no trusted setup, PCS-based, recursive
		\item Other resources: 
	\end{itemize}

	\item Marlin (2019): \href{https://eprint.iacr.org/2019/1047.pdf}{Paper}
	\begin{itemize}[label={--}]
		\item Notes: SNORK, PCS-based
		\item Other resources: \href{https://github.com/arkworks-rs/marlin}{Rust Github}
	\end{itemize}

	\item Fractal (2019): \href{https://eprint.iacr.org/2019/1076.pdf}{Paper}
	\begin{itemize}[label={--}]
		\item Notes: Recursive, transparent SNARK
		\item Other resources: 
	\end{itemize}

	\item SuperSonic (2019): \href{https://eprint.iacr.org/2019/1229.pdf}{Paper}
	\begin{itemize}[label={--}]
		\item Notes: transparent SNARK, PCS-based
		\item Other resources: \href{https://eprint.iacr.org/2021/358}{Attack on DARK compiler in 2021}
	\end{itemize}

	\item Redshift (2019): \href{https://eprint.iacr.org/2019/1400}{Paper}
	\begin{itemize}[label={--}]
		\item Notes: SNORK, PCS-based
		\item Other resources: 
	\end{itemize}



\end{enumerate}

\textbf{Other resources on the zoo: } \href{https://github.com/matter-labs/awesome-zero-knowledge-proofs}{Awesome ZKP list on Github}, \href{https://zkp.science/}{ZKP community} with the \href{https://docs.zkproof.org/reference.pdf}{reference document}

}

\paragraph{To Do List}
\begin{itemize}
	\item Make table for prover time, verifier time, and proof size
	\item Think of categories - \textit{Achieved Goals}: Trusted setup or not, Post-quantum or not, \dots
	\item Think of categories - \textit{Mathematical background}: Polynomial commitment scheme, \dots
	\item \dots while we discuss the points above, we should also discuss a common notation/language for all these things. (E.g. transparent SNARK/no trusted setup/STARK)
\end{itemize}

\paragraph{Points to cover while writing}
\begin{itemize}
	\item Make a historical overview over the "discovery" of the different ZKP systems
	\item Make reader understand what paper is build on what result etc. - the tree of publications!
	\item Make reader understand the different terminology, e.g. SNARK/SNORK/STARK, PCS, R1CS, updateable, universal, $\dots$
	\item Make reader understand the mathematical assumptions - and what this means for the zoo.
	\item Where will the development/evolution go? What are bottlenecks?
\end{itemize}

\vspace*{1em}
{\footnotesize
\hspace*{-1em}\textbf{Other topics I fell into while compiling this list}
\begin{itemize}
	\item Vector commitments: \url{https://eprint.iacr.org/2020/527.pdf}
	\item Snarkl: \url{http://ace.cs.ohio.edu/~gstewart/papers/snaarkl.pdf}
	\item Virgo?: \url{https://people.eecs.berkeley.edu/~kubitron/courses/cs262a-F19/projects/reports/project5_report_ver2.pdf}
\end{itemize} 
}

\end{comment}

